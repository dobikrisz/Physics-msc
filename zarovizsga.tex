\documentclass[a4paper,12pt]{article}

\usepackage[utf8]{inputenc}
\usepackage[T1]{fontenc}
\usepackage{lmodern}
\usepackage{geometry}
\geometry{margin=1in}
\usepackage{amsmath, amssymb}
\usepackage{graphicx}
\usepackage{hyperref}
\usepackage{float}
\usepackage[makeroom]{cancel}
\usepackage{caption}
\usepackage{subcaption}
\usepackage{gensymb}

\title{Fizika BSc. tétel kidolgozás}
\author{Dobos Krisztián}
\date{\today}

\begin{document}

\maketitle

\section{A klasszikus mechanika alapjai}

Kinematikai alapfogalmak, mozgás leírása különböző koordináta-rendszerekben.
Newton-törvények, mozgásegyenlet, tehetetlen és súlyos tömeg. Gyorsuló koordináta-rendszerek (jelenségek a forgó Földön).
Munkatétel. Pontrendszerek. Merev testek: egyensúly feltétele, tehetetlenségi tenzor, pörgettyűk. Energia-, impulzus- és 
impulzusmomentum-megmaradási tételek tömegpontra és pontrendszerre. Galilei-, Lorentz-transzformáció, relativisztikus kinematika,
relativisztikus dinamika. Négyesimpulzus.
\subsection{Kinematikai alapfogalmak}
A kinematika a mozgások leírásával foglalkozó ága a fizikának. A szó maga Ampière francia fizikustól származik, aki a görög $\kappa \acute{\iota} \eta \varepsilon \mu \alpha$ (kinéma - mozgás) szóból alkotta meg a kinematika kifejezést.
A kinematika a mozgások "geometriájával" foglalkozik, azaz a mozgások leírására szolgáló matematikai eszközöket dolgozza ki. A kinematika nem foglalkozik a mozgás okával, azaz a dinamikával,
amely a mozgásokat előidéző erőkkel foglalkozik és a rendszer időfeejlődését írja le.

A kinematika alapvető egysége a pont, amelynek helyzetét egy adott időpillanatban egy koordináta-rendszerben megadott koordinátákkal jellemezhetjük.
\begin{equation}
    \bold{r}(t) = x(t)\hat{i} + y(t)\hat{j} + z(t)\hat{k}\\
\end{equation}
\begin{equation}
    | \bold{r}(t) | = \sqrt{x(t)^2 + y(t)^2 + z(t)^2}
\end{equation}

A pont helyzetének időbeli változását a sebesség és gyorsulás vektoraival adhatjuk meg:
\begin{equation}
    \bold{v}(t) = \lim_{\Delta t \rightarrow 0} \frac{\bold{r}(t + \Delta t) - \bold{r}(t)}{\Delta t} = \frac{d\bold{r}(t)}{dt} = \dot{\bold{r}}(t)
\end{equation}
\newline
\begin{equation}
    \bold{a}(t) = \lim_{\Delta t \rightarrow 0} \frac{\bold{v}(t + \Delta t) - \bold{v}(t)}{\Delta t} = \frac{d\bold{v}(t)}{dt} = \frac{d^2\bold{r}(t)}{dt^2} = \ddot{\bold{r}}(t)
\end{equation}
\newline
\begin{figure}[H]
    \centering
    \includegraphics[width=0.5\textwidth]{imgs/1-tetel/kinematika-pont.png}
    \caption{Pont mozgásának kinematikai jellemzői}
    \label{fig:kinematika}
\end{figure}
Ezek alapján látható, hogy a sebesség a pont helyzetének változása $t$ és $\Delta t$ között. Ha ezt a $\Delta t$ időintervallumot végtelenül kicsire csökkentjük, akkor megkapjuk a pont pillanatnyi sebességét.
Hasonlóképpen járhatunk el a sebességgel, hogy megkapjuk a gyorsulást, amely a sebesség időbeli változását jellemzi. Érdekesség, hogy a jelenlegi tudásunk szerint, a magasabb rendű időbeli deriváltaknak (pl. gyorsulás időbeli változása) nincs fizikai jelentése.
\newline
Mi van akkor, ha a gyorsulást ismerem? Például ha leejtek egy testet akkor empirikusan meg tudom határozni a gravitációs gyorsulást, ami a Föld felszínén nagyjából $g = 9.81 \frac{m}{s^2}$.
Ekkor a gyorsulás időbeli integrálásával megkaphatom a sebességet, majd a sebesség időbeli integrálásával a helyzetet.
\begin{equation}
    \bold{a}(t) = (0, 0, -g)
\end{equation}
\begin{equation}
    \int \bold{a}(t) dt = \bold{v}(t) = \bold{a}t + \bold{v_0} = (v_{x0}, v_{y0}, v_{z0}-gt)
\end{equation}
\begin{equation}
    \bold{r}(t) = \int \bold{v}(t) dt = \frac{1}{2}\bold{a}t^2 + \bold{v_0}t + \bold{r_0} = (v_{x0}t + x_0, v_{y0}t + y_0, -\frac{1}{2}gt^2 + v_{z0}t + z_0)
\end{equation}
Ahol $\bold{v_0}$ és $\bold{r_0}$ a kezdeti sebesség és helyzet vektora, vagyis a kezdőfeltételek. 
\subsubsection*{Példa: Harmonikus oszcillátor}
Vegyünk egy olyan rendszert, ahol egy testet egy ideális rugóval kötünk egy falhoz. Empirikusan megfigyelhető, hogy a rugó visszatérítő erőt fejt ki a testre, amely arányos a kitéréssel és ellentétes irányú vele. 
Ez alapján fel tudjuk írni a mozgásegyenletet:
\begin{equation}
    m\ddot{z} = -\omega_0^2 z
\end{equation}
Ahol $m$ a test tömege, $z$ a kitérés a nyugalmi helyzettől és $\omega_0$ a körfrekvencia, amely a rugóállandótól és a test tömegétől függ. A mozgásegyenlet egy másodrendű lineáris homogén differenciálegyenlet, amelynek megoldása: 
\begin{equation}
    z(t) = A \sin(\omega_0 t + \varphi) = Z_0\cos(\omega_0 t) + \frac{v_0}{\omega_0}\sin(\omega_0 t)
\end{equation}
Ahol $A$ a amplitúdó, $\varphi$ a kezdeti fázis, $Z_0$ a kezdeti kitérés és $v_0$ a kezdeti sebesség. A megoldás egy harmonikus oszcillációt ír le, amelynek frekvenciája $\omega_0$. Ez a megoldás egy "sejtésen" alapul, 
és a megoldást még validálni kell:
\begin{equation}
    \dot{z}(t) = A\omega_0 \cos(\omega_0 t + \varphi)
\end{equation}
\begin{equation}
    \ddot{z}(t) = -A\omega_0^2 \sin(\omega_0 t + \varphi)
\end{equation}
Látható, hogy a sejtés helyes, hiszen a második derivált megegyezik a mozgásegyenlet jobb oldalával. A megoldásban szereplő $\omega_0$-t a rugó határozza meg, míg $A$ és $\varphi$ a kezdeti feltételektől függ, amelyeket a következőképpen határozhatunk meg:
\begin{equation}
    z(0) = A \sin(\varphi) = Z_0
\end{equation}
\begin{equation}
    \dot{z}(0) = A\omega_0 \cos(\varphi) = v_0
\end{equation}
Ebből az $A$ és $\varphi$ kifejezhető a kezdeti feltételek segítségével:
\begin{equation}
    A = \sqrt{Z_0^2 + \frac{v_0^2}{\omega_0^2}}
\end{equation}
\begin{equation}
    \varphi = \arctan\left(\frac{Z_0 \omega_0}{v_0}\right)
\end{equation}
Ha pedig kicsit átalakítjuk a megoldásokat, akkor a következő alakot kapjuk:
\begin{equation}
    (\frac{z}{A})^2 = \sin^2(\omega_0 t + \varphi)
\end{equation}
\begin{equation}
    (\frac{\dot{z}}{A\omega_0})^2 = \cos^2(\omega_0 t + \varphi)
\end{equation}
Amiket ha összeadunk (ugyanis a differenciálegyenlet megoldásainak bármely lineárkombinációja is megoldás), akkor a következőt kapjuk és alkalmazzuk a trigonometrikus azonosságot ($\sin^2(x) + \cos^2(x) = 1$):
\begin{equation}
    (\frac{z}{A})^2 + (\frac{\dot{z}}{A\omega_0})^2 = 1
\end{equation}
Ez egy ellipszis egyenlete a $z$ és $\dot{z}$ koordinátákban, amely azt jelenti, hogy a harmonikus oszcillátor mozgása egy ellipszis pályán történik a fázistérben.

\subsection{mozgás leírása különböző koordináta-rendszerekben}
A mozgások leírására használt koordináta-rendszer mindig egy választás kérdése, amely a probléma természetétől és a kényelmi szempontoktól függ. A különböző koordináta-rendszerek feltétele, hogy egymással ekvivalensek legyenek,
azaz ugyanazt a fizikai helyzetet írják le. A különbséget a koordináták módjának megadása jelenti, hogy mely tulajdonságaival jellemzünk egy kijelölt pontot a térben. Mindig annyi koordinátára van szükség, ahány szabadságfoka van a rendszernek.
Például egy pont a háromdimenziós térben három szabadságfokkal rendelkezik, így három koordinátára van szükség a helyzetének meghatározásához. Egy merev testnek hat szabadságfoka van (három transzlációs és három rotációs),
így hat koordinátára van szükség a helyzetének meghatározásához. A leggyakrabban használt koordináta-rendszerek a következők:
\begin{itemize}
    \item Descartes-féle (kartéziánus) koordinátarendszer: A leggyakrabban használt koordináta-rendszer, amely három egymásra merőleges tengelyből áll (x, y, z). A pont helyzetét a három tengely mentén mért távolságokkal jellemezzük.
    \item Hengerkoordináta-rendszer: Egy olyan koordináta-rendszer, amely egy henger felületén helyezkedik el. A pont helyzetét egy sugárral (r), egy szöggel ($\varphi$) és egy magassággal (z) jellemezzük.
    \item Gömbkoordináta-rendszer: Egy olyan koordináta-rendszer, amely egy gömb felületén helyezkedik el. A pont helyzetét egy sugárral (r), egy polárisszöggel ($\theta$) és egy azimutszöggel ($\varphi$) jellemezzük.
\end{itemize}
A koordináta-rendszerek közötti átváltásokat a következő képletek segítségével végezhetjük el:
\begin{itemize}
    \item Descartes-féle és hengerkoordináta-rendszer közötti átváltás:
    \begin{equation}
        x = r \cos(\varphi)
    \end{equation}
    \begin{equation}
        y = r \sin(\varphi)
    \end{equation}
    \begin{equation}
        z = z
    \end{equation}
    \noindent\hfil\rule{0.5\textwidth}{.4pt}\hfil
    \begin{equation}
        r = \sqrt{x^2 + y^2}
    \end{equation}
    \begin{equation}
        \varphi = \arctan\left(\frac{y}{x}\right)
    \end{equation}
    \begin{equation}
        z = z
    \end{equation}
    \begin{figure}[H]
    \centering
    \includegraphics[width=0.5\textwidth]{imgs/1-tetel/descartes-henger.png}
    \caption{Descartes-féle és hengerkoordináta-rendszer átváltás}
    \label{fig:descartes-henger}
\end{figure}
    \item Descartes-féle és gömbkoordináta-rendszer közötti átváltás:
    \begin{equation}
        x = r \sin(\theta) \cos(\varphi)
    \end{equation}
    \begin{equation}
        y = r \sin(\theta) \sin(\varphi)
    \end{equation}
    \begin{equation}
        z = r \cos(\theta)
    \end{equation}
    \noindent\hfil\rule{0.5\textwidth}{.4pt}\hfil
    \begin{equation}
        r = \sqrt{x^2 + y^2 + z^2}
    \end{equation}
    \begin{equation}
        \theta = \arccos\left(\frac{z}{r}\right)
    \end{equation}
    \begin{equation}
        \varphi = \arctan\left(\frac{y}{x}\right)
    \end{equation}
\end{itemize}
\begin{figure}[H]
    \centering
    \includegraphics[width=0.5\textwidth]{imgs/1-tetel/coord-systems.png}
    \caption{különböző koordináta-rendszerek}
    \label{fig:koordinata-rendszerek}
\end{figure}
A megfelelő koordinátarendszer kiválasztása a probléma geometriájától és a szimmetriáitól függ. Például transzlációs mozgás esetén a Descartes-féle koordinátarendszer a lehet az alkalmasabb,
míg forgó mozgás esetén a henger- vagy gömbkoordináta-rendszer lehet előnyösebb. A koordináta-rendszerek közötti átváltások során ügyelni kell arra, hogy a vektorok komponensei is megváltoznak,
és a deriváltak is másképp néznek ki az adott koordináta-rendszerben. Például a sebesség és gyorsulás vektorok hengerkoordináta-rendszerben a következőképpen néznek ki: 
\begin{equation}
    \bold{r} = r\hat{e_r} + z\hat{e_z}
\end{equation}
\begin{equation}
     \hat{e_r} = \cos(\varphi)\hat{i} + \sin(\varphi)\hat{j}, \quad \hat{e_\varphi} = -\sin(\varphi)\hat{i} + \cos(\varphi)\hat{j}, \quad \hat{e_z} = \hat{k}
\end{equation}
\noindent\hfil\rule{0.5\textwidth}{.4pt}\hfil
\begin{equation}
    \bold{v} = \frac{d\bold{r}}{dt} = \dot{r}\hat{e_r} - r \cdot \dot{\varphi}sin(\varphi)\hat{i} + r \cdot \dot{\varphi}\cos(\varphi)\hat{j} + \dot{z}\hat{k} 
\end{equation}
\begin{equation}
    \bold{v} = \dot{r}\hat{e_r} + r\dot{\varphi}(- sin(\varphi)\hat{i} + \cos(\varphi)\hat{j}) + \dot{z}\hat{e_z}
\end{equation}
\begin{equation}
    \bold{v} = \dot{r}\hat{e_r} + r\dot{\varphi}\hat{e_\varphi} + \dot{z}\hat{e_z}
\end{equation}
Hasonlóan a gyorsulás vektora: 
\begin{equation}
    \bold{a} = (\ddot{r} - r\dot{\varphi}^2)\hat{e_r} + (r\ddot{\varphi} + 2\dot{r}\dot{\varphi})\hat{e_\varphi} + \ddot{z}\hat{e_z}
\end{equation}
Ahol $\hat{e_r}$, $\hat{e_\varphi}$ és $\hat{e_z}$ a hengerkoordináta-rendszer egységvektorai. A $\hat{e_\varphi}$ úgy van definiálva, hogy merőleges legyen a $\hat{e_r}$-re és a $\hat{e_z}$-re, és a jobbkéz-szabályt követi. 
A gömbkoordináta-rendszerben a sebesség és gyorsulás vektorok a következőképpen néznek ki:
\begin{equation}
    \bold{r} = r\hat{e_r}
\end{equation}
\begin{equation}
     \hat{e_r} = \sin(\theta)\cos(\varphi)\hat{i} + \sin(\theta)\sin(\varphi)\hat{j} + \cos(\theta)\hat{k}
\end{equation}
\begin{equation}
     \hat{e_\theta} = \cos(\theta)\cos(\varphi)\hat{i} + \cos(\theta)\sin(\varphi)\hat{j} - \sin(\theta)\hat{k}
\end{equation}
\begin{equation}
    \hat{e_\varphi} = -\sin(\varphi)\hat{i} + \cos(\varphi)\hat{j}
\end{equation}
\noindent\hfil\rule{0.5\textwidth}{.4pt}\hfil
\begin{equation}
    \bold{v} = \dot{r}\hat{e_r} + r\dot{\theta}\hat{e_\theta} + r\sin(\theta)\dot{\varphi}\hat{e_\varphi}
\end{equation}
\begin{equation}
    \bold{a} = (\ddot{r} - r\dot{\theta}^2 - r\sin^2(\theta)\dot{\varphi}^2)\hat{e_r} + (r\ddot{\theta} + 2\dot{r}\dot{\theta} - r\sin(\theta)\cos(\theta)\dot{\varphi}^2)\hat{e_\theta} + (r\sin(\theta)\ddot{\varphi} + 2\dot{r}\sin(\theta)\dot{\varphi} + 2r\cos(\theta)\dot{\theta}\dot{\varphi})\hat{e_\varphi}
\end{equation}
Ahol $\hat{e_r}$, $\hat{e_\theta}$ és $\hat{e_\varphi}$ a gömbkoordináta-rendszer egységvektorai. Ezek a kifejezések bonyolultabbak, mint a Descartes-féle koordináta-rendszerben, de bizonyos problémák esetén egyszerűbbé teszik a számításokat.
\subsubsection{természetes koordináta-rendszer}
A természetes koordináta-rendszer egy olyan koordináta-rendszer, amely a mozgó test pillanatnyi helyzetéhez és mozgásához igazodik. A természetes koordináta-rendszerben a mozgó test helyzetét a következő három koordinátával jellemezzük:
\begin{itemize}
    \item $s$: a pálya mentén mért távolság, amely a test helyzetét a pálya mentén határozza meg.
    \item $n$: a pályára merőleges irányú távolság, amely a test helyzetét a pályára merőlegesen határozza meg.
    \item $b$: a pályára merőleges irányú távolság, amely a test helyzetét a pályára merőlegesen határozza meg.
\end{itemize}
A természetes koordináta-rendszer előnye, hogy a mozgó test helyzetét és mozgását a pálya mentén és a pályára merőlegesen is jellemezhetjük, ami bizonyos problémák esetén egyszerűbbé teszi a számításokat.
A természetes koordináta-rendszerben a sebesség és gyorsulás vektorok a következőképpen néznek ki:
\begin{equation}
    \bold{v} = \dot{s}\hat{e_s} + \dot{n}\hat{e_n} + \dot{b}\hat{e_b}
\end{equation}
\begin{equation}
    \bold{a} = (\ddot{s} - \kappa \dot{s}^2)\hat{e_s} + (\ddot{n} + \kappa \dot{s}\dot{n})\hat{e_n} + (\ddot{b} + \kappa \dot{s}\dot{b})\hat{e_b}
\end{equation}
Ahol $\hat{e_s}$, $\hat{e_n}$ és $\hat{e_b}$ a természetes koordináta-rendszer egységvektorai, és $\kappa$ a pálya görbületi sugara. A természetes koordináta-rendszerben a sebesség és gyorsulás vektorok kifejezése bonyolultabb, mint a Descartes-féle koordináta-rendszerben, de bizonyos problémák esetén egyszerűbbé teszik a számításokat.
\begin{figure}[H]
    \centering
    \includegraphics[width=0.5\textwidth]{imgs/1-tetel/natural-coord.jpg}
    \caption{természetes koordináta-rendszer}
    \label{fig:natural-coord}
\end{figure}

\subsection{Newton-törvények, mozgásegyenlet, tehetetlen és súlyos tömeg}
A Mechanika alapjait Sir Isaac Newton fektette le a 17. században, amikor megalkotta a klasszikus mechanika három alapvető törvényét, amelyek a mozgások és erők közötti kapcsolatot írják le. Ezeket a tövényeket Newton a bolygók mozgásának megfigyelései alapján fogalmazta meg,
és azóta is a fizika egyik legfontosabb alapkövei. Newton a megfigyelései alapján négy axiómát fogalmazott meg, amelyek a következők:
\begin{itemize}
    \item Newton első törvénye (tehetetlenség törvénye): Létezik inerciarendszer. Az inerciarendszer egy olyan vonatkoztatási rendszer, amelyben egy test nyugalomban marad vagy egyenes vonalú egyenletes mozgást végez, ha nem hat rá külső erő. Ez azt jelenti, hogy egy test csak akkor változtatja meg mozgását, ha külső erő hat rá.
    \item Newton második törvénye (mozgásegyenlet): A testre ható erő egyenlő a test tömegének és gyorsulásának szorzatával. Matematikailag ez a következőképpen írható fel: $\bold{F} = m\bold{a}$, ahol $\bold{F}$ az erő, $m$ a tömeg és $\bold{a}$ a gyorsulás.
    \item Newton harmadik törvénye (hatás-ellenhatás törvénye): Minden hatásra van egy egyenlő és ellentétes ellenhatás. Ez azt jelenti, hogy ha egy test erőt fejt ki egy másik testre, akkor a másik test is erőt fejt ki az első testre, amely egyenlő nagyságú és ellentétes irányú. Tehát minden erőre létezik: $\bold{F_{12}} = - \bold{F_{21}}$
    \item Newton negyedik törvénye (a hatások függetlenségének elve): A testekre ható erők függetlenek és összeadódnak. Ez azt jelenti, hogy ha egy testre több erő hat, akkor a test mozgását a rá ható erők vektori összege határozza meg. Matematikailag ez a következőképpen írható fel: $\bold{F_{net}} = \sum \bold{F_i}$, ahol $\bold{F_{net}}$ a testre ható erők eredője, és $\bold{F_i}$ az egyes erők.
\end{itemize}

\subsubsection{mozgásegyenlet}
A mozgásegyenlet a Newton második törvényéből származtató egyenlet, amely a test mozgását írja le a rá ható erők alapján. Minden dinamikai problémára fel lehet írni a mozgásegyenletet, és bár $\bold{F} = m\bold{a}$ egy általános alak, mely inerciarendszerben érvényes,
a mozgásegyenletet különböző koordináta-rendszerekben is fel lehet írni és különböző kényszererők is megjelenhetnek benne. A mozgásegyenlet felírásának lépései a következők:
\begin{itemize}
    \item Válasszuk ki a vizsgálandó testet vagy rendszert, és határozzuk meg a tömegét ($m$).
    \item Válasszuk ki a megfelelő koordináta-rendszert, amelyben a mozgást leírjuk (pl. Descartes-féle, henger, gömb, természetes).
    \item Határozzuk meg a testre ható összes erőt ($\bold{F_i}$), beleértve a külső erőket (pl. gravitáció, súrlódás, rugóerő) és a kényszererőket (pl. kötél, felület).
    \item Írjuk fel a mozgásegyenletet a Newton második törvénye alapján: $\bold{F_{net}} = m\bold{a}$, ahol $\bold{F_{net}} = \sum \bold{F_i}$ az összes erő vektori összege.
    \item Oldjuk meg a mozgásegyenletet a gyorsulás ($\bold{a}$) kifejezésére, majd integráljuk idő szerint, hogy megkapjuk a sebességet ($\bold{v}$) és helyzetet ($\bold{r}$).
\end{itemize}
A néhány példa a mozgásegyenletre:
\begin{itemize}
    \item Lineáris mozgásegyenlet: $\bold{F} = m\bold{a}$, ahol $\bold{F}$ az erő, $m$ a tömeg és $\bold{a}$ a gyorsulás.
    \item Forgó mozgásegyenlet: $\bold{\tau} = \Theta\bold{\alpha}$, ahol $\bold{\tau}$ a nyomaték, $\Theta$ a tehetetlenségi nyomaték és $\bold{\alpha}$ a szöggyorsulás.
    \item Rugó mozgásegyenlet: $m\ddot{x} = -kx$, ahol $m$ a tömeg, $k$ a rugóállandó és $x$ a kitérés a nyugalmi helyzettől.
\end{itemize}
\subsubsection*{Példa: Test esése közegben}
Vegyünk egy olyan rendszert, ahol egy test esik szabadon a Föld felszínén, és közegellenállás is hat rá. A testre ható erők a következők:
\begin{itemize}
    \item Gravitációs erő: $\bold{F_g} = m\cdot \bold{g}$, ahol $\bold{g} = (0, 0, -9.81 \frac{m}{s^2})$ a gravitációs gyorsulás.
    \item Közegellenállási erő: $\bold{F_d} = -\lambda\bold{v}$, ahol $\lambda$ a közegellenállási együttható és $\bold{v}$ a test sebessége.
\end{itemize}
A mozgásegyenlet felírása:
\begin{equation}
    \bold{F_{net}} = \bold{F_g} + \bold{F_d} = m\bold{a}
\end{equation}
\begin{equation}
    m\cdot \bold{g} - \lambda\bold{v} = m\bold{a} = m\bold{\dot{v}}
\end{equation}
\begin{equation}
    \bold{\dot{v}} = \bold{g} - \frac{\lambda}{m}\bold{v}
\end{equation}
Ez egy elsőrendű lineáris differenciálegyenlet a sebességre, amelyet megoldhatunk a következőképpen:
\begin{equation}
    \frac{\lambda}{m} = \beta
\end{equation}
\begin{equation}
    \bold{\dot{v}} = \bold{g} - \beta\bold{v}
\end{equation}
Ahol $\beta$ a Boltzmann-állandó. A megoldáshoz először találjuk meg az integrálási faktort ami az ODE általános alakjából következik:
\begin{equation}
    \bold{\dot{v}} + P(t)\bold{v} = Q(t)
\end{equation}
Az integrrálási faktor pedig a következőképpen van definiálva:
\begin{equation}
    \mu(t) = e^{\int P(t) dt}
\end{equation}
Tehát ha átrendezzük az eredeti egyenletet, akkor a következőt kapjuk:
\begin{equation}
    \bold{\dot{v}} + \beta\bold{v} = \bold{g}
\end{equation}
Ahol az $P(t) = \beta$ és $Q(t) = \bold{g}$. Innen az integrálási faktor:
\begin{equation}
    \mu(t) = e^{\int \beta dt} = e^{\beta t}
\end{equation}
Most szorozzuk meg az eredeti egyenletet az integrálási faktorral:
\begin{equation}
    e^{\beta t}\bold{\dot{v}} + \beta e^{\beta t}\bold{v} = \bold{g}e^{\beta t}
\end{equation}
A bal oldalon alkalmazzuk a szorzat derivált szabályt:
\begin{equation}
    \frac{d}{dt}(e^{\beta t}\bold{v}) = \bold{g}e^{\beta t}
\end{equation}
Most integráljuk mindkét oldalt idő szerint:
\begin{equation}
    \int \frac{d}{dt}(e^{\beta t}\bold{v}) dt = \int \bold{g}e^{\beta t} dt
\end{equation}
\begin{equation}
    e^{\beta t}\bold{v} = \frac{\bold{g}}{\beta}e^{\beta t} + \bold{C}
\end{equation}
Ahol $\bold{C}$ az integrálási állandó, amelyet a kezdeti feltételekből határozhatunk meg. Most szorozzuk meg mindkét oldalt $e^{-\beta t}$-tel, hogy kifejezzük a sebességet:
\begin{equation}
    \bold{v} = \frac{\bold{g}}{\beta} + \bold{C}e^{-\beta t}
\end{equation}
Most alkalmazzuk a kezdeti feltételt, hogy meghatározzuk az integrálási állandót. Tegyük fel, hogy a test kezdeti sebessége $\bold{v_0}$, amikor $t = 0$:
\begin{equation}
    \bold{v_0} = \frac{\bold{g}}{\beta} + \bold{C}e^{0}
\end{equation}
\begin{equation}
    \bold{C} = \bold{v_0} - \frac{\bold{g}}{\beta}
\end{equation}
Most helyettesítsük vissza az integrálási állandót a sebesség egyenletbe:
\begin{equation}
    \bold{v} = \frac{\bold{g}}{\beta} + \left(\bold{v_0} - \frac{\bold{g}}{\beta}\right)e^{-\beta t}
\end{equation}
\begin{equation}
    \bold{v} = \frac{\bold{g}}{\beta}(1 - e^{-\beta t}) + \bold{v_0}e^{-\beta t}
\end{equation}
Ez a sebesség időbeli változását írja le egy test esése közegben, figyelembe véve a gravitációs erőt és a közegellenállást. Ebben az egyenletben a $\frac{\bold{g}}{\beta}$ a test végsebességét határozza meg.
A helyzetet úgy kaphatjuk meg, hogy integráljuk a sebességet idő szerint:
\begin{equation}
    \bold{r}(t) = \int \bold{v}(t) dt
\end{equation}
\begin{equation}
    \bold{r}(t) = \int \left[\frac{\bold{g}}{\beta}(1 - e^{-\beta t}) + \bold{v_0}e^{-\beta t}\right] dt
\end{equation}
\begin{equation}
    \bold{r}(t) = \frac{\bold{g}}{\beta}\left(t + \frac{1}{\beta}e^{-\beta t}\right) - \frac{\bold{g}}{\beta^2} + \frac{\bold{v_0}}{\beta}e^{-\beta t} + \bold{r_0} 
\end{equation}
Ahol $\bold{r_0}$ a kezdeti helyzet, amikor $t = 0$. Ez a helyzet időbeli változását írja le egy test esése közegben, figyelembe véve a gravitációs erőt és a közegellenállást.

\subsubsection{tehetetlen és súlyos tömeg}
A tehetetlen tömeg és a súlyos tömeg két különböző fogalom a fizikában, amelyek azonban meglepő módon numerikusan egyenlők egymással.
\begin{itemize}
    \item Tehetetlen tömeg ($m_i$): A tehetetlen tömeg Newton második törvényéből származik ($\bold{F} = m_i\bold{a}$), és azt méri, hogy egy test mennyire ellenáll a mozgásállapotának megváltoztatására irányuló erőnek. Minél nagyobb a tehetetlen tömeg, annál nehezebb megváltoztatni a test mozgását.
    \item Súlyos tömeg ($m_g$): A súlyos tömeg Newton gravitációs törvényéből származik ($\bold{F_g} = \gamma \frac{m_{g1}m_{g2}}{r^2}$), és azt méri, hogy egy testet mennyire vonzza a gravitációs mező. A súlyos tömeg határozza meg a test súlyát, amely a gravitációs erő és a test tömegének szorzata ($\bold{F_g} = m_g\cdot \bold{g}$).
\end{itemize}
Jelenlegi ismereteink szerint, és a legmodernebb mérések alapján a két tömeg megegyezik egymással. Ez az egyezés vezette rá Albert Einsteint az általános relativitáselmélet megalkotására,
amelyben a gravitációt a téridő görbületeként értelmezte. Az általános relativitáselmélet szerint a tehetetlen és súlyos tömeg közötti egyezés nem véletlen, hanem egy mélyebb kapcsolatot jelez a gravitáció és a mozgás között.
Ez az egyezés lehetővé teszi számunkra, hogy a gravitációt és a mozgást egyetlen elméleti keretben kezeljük, és megértsük a világegyetem működését.

\subsection{Gyorsuló koordináta-rendszerek, jelenségek a forgó földön}
\subsubsection{gyorsuló koordináta-rendszerek}
Az alapvető Newtoni mechanika inerciarendszerekben érvényes, ahol a Newton-törvények közvetlenül alkalmazhatók. Azonban a valóságban sok helyzetben praktikusabb egy olyan rendszerhez igazítani a koordináta-rendszerünket,
amely maga is gyorsul, például egy járműhöz vagy a Földhöz kötött rendszerhez. Ezeket a rendszereket gyorsuló koordináta-rendszereknek nevezzük. Ezekben a rendszerekben a rendszerre ható külső erők miatt a rendszeren belül
megjelennek úgynevezett virtuális erők vagy tehetetlenségi erők, amelyek nem valódi erők, hanem a gyorsuló rendszer hatásai. Ezeket a virutális erőket ugyanúgy figyelembe kell venni a mozgásegyenlet felírásakor, mint a valódi erőket.
A gyorsuló koordináta-rendszerek között megkülönböztetünk lineárisan gyorsuló vagy transzlációs rendszereket és forgó rendszereket. E két rendszer segítéségvel tetszőlegesen mozgó rendszert leírhatunk. Egy lineárisan gyorsuló rendszerben
a gyorsulás $\bold{a_0}$, és a testre ható virtuális erő a következőképpen írható fel:
\begin{equation}
    \bold{F_{in}} = -m\bold{a_0}
\end{equation}
Ahol $m$ a test tömege. Ez az erő ellentétes irányú a gyorsulással, és a test tehetetlenségét jelzi a gyorsuló rendszerhez képest. Ezek alapján a mozgásegyenlet a gyorsuló rendszerben a következőképpen írható fel:
\begin{equation}
    \bold{F_{net}} + \bold{F_{in}} = m\bold{a}
\end{equation}
\begin{equation}
    \bold{F_{net}} - m\bold{a_0} = m\bold{a}
\end{equation}
\begin{equation}
    \bold{F_{net}} = m(\bold{a} + \bold{a_0})
\end{equation}
A forgó koordináta-rendszerek esetén a helyzet bonyolultabb, mivel a forgás miatt további virtuális erők is megjelennek. A forgás leírásához vegyünk egy forgási operátort, amely az inerciarendszerből a forgó rendszerbe visz át:
\begin{equation}
    \bold{r} = \hat{O}\bold{r'}(t)
\end{equation}
Ahol $\bold{r}$ a test helyvektora az inerciarendszerben, $\bold{r'}$ a test helyvektora a forgó rendszerben, és $\hat{O}$ a forgási operátor. A sebesség vektor a következőképpen kapható meg:
\begin{equation}
    \bold{\dot{r}}(t) = \frac{d}{dt}(\hat{O}\bold{r'}(t)) = \hat{O}\dot{\bold{r'}(t)} + \dot{\hat{O}}\bold{r'}(t)
\end{equation}
Szorozzuk meg mindkét oldalt $\hat{O}^T$-vel (a transzponált operátorral), hogy a sebességet a forgó rendszerben kifejezzük:
\begin{equation}
    \hat{O}^T\bold{\dot{r}}(t) = \hat{O}^T\hat{O}\dot{\bold{r'}(t)} + \hat{O}^T\dot{\hat{O}}\bold{r'}(t)
\end{equation}
Itt a következőket használjuk ki:
\begin{itemize}
    \item $\hat{O}^T\bold{\dot{r}}(t) = \bold{\dot{r'}}(t)$, Mivel az operátor nem függ az időtől, így bevihető a derivált alá.
    \item $\hat{O}^T\hat{O} = \hat{I}$, ahol $\hat{I}$ az egységmátrix. Tehát a jobboldal első tagja egyszerűsödik $\dot{\bold{r'}(t)}$-re.
    \item $\hat{O}^T\dot{\hat{O}} = \hat{\Omega}$, ahol $\hat{\Omega}$ egy antiszimmetrikus mátrix, amely a forgás szögsebességét jellemzi.
\end{itemize}
$\hat{\Omega}$-nak fontos tulajdonsága, hogy antiszimmetrikus, azaz $\hat{\Omega}^T = -\hat{\Omega}$. Ez azt jelenti, hogy a mátrix főátlójában minden elem nulla, és a mátrix elemei a főátló alatt és felett ellentétes előjelűek.
Tehát $\hat{\Omega}$ a következő alakú:
\begin{equation}
    \hat{\Omega} = \begin{pmatrix}
    0 & -\omega_z & \omega_y \\
    \omega_z & 0 & -\omega_x \\
    -\omega_y & \omega_x & 0
    \end{pmatrix}
\end{equation}
Ahol $\omega_x$, $\omega_y$ és $\omega_z$ a forgás szögsebességének komponensei az x, y és z tengelyek mentén.
Most helyettesítsük vissza ezeket az eredményeket a sebesség egyenletbe:
\begin{equation}
    \bold{\dot{r'}}(t) = \dot{\bold{r'}(t)} + \hat{\Omega}\bold{r'}(t)
\end{equation}
Ezt a mátrix tulajdonságai miatt felírhatjuk vektoriális szorzatként is:
\begin{equation}
    \bold{\dot{r'}}(t) = \dot{\bold{r'}(t)} + \bold{\omega} \times \bold{r'}(t)
\end{equation}
Ahol $\bold{\omega} = (\omega_x, \omega_y, \omega_z)$ a forgás szögsebesség vektora. Ezt az alakot átalakíthatjuk egy operátorrá:
\begin{equation}
    \bold{\dot{r'}}(t) = \left(\frac{d}{dt} + \bold{\omega} \times \right)\bold{r'}(t) \rightarrow \frac{d}{dt} = \frac{d'}{dt} + \bold{\omega} \times
\end{equation}
Itt a $\frac{d'}{dt}$ a forgó rendszerben vett időderiváltat jelöli. Most alkalmazzuk ezt a sebességre, hogy megkapjuk a gyorsulást:
\begin{equation}
    \bold{\ddot{r}}(t) = \frac{d}{dt}(\bold{\dot{r}}(t)) = \frac{d}{dt}\left(\left(\frac{d'}{dt} + \bold{\omega} \times \right)\bold{r'}(t)\right)
\end{equation}
\begin{equation}
    \bold{a}(t) = \left(\frac{d'}{dt} + \bold{\omega} \times \right)\left(\frac{d'\bold{r'}}{dt} + \bold{\omega} \times \bold{r'} \right)
\end{equation}
\begin{equation}
    \bold{a}(t) = \frac{d'^2\bold{r'}}{dt^2} + 2\bold{\omega} \times \frac{d'\bold{r'}}{dt} + \bold{\omega} \times (\bold{\omega} \times \bold{r'}) + \dot{\bold{\omega}} \times \bold{r'}
\end{equation}
\begin{equation}
    \bold{a}(t) = \bold{a'} + 2\bold{\omega} \times \bold{v'} + \bold{\omega} \times (\bold{\omega} \times \bold{r'}) + \bold{\beta} \times \bold{r'}
\end{equation}
Ahol $\bold{a'} = \frac{d'^2\bold{r'}}{dt^2}$ a gyorsulás a forgó rendszerben, $\bold{v'} = \frac{d'\bold{r'}}{dt}$ a sebesség a forgó rendszerben, és $\bold{\beta} = \dot{\bold{\omega}}$ a szöggyorsulás. Most helyettesítsük be ezt a gyorsulást a Newton második törvényébe:
\begin{equation}
    \bold{F_{net}} = m\bold{a} = m\left(\bold{a'} + 2\bold{\omega} \times \bold{v'} + \bold{\omega} \times (\bold{\omega} \times \bold{r'}) + \bold{\beta} \times \bold{r'}\right)
\end{equation}
Most rendezzük át az egyenletet, hogy a gyorsulást a forgó rendszerben kifejezzük:
\begin{equation}
    m\bold{a'} = \bold{F_{net}} - m\left(2\bold{\omega} \times \bold{v'} + \bold{\omega} \times (\bold{\omega} \times \bold{r'}) + \bold{\beta} \times \bold{r'}\right)
\end{equation}
Ahol a következő virtuális erők jelennek meg:
\begin{itemize}
    \item Coriolis-erő: $\bold{F_{C}} = -2m(\bold{\omega} \times \bold{v'})$, amely a mozgó test sebességéből ered a forgó rendszerben.
    \item Centrifugális erő: $\bold{F_{cf}} = -m(\bold{\omega} \times (\bold{\omega} \times \bold{r'}))$, amely a test helyzetéből ered a forgó rendszerben.
    \item Euler-erő: $\bold{F_{E}} = -m(\bold{\beta} \times \bold{r'})$, amely a forgás szöggyorsulásából ered a forgó rendszerben.
\end{itemize}
Így a mozgásegyenlet a forgó rendszerben a következőképpen írható fel:
\begin{equation}
    m\bold{a'} = \bold{F_{net}} + \bold{F_{C}} + \bold{F_{cf}} + \bold{F_{E}}
\end{equation}
A centrifugális erő kifejezhető a következőképpen is:
\begin{equation}
    \bold{F_{cf}} = \bold{\omega} \times (\bold{\omega} \times \bold{r'}) = \left(\bold{\omega} \cdot \bold{r'}\right)\bold{\omega} - \omega^2\bold{r'}
\end{equation}
\begin{equation}
    \bold{r_{\omega}} = \frac{\bold{\omega} \cdot \bold{r'}}{|\omega|^2}\bold{\omega} \rightarrow \left(\bold{\omega} \cdot \bold{r'}\right)\bold{\omega} = \omega^2\bold{r_{\omega}}
\end{equation}
\begin{equation}
    \left(\bold{\omega} \cdot \bold{r'}\right)\bold{\omega} - \omega^2\bold{r'} = \omega^2\bold{r_{\omega}} - \omega^2\bold{r'} = -\omega^2(\bold{r'} - \bold{r_{\omega}})
\end{equation}
\begin{equation}
    \quad \bold{r'} - \bold{r_{\omega}} = \bold{s} \quad \rightarrow \quad \bold{F_{cf}} = -m\omega^2\bold{s}
\end{equation}
Ahol $\bold{s}$ a forgás tengelyétől mért távolságvektor. Ez az erő mindig kifelé mutat a forgás tengelyétől, és a forgás sebességének négyzetével arányos.
\begin{figure}[H]
    \centering
    \includegraphics[width=0.5\textwidth]{imgs/1-tetel/Centrifugal.png}
    \caption{centrifugális erő}
\end{figure}
\subsubsection{jelenségek a forgó földön}
A Föld egy forgó koordináta-rendszer, amelynek következtében a Föld felszínén számos jelenség figyelhető meg, amelyek a Coriolis-erő és a centrifugális erő hatására alakulnak ki. Ezek a jelenségek jelentős hatással vannak a légkör és az óceánok mozgására, valamint a földi élet számos aspektusára. Néhány példa ezekre a jelenségekre:
\begin{itemize}
    \item Coriolis-erő hatása a légkörben: A Coriolis-erő miatt a légáramlatok és a szelek eltérülnek az egyenes vonalú mozgástól. Az északi féltekén a szelek jobbra, míg a déli féltekén balra térülnek el. Ez a jelenség hozzájárul a ciklonok és anticiklonok kialakulásához, valamint a globális légkörzési mintákhoz.
    \item Coriolis-erő hatása az óceánokban: Az óceáni áramlatok is eltérülnek a Coriolis-erő hatására. Az északi féltekén az óceáni áramlatok jobbra, míg a déli féltekén balra térülnek el. Ez a jelenség hozzájárul a nagy óceáni áramlási rendszerek, például a Golf-áramlat kialakulásához.
    \item Földi forgás és a nap mozgása: A Föld forgása miatt a Nap látszólagos mozgása az égen is változik. A Nap reggel keleten kel fel, délben a legmagasabban van, és este nyugaton nyugszik le. Ez a jelenség a Föld forgásának következménye.
    \item Földi forgás és a gravitációs: A Föld forgása miatt a gravitációs erő nem pontosan függőleges a Föld felszínén. A centrifugális erő miatt a gravitációs erő kissé eltérül a függőlegestől, ami a Föld alakját is befolyásolja, amely geoid formájú.
    \item Földi forgás és a Coriolis-erő hatása a repülőgépek és rakéták mozgására: A repülőgépek és rakéták útvonala is eltérül a Coriolis-erő hatására. Ezért a pilóták és a rakétakilövő központok figyelembe veszik ezt a hatást a navigáció során.
\end{itemize}
Ezek a jelenségek mind a Föld forgásának következményei, és jelentős hatással vannak a földi életre és a környezetre. A Föld forgása és a hozzá kapcsolódó erők megértése kulcsfontosságú a meteorológia, az óceánográfia és a földtudományok területén.

\subsubsection*{Példa: Szabadesés a forgó Földön}
Vegyünk egy olyan rendszert, ahol egy test szabadon esik a Föld felszínén, figyelembe véve a Föld forgását. A testre ható erők a következők:
\begin{itemize}
    \item Gravitációs erő: $\bold{F_g} = m\cdot \bold{g}$, ahol $\bold{g} = (0, 0, -9.81 \frac{m}{s^2})$ a gravitációs gyorsulás.
    \item Coriolis-erő: $\bold{F_{C}} = -2m(\bold{\omega} \times \bold{v'})$, ahol $\bold{v'}$ a test sebessége a Földhöz kötött rendszerben.
\end{itemize}
Az Euler-erő és a centrifugális erő hatását most elhanyagoljuk, mivel a szabad esés során ezek az erők kisebbek, mint a gravitációs és Coriolis-erő. A mozgásegyenlet felírása:
\begin{equation}
    \bold{F_{net}} = \bold{F_g} + \bold{F_{C}} = m\bold{a}
\end{equation}
\begin{equation}
    m\cdot \bold{g} - 2m(\bold{\omega} \times \bold{v'}) = m\bold{a}
\end{equation}
A vektokat a következőképpen definiáljuk:
\begin{itemize}
    \item $\bold{r'} = (x, y, z)$ a test helyzete a Földhöz kötött rendszerben.
    \item $\bold{v'} = \dot{\bold{r'}} = (\dot{x}, \dot{y}, \dot{z})$ a test sebessége a Földhöz kötött rendszerben.
    \item $\bold{a'} = \ddot{\bold{r'}} = (\ddot{x}, \ddot{y}, \ddot{z})$ a test gyorsulása a Földhöz kötött rendszerben.
    \item $\bold{\omega} = (-\omega\cos\varphi, 0, \omega\sin\varphi)$ a Föld forgásának szögsebesség vektora, ahol $\phi$ a földrajzi szélesség és $\omega \approx 7.2921 \times 10^{-5} \frac{rad}{s}$ a Föld forgásának szögsebessége.
    \item $\bold{g} = (0, 0, -9.81 \frac{m}{s^2})$ a gravitációs gyorsulás vektora.
\end{itemize}
Ekkor a Coriolis-erő kifejezhető a következőképpen:
\begin{equation}
    \bold{F_{C}} = -2m(\bold{\omega} \times \bold{v'}) = -2m \begin{vmatrix}
    \hat{i} & \hat{j} & \hat{k} \\
    -\omega\cos\varphi & 0 & \omega\sin\varphi \\
    \dot{x} & \dot{y} & \dot{z}
    \end{vmatrix}
\end{equation}
\begin{equation}
    \bold{F_{C}} = -2m\left((\dot{y}\omega\sin\varphi)\hat{i} + (\dot{z}\omega\cos\varphi + \dot{x}\omega\sin\varphi)\hat{j} - (\dot{y}\omega\cos\varphi)\hat{k}\right)
\end{equation}
Most helyettesítsük be ezt a Coriolis-erőt a mozgásegyenletbe:
\begin{equation}
    m\cdot \bold{g} - 2m\left((\dot{y}\omega\sin\varphi)\hat{i} + (\dot{z}\omega\cos\varphi + \dot{x}\omega\sin\varphi)\hat{j} - (\dot{y}\omega\cos\varphi)\hat{k}\right) = m\bold{a'}
\end{equation}
Ez csak úgy lehet egyenlő minden komponensében, ha a következő három egyenlet teljesül:
\begin{equation}
    \ddot{x} = -2\dot{y}\omega\sin\varphi
\end{equation}
\begin{equation}
    \ddot{y} = -2\dot{z}\omega\cos\varphi + 2\dot{x}\omega\sin\varphi
\end{equation}
\begin{equation}
    \ddot{z} = -g + 2\dot{y}\omega\cos\varphi
\end{equation}
Most oldjuk meg ezeket az egyenleteket a kezdeti feltételekkel. Tegyük fel, hogy a test kezdeti helyzete $\bold{r'}(0) = (0, 0, h)$ és kezdeti sebessége $\bold{v'}(0) = (0, 0, 0)$, ahol $h$ a kezdeti magasság.
A Coriolis-erő hatása csak y irányban lesz jelentős, így vizsgáljuk meg azt. Észrevehetjük, hogy, ha $\ddot{y}$ egyenletet még egyszer deriváljuk, akkor a következőt kapjuk:
\begin{equation}
    \dddot{y} = -2\ddot{z}\omega\cos\varphi + 2\ddot{x}\omega\sin\varphi
\end{equation}
behelyettesítve $\ddot{z}$ és $\ddot{x}$ értékét a korábbi egyenletekből:
\begin{equation}
    \dddot{y} = -2(-g + 2\dot{y}\omega\cos\varphi)\omega\cos\varphi - 2\omega\sin\varphi(-2\dot{y}\omega\sin\varphi)
\end{equation}
\begin{equation}
    \dddot{y} = 2g\omega\cos\varphi + 4\dot{y}\omega^2\cos^2\varphi + 4\dot{y}\omega^2\sin^2\varphi
\end{equation}
Mivel $\cos^2\varphi + \sin^2\varphi = 1$, így a végső egyenlet a következő lesz:
\begin{equation}
    \dddot{y} = 2g\omega\cos\varphi + 4\dot{y}\omega^2
\end{equation}
Ha ezt megoldjuk az Y irányú sebességre, akkor a következőt kapjuk:
\begin{equation}
    \ddot{v_y} = 2g\omega\cos\varphi + 4\dot{y}\omega^2
\end{equation}
A megoldást ebben az alakban keresem:
\begin{equation}
    v_y = A\cos(2\omega t + \varphi) + B
\end{equation}
Ahol $A$ és $B$ integrálási állandók, amelyeket a kezdeti feltételekből határozhatunk meg. A kezdeti feltételek alapján:
\begin{equation}
    v_y(0) = 0 \quad \text{és} \quad \dot{v_y}(0) = 0
\end{equation}
Ezek alapján az integrálási állandók a következők lesznek:
\begin{equation}
    A = -\frac{g\cos\varphi}{2\omega} \quad \text{és} \quad B = \frac{g\cos\varphi}{2\omega}
\end{equation}
Így a y irányú sebesség időbeli változása a következő lesz:
\begin{equation}
    v_y = \frac{g\cos\varphi}{2\omega}(1 - \cos(2\omega t))
\end{equation}
$\omega$ nagyon kicsi érték, így a $\cos(2\omega t)$ közelíthető $1 - 2\omega^2 t^2$-re Taylor-sorral, ha $t$ nem túl nagy. Így a y irányú sebesség közelítése:
\begin{equation}
    v_y \approx \frac{g\cos\varphi}{2\omega}(1 - (1 - 2\omega^2 t^2)) = g\cos\varphi \cdot \omega t^2
\end{equation}
Most integráljuk ezt a sebességet, hogy megkapjuk a y irányú elmozdulást:
\begin{equation}
    y(t) = \int v_y dt = \int g\cos\varphi \cdot \omega t^2 dt = \frac{1}{3}g\cos\varphi \cdot \omega t^3
\end{equation}
Innen megkaptuk a test y irányú elmozdulását a szabad esés során a Föld forgásának hatására. Ez az elmozdulás a földrajzi szélességtől ($\varphi$) és az esési időtől ($t$) függ.

\subsubsection*{Példa: Inga mozgása a forgó Földön (Foucault-inga)}
A Foucault-inga egy híres kísérlet, amely bemutatja a Föld forgásának hatását egy ingára. Az inga mozgása a Földhöz kötött rendszerben történik, így a mozgásegyenlet felírásához figyelembe kell venni a Coriolis-erőt és a centrifugális
erőt is. Tegyük fel, hogy az inga hossza $L$, és a lengés síkja a földrajzi szélesség $\varphi$-vel bezárt szöget zár be. Az inga helyzetét a következőképpen definiáljuk:
\begin{itemize}
    \item $\theta(t)$: az inga kitérése a függőleges helyzettől.
    \item $\phi(t)$: az inga lengés síkjának elfordulása a földrajzi északi iránytól.
\end{itemize}
Az inga mozgásegyenlete a következőképpen írható fel a Földhöz kötött rendszerben:
\begin{equation}
    m\bold{a'} = \bold{F_g} + \bold{F_C} + \bold{F_{cf}}
\end{equation}
Ahol:
\begin{itemize}
    \item $\bold{F_g} = m\cdot \bold{g}$ a gravitációs erő.
    \item $\bold{F_C} = -2m(\bold{\omega} \times \bold{v'})$ a Coriolis-erő.
\end{itemize}
Az inga gyorsulása a következőképpen kapható meg:
\begin{equation}
    \ddot{x} = 2\dot{y}\omega\sin\varphi + \lambda x
\end{equation}
\begin{equation}
    \ddot{y} = -2\dot{z}\omega\cos\varphi -2\dot{x}\omega\sin\varphi + \lambda y
\end{equation}
\begin{equation}
    \ddot{z} = -g + 2 \dot{y}\omega\cos\varphi + \lambda z
\end{equation}
Ahol $\lambda$ a feszítőerő per tömeg, amely az inga feszítéséből származik. Az inga mozgását a következő kezdeti feltételekkel vizsgáljuk:
\begin{equation}
    x^2 + y^2 + z^2 = L^2
\end{equation}
\begin{equation}
    Z = \pm \sqrt{L^2 - x^2 - y^2}
\end{equation}
\begin{equation}
    Z = \pm l \sqrt{1 - \frac{x^2 + y^2}{L^2}} \rightarrow \frac{x^2 + y^2}{L^2} \approx 0, \quad \text{Mivel L-hez képest kicsi a kitérés}
\end{equation}
\begin{equation}
    Z \approx -L
\end{equation}
Innen a következő közelítést kapjuk:
\begin{equation}
    \ddot{Z} = -g + 2\dot{y}\omega\cos\varphi + \lambda l
\end{equation}
Mivel $Z \approx -L$ így $\ddot{Z} \approx 0$ és $2\dot{y}\omega\cos\varphi \approx 0$, mivel $\dot{y}$ kicsi a kis kimozdítás miatt, így a feszítőerő per tömeg kifejezhető a következőképpen:
\begin{equation}
    \lambda = \frac{g}{L}
\end{equation}
Most helyettesítsük be ezt a feszítőerő per tömeget az x és y irányú gyorsulás egyenletekbe:
\begin{equation}
    \ddot{x} = 2\dot{y}\omega\sin\varphi - \frac{g}{L}x
\end{equation}
\begin{equation}
    \ddot{y} = \cancel{-2\dot{z}\omega\cos\varphi} - 2\dot{x}\omega\sin\varphi - \frac{g}{L}y
\end{equation}
Mivel $\dot{z} \approx 0$, így elhanyagolható. Most oldjuk meg ezeket az egyenleteket. 
\begin{equation}
    \omega_1 = \omega \sin \varphi
\end{equation}
\begin{equation}
    \ddot{x} - 2 \omega_1 \dot{y} + \frac{g}{l} x = 0
\end{equation}
\begin{equation}
    \ddot{y} - 2 \omega_1 \dot{x} + \frac{g}{l} y = 0
\end{equation}
Szorrozzuk meg i-vel a két egyenletet:
\begin{equation}
   i \ddot{x} - 2i \omega_1 \dot{y} + \frac{g}{l} ix = 0
\end{equation}
\begin{equation}
   i \ddot{y} - 2i \omega_1 \dot{x} + \frac{g}{l} iy = 0
\end{equation}
És vezessük be a $z = x + iy$ jelölést és adjuk össze a két egyenletet:
\begin{equation}
    \ddot{z} + 2 \omega_1 \dot{z} + \frac{g}{l} z = 0
\end{equation}
Keressük a megoldást a következő alakban:
\begin{equation}
    z = z_0 e^{i  \alpha t}
\end{equation}
Ezt visszahelyettesítve:
\begin{equation}
    - \alpha^2 z - 2 \omega_1 \alpha z + \frac{g}{l} z = 0
\end{equation}
Megoldva a másodfokú egyenletet:
 \begin{equation}
    \alpha_{1,2} = \omega_1 \pm \sqrt{\cancel{\omega_1^2} + \frac{g}{l}}
 \end{equation}
A $\omega_1^2$ elhanyagolható a $\frac{g}{l}$ mellett. Az egészet összetéve megkaphatjuk a két megoldást, de mivel a két megoldás lineárkombinációja is megoldás, így a megoldást a következő alakban is felírhatjuk:
\begin{equation}
    z = z_1 e^{-i\omega_1 t} e^{i\sqrt{\frac{g}{l}}t} + z_2 e^{-i\omega_1 t} e^{-i\sqrt{\frac{g}{l}}t} 
\end{equation}
\begin{equation}
    z = e^{-i\omega_1 t} \left(z_1 e^{i\sqrt{\frac{g}{l}}t} + z_2 e^{-i\sqrt{\frac{g}{l}}t}\right)
\end{equation}
A kezdőfeltételekkel meghatározhatóak az együtthatók:
\begin{equation}
    z(0) = a \leftarrow \text{valós szám}
\end{equation}
\begin{equation}
    \dot{z}(0) = 0
\end{equation}
\begin{equation}
    z_1 + z_2 = a
\end{equation}
\begin{equation}
    \dot{z} = z_1 \left(i\sqrt{\frac{g}{l}} - i\omega_1\right) e^{-i\omega_1t + i\sqrt{\frac{g}{l}}t} + z_2 \left(-i\sqrt{\frac{g}{l}} - i\omega_1\right) e^{-i\omega_1t - i\sqrt{\frac{g}{l}}t}
\end{equation}
Itt az egyenlet t=0-ban csak akkor lesz nulla, ha:
\begin{equation}
    z_1 \left(\sqrt{\frac{g}{l}} - \omega_1\right) + z_2 \left(-\sqrt{\frac{g}{l}} - \omega_1\right) = 0
\end{equation}
\begin{equation}
    z_1 \left(-\sqrt{\frac{g}{l}} + \omega_1\right) + z_2 \left(\sqrt{\frac{g}{l}} + \omega_1\right) = 0
\end{equation}
Ha ebbe behelyettesítjük a $z_1 + z_2 = a$ egyenletet, akkor:
\begin{equation}
    \omega_1 a + \sqrt{\frac{g}{l}}(z_2 - z_1) = 0
\end{equation}
\begin{equation}
    z_1 - z_2 = \frac{\omega_1 a}{\sqrt{\frac{g}{l}}}
\end{equation}
Tehát ezt a két egyenletet kell megoldanunk:
\begin{gather}
    z_1 + z_2 = a \\
    z_1 - z_2 = \frac{\omega_1 a}{\sqrt{\frac{g}{l}}}
\end{gather}
$z_1$ és $z_2$ pedig a következőképpen néz ki:
\begin{equation}
    z_1 = \frac{a}{2}\left(1 + \frac{\omega_1}{\sqrt{\frac{g}{l}}}\right)
\end{equation}
\begin{equation}
    z_2 = \frac{a}{2}\left(1 - \frac{\omega_1}{\sqrt{\frac{g}{l}}}\right)
\end{equation}
Így a megoldás a következő lesz:
\begin{equation}
    z = e^{-i\omega_1 t} \left[\frac{a}{2}\left(1 + \frac{\omega_1}{\sqrt{\frac{g}{l}}}\right) e^{i\sqrt{\frac{g}{l}}t} + \frac{a}{2}\left(1 - \frac{\omega_1}{\sqrt{\frac{g}{l}}}\right) e^{-i\sqrt{\frac{g}{l}}t}\right]
\end{equation}
\begin{equation}
    z = \frac{a}{2} e^{-i\omega_1 t} \left[\left(1 + \frac{\omega_1}{\sqrt{\frac{g}{l}}}\right) e^{i\sqrt{\frac{g}{l}}t} + \left(1 - \frac{\omega_1}{\sqrt{\frac{g}{l}}}\right) e^{-i\sqrt{\frac{g}{l}}t}\right]
\end{equation}
\begin{equation}
    z = \frac{a}{2} e^{-i\omega_1 t} \left[2\cos\left(\sqrt{\frac{g}{l}}t\right) + 2i\frac{\omega_1}{\sqrt{\frac{g}{l}}}\sin\left(\sqrt{\frac{g}{l}}t\right)\right]
\end{equation}
\begin{equation}
    z = a e^{-i\omega_1 t} \left[\cos\left(\sqrt{\frac{g}{l}}t\right) + i\frac{\omega_1}{\sqrt{\frac{g}{l}}}\sin\left(\sqrt{\frac{g}{l}}t\right)\right]
\end{equation}
Most szedjük szét a valós és képzetes részre:
\begin{equation}
    x = a \cos(\omega_1 t) \cos\left(\sqrt{\frac{g}{l}}t\right) + a \frac{\omega_1}{\sqrt{\frac{g}{l}}} \sin(\omega_1 t) \sin\left(\sqrt{\frac{g}{l}}t\right)
\end{equation}
\begin{equation}
    y = -a \sin(\omega_1 t) \cos\left(\sqrt{\frac{g}{l}}t\right) + a \frac{\omega_1}{\sqrt{\frac{g}{l}}} \cos(\omega_1 t) \sin\left(\sqrt{\frac{g}{l}}t\right)
\end{equation}
Ez a megoldás azt mutatja, hogy az inga lengés síkja idővel elfordul a földrajzi északi iránytól, ami a Foucault-inga jelenségét magyarázza. Az inga lengésének frekvenciája közelítőleg $\sqrt{\frac{g}{l}}$, míg a lengés síkjának elfordulási sebessége $\omega \sin \varphi$, ahol $\varphi$ a földrajzi szélesség. 

\subsection{Munkatétel, Energia-, impulzus- és impulzusmomentum megmaradás tömegpontra}

Ha a Newton második törvényét vizsgáljuk, akkor megkisérelhetjük megnézni, hogy mi történik, ha különböző vektorokkal megszorozzuk az egyenletet. Például, ha a sebességvektoral szozunk be,
akkor a következőt kapjuk:
\begin{equation}
    \bold{F_{net}} \cdot \bold{v} = m\bold{a} \cdot \bold{v}
\end{equation}
A sebesség és az erő szorzatát elnevezhetük egy új egységnek, amit teljesítménynek hívunk:
\begin{equation}
    P = \bold{F_{net}} \cdot \bold{v}
\end{equation}
Az egyenlet másik oldalát megvizsgálva Észrevehetjük hogy a gyorsulás és a sebesség szorzata kifejezhető az idő szerinti deriváltként:
\begin{equation}
    \frac{d}{dt}\left(\bold{v} \cdot \bold{v}\right) = 2 \bold{v}\bold{a}
\end{equation}
\begin{equation}
    \bold{v}\bold{a} = \frac{1}{2} \frac{d}{dt}\left(\bold{v} \cdot \bold{v}\right) = \frac{d}{dt}\left(\frac{1}{2} m \bold{v}^2\right)
\end{equation}
Így a teljesítmény kifejezhető a következőképpen:
\begin{equation}
    P = \bold{F_{net}} \cdot \bold{v} = \frac{d}{dt}\left(\frac{1}{2} m \bold{v}^2\right)
\end{equation}
Ahol $\frac{1}{2} m \bold{v}^2$ az ún. kinetikus energia, amit $K$-val jelölünk:
\begin{equation}
    K = \frac{1}{2} m \bold{v}^2
\end{equation}
Így a teljesítmény a kinetikus energia idő szerinti változásaként értelmezhető:
\begin{equation}
    P = \frac{dK}{dt}
\end{equation}
Most integráljuk ezt az egyenletet egy időintervallumon:
\begin{equation}
    \int_{t_1}^{t_2} P dt = \int_{t_1}^{t_2} \frac{dK}{dt} dt = K(t_2) - K(t_1)
\end{equation}
A teljesítmény idő szerinti integrálja a kinetikus energia változását adja meg az adott időintervallumban. Ezt az integrált munkának nevezzük, amit $W$-vel jelölünk:
\begin{equation}
    W = \int_{t_1}^{t_2} P dt = K(t_2) - K(t_1)
\end{equation}
A munka tehát a kinetikus energia változását méri egy adott időintervallumban. Ez az összefüggés a munka-energia tétel, amely kimondja, hogy a testre ható erők által végzett munka megegyezik a test kinetikus energiájának változásával.
Most nézzük meg a munka kifejezést egy kis időintervallumra:
\begin{equation}
    W = \int_{t_1}^{t_2} P dt = \int_{t_1}^{t_2} \bold{F_{net}} \cdot \bold{v} dt
\end{equation}
\begin{equation}
    W = \int_{t_1}^{t_2} \bold{F_{net}} \cdot \frac{d\bold{r}}{dt} dt = \int_{\bold{r_1}}^{\bold{r_2}} \bold{F_{net}} \cdot d\bold{r}
\end{equation}
Ahol $\bold{r_1}$ és $\bold{r_2}$ a test helyzete az időintervallum elején és végén. Ez az integrál a testre ható erők által végzett munkát méri a test útja mentén. Ez a munka definíciója:
\begin{equation}
    W = \int_{\bold{r_1}}^{\bold{r_2}} \bold{F_{net}} \cdot d\bold{r}
\end{equation}
Ez az egyenlet azt mutatja, hogy a munka a testre ható erők és a test elmozdulása közötti skaláris szorzat integrálja az út mentén. Ez az összefüggés a munka-energia tétel egyik formája,
amely kimondja, hogy a testre ható erők által végzett munka megegyezik a test kinetikus energiájának változásával. Tehát a munka legegyszerűbb alakjában a következő:
\begin{equation}
    W = \bold{F} \cdot \bold{s} = F s \cos \alpha
\end{equation}
Ahol $\bold{F}$ az erő, $\bold{s}$ az elmozdulás, és $\alpha$ az erő és az elmozdulás közötti szög.
\subsubsection{Erőterek}
Elképzelhetünk egy olyan környezetet, ahol a testre ható erő függ a helyzetétől, de nem függ az időtől. Ilyen esetben az erőt erőtérnek nevezzük, és az erő a helyzet függvényeként írható fel:
\begin{equation}
    \bold{F} = \bold{F}(\bold{r})
\end{equation}
Vizsgáljuk meg egy ilyen rendszerben, hogy mi történik egy kis kimozdulás esetén. Ekkor az munka így változik:
\begin{equation}
    W = \bold{F}(\bold{r}) \cdot \bold{s}
\end{equation}
\begin{equation}
    \bold{s} = \bold{r_{i+1}} - \bold{r_i}
\end{equation}
\begin{equation}
    \sum_{i=0}^{N-1} \bold{F}(\bold{r_i}) \cdot (\bold{r_{i+1}} - \bold{r_i}) = \sum_{i=0}^{N-1} \bold{F}(\bold{r_i}) \cdot \Delta \bold{r_i}
\end{equation}
Ha a kimozdulás elég kicsi, akkor a következő határértéket vehetjük:
\begin{equation}
    W = \lim_{N \to \infty} \sum_{i=0}^{N-1} \bold{F}(\bold{r_i}) \cdot \Delta \bold{r_i} = \int_{G}^{} \bold{F}(\bold{r}) \cdot d\bold{r}
\end{equation}
Ez az integrál egy görbe menti integrál és a testre ható erők által végzett munkát méri a test útja mentén egy erőtérben. Ez az összefüggés a munka-energia tétel egyik formája, amely kimondja, hogy a testre ható erők által végzett munka megegyezik a test kinetikus energiájának
változásával.
\subsubsection{Konzervatív erőtér}
Egy erőtér akkor konzervatív, ha a testre ható erők által végzett munka csak a kezdő és végpont helyzetétől függ, és nem függ az útvonal alakjától. Matematikailag ez azt jelenti, 
hogy két különböző útvonal esetén, amelyek ugyanazon a kezdő és végponton haladnak át, a munka ugyanaz lesz:
\begin{equation}
    \int_{G_1}^{} \bold{F}(\bold{r}) \cdot d\bold{r} = \int_{G_2}^{} \bold{F}(\bold{r}) \cdot d\bold{r}
\end{equation}
\begin{equation}
    W_{A \to B}^{(1)} = W_{A \to B}^{(2)}
\end{equation}
Ez azt is jelenti, hogy egy zárt görbe mentén végzett munka nulla:
\begin{equation}
    \oint \bold{F}(\bold{r}) \cdot d\bold{r} = 0
\end{equation}
Ilyen erőtér esetén a test munkavégző képessége csak a helyzetétől függ, és nem függ az útvonal alakjától. Ezért bevezethetünk egy skaláris mennyiséget, amit potenciális energiának nevezünk, és $U(\bold{r})$-vel jelölünk. A potenciális energia a következőképpen definiálható:
\begin{equation}
    U(\bold{r}) = - \int_{r_0}^{\bold{r}} \bold{F}(\bold{r'}) \cdot d\bold{r'}
\end{equation}
Ahol $r_0$ egy tetszőleges referencia pont, ahol a potenciális energia nulla. A negatív előjel konvenció által lett meghatozva és azt jelenti, hogy a potenciális energia csökken, amikor a test a potenciális energia minimuma felé mozog.
A potenciális energia tehát a test helyzetétől függ, és a következő kapcsolat áll fenn az erő és a potenciális energia között:
\begin{equation}
    \bold{F}(\bold{r}) = - \nabla U(\bold{r})
\end{equation}
Ahol $\nabla U(\bold{r})$ a potenciális energia gradiensét jelenti. Ez az összefüggés azt mutatja, hogy a konzervatív erőtérben az erő a potenciális energia csökkenésének irányába mutat. A munka-energia tétel konzervatív erőtér esetén a következőképpen írható fel:
\begin{equation}
    W = K(t_2) - K(t_1) = U(\bold{r_1}) - U(\bold{r_2})
\end{equation}
\begin{equation}
    K(t_1) + U(\bold{r_1}) = K(t_2) + U(\bold{r_2})
\end{equation}
Ez azt jelenti, hogy a test teljes mechanikai energiája (kinetikus + potenciális) állandó marad egy konzervatív erőtérben:
\begin{equation}
    E = K + U = \text{állandó}
\end{equation}

\subsubsection*{Példa: Inga mozgása konzervatív erőtérben}
Vizsgáljuk meg egy inga mozgását egy konzervatív erőtérben, például a gravitációs erőtérben. Tegyük fel, hogy az inga hossza $L$, és a lengés síkja a függőlegeshez képest egy szöget zár be, amit $\varphi$-val jelölünk.
Az inga potenciális energiája a következőképpen írható fel:
\begin{equation}
    U(\varphi) = mgh
\end{equation}
\begin{equation}
    h = L(1 - \cos\varphi)
\end{equation}
Ahol $h$ az inga tömegközéppontjának magassága a referencia szinthez képest. A h-t úgy kapjuk meg, hogy a teljes hosszából kivonjuk a kitérés függőleges komponensét.
Az inga sebessége a következőképpen kapható meg:
\begin{equation}
    \bold{v} = \bold{\dot{r}} \bold{e_r} + r\dot{\varphi} \bold{e_\varphi}
\end{equation}
Ahol $\bold{\dot{r}} \bold{e_r} = 0$ mivel az inga hossza állandó, így csak a szögsebesség komponens marad:
\begin{equation}
    \bold{v} = L\dot{\varphi} \bold{e_\varphi}
\end{equation}
Így a kinetikus energia a következőképpen írható fel:
\begin{equation}
    K = \frac{1}{2} m (L\dot{\varphi})^2 = \frac{1}{2} m L^2 \dot{\varphi}^2
\end{equation}
Most alkalmazzuk a munka-energia tételt az inga mozgására:
\begin{equation}
    E = K + U = \text{állandó}
\end{equation}
\begin{equation}
    \frac{1}{2} m L^2 \dot{\varphi}^2 + mgL(1 - \cos\varphi) = \text{állandó}
\end{equation}
Ez az egyenlet leírja az inga mozgását egy konzervatív erőtérben. Az inga lengésének frekvenciája és amplitúdója a kezdeti feltételektől függ, de a teljes mechanikai energia állandó marad a mozgás során.
Ha az energia változását vizsgáljuk, akkor a következő egyenletet kapjuk:
\begin{equation}
    \frac{dE}{dt} = \frac{1}{2} m L^2 \cdot 2\dot{\varphi}\ddot{\varphi} + mgL \sin\varphi \cdot \dot{\varphi} = 0
\end{equation}
\begin{equation}
    L \ddot{\varphi} + g \sin\varphi = 0
\end{equation}
Ha kis kitéréssel dolgozunk, akkor a $\sin\varphi \approx \varphi$ közelítést alkalmazhatjuk, így a következő egyszerűsített egyenletet kapjuk:
\begin{equation}
    \ddot{\varphi} \approx -\frac{g}{L} \varphi
\end{equation}
Ez egy harmonikus oszcillátor egyenlete, ahol a freqvencia:
\begin{equation}
    \omega = \sqrt{\frac{g}{L}}
\end{equation}
Így az inga mozgása kis kitérés esetén harmonikus rezgésként viselkedik, és a teljes mechanikai energia állandó marad a mozgás során.

\subsubsection{Impulzus}
Ha visszatérünk a Newton második törvényére, akkor átírhatjuk az egyenletet a következőképpen:
\begin{equation}
    \bold{F_{net}} = m\bold{a} = m\frac{d\bold{v}}{dt} = \frac{d}{dt}(m\bold{v}) = \frac{d\bold{p}}{dt}
\end{equation}
Ahol $\bold{p} = m\bold{v}$ az impulzusvektor, amit a tömeg és a sebesség szorzataként definiálunk. Így a Newton második törvénye az impulzus idő szerinti változását írja le:
\begin{equation}
    \bold{F_{net}} = \frac{d\bold{p}}{dt}
\end{equation}
Ha integráljuk ezt az egyenletet egy időintervallumon, akkor a következőt kapjuk:
\begin{equation}
    \int_{t_1}^{t_2} \bold{F_{net}} dt = \int_{t_1}^{t_2} \frac{d\bold{p}}{dt} dt
\end{equation}
\begin{equation}
    \bold{J} = \bold{p}(t_2) - \bold{p}(t_1)
\end{equation}
Ahol $\bold{J}$ az impulzusváltozás, amit az erő idő szerinti integráljaként definiálunk:
\begin{equation}
    \bold{J} = \int_{t_1}^{t_2} \bold{F_{net}} dt
\end{equation}
Ez az összefüggés az impulzus-momentum tétel, amely kimondja, hogy a test impulzusának változása megegyezik a testre ható erők idő szerinti integráljával. Ha a testre ható erők eredője nulla, akkor az impulzus állandó marad:
\begin{equation}
    \bold{F_{net}} = 0 \Rightarrow \frac{d\bold{p}}{dt} = 0 \Rightarrow \bold{p} = \text{állandó}
\end{equation}
Ez az impulzus megmaradás törvénye, amely kimondja, hogy zárt rendszerben az impulzus állandó marad, ha nincsenek külső erők.

Az impulzus fogalma bár triviálisan következik Newton második törvényéből, mégis hasznos bevezni a mennyiséget, ugyanis a segítségével sokkal egyszerűbben lehet kezelni bizonyos problémákat, mint a Newton törvényeivel.
Például ütközések esetén, ahol az erők nagyon nagyok és rövid ideig hatnak, az impulzus megmaradás törvénye sokkal egyszerűbbé teszi a számításokat.

\subsubsection{Impulzusmomentum}
Nem csak a transzlációs mozgásra, hanem a forgó mozgásra is létezik egy hasonló mennyiség, amit impulzusmomentumnak nevezünk. Az impulzusmomentumot a következőképpen definiáljuk:
\begin{equation}
    \bold{N} = \bold{r} \times \bold{p}
\end{equation}
Ahol $\bold{r}$ a test helyvektora egy adott ponttól mérve, és $\bold{p}$ az impulzusvektor. Az impulzusmomentum tehát a helyvektor és az impulzusvektor vektoriális szorzataként definiálható.
Most vizsgáljuk meg az impulzusmomentum idő szerinti változását. A Newton második törvényét felhasználva:
\begin{equation}
    \frac{d\bold{N}}{dt} = \frac{d}{dt}(\bold{r} \times \bold{p}) = \frac{d\bold{r}}{dt} \times \bold{p} + \bold{r} \times \frac{d\bold{p}}{dt}
\end{equation}
\begin{equation}
    \frac{d\bold{N}}{dt} = \bold{v} \times m\bold{v} + \bold{r} \times \bold{F_{net}}
\end{equation}
Az első tag nulla, mivel a sebességvektor és az impulzusvektor párhuzamosak. Így az impulzusmomentum idő szerinti változása a következőképpen írható fel:
\begin{equation}
    \frac{d\bold{N}}{dt} = \bold{r} \times \bold{F_{net}} = \bold{M}
\end{equation}
Ahol $\bold{M}$ a forgatónyomaték, amit a helyvektor és a nettó erő vektoriális szorzataként definiálunk. Ez az összefüggés az impulzusmomentum tétel, amely kimondja, hogy a test impulzusmomentumának változása megegyezik a testre ható forgatónyomatékkal.
Ha a testre ható erők eredője nulla, akkor az impulzusmomentum állandó marad:
\begin{equation}
    \bold{F_{net}} = 0 \Rightarrow \bold{M} = 0 \Rightarrow \frac{d\bold{N}}{dt} = 0 \Rightarrow \bold{N} = \text{állandó}
\end{equation}
Ez az impulzusmomentum megmaradás törvénye, amely kimondja, hogy zárt rendszerben az impulzusmomentum állandó marad, ha nincsenek külső erők.

\subsection{Pontrendszerek}
Eddig pontszerű testekkel foglalkoztunk, de a Newton törvények általánosíthatók pontrencekekre is. Azokat a pontrencereket melyeknél a pontok távolsága jó közelítéssel állandó, merev testeknek nevezzük.
Egy pontrendszer pontjait a következőképpen jellemezhetjük:
\begin{itemize}
    \item $m_i$: az i-edik pont tömege.
    \item $\bold{r_i}$: az i-edik pont helyvektora.
    \item $\bold{v_i}$: az i-edik pont sebességvektora.
    \item $\bold{a_i}$: az i-edik pont gyorsulásvektora.
\end{itemize}
A pontrendszer teljes tömege a pontok tömegének összege:
\begin{equation}
    M = \sum_{i} m_i
\end{equation}
A pontrendszer tömegközéppontja a következőképpen definilható:
\begin{equation}
    \bold{R} = \frac{1}{M} \sum_{i} m_i \bold{r_i}
\end{equation}
Ezek alapján a Newton második törvénye egy pontra a pontrendszerben:
\begin{equation}
    m_i \bold{a_i} = \bold{F_i^{ext}} + \sum_{j=1}^{N} \bold{F_{ij}}
\end{equation}
Ahol $\bold{F_i^{ext}}$ az i-edik pontra ható külső erő, és $\bold{F_{ij}}$ az i-edik pontra ható j-edik pont által kifejtett belső erő. Ha összeadjuk az összes pontra vonatkozó egyenletet, akkor a következőt kapjuk:
\begin{equation}
    \sum_{i} m_i \bold{a_i} = \sum_{i} \bold{F_i^{ext}} + \sum_{i,j}^{N} \bold{F_{ij}}
\end{equation}
A belső erők összege nulla, mivel a Newton harmadik törvénye szerint az i-edik pontra ható j-edik pont erője és a j-edik pontra ható i-edik pont erője ellentétes irányúk és egyenlő nagyságúak:
\begin{equation}
    \sum_{i,j}^{N} \bold{F_{ij}} = \sum_{j,i}^{N} \bold{F_{ji}} = -\sum_{i,j}^{N} \bold{F_{ij}} \Rightarrow \sum_{i,j}^{N} \bold{F_{ij}} = 0
\end{equation}
Így a pontrendszer mozgásegyenlete a következőképpen írható fel:
\begin{equation}
    \sum_{i=1}^{N} m_i \bold{\ddot{r_i}} = \sum_{i=1}^{N} \bold{F_i^{ext}}
\end{equation}
Ezek alapján a rendszer impulzusának időbeli változása a külső erők eredőjével egyenlő:
\begin{equation}
    \frac{d\bold{P}}{dt} = \sum_{i=1}^{N} \bold{F_i^{ext}}
\end{equation}
Ahol az impulzus a következőképpen definiálható:
\begin{equation}
    \bold{P} = \sum_{i=1}^{N} m_i \bold{v_i} = M \bold{V}
\end{equation}
Ahol $\bold{V}$ a pontrendszer tömegközéppontjének sebessége:
\begin{equation}
    \bold{V} = \frac{1}{M} \sum_{i=1}^{N} m_i \bold{v_i}
\end{equation}
Tehát a rendszer össz-impulzusát csak a külső erők változtatják meg, a belső erők nem befolyásolják azt. A rendszer tehát jellemezhető a tömegközéppontjának mozgásával, amelyre a következő egyenlet érvényes:
\begin{equation}
    M \bold{\ddot{R}} = \sum_{i=1}^{N} \bold{F_i^{ext}}
\end{equation}
Ez az egyenlet azt mutatja, hogy a pontrendszer tömegközéppontjának gyorsulása a külső erők eredőjével arányos. Így a pontrendszer mozgása a tömegközéppont mozgásával jellemezhető, amelyre a Newton második törvénye érvényes.

\subsubsection{forgás}
forgassuk meg a pontrendszert egy $\bold{r}$ vektorral:
\begin{equation}
    m_i \bold{\ddot{r_i}} = \bold{F_i}^{ext} + \sum_{j=1}^{N} \bold{F_{ij}} \quad  \setminus \bold{r_i} \times
\end{equation}
\begin{equation}
    m_i (\bold{r_i} \times \bold{\ddot{r_i}}) = \bold{r_i} \times \bold{F_i}^{ext} + \sum_{j=1}^{N} (\bold{r_i} \times \bold{F_{ij}})
\end{equation}
\begin{equation}
    \frac{d}{dt} \bold{N_i} = \bold{M_i} + \sum_{j=1}^{N} (\bold{r_i} \times \bold{F_{ij}})
\end{equation}
Összegessük az összes pontra:
\begin{equation}
    \sum_{i=1}^{N} \frac{d}{dt} \bold{N_i} = \sum_{i=1}^{N} \bold{M_i} + \sum_{i,j}^{N} (\bold{r_i} \times \bold{F_{ij}})
\end{equation}
A belső erők összege nulla, mivel a Newton harmadik törvénye szerint az i-edik pontra ható j-edik pont erője és a j-edik pontra ható i-edik pont erője ellentétes irányúk és egyenlő nagyságúak:
\begin{equation}
    \sum_{i,j}^{N} (\bold{r_i} \times \bold{F_{ij}}) = \sum_{j,i}^{N} (\bold{r_j} \times \bold{F_{ji}}) = -\sum_{i,j}^{N} (\bold{r_i} \times \bold{F_{ij}}) \Rightarrow \sum_{i,j}^{N} (\bold{r_i} \times \bold{F_{ij}}) = 0
\end{equation}
Itt fontos megjegyezni, hogy ez csak centrális erőtér esetén igaz, ahol az erő iránya a két pontot összekötő egyenes mentén hat. Ha ez nem így van, akkor a belső erők összege nem feltétlenül nulla,
és a pontrendszer forgása befolyásolható a belső erők által is. 
\begin{equation}
    2 \cdot \sum_{i,j}^{N} (\bold{r_i} \times \bold{F_{ij}}) = \sum_{i,j}^{N} (\bold{r_i} \times \bold{F_{ij}}) + \sum_{j,i}^{N} (\bold{r_j} \times \bold{F_{ji}}) = \sum_{i,j}^{N} (\bold{r_i} - \bold{r_j}) \times \bold{F_{ij}} = 0
\end{equation}
\begin{equation}
    \text{Csak ha } \bold{F_{ij}} \parallel (\bold{r_i} - \bold{r_j}) \Rightarrow \sum_{i,j}^{N} (\bold{r_i} \times \bold{F_{ij}}) = 0
\end{equation}
Így a pontrendszer forgásának mozgásegyenlete a következőképpen írható fel:
\begin{equation}
    \frac{d\bold{N}}{dt} = \sum_{i=1}^{N} \bold{M_i}
\end{equation}
Ahol az össz-impulzusmomentum a következőképpen definiálható centrális erőtér esetén:
\begin{equation}
    \bold{N} = \sum_{i=1}^{N} \bold{N_i} = \sum_{i=1}^{N} (\bold{r_i} \times m_i \bold{v_i})
\end{equation}
Ahol az össz-nyomaték a következőképpen definiálható:
\begin{equation}
    \bold{M} = \sum_{i=1}^{N} \bold{M_i} = \sum_{i=1}^{N} (\bold{r_i} \times \bold{F_i}^{ext})
\end{equation}
Ez az egyenlet azt mutatja, hogy a pontrendszer impulzusmomentumának időbeli változása a külső nyomaték eredőjével egyenlő. Így a pontrendszer forgása a külső nyomaték hatására változik,
a belső erők nem befolyásolják azt centrális erőtér esetén. Viszont van egy probléma ezzel a megoldással, mégpedig, hogy függ a koordinátarendszertől. Ha más vonatkoztatási rendszert választunk,
akkor a helyvektorok is megváltoznak. Ezért érdemes a pontrendszer helyzetét a tömegközéppontjához képest vizsgálni. Ekkor a helyvektorokat a következőképpen definiálhatjuk:
\begin{equation}
    \bold{r_i} = \bold{R} + \bold{\varrho_i}
\end{equation}
\begin{equation}
    \bold{v} = \bold{V} + \bold{\dot{\varrho_i}}
\end{equation}
Ahol $\bold{R}$ a pontrendszer tömegközéppontjának helyvektora, $\bold{V}$ a tömegközéppont sebességvektora, és $\bold{\varrho_i}$ az i-edik pont helyvektora a tömegközépponthoz képest.
Most helyettesítsük be ezeket az egyenleteket az impulzusmomentum definíciójába:
\begin{equation}
    \bold{N} = \sum_{i=1}^{N} (\bold{R} + \bold{\varrho_i}) \times m_i (\bold{V} + \bold{\dot{\varrho_i}})
\end{equation}
\begin{equation}
    \bold{N} = \sum_{i=1}^{N} \left[\bold{R} \times m_i \bold{V} + \bold{R} \times m_i \bold{\dot{\varrho_i}} + \bold{\varrho_i} \times m_i \bold{V} + \bold{\varrho_i} \times m_i \bold{\dot{\varrho_i}}\right]
\end{equation}
\begin{equation}
    \bold{N} = \bold{R} \times M \bold{V} + \bold{R} \times \sum_{i=1}^{N} m_i \bold{\dot{\varrho_i}} + \sum_{i=1}^{N} \bold{\varrho_i} \times m_i \bold{V} + \sum_{i=1}^{N} \bold{\varrho_i} \times m_i \bold{\dot{\varrho_i}}
\end{equation}
Ahol $M$ a pontrendszer teljes tömege. A második és harmadik tagok nullák, mivel a tömegközéppont definíciója szerint:
\begin{equation}
    \sum_{i=1}^{N} m_i \bold{\varrho_i} = 0 \quad \Rightarrow \quad \sum_{i=1}^{N} m_i \bold{\dot{\varrho_i}} = 0
\end{equation}
\begin{equation}
    \sum_{i=1}^{N} \bold{\varrho_i} \times m_i \bold{V} = \left(\sum_{i=1}^{N} m_i \bold{\varrho_i}\right) \times \bold{V} = 0
\end{equation}
Így az impulzusmomentum a következőképpen egyszerűsödik:
\begin{equation}
    \bold{N} = \bold{R} \times M \bold{V} + \sum_{i=1}^{N} \bold{\varrho_i} \times m_i \bold{\dot{\varrho_i}}
\end{equation}
Ez az egyenlet azt mutatja, hogy a pontrendszer impulzusmomentuma két részből áll: az egyik a tömegközéppont mozgásásából származik, a másik pedig a pontrendszer belső mozgásából a tömegközépponthoz képest.
Most nézzük meg a nyomaték definícióját:
\begin{equation}
    \bold{M} = \sum_{i=1}^{N} (\bold{R} + \bold{\varrho_i}) \times \bold{F_i}^{ext}
\end{equation}
\begin{equation}
    \bold{M} = \bold{R} \times \sum_{i=1}^{N} \bold{F_i}^{ext} + \sum_{i=1}^{N} \bold{\varrho_i} \times \bold{F_i}^{ext}
\end{equation}
Ez az egyenlet azt mutatja, hogy a pontrendszer nyomatéka két részből áll: az egyik a tömegközéppont helyzetéből származik, a másik pedig a pontrendszer belső elrendeződéséből a tömegközépponthoz képest.
Most helyettesítsük be ezeket az egyenleteket a pontrendszer forgásának mozgásegyenletébe:
\begin{equation}
    \frac{d}{dt} \left(\bold{R} \times M \bold{V} + \sum_{i=1}^{N} \bold{\varrho_i} \times m_i \bold{\dot{\varrho_i}}\right) = \bold{R} \times \sum_{i=1}^{N} \bold{F_i}^{ext} + \sum_{i=1}^{N} \bold{\varrho_i} \times \bold{F_i}^{ext}
\end{equation}
\begin{equation}
    \frac{d}{dt} (\bold{R} \times M \bold{V}) + \frac{d}{dt} \left(\sum_{i=1}^{N} \bold{\varrho_i} \times m_i \bold{\dot{\varrho_i}}\right) = \bold{R} \times \sum_{i=1}^{N} \bold{F_i}^{ext} + \sum_{i=1}^{N} \bold{\varrho_i} \times \bold{F_i}^{ext}
\end{equation}
Ahol az első tag a tömegközéppont mozgásából származó impulzusmomentum időbeli változását jelenti, a második tag pedig a pontrendszer belső mozgásából származó impulzusmomentum időbeli változását jelenti.
Ez az egyenlet tehát a pontrendszer forgásának mozgásegyenlete a tömegközéppont helyzetéhez képest. Az egyenlet két részre bontható: az egyik a tömegközéppont mozgására vonatkozik, a másik pedig a pontrendszer belső mozgására a tömegközépponthoz képest.

\subsection{Merev testek egyensúlyfeltétele, tehetetlenségi tenzor, pörgettyűk}
A merev testet úgy definiáljuk, hogy a test pontjai közötti távolságok állandóak maradnak a test mozgása során. Ez azt jelenti, hogy a test alakja és mérete nem változik meg, függetlenül attól, hogy a test hogyan mozog.
Ennek alapján egy merev test pontos pozícióját három koordinátával és három forgási szöggel lehet meghatározni. A test mozgását tehát hat paraméter írja le: három transzlációs és három rotációs.
A pontrendszerek egyenleteit peddig a következőképpen írhatjuk fel:
\begin{equation}
    M \bold{\ddot{R}} = \sum_{i=1}^{N} \bold{F_i}^{ext}
\end{equation}
\begin{equation}
    \frac{d\bold{N}}{dt} = \sum_{i=1}^{N} \bold{M_i}^{ext}
\end{equation}
Ahol $M$ a merev test tömege, $\bold{R}$ a tömegközéppont helyvektora, $\bold{N}$ a merev test impulzusmomentuma, $\bold{F_i}^{ext}$ az i-edik pontra ható külső erő, és $\bold{M_i}^{ext}$ az i-edik pontra ható külső nyomaték.
Ez pontosan hat daab egyenletet jelent (3 transzlációs és 3 rotációs), ami megegyezik a merev test mozgását leíró paraméterek számával. Ezért a merev test mozgása teljesen meghatározható ezekkel az egyenletekkel.
Tehát a merev test mozgását a következő egyenlettel lehet leírni:
\begin{equation}
    \varDelta \bold{r} = \varDelta \bold{R} + \varDelta \bold{\varphi} \times \bold{\bold{r} - \bold{R}}
\end{equation}
Ahol $\varDelta \bold{r}$ a test egy pontjának elmozdulása, $\varDelta \bold{R}$ a tömegközéppont elmozdulása, $\bold{\varphi}$ a test szögelfordulása, és $\bold{r} - \bold{R}$ a pont helyvektora a tömegközépponthoz képest.
Az a pont, amelyiket a $\bold{r}$ vektorral jelölünk, a test bármely pontja lehet. Ez az egyenlet tehát azt mutatja, hogy a test bármely pontjának elmozdulása a tömegközéppont elmozdulásából és a test forgásából származik. 
Innen a test sebessége a következőképpen kapható meg:
\begin{equation}
    \frac{d}{dt}\bold{r} = \frac{d}{dt}\bold{R} + \frac{d}{dt}\bold{\varphi} \times (\bold{r} - \bold{R})
\end{equation}
\begin{equation}
    \bold{v} = \bold{V} + \bold{\omega} \times (\bold{r} - \bold{R})
\end{equation}
Ahol $\bold{v}$ a test egy pontjának sebességvektora, $\bold{V}$ a tömegközéppont sebességvektora, és a többi jelölés megegyezik az előző egyenletben. Itt $\bold{V}$ és $\bold{\omega}$ 3-3 komponensű vektorok, vagyis összesen 6 komponenssel rendelkeznek, ami megegyezik a merev test mozgását leíró paraméterek számával. 

Mi történik, ha meg akarom változtatni a kitüntetett pontot?
\begin{equation}
    \varDelta \bold{r} = \varDelta \bold{R} + \varDelta \bold{\varphi} \times (\bold{r} - \bold{R})
\end{equation}
\begin{equation}
    \varDelta \bold{r} = \varDelta (\bold{R} + \bold{a}) + \varDelta \bold{\varphi'} \times (\bold{r} - \bold{R} + \bold{a})
\end{equation}
ahol $\bold{a}$ a kitüntetett pont a kitüntetett pont eltolásvektora.
\begin{equation}
    \varDelta \bold{r} = \varDelta \bold{R} + \varDelta \bold{a} + \varDelta \bold{\varphi'} \times (\bold{r} - \bold{R} + \bold{a})
\end{equation}
$\varDelta \bold{a} = 0$, mert a kitüntetett pont nem változik.
\begin{equation}
    \varDelta \bold{r} = \varDelta \bold{R} + \varDelta \bold{\varphi'} \times (\bold{r} - \bold{R}) + \varDelta \bold{\varphi'} \times \bold{a}
\end{equation}
Itt tetszőleges $\varDelta \bold{r}$ csak akkor lehet egyenlő az eredetivel, ha $\varDelta \bold{\varphi'} = \varDelta \bold{\varphi}$. Az egyenletbe bejön egy plusz tag ami a koordinátarendszer eltolásából származik,
de a szögselfordulásnak meg kell egyeznie, bármelyik kitüntetett pontot is választjuk. Ezért a szögelfordulás vektora független a kitüntetett pont helyzetétől.

\subsubsection{Merev test egyensúlyfeltétele}
Egy merev test akkor van egyensúlyban, ha a testre ható erők és nyomatékok eredője nulla. Matematikailag ez a következőképpen írható fel:
\begin{equation}
    \sum_{i=1}^{N} \bold{F_i}^{ext} = 0
\end{equation}
\begin{equation}
    \sum_{i=1}^{N} \bold{M_i}^{ext} = 0
\end{equation}
Ahol $\bold{F_i}^{ext}$ az i-edik pontra ható külső erő, és $\bold{M_i}^{ext}$ az i-edik pontra ható külső nyomaték. Ezek az egyenletek azt jelentik, hogy a test nem gyorsul és nem forog.
Ezek az egyenletek hat független feltételt jelentenek (3 erő és 3 nyomaték komponens), ami megegyezik a merev test mozgását leíró paraméterek számával. Ezért a merev test egyensúlyi állapota teljesen meghatározható ezekkel az egyenletekkel.

\subsubsection{Tehetetlenségi tenzor}
A fogatónyomatékot egy merev testre a következőképpen definiálhatjuk:
\begin{equation}
    \bold{N} = \bold{R} \times M \bold{V} + \bold{N_s}
\end{equation}
Ahol $\bold{N_s}$ a test belső impulzusmomentuma a tömegközépponthoz képest:
\begin{equation}
    \bold{N_s} = \sum_{i=1}^{N} \bold{\varrho_i} \times m_i \bold{\dot{\varrho_i}}
\end{equation}
Ahol $\bold{\varrho_i}$ az i-edik pont helyvektora a tömegközépponthoz képest, és $\bold{\dot{\varrho_i}}$ az i-edik pont sebességvektora a tömegközépponthoz képest.
$\bold{\dot{\varrho_i}}$ a következőképpen adható meg:
\begin{equation}
    \bold{\dot{\varrho_i}} = \bold{\omega} \times \bold{\varrho_i}
\end{equation}
Ezt behelyettesítve az impulzusmomentum definíciójába:
\begin{equation}
    \bold{N_s} = \sum_{i=1}^{N} \bold{\varrho_i} \times m_i (\bold{\omega} \times \bold{\varrho_i})
\end{equation}
Ezt a vektoriális szorzat azonosság segítségével tovább egyszerűsíthetjük:
\begin{equation}
    \bold{N_s} = \sum_{i=1}^{N} m_i \left[\bold{\varrho_i} (\bold{\varrho_i} \cdot \bold{\omega}) - \bold{\omega} (\bold{\varrho_i} \cdot \bold{\varrho_i})\right]
\end{equation}
\begin{equation}
    \bold{N_s} = \sum_{i=1}^{N} m_i \left[(\bold{\varrho_i} \bold{\varrho_i}) \cdot \bold{\omega} - (\bold{\varrho_i} \circ \bold{\varrho_i}) \cdot \bold{\omega}\right]
\end{equation}
ahol a $(\bold{\varrho_i} \circ \bold{\varrho_i})_{j} = (\bold{\varrho_i})_{j} (\bold{\varrho_i})_{j}$ a hadamard szorzatot jelenti.
\begin{equation}
    \bold{N_s} = \left[\sum_{i=1}^{N} m_i (\bold{\varrho_i} \bold{\varrho_i}) - \sum_{i=1}^{N} m_i (\bold{\varrho_i} \circ \bold{\varrho_i})\right] \cdot \bold{\omega}
\end{equation}
Ezt a kifejezést a tehetetlenségi tenzor segítségével is felírhatjuk:
\begin{equation}
    \bold{N_s} = \bold{\hat{I}} \cdot \bold{\omega}
\end{equation}
Ahol a tehetetlenségi tenzor a következőképpen definiálható:
\begin{equation}
    \bold{\hat{I}} = \sum_{i=1}^{N} m_i \left[(\bold{\varrho_i} \circ \bold{\varrho_i}) - (\bold{\varrho_i} \bold{\varrho_i})\right]
\end{equation}
A tehetetlenségi tenzor egy szimmetrikus mátrix, amely a merev test tömegének eloszlását jellemzi a tömegközéppont körül. A tehetetlenségi tenzor komponensei a következőképpen számíthatók ki:
\begin{equation}
\begin{bmatrix}
\sum_{i=1}^{N} m_i (y_i^2 + z_i^2) & -\sum_{i=1}^{N} m_i x_i y_i & -\sum_{i=1}^{N} m_i x_i z_i \\
-\sum_{i=1}^{N} m_i y_i x_i & \sum_{i=1}^{N} m_i (x_i^2 + z_i^2) & -\sum_{i=1}^{N} m_i y_i z_i \\
-\sum_{i=1}^{N} m_i z_i x_i & -\sum_{i=1}^{N} m_i z_i y_i & \sum_{i=1}^{N} m_i (x_i^2 + y_i^2)
\end{bmatrix}
\end{equation}
Ahol $x_i$, $y_i$, és $z_i$ az i-edik pont koordinátái a tömegközépponti koordinátarendszerben. A szimmetria miatt ennek a mátrixnak mindig valós a sajátértéke, és a sajátvektorai ortogonálisak.
Végül tehát a merev test impulzusmomentuma a következőképpen írható fel:
\begin{equation}
    \bold{N_s} = \bold{\hat{I}} \cdot \bold{\omega}
\end{equation}
Ez az egyenlet azt mutatja, hogy a merev test impulzusmomentuma a tehetetlenségi tenzor és a szögsebesség vektor szorzataként kapható meg. A tehetetlenségi tenzor tehát a merev test forgási tulajdonságait jellemzi a tömegközéppont körül.

\subsubsection{pörgettyűk}
A pörgettyűk olyan merev, kiterjedt testek, amelyek egy adott tengely körül forognak. Két fajtájukat különböztetjük meg: a súlytalan pörgettyűt és a súlyos pörgettyűt.

\subsubsection*{súlytalan pörgettyű}
A súlytalan pörgettyű egy olyan ideális test, amelynek nincs tömege, és csak a forgási mozgását vizsgáljuk. A súlytalan pörgettyűre ható erők és nyomatékok a következőképpen írhatók fel:
\begin{equation}
\frac{d\bold{N}}{dt} = \sum_{i=1}^{N} \bold{M_i}^{ext}
\end{equation}
\begin{equation}
\bold{N} = \bold{\hat{I}^{**}} \cdot \bold{\omega}
\end{equation}
Ahol $\bold{\hat{I}^{**}}$ a súlytalan pörgettyű tehetetlenségi tenzora nem súlyponti koordinátarendszerben, amely a pörgettyű tömegének eloszlását jellemzi a forgástengely körül.
Innen ha felhasználjuk a gyorsuló kordináta rendszerre vonatkozó összefüggést, akkor a következőt kapjuk:
\begin{equation}
    \frac{d\bold{A}}{dt} = \frac{d'\bold{A}}{dt} + \bold{\omega} \times \bold{A} \quad \text{ bármely vektorra } \bold{A}
\end{equation}
Tehát az impulzusmomentum időbeli változása a következőképpen írható fel:
\begin{equation}
    \frac{d\bold{N}}{dt} = \frac{d'\bold{N}}{dt} + \bold{\omega} \times \bold{N} = \bold{M}
\end{equation}
Innen komponenseire bontva felírhatjuk az impulzusmomentum deriváljtainak értékét:
\begin{equation}
    \frac{d'N_x}{dt} + \omega_y N_z - \omega_z N_y = M_x
\end{equation}
\begin{equation}
    \frac{d'N_y}{dt} + \omega_z N_x - \omega_x N_z = M_y
\end{equation}
\begin{equation}
    \frac{d'N_z}{dt} + \omega_x N_y - \omega_y N_x = M_z
\end{equation}
Ahol $N_x$, $N_y$, és $N_z$ az impulzusmomentum komponensei, $M_x$, $M_y$, és $M_z$ a nyomaték komponensei, és $\omega_x$, $\omega_y$, és $\omega_z$ a szögsebesség komponensei.
Ezek az egyenleteket Euler egyenleteknek nevezzük, amelyek a súlytalan pörgettyű forgási mozgását írják le. Ezek az egyenletek hat független egyenletet jelentenek (3 impulzusmomentum és 3 nyomaték komponens), ami megegyezik a pörgettyű mozgását leíró paraméterek számával.
Ha ide behelyettesítjük a tehetetlenségi tenzor és az impulzusmomentum közötti összefüggést ($\bold{N} = \bold{\hat{I}} \cdot \bold{\omega}$), akkor a következőt kapjuk:
\begin{equation}
    \hat{I}_x \frac{d'\omega_x}{dt} + (\hat{I}_z - \hat{I}_y) \omega_y \omega_z = M_x
\end{equation}
\begin{equation}
    \hat{I}_y \frac{d'\omega_y}{dt} + (\hat{I}_x - \hat{I}_z) \omega_z \omega_x = M_y
\end{equation}
\begin{equation}
    \hat{I}_z \frac{d'\omega_z}{dt} + (\hat{I}_y - \hat{I}_x) \omega_x \omega_y = M_z
\end{equation}
Ahol $\hat{I}_x$, $\hat{I}_y$, és $\hat{I}_z$ a tehetetlenségi tenzor főtengelyei. Ezek az egyenletek a súlytalan pörgettyű forgási mozgását írják le a főtengelyek mentén, gyorsuló koordináta rendszerben ami a forgási tengelyhez van rögzítve.

Ha egy súlytalan pörgettyűről van szó, akkor a forgatónyomaték nulla ($M_x = M_y = M_z = 0$), így az egyenletek a következőképpen egyszerűsödnek:
\begin{equation}
    \hat{I}_x \frac{d'\omega_x}{dt} + (\hat{I}_z - \hat{I}_y) \omega_y \omega_z = 0
\end{equation}
\begin{equation}
    \hat{I}_y \frac{d'\omega_y}{dt} + (\hat{I}_x - \hat{I}_z) \omega_z \omega_x = 0
\end{equation}
\begin{equation}
    \hat{I}_z \frac{d'\omega_z}{dt} + (\hat{I}_y - \hat{I}_x) \omega_x \omega_y = 0
\end{equation}
Továbbá ha szimmetrikus pörgettyűről van szó, ahol két tehetetlenségi tenzor főtengely megegyezik ($\hat{I}_x = \hat{I}_y$), akkor az egyenletek a következőképpen egyszerűsödnek:
\begin{equation}
    \hat{I}_x \frac{d'\omega_x}{dt} + (\hat{I}_z - \hat{I}_x) \omega_y \omega_z = 0
\end{equation}
\begin{equation}
    \hat{I}_x \frac{d'\omega_y}{dt} + (\hat{I}_x - \hat{I}_z) \omega_z \omega_x = 0
\end{equation}
\begin{equation}
    \hat{I}_z \frac{d'\omega_z}{dt} = 0
\end{equation}
Ebből egyből látszik, hogy a $\omega_z$ szögsebesség komponens állandó, mivel a harmadik egyenlet szerint a $\frac{d'\omega_z}{dt} = 0$. 
Ha az $\omega_x$ és $\omega_y$ egyenleteketmegszorrozzuk $\omega_y$ illetve $\omega_x$-szel:
\begin{equation}
    \hat{I}_x \frac{d'\omega_x}{dt} + (\hat{I}_z - \hat{I}_x) \omega_y \omega_z = 0 \quad \cdot \omega_x
\end{equation}
\begin{equation}
    \hat{I}_x \frac{d'\omega_y}{dt} + (\hat{I}_x - \hat{I}_z) \omega_z \omega_x = 0 \quad \cdot \omega_y
\end{equation}
Akkor a következőt kapjuk:
\begin{equation}
    \hat{I}_x \omega_x \frac{d'\omega_x}{dt} + (\hat{I}_z - \hat{I}_x) \omega_x \omega_y \omega_z = 0
\end{equation}
\begin{equation}
    \hat{I}_x \omega_y \frac{d'\omega_y}{dt} + (\hat{I}_x - \hat{I}_z) \omega_y \omega_z \omega_x = 0
\end{equation}
Most adjuk össze ezeket az egyenleteket:
\begin{equation}
    \hat{I}_x \left(\omega_x \frac{d'\omega_x}{dt} + \omega_y \frac{d'\omega_y}{dt}\right) + (\hat{I}_z - \hat{I}_x) \omega_x \omega_y \omega_z + (\hat{I}_x - \hat{I}_z) \omega_y \omega_z \omega_x = 0
\end{equation}
\begin{equation}
    \hat{I}_x \left(\omega_x \frac{d'\omega_x}{dt} + \omega_y \frac{d'\omega_y}{dt}\right) = 0
\end{equation}
\begin{equation}
    \frac{d'}{dt} \left(\frac{1}{2} \hat{I}_x (\omega_x^2 + \omega_y^2)\right) = 0
\end{equation}
Ebből látszik, hogy a $\frac{1}{2} \hat{I}_x (\omega_x^2 + \omega_y^2)$ mennyiség állandó, és mivel azt már előbb beláttuk, hogy a $\omega_z$ is állandó, ezért a teljes szögsebesség vektor nagysága is állandó:
\begin{equation}
    \omega^2 = \omega_x^2 + \omega_y^2 + \omega_z^2 = \text{állandó}
\end{equation}
Ez azt jelenti, hogy a súlytalan szimmetrikus pörgettyű szögsebesség vektora állandó nagyságú. Mivel $\bold{M}$ nulla, ezért az impulzusmomentum is állandó:
\begin{equation}
    \frac{d}{dt} \bold{N} = \bold{M} = 0
\end{equation}
\begin{equation}
    \bold{N} = \text{állandó}
\end{equation}
Ebből következik, hogy az egyetlen dolog ami meg tud változni az a szögsebesség iránya, mivel a nagysága állandó. Ez azt jelenti, hogy a szögsebesség vektor a az impulzusmomentum vektor körül körbeforog, ezt nevezik nutációnak.

\subsubsection*{súlyos pörgettyű}
A súlyos pörgettyű egy olyan merev test, amelynek van tömege, és a gravitációs erő hat rá. A súlyos pörgettyűre ható erők és nyomatékok a következőképpen írhatók fel:
\begin{equation}
    \frac{d\bold{N}}{dt} = \sum_{i=1}^{N} \bold{M_i}^{ext}
\end{equation}
\begin{equation}
    \bold{N} = \bold{\hat{I}^{**}} \cdot \bold{\omega}
\end{equation}
Ahol $\bold{\hat{I}^{**}}$ a súlyos pörgettyű tehetetlenségi tenzora nem súlyponti koordinátarendszerben, amely a pörgettyű tömegének eloszlását jellemzi a forgástengely körül.
A súlyos pörgettyűre ható külső nyomaték a következőképpen írható fel:
\begin{equation}
    \bold{M} = \bold{r_c} \times M \bold{g}
\end{equation}
Ahol $\bold{r_c}$ a tömegközéppont helyvektora a forgástengelyhez képest, $M$ a pörgettyű tömege, és $\bold{g}$ a gravitációs gyorsulás vektora.
Ez az egyenlet azt mutatja, hogy a súlyos pörgettyű forgási mozgását a tehetetlenségi tenzor, a szögsebesség vektor, és a gravitációs erő határozza meg. A súlyos pörgettyű mozgása tehát bonyolultabb, mint a súlytalan pörgettyűé, mivel a gravitációs erő hat rá.
Ezek alapján az Euler egyenletek a súlyos pörgettyűre a következőképpen írhatók fel:
\begin{equation}
    \hat{I}_x \frac{d'\omega_x}{dt} + (\hat{I}_z - \hat{I}_y) \omega_y \omega_z = M_x
\end{equation}
\begin{equation}
    \hat{I}_y \frac{d'\omega_y}{dt} + (\hat{I}_x - \hat{I}_z) \omega_z \omega_x = M_y
\end{equation}
\begin{equation}
    \hat{I}_z \frac{d'\omega_z}{dt} + (\hat{I}_y - \hat{I}_x) \omega_x \omega_y = M_z
\end{equation}
Ahol $M_x$, $M_y$, és $M_z$ a külső nyomaték komponensei, amelyeket a gravitációs erő határoz meg.

\subsubsection{Energia, impulzus és impulzusmomentum megmaradás merev testek esetén}
Egy merev test esetén ugyan úgy igaz, hogy a teljes energia a kinematikai és potenciális energia összege:
\begin{equation}
    E = E_{kin} + E_{pot}
\end{equation}
A kinematikai energia esetében figyelembe kell venni, hogy a test nem pontszrű, tehát a tömegét összegezni kell a test minden pontján:
\begin{equation}
    E_{kin} = \frac{1}{2} m v^2 \quad \rightarrow \quad E_{kin} = \int_{V} \frac{1}{2} \bold{v}^2 \varrho dV
\end{equation}
\begin{equation}
    \bold{v} = \bold{v_0} + \bold{\omega} \times \bold{r}
\end{equation}
\begin{equation}
    E_{kin} = \int_{V} \frac{1}{2} \bold{v}^2 \varrho dV = \frac{1}{2} \int_{V} \bold{v_0}^2 \varrho dV + \frac{1}{2} \int_{V} (\bold{\omega} \times \bold{r})^2 \varrho dV + \int_{V} \bold{v_0} \cdot (\bold{\omega} \times \bold{r}) \varrho dV
\end{equation}
Ahol $\bold{v_0}$ a test tömegközéppontjának sebességvektora, $\bold{\omega}$ a test szögsebesség vektora, $\bold{r}$ a test egy pontjának helyvektora a tömegközépponthoz képest, és $\varrho$ a test sűrűsége.
Itt a kinematikai energia három részre bontható: az első tag a test tömegközéppontjának transzlációs mozgásából származik, a második tag a test forgási mozgásából származik a tömegközépponthoz képest, és a harmadik tag a két mozgás közötti kölcsönhatásból származik.
A harmadik tag nullává válik tömegközépponti rendszerben, mivel a tömegközéppont definíciója szerint:
\begin{equation}
    \int_{V} \bold{r} \varrho dV = 0
\end{equation}
Ha a kapott eredménybe behelyettesítjük a tehetetlenségi tenzor definícióját, akkor a kinematikai energia a következőképpen írható fel:
\begin{equation}
    \bold{\hat{I}} = \int_{V} [(\bold{r} \cdot \bold{r}) - \bold{r} \circ \bold{r}] \varrho dV
\end{equation}
A forgási energia képletét átalakíthatjuk a következő alakra:
\begin{equation}
    \frac{1}{2} \int_{V} (\bold{\omega} \times \bold{r})^2 \varrho dV = \frac{1}{2} \int_{V} \omega^2 \bold{r}^2 - (\bold{\omega} \cdot \bold{r})^2 \varrho dV
\end{equation}
A $(\bold{\omega} \cdot \bold{r})^2$-t komponenseire bontjuk:
\begin{equation}
    (\bold{\omega} \cdot \bold{r})^2 = (\omega_x r_x + \omega_y r_y + \omega_z r_z)^2 = \omega_x^2 r_x^2 + \omega_y^2 r_y^2 + \omega_z^2 r_z^2 + 2 \omega_x \omega_y r_x r_y + 2 \omega_y \omega_z r_y r_z + 2 \omega_z \omega_x r_z r_x
\end{equation}
Itt a keresztszorzatok nullává válnak, mivel a tömegközépponti rendszerben a következő feltétel teljesül:
\begin{equation}
    \int_{V} r_i r_j \varrho dV = 0 \quad \text{ ha } i \neq j
\end{equation}
Így a képlet a következőképpen egyszerűsödik:
\begin{equation}
    \int_{V} (\bold{\omega} \cdot \bold{r})^2 \varrho dV = \int_{V}\omega_x^2 r_x^2 + \omega_y^2 r_y^2 + \omega_z^2 r_z^2 \varrho dV
\end{equation}
Ez pedig pontosan ugyanaz mint, ha a hadamard szorzatot alkalmazzuk:
\begin{equation}
    (\bold{\omega} \cdot \bold{r})^2 = \omega^T \cdot (\bold{r} \circ \bold{r}) \cdot \omega
\end{equation}
Az egészet összetéve és az $\omega$-kat kiemelve pedig:
\begin{equation}
    \frac{1}{2} \int_{V} (\bold{\omega} \times \bold{r})^2 \varrho dV = \frac{1}{2} \omega^T \cdot \left(\int_{V} [(\bold{r} \cdot \bold{r}) - (\bold{r} \circ \bold{r})] \varrho dV\right) \cdot \omega = \frac{1}{2} \omega^T \cdot \bold{\hat{I}} \cdot \omega
\end{equation}
Tehát a teljes kinematikai energia a következőképpen írható fel:
\begin{equation}    
    E_{kin} = \frac{1}{2} M \bold{v_0}^2 + \frac{1}{2} \bold{\omega}^T \cdot \bold{\hat{I}} \cdot \bold{\omega}
\end{equation}
Ha az energiamegmaradás törvényét alkalmazzuk Konzervatív erőtérre, akkor a következőt kapjuk:
\begin{equation}
    \frac{dE}{dt} = \frac{dE_{kin}}{dt} + \frac{dE_{pot}}{dt} = 0
\end{equation}
\begin{equation}
    \frac{dE_{kin}}{dt} = \frac{d}{dt} \left(\frac{1}{2} M \bold{v_0}^2 + \frac{1}{2} \bold{\omega}^T \cdot \bold{\hat{I}} \cdot \bold{\omega}\right) = M \bold{v_0} \cdot \frac{d\bold{v_0}}{dt} + \bold{\omega}^T \cdot \bold{\hat{I}} \cdot \frac{d\bold{\omega}}{dt}
\end{equation}
Tehát látszódik, hogy az energiamegmaradás feltétele úgy módosul, hogy a merev test transzlációs, rotációs és potenciális enegiájának összege állandó marad:
\begin{equation}
    \frac{d}{dt} \left(\frac{1}{2} M \bold{v_0}^2 + \frac{1}{2} \bold{\omega}^T \cdot \bold{\hat{I}} \cdot \bold{\omega} + E_{pot}\right) = 0
\end{equation}

\subsubsection*{Impulzus és impulzusmomentum megmaradás}
Egy merev testre az impulzus a következőképpen írható fel:
\begin{equation}
    \bold{p} = M \bold{v_0} + \sum_{i=1}^{N} m_i \bold{\dot{\varrho_i}} = M \bold{v_0} + \sum_{i=1}^{N} m_i (\bold{\omega} \times \bold{\varrho_i}) = M \bold{v_0} + \bold{\omega} \times M \bold{\varrho_i}
\end{equation}
Az impulzusmomentum pedig hasonlóképpen:
\begin{equation}
    \bold{N} = \bold{R} \times M \bold{v_0} + \int_{V} \bold{r} \times \varrho (\bold{\omega} \times \bold{r}) dV = \bold{R} \times M \bold{v_0} + \bold{\hat{I}} \cdot \bold{\omega}
\end{equation}
Ha a testre ható külső erők eredője nulla, akkor az impulzus megmarad:
\begin{equation}
    \frac{d\bold{p}}{dt} = \sum_{i=1}^{N} \bold{F_i}^{ext} = 0
\end{equation}
\begin{equation}
    \bold{p} = \text{állandó}
\end{equation}
Ha a testre ható külső nyomatékok eredője nulla, akkor az impulzusmomentum megmarad:
\begin{equation}
    \frac{d\bold{N}}{dt} = \sum_{i=1}^{N} \bold{M_i}^{ext} = 0
\end{equation}
\begin{equation}
    \frac{d\bold{N}}{dt} = \frac{d}{dt} \left(\bold{R} \times M \bold{v_0} + \bold{\hat{I}} \cdot \bold{\omega}\right) = 0
\end{equation}
\begin{equation}
    \bold{N} = \text{állandó}
\end{equation}  

\subsection{Galilei-, Lorentz-transzformáció, relativisztikus kinematika, relativisztikus dinamika. Négyesimpulzus.}

\subsubsection{Relativitás elve}
 A relativitás elve az inerciarendszerek közötti kapcsolatot vizsgálja. Newton törvényei alapján kijelentettük, hogy léteznek inercia rendszerek, és ezek a rendszerek amikben a 
 Newton törvények érvényesek. Ezek alapján, a fizika tövényei két tetszőleges inerciarendszerben ugyan azok. Emellett Einstein speciális relativitáselmélete szerint, a fénysebeség ($c$)
 vákuumbeli nagysága is megegyezik minden inerciarendszerben. Ez két stacionárius rendsze között tiviális, ha azt a posztulátumot is elfogadjuk, hogy nincs kitüntetett pont a világegyetemben.
 De mi van akkor, ha a két inerciarendszer egymáshoz képest $\bold{v}$ sebességgel mozog? Ekkor nincs gyorsulás, ergo továbbra is inerciarendszerekről beszélünk, de a koordináták átváltásánál figyelembe kell venni a relatív mozgást is.
 Ennek a módszerét a Galilei transzformációval írhatjuk le.

\subsubsection{Galilei transzformáció}
    A Galilei transzformáció egy olyan matematikai eszköz, amely lehetővé teszi a koordináták és idő átváltását két inerciarendszer között, amelyek egymáshoz képest állandó sebességgel mozognak.
    Legyen két inerciarendszer: az egyik a $S$ rendszer, amelyben egy esemény helyét és idejét $(x, y, z, t)$-vel jelöljük, és a másik a $S'$ rendszer, amely a $S$ rendszerhez képest $\bold{v}$ sebességgel mozog.
    A Galilei transzformáció a következőképpen írható fel:
    \begin{equation}
        x' = x - vt
    \end{equation}
    \begin{equation}
        y' = y
    \end{equation}
    \begin{equation}
        z' = z
    \end{equation}
    \begin{equation}
        t' = t
    \end{equation}
    Itt az $x'$ koordináta a $S'$ rendszerben az $x$ koordinátából származik úgy, hogy levonjuk a $vt$ értéket, ami a $S'$ rendszer elmozdulását jelenti az idő $t$ alatt. Az $y$ és $z$ koordináták nem változnak,
    mivel a mozgás csak az $x$ tengely mentén történik. Az idő pedig mindkét rendszerben ugyanaz. Viszont van egy probléma a Galilei transzformációval, mégpedig az, hogy nem veszi figyelembe a fénysebesség állandóságát.
    Ha a Galilei transzformációt alkalmazzuk egy fényimpulzusra, amely a $S$ rendszerben $c$ sebességgel mozog, akkor a $S'$ rendszerben a fénysebesség $c - v$ lesz, ami ellentmond Einstein posztulátumának.
    Emiatt a Galilei transzformációt nem lehet alkalmazni relativisztikus sebességek esetén, ahol a fénysebességhez közeli sebességekről van szó. Ilyen esetekben a Lorentz transzformációt kell használni.
\subsubsection{Lorentz transzformáció}
Tegyük fel, hogy az álló inerciarendszerben, egy esemény történik (pl. egy villanás), amely a tér minden irányába terjed $c$ fénysebességgel. Az esemény helyét és idejét az álló rendszerben $(x, y, z, t)$-vel jelöljük.
Ekkor a következő feltétel teljesül:
\begin{equation}
    |\bold{r}| = ct
\end{equation}
\begin{equation}
    \sqrt{x^2 + y^2 + z^2} = ct
\end{equation}
\begin{equation}
    x^2 + y^2 + z^2 = c^2 t^2
\end{equation}
\begin{equation}
    x^2 + y^2 + z^2 - c^2 t^2 = 0
\end{equation}
Ugyanezt felírhatjuk a $\bold{v}$ sebességgel mozgó rendszerben is, ahol az esemény helyét és idejét $(x', y', z', t')$-vel jelöljük:
\begin{equation}
    x'^2 + y'^2 + z'^2 - c^2 t'^2 = 0
\end{equation}
Mivel mind a két egyenlet nulla, ezért egyenlőek egymással:
\begin{equation}
    x^2 + y^2 + z^2 - c^2 t^2 = x'^2 + y'^2 + z'^2 - c^2 t'^2
\end{equation}
Mivel a két rendszer között csak az $x$ tengely mentén van relatív mozgás, ezért $y = y'$ és $z = z'$:
\begin{equation}
    x^2 - c^2 t^2 = x'^2 - c^2 t'^2
\end{equation}
Ezt az egyenletet nem lehet úgy kielégíteni, hogy $x' = x - vt$ és $t' = t$, mert akkor a fénysebesség nem lenne állandó. Ezért nézzük meg, hogy mi történik, ha az esemény, a mozgás síkjában történik, tehát $y = z = 0$,
akkor amikor a két koordináta redszer középpontja pont egybeesik.
\begin{figure}[H]
    \centering
    \includegraphics[width=0.5\textwidth]{imgs/1-tetel/lorentz.png}
    \caption{Lorentz transzformáció}
    \label{fig:lorentz}
\end{figure}
Ekkor $t$ időpillanatban $S'$ koordináta-rendszer középpontja a $S$ koordináta-rendszer $x = vt$ helyén van. A mozgó rendszer középpontja és az esemény közötti távolság a $S$ rendszerben $x - vt$,
míg a $S'$ rendszerben $x'$. Tehát van egy problémánk ami szerint:
\begin{equation}
    x' \neq x - vt
\end{equation}
A fénynek mind a két rendszerben ugyan olyan gyorsan kell terjednie, tehát ez nem lehet igaz, ha a távolságokat és az időt állandnak vesszük. Ezért feltételezzük, hogy a távolságok megváltozik a két rendszer között.
Tehát legyen:
\begin{equation}
    x' = \gamma (x - vt)
\end{equation}
az idő pedig:
\begin{equation}
    t' = \beta x + \alpha t
\end{equation}
ahol $\gamma$, $\beta$ és $\alpha$ olyan állandók, amelyeket meg kell határozni. Ezeket az állandókat úgy határozhatjuk meg, hogy a fénysebesség állandóságának feltételét alkalmazzuk:
\begin{equation}
    x^2 - c^2 t^2 = x'^2 - c^2 t'^2
\end{equation}
Behelyettesítve az $x'$ és $t'$ kifejezéseket:
\begin{equation}
    x^2 - c^2 t^2 = \gamma^2 (x - vt)^2 - c^2 (\beta x + \alpha t)^2
\end{equation}
Kibontva és összegyűjtve a tagokat:
\begin{equation}
    x^2 - c^2 t^2 = \gamma^2 (x^2 - 2vxt + v^2 t^2) - c^2 (\beta^2 x^2 + 2 \alpha \beta xt + \alpha^2 t^2)
\end{equation}
\begin{equation}
    x^2 - c^2 t^2 = (\gamma^2 - c^2 \beta^2) x^2 + (-2 \gamma^2 v - 2 c^2 \alpha \beta) xt + (\gamma^2 v^2 - c^2 \alpha^2) t^2
\end{equation}
Most összehasonlítva a két oldalt, három egyenletet kapunk:
\begin{equation}
    1 = \gamma^2 - c^2 \beta^2
\end{equation}
\begin{equation}
    0 = -2 \gamma^2 v - 2 c^2 \alpha \beta
\end{equation}
\begin{equation}
    -c^2 = \gamma^2 v^2 - c^2 \alpha^2
\end{equation}
Ebből a második egyenletből kifejezhetjük az $\alpha$-t:
\begin{equation}
    \alpha = -\frac{\gamma^2 v}{c^2 \beta}
\end{equation}
Ezt behelyettesítve a harmadik egyenletbe:
\begin{equation}
    -c^2 = \gamma^2 v^2 - c^2 \left(-\frac{\gamma^2 v}{c^2 \beta}\right)^2
\end{equation}
\begin{equation}
    -c^2 = \gamma^2 v^2 - \frac{\gamma^4 v^2}{\beta^2 c^2}
\end{equation}
\begin{equation}
    -c^4 \beta^2 = \gamma^2 v^2 \beta^2 - \gamma^4 v^2
\end{equation}
\begin{equation}
    \beta^2 (c^4 + \gamma^2 v^2) = \gamma^4 v^2
\end{equation}
\begin{equation}
    \beta^2 = \frac{\gamma^4 v^2}{c^4 + \gamma^2 v^2}
\end{equation}
Ezt behelyettesítve az első egyenletbe:
\begin{equation}
    1 = \gamma^2 - c^2 \frac{\gamma^4 v^2}{c^4 + \gamma^2 v^2}
\end{equation}
\begin{equation}
    1 = \gamma^2 - \frac{\gamma^4 v^2}{c^2 + \frac{\gamma^2 v^2}{c^2}}
\end{equation}
\begin{equation}
    1 = \gamma^2 - \frac{\gamma^4 v^2}{c^2 + \frac{\gamma^2 v^2}{c^2}} = \gamma^2 - \frac{\gamma^4 v^2 c^2}{c^4 + \gamma^2 v^2}
\end{equation}
\begin{equation}
    (c^4 + \gamma^2 v^2) = \gamma^2 (c^4 + v^2) - \gamma^4 v^2 c^2
\end{equation}
\begin{equation}
    c^4 + \gamma^2 v^2 = \gamma^2 c^4 + \gamma^2 v^2 - \gamma^4 v^2 c^2
\end{equation}
\begin{equation}
    c^4 = \gamma^2 c^4 - \gamma^4 v^2 c^2
\end{equation}
\begin{equation}
    c^4 = \gamma^2 c^2 (c^2 - \gamma^2 v^2)
\end{equation}
\begin{equation}
    \gamma^2 = \frac{c^2}{c^2 - v^2}
\end{equation}
\begin{equation}
    \gamma = \frac{1}{\sqrt{1 - \frac{v^2}{c^2}}}
\end{equation}
Ezt visszahelyettesítve a $\beta$ és $\alpha$ kifejezésekbe:
\begin{equation}
    \beta = -\frac{\gamma^2 v}{c^2} = -\frac{v}{c^2 (1 - \frac{v^2}{c^2})} = -\frac{v}{c^2 - v^2}
\end{equation}
\begin{equation}
    \alpha = -\frac{\gamma^2 v}{c^2 \beta} = -\frac{\frac{c^2}{c^2 - v^2} v}{c^2 \left(-\frac{v}{c^2 - v^2}\right)} = -\frac{c^2}{c^2 - v^2}
\end{equation}
Így a Lorentz transzformáció végső alakja a következő:
\begin{equation}
    x' = \frac{x - vt}{\sqrt{1 - \frac{v^2}{c^2}}}
\end{equation}
\begin{equation}
    y' = y
\end{equation}
\begin{equation}
    z' = z
\end{equation}
\begin{equation}
    t' = \frac{t - \frac{vx}{c^2}}{\sqrt{1 - \frac{v^2}{c^2}}}
\end{equation}
A Lorentz transzformáció tehát figyelembe veszi a fénysebesség állandóságát, és lehetővé teszi a koordináták és idő átváltását két inerciarendszer között, amelyek egymáshoz képest állandó sebességgel mozognak.
A másik irányú transzformáció pedig a következő:
\begin{equation}
    x = \frac{x' + vt'}{\sqrt{1 - \frac{v^2}{c^2}}}
\end{equation}
\begin{equation}
    y = y'
\end{equation}
\begin{equation}
    z = z'
\end{equation}
\begin{equation}
    t = \frac{t' + \frac{vx'}{c^2}}{\sqrt{1 - \frac{v^2}{c^2}}}
\end{equation}

\subsection{Relativisztikus kinematika}
A relativizmus hatásai nem csak a koordinátákra és időre korlátozódnak, hanem a sebességekre is. A Galilei transzformáció szerint a sebességek egyszerűen összeadódnak, de a Lorentz transzformáció szerint a sebességek összeadásának módja megváltozik.
tegyük fel, hogy egy test a $\bold{w}$ sebességgel mozog a $S$ rendszerben:
\begin{equation}
    \bold{w} = \frac{\bold{dr}}{dt} = \left(\frac{dx}{dt}, \frac{dy}{dt}, \frac{dz}{dt}\right) = (w_x, w_y, w_z)
\end{equation}
Ha megnézzzük az x komponensét a mozgó rendszerben:
\begin{equation}
    w'_x = \frac{dx'}{dt}
\end{equation}
Ebbe behelyettesítve a Lorentz transzformációt:
\begin{equation}
    w'_x = \frac{d}{dt} \left(\frac{x - vt}{\sqrt{1 - \frac{v^2}{c^2}}}\right)
\end{equation}
\begin{equation}
    w'_x = \frac{1}{\sqrt{1 - \frac{v^2}{c^2}}} \left(\frac{dx}{dt} - v\frac{dt}{dt}\right)
\end{equation}
\begin{equation}
    w'_x = \frac{w_x - v}{\sqrt{1 - \frac{v^2}{c^2}}}
\end{equation}
Ez a relativisztikus sebességösszeadás x komponense. Hasonlóan kiszámítható a y és z komponens is:
\begin{equation}
    w'_y = \frac{dy'}{dt} = \frac{d}{dt} (y) = w_y
\end{equation}
\begin{equation}
    w'_z = \frac{dz'}{dt} = \frac{d}{dt} (z) = w_z
\end{equation}
De itt figyelembe kell venni, hogy az idő is megváltozik a két rendszer között, tehát a teljes kifejezés a következő lesz:
\begin{equation}
    w'_y = \frac{dy'}{dt'} = \frac{dy}{dt'} = \frac{dy}{dt} \cdot \frac{dt}{dt'} = w_y \cdot \frac{dt}{dt'}
\end{equation}
\begin{equation}
    w'_z = \frac{dz'}{dt'} = \frac{dz}{dt'} = \frac{dz}{dt} \cdot \frac{dt}{dt'} = w_z \cdot \frac{dt}{dt'}
\end{equation}
Most kiszámíthatjuk a $\frac{dt}{dt'}$ kifejezést a Lorentz transzformációból:
\begin{equation}
    t' = \frac{t - \frac{vx}{c^2}}{\sqrt{1 - \frac{v^2}{c^2}}}
\end{equation}
\begin{equation}
    \frac{dt'}{dt} = \frac{1 - \frac{v}{c^2} \frac{dx}{dt}}{\sqrt{1 - \frac{v^2}{c^2}}} = \frac{1 - \frac{v w_x}{c^2}}{\sqrt{1 - \frac{v^2}{c^2}}}
\end{equation}
\begin{equation}
    \frac{dt}{dt'} = \frac{\sqrt{1 - \frac{v^2}{c^2}}}{1 - \frac{v w_x}{c^2}}
\end{equation}
Ezt visszahelyettesítve a $w'_y$ és $w'_z$ kifejezésekbe:
\begin{equation}
    w'_y = w_y \cdot \frac{\sqrt{1 - \frac{v^2}{c^2}}}{1 - \frac{v w_x}{c^2}}
\end{equation}
\begin{equation}
    w'_z = w_z \cdot \frac{\sqrt{1 - \frac{v^2}{c^2}}}{1 - \frac{v w_x}{c^2}}
\end{equation}
Tehát a relativisztikus sebességösszeadás végső képletei a következők:
\begin{equation}
    w'_x = \frac{w_x - v}{1 - \frac{v w_x}{c^2}}
\end{equation}
\begin{equation}
    w'_y = \frac{w_y \sqrt{1 - \frac{v^2}{c^2}}}{1 - \frac{v w_x}{c^2}}
\end{equation}
\begin{equation}
    w'_z = \frac{w_z \sqrt{1 - \frac{v^2}{c^2}}}{1 - \frac{v w_x}{c^2}}
\end{equation}
Ezek a képletek biztosítják, hogy a fénysebesség minden inerciarendszerben állandó maradjon, és hogy a sebességek ne haladják meg a fénysebességet. Ebből látszódik, hogy
Relativisztikus kinematika esetén hiába csak egy irányba mozog a két rendszer egymáshoz képest, a másik két irány komponensei is megváltoznak, ami fontos különbség a klasszikus kinematikához képest.

A gyorsulással hasonlóan lehet eljárni és a következő képleteket kapjuk:
\begin{equation}
    a'_x = \frac{a_x (1 - \frac{v w_x}{c^2})^3}{(1 - \frac{v^2}{c^2})(1 - \frac{w^2}{c^2})^{3/2}} 
\end{equation}
\begin{equation}
    a'_y = \frac{a_y \sqrt{1 - \frac{v^2}{c^2}} (1 - \frac{v w_x}{c^2})^2}{(1 - \frac{w^2}{c^2})^{3/2}}
\end{equation}
\begin{equation}
    a'_z = \frac{a_z \sqrt{1 - \frac{v^2}{c^2}} (1 - \frac{v w_x}{c^2})^2}{(1 - \frac{w^2}{c^2})^{3/2}}
\end{equation}

\subsubsection*{Forgások téridőben. Minkowski téridő.}
Láthattuk tehát, hogy a Relativizmus esetén 4D téridővel kell dolgoznunk, ahol az időt is koordinátaként kezeljük.  
A transzlációs mozgást egy 4D-s mátrixal írhatjuk le, amely a következőképpen néz ki:
\begin{equation}
    \begin{pmatrix}
        ct' \\
        x' \\
        y' \\
        z'
    \end{pmatrix} =
    \begin{pmatrix}
        \gamma & -\gamma \frac{v}{c} & 0 & 0 \\
        -\gamma \frac{v}{c} & \gamma & 0 & 0 \\
        0 & 0 & 1 & 0 \\
        0 & 0 & 0 & 1
    \end{pmatrix}
    \begin{pmatrix}
        ct \\
        x \\
        y \\
        z
    \end{pmatrix}
\end{equation}
Ahol a mátrix első sora az idő transzformációját, a második sora az x koordináta transzformációját, a harmadik és negyedik sora pedig az y és z koordináták transzformációját írja le.
Ezt a formalizmust Hermann Minkowski vezette be, aki rájött, hogy a speciális relativitáselméletet egy négy dimenziós téridőben lehet legjobban leírni. Ha megnézzük ezt a mátrixot,
akor feltűnhet, hogy rettentően hasonlít a forgatási mátrixokra, amelyeket a klasszikus mechanikában használtunk:
\begin{equation}
    \begin{pmatrix}
        x' \\
        y' \\
        z'
    \end{pmatrix} =
    \begin{pmatrix}
        \cos \alpha & -\sin \alpha & 0 \\
        \sin \alpha & \cos \alpha & 0 \\
        0 & 0 & 1
    \end{pmatrix}
    \begin{pmatrix}
        x \\
        y \\
        z
    \end{pmatrix}
\end{equation}
Ahol $\alpha$ a forgatási szög. A különbség annyi, hogy a mi mátrixunk 4D-s, míg a klasszikus forgatási mátrix 3D-s.
Módosítsuk egy kicsit a jelöléseinket az alábbi módon:
\begin{equation}
    x_0 = ct, \quad x_1 = x, \quad x_2 = y, \quad x_3 = z
\end{equation}
\begin{equation}
    x'_0 = ct', \quad x'_1 = x', \quad x'_2 = y', \quad x'_3 = z'
\end{equation}
\begin{equation}
    \beta = \frac{v}{c}, \quad \gamma = \frac{1}{\sqrt{1 - \beta^2}}
\end{equation}
\begin{equation}
    \Lambda =
    \begin{pmatrix}
        \gamma & -\gamma \beta & 0 & 0 \\
        -\gamma \beta & \gamma & 0 & 0 \\
        0 & 0 & 1 & 0 \\
        0 & 0 & 0 & 1
    \end{pmatrix}
\end{equation}

Ekkor az egész műveletet felírhatjuk a következőképpen:
\begin{equation}
    x' = \Lambda x
\end{equation}
A 4 komponenst négyesvektornak nevezzük, és ezt a teret ami a 3 tér és 1 idő dimenzióból áll, Minkowski térnek nevezzük.

A Lorentz transzformációkat tehát tekinthetjük úgy, mint a Minkowski térben végzett forgatásokat, ahol a forgatási szög a relatív sebességtől függ. Ez a megközelítés segít megérteni a relativisztikus jelenségeket, mint például az idődilatációt és a hosszuskontrakciót
Ennek az analógiának a még egyértelműbbé tételéhez bevezethetjük a következő jelölést:
\begin{equation}
    \cosh \phi = \gamma, \quad \sinh \phi = \gamma \beta
\end{equation}
ahol $\phi$ a rapiditás, amely a relativisztikus sebesség egy alternatív mértéke, $cosh$ és $sinh$ pedig a hiperbolikus koszinusz és szinusz függvények. Ekkor a Lorentz transzformáció mátrixa a következőképpen néz ki:
\begin{equation}
    \Lambda =
    \begin{pmatrix}
        \cosh \phi & -\sinh \phi & 0 & 0 \\
        -\sinh \phi & \cosh \phi & 0 & 0 \\
        0 & 0 & 1 & 0 \\
        0 & 0 & 0 & 1
    \end{pmatrix}
\end{equation}
Ez a mátrix hasonló a klasszikus forgatási mátrixhoz, de itt a trigonometrikus függvények helyett hiperbolikus függvényeket használunk. Ez a hasonlóság még inkább alátámasztja,
hogy a Lorentz transzformációk a Minkowski térben végzett forgatások reprezentálják.

\subsubsection*{Invariáns skaláris szorzat}
A minkowski formalizmus segítségével könnyedén bevezethetjük a forgásokat 3D térben is:
\begin{equation}
    \Lambda =
    \begin{pmatrix}
        1 & 0 & 0 & 0 \\
        0 & R_{11} & R_{12} & R_{13} \\
        0 & R_{21} & R_{22} & R_{23} \\
        0 & R_{31} & R_{32} & R_{33} \\
    \end{pmatrix}
    = 
    \begin{pmatrix}
        1 & 0 & 0 & 0 \\
        0 &  &  &  \\
        0 &  & \hat{R} &  \\
        0 &  &  & 
    \end{pmatrix}
\end{equation}
Innen látható, hogy a 3D forgatások a Lorentz transzformációk speciális esetei, ahol az idő komponens változatlan marad. Ez azt jelenti, hogy a térbeli forgatások nem befolyásolják az időt, Ezt gyakran tiszta térbeli forgatásnak nevezzük.
Ez a megállapítás nem igaz általánosan, és ez sebességváltozással jár, amit Lorentz boostnak nevezünk. 

Ez a matematikai hasonlóság viszont motiválhat minket arra, hogy bevezessünk egy új skaláris szorzatot a minkowski térben, amely hasonló a klasszikus euklideszi skaláris szorzathoz, de figyelembe veszi az idő dimenzió különleges szerepét.
Ezt a skaláris szorzatot minkowski skaláris szorzatnak nevezzük, és a következőképpen definiáljuk két négyesvektor, $a$ és $b$ között:
\begin{equation}
    x \cdot y = -x_0 y_0 + x_1 y_1 + x_2 y_2 + x_3 y_3 = -x_0 y_0 + \bold{x} \cdot \bold{y} 
\end{equation}
ahol $x_0 = ct_a$, $x_1 = x_a$, $x_2 = y_a$, $x_3 = z_a$ és hasonlóan $y$-re is. Ez a skaláris szorzat különbözik a klasszikus euklideszi skaláris szorzattól, mivel szerepel egy korrekciós tag is, amely az idő komponensek szorzatát tartalmazza negatív előjellel.
Ez a korrekciós tag biztosítja, hogy a minkowski skaláris szorzat invariáns maradjon a Lorentz transzformációk alatt, ami azt jelenti, hogy a skaláris szorzat értéke nem változik meg, ha a négyesvektorokat egy másik inerciarendszerbe transzformáljuk.
Ez az invariancia fontos szerepet játszik a relativisztikus fizika alapelveiben, mivel lehetővé teszi, hogy a fizikai törvények formája ugyanaz maradjon minden inerciarendszerben.

\subsubsection*{Fénykúp}
Az invariánsa skaláris szorzat egyik tulajdonsága, hogy a vektorok sazorzata lehet negatív, akkor is, ha a vektorok maguk nem azok. Ahhoz, hogy ennek az implikációit láthassuk, rajzoljunk fel egy téidő diagramot, ahol az  $x_1$ és  $x_2$ tengely a térbeli dimenziót, az  $x_4$ tengely pedig az idő dimenziót jelöli.
\begin{figure}[H]
    \centering
    \includegraphics[width=0.5\textwidth]{imgs/1-tetel/lorentz2.png}
    \caption{Fénykúp}
    \label{fig:lorentz2}
\end{figure}

ekkor definiálhatjuk a téridő távolságot két esemény között a következőképpen:
\begin{equation}
    s^2 = -(c \Delta t)^2 + (\Delta x)^2 + (\Delta y)^2 + (\Delta z)^2
\end{equation}
ahol $\Delta t$, $\Delta x$, $\Delta y$ és $\Delta z$ a két esemény közötti idő és térbeli különbségek. Ez a téridő távolság invariáns a Lorentz transzformációk alatt, ami azt jelenti, hogy minden inerciarendszerben ugyanazt az értéket kapjuk. Ez a téridő távolság három különböző esetet különböztet meg:
\begin{itemize}
    \item Ha $s^2 > 0$, akkor a két esemény között időszerű kapcsolat van, és az események között létezhet ok-okozati kapcsolat. Ezt időszerű távolságnak nevezzük.
    \item Ha $s^2 < 0$, akkor a két esemény között térbeli kapcsolat van, és az események között nem létezhet ok-okozati kapcsolat. Ezt térbeli távolságnak nevezzük.
    \item Ha $s^2 = 0$, akkor a két esemény között fénysebességű kapcsolat van, és az események között létezhet ok-okozati kapcsolat, de csak fénysebességgel. Ezt null távolságnak nevezzük.
\end{itemize}
Ezek a különbségek fontosak a relativisztikus fizika megértéséhez, mivel meghatározzák, hogy mely események között létezhet ok-okozati kapcsolat, és hogyan viselkednek a fizikai törvények a különböző inerciarendszerekben.

\subsection{Relativisztikus dinamika. Négyesimpulzus.}
A relativisztikus dinamika a klasszikus mechanika kiterjesztése, amely figyelembe veszi a fénysebesség korlátait és a relativisztikus hatásokat. A klasszikus mechanikában az impulzust a tömeg és a sebesség szorzataként definiáljuk:
\begin{equation}
    \bold{p} = m \bold{v}
\end{equation}
ahol $\bold{p}$ az impulzus, $m$ a tömeg és $\bold{v}$ a sebesség. A relativisztikus mechanikában azonban a tömeg nem állandó, hanem a sebességtől függ, és a következőképpen definiáljuk:
\begin{equation}
    m = \frac{m_0}{\sqrt{1 - \frac{v^2}{c^2}}}
\end{equation}
ahol $m_0$ a nyugalmi tömeg, $v$ a sebesség és $c$ a fénysebesség. Ebből következik, hogy az impulzus a következőképpen alakul:
\begin{equation}
    \bold{p} = \frac{m_0 \bold{v}}{\sqrt{1 - \frac{v^2}{c^2}}}
\end{equation}
Ez az impulzus definíciója azonban nem elégséges a relativisztikus dinamika teljes leírásához, mivel az idő is fontos szerepet játszik. Ezért bevezetjük a négyesimpulzus fogalmát, amely a következőképpen definiáljuk:
\begin{equation}
    P^\mu = \left(\frac{E}{c}, \bold{p}\right) = \left(\frac{E}{c}, p_x, p_y, p_z\right)
\end{equation}
ahol $E$ az energia, $\bold{p}$ az impulzus és $c$ a fénysebesség. Az energia és az impulzus közötti kapcsolatot a következőképpen írhatjuk fel:
\begin{equation}
    E^2 = (pc)^2 + (m_0 c^2)^2
\end{equation}
Ez az egyenlet azt mutatja, hogy az energia nem csak az impulzustól, hanem a nyugalmi tömegtől is függ. Ez az egyenlet a relativisztikus energiamegmaradás alapja, és fontos szerepet játszik a részecskefizikában és a kozmológiában.
A négyesimpulzus komponensei a következőképpen transzformálódnak a Lorentz transzformációk alatt:
\begin{equation}
    P'^\mu = \Lambda^\mu_\nu P^\nu
\end{equation}
ahol $\Lambda^\mu_\nu$ a Lorentz transzformáció mátrixa. Ez az invariancia biztosítja, hogy a fizikai törvények formája ugyanaz maradjon minden inerciarendszerben, és lehetővé teszi a relativisztikus dinamika alkalmazását különböző helyzetekben.

\section{A klasszikus mechanika elvei}

Virtuális munka elve, Hamilton-elv. Legkisebb hatás elve. Lagrange-féle elsőfajú és másodfajú mozgásegyenletek. Hamilton-függvény, kanonikus egyenletek. Kanonikus transzformációk. Szimmetriák és megmaradási tételek.

\subsection{Bevezető}
A klasszikus mechanika alapját a Newtoni formalizumus képzi, ami a Newton törvények és megmaradási törvények segítségével írja le a testek mozgását. Ennek alapját képzi a Newton törvények, amelyek a következők:
\begin{itemize}
    \item Inerciarendszer: Minden test, amelyre nem hat külső erő, egyenes vonalú egyenletes mozgást végez.
    \item Erő és gyorsulás: A testre ható erő egyenlő a test tömegének és gyorsulásának szorzatával: $\bold{F} = m \bold{a}$.
    \item Akció és reakció: Minden erőhatásra van egy vele ellentétes irányú és egyenlő nagyságú erőhatás.
\end{itemize}
És ezekből levezethetők a megmaradási törvények:
\begin{itemize}
    \item Energia megmaradás: Egy zárt rendszerben az összenergia állandó $E = K + U = const$, ahol $K$ a kinetikus energia és $U$ a potenciális energia.
    \item Impulzus megmaradás: Egy zárt rendszerben az összimpulzus állandó $\bold{P} = \sum m_i \bold{v}_i = const$.
    \item Forgásimpulzus megmaradás: Egy zárt rendszerben az összforgásimpulzus állandó $\bold{L} = \sum \bold{r}_i \times m_i \bold{v}_i = const$.
\end{itemize}
Ezeket a törvényeket felhasználva minden mechanikai problémára fel lehet írni a mozgásegyenleteket, amelyet egy differenciálegyenlet segítségével megoldhatunk.

Ez viszont gyakran olyan egyenleteket erredményez, amelyeket nagyon nehéz vagy analitikusan akár lehetetlen megoldani. Ezért az évszázadok során megpróbálták ezeket a fundamentális törvényeket más szempontból megközelíteni,
és új fomalizmusokat létrehozni. Így járt el például Lagrange és Hamilton is a 18-19. században, akik a mechanikát egy új megközelítésből írták le, amely a mozgásegyenletek helyett a rendszer energiájára és impulzusára fókuszált.
Ők abból az irányból közelítették meg a problémát, hogy a fizikának látszólag van egy olyan hajlama, hogy egy rendszer egy adott állapotból egy másikba úgy menjen át, hogy mindig a "legrövidebb" utat válassza, a két pont közötti
végtelen lehetőség közül. Ezt a "legrövidebb" utat pedig úgy definiálták, hogy a rendszer egy adott állapotból egy másikba úgy menjen át, hogy a hatás (action) nevű mennyiség minimális legyen. Ezt a megközelítést Hamilton vezette be, és ezért nevezzük Hamilton-elvnek.
Ebből a megközelítésből kiindulva Lagrange is kidolgozta a saját formalizmusát, amely a Lagrange-féle mozgásegyenleteket eredményezte. Ezek az egyenletek a kinetikus és potenciális energia különbségéből származnak, és lehetővé teszik a
mozgásegyenletek felírását anélkül, hogy közvetlenül az erőket kellene figyelembe venni. Ez különösen hasznos olyan rendszerek esetén, ahol az erők bonyolultak vagy nehezen meghatározhatók.
A Hamilton és Lagrange formalizmusok tehát alternatív megközelítést kínálnak a klasszikus mechanika problémáinak megoldására, és gyakran egyszerűbbé teszik a mozgásegyenletek felírását és megoldását.

\subsection{Virtuális munka elve, Hamilton-elv. Legkisebb hatás elve.}
\subsubsection{Virtuális munka elve}
A virtuális munka elve azon a megfigyelésen alapszik, hogy ha egy testre erő hat, és emiatt $A$ pontból $B$ pontba mozog, akkor a két pont között úgy választja ki a pályát, hogy a pálya és a körülötte lévő "közeli" pályák között a a munka nulla legyen (elsőrendűen).
Ez tehát a munka függvényének egy szélsőértékét kell megtalálni  (lehet inflexiós pont is) ami azt jelenti, hogy kis változásokra munka nem változik elsőrendűen. A munkát úgy definiáltuk, hogy:
\begin{equation}
    W = \int_{\bold{r}(t_0) = A}^{\bold{r}(t_1) = B} \bold{F} \cdot d\bold{r} = \int_{t_0}^{t_1} \bold{F} \cdot \frac{d\bold{r}}{dt} dt = \int_{t_0}^{t_1} \bold{F} \cdot \bold{v} dt
\end{equation}
ahol $\bold{F}$ a testre ható erő, és $d\bold{r}$ a test kis elmozdulása. Most tegyük fel, hogy a test egy kicsit eltér a valódi pályától, és egy közeli pályán mozog, amelyet $\bold{r}(t) + \delta \bold{r}(t)$-vel jelölünk.
Ekkor a munka változása a két pálya között a következőképpen írható fel:
\begin{equation}
    \delta W = \int_{t_0}^{t_1} \bold{F} \cdot \frac{d}{dt} (\bold{r} + \delta \bold{r}) dt - \int_{t_0}^{t_1} \bold{F} \cdot \frac{d\bold{r}}{dt} dt
\end{equation}
\begin{equation}
    \delta W = \int_{t_0}^{t_1} \bold{F} \cdot \frac{d (\delta \bold{r})}{dt} dt
\end{equation}
Ezt nevezik virtuális munkának, és ez a munka egy olyan kis virtuális elmozdulásra vonatkozik, amely nem valósul meg a tényleges rendszeben de elképzelhető.

Ha a rendszer statikus egyensúlyi állapotban van, akkor a virtuális munka elve szerint a virtuális munka nulla kell legyen minden lehetséges virtuális elmozdulásra:
\begin{equation}
    \delta W = \sum_i \bold{F}_i \cdot \delta \bold{r}_i = 0
\end{equation}
ahol $\delta W$ a virtuális munka, $\bold{F}_i$ a rendszerre ható erők, és $\delta \bold{r}_i$ a testek virtuális elmozdulásai. 

Szabad mozgás esetén a feltétel csupán annyi, hogy minden i-re igaz legyen, hogy $\bold{F}_i = 0$. Ha a testeknek viszont vannak kényszerei (pl. egy kötött pályán mozognak), akkor a kényszererők is hatnak a testekre,
és ezeket is figyelembe kell venni a virtuális munka elvében. Tegyük fel, hogy a kényszereket megadhatjuk a következő alakban:
\begin{equation}
    \phi_j(\bold{r}_1, \bold{r}_2, \ldots, \bold{r}_N) = 0, \quad j = 1, 2, \ldots, k
\end{equation}
Ahol $k$ a kényszerek száma. Ezeknek a kényszereknek kis elmozdulás esetén is teljesülniük kell, tehát a kényszerfeltételek teljesülnek a virtuális elmozdulásokra is:
\begin{equation}
    \phi_j (\bold{r}_1 + \delta \bold{r}_1, \bold{r}_2 + \delta \bold{r}_2, \ldots, \bold{r}_N + \delta \bold{r}_N) = 0
\end{equation}
Vezessünk be egy egyszerűbb jelölést a kényszerfeltételekhez:
\begin{equation}
    \{\bold{r}_1, \bold{r}_2, \ldots, \bold{r}_N\} = \bold{r}_i
\end{equation}
\begin{equation}
    \{\bold{r}_1 + \delta \bold{r}_1, \bold{r}_2 + \delta \bold{r}_2, \ldots, \bold{r}_N + \delta \bold{r}_N\} = \bold{r}_i + \delta \bold{r}_i
\end{equation}
Ekkor a kényszerfeltételek a következőképpen írhatók fel:
\begin{equation}
    \phi_j (\bold{r}_i) = 0
\end{equation}
\begin{equation}
    \phi_j (\bold{r}_i + \delta \bold{r}_i) = 0
\end{equation}
Most alkalmazzuk a Taylor-sorfejtést a kényszerfeltételekre, és hagyjuk el a másod- és magasabb rendű tagokat, mivel a virtuális elmozdulások kicsik:
\begin{equation}
    \phi_j (\bold{r}_i + \delta \bold{r}_i) = \phi_j (\bold{r}_i) + \sum_i \frac{\partial \phi_j}{\partial \bold{r}_i} \cdot \delta \bold{r}_i = 0
\end{equation}
Mivel $\phi_j (\bold{r}_i) = 0$, ezért a fenti egyenlet egyszerűsödik:
\begin{equation}
    \sum_i \frac{\partial \phi_j}{\partial \bold{r}_i} \cdot \delta \bold{r}_i = 0
\end{equation}
Vagy a Nabla jelöléssel:
\begin{equation}
    \sum_i \nabla_i \phi_j(\bold{r}_i) \cdot \delta \bold{r}_i = 0
\end{equation}
Ezt az egyenletet a Lagrange multiplikátor módszerével oldhatjuk meg, amelynek használatával megtaláhatjuk egy függvény szélsőértékét kényszerfeltételek mellett. A módszer lényege, hogy bevezetünk egy új függvényt, amely a keresett függvény és a kényszerfeltételek lineáris kombinációja:
\begin{equation}
    \mathcal{L} = \delta W + \sum_j \lambda_j \sum_i \nabla_i \phi_j(\bold{r}_i) \cdot \delta \bold{r}_i
\end{equation}
ahol $\lambda_j$ a Lagrange multiplikátorok. Most már alkalmazhatjuk a virtuális munka elvét a $\mathcal{L}$ függvényre:
\begin{equation}
    \mathcal{L} = \sum_i \bold{F}_i \cdot \delta \bold{r}_i + \sum_j \lambda_j \sum_i \nabla_i \phi_j(\bold{r}_i) \cdot \delta \bold{r}_i = 0
\end{equation}
Mivel a virtuális elmozdulások tetszőlegesek, ezért a fenti egyenlet csak akkor teljesülhet, ha az egyes tagok külön-külön is nulla:
\begin{equation}
    \bold{F}_i + \sum_j \lambda_j \nabla_i \phi_j(\bold{r}_i) = 0
\end{equation}
Ez az egyenlet a rendszer mozgásegyenleteit írja le kényszerfeltételek mellett, ahol a $\lambda_j$ multiplikátorok a kényszererőket reprezentálják.

\subsubsection{Hamilton-elv, Legkisebb hatás elve}
Amikor egy fizikai rendszer változását vizsgáljuk, gyakran előfordul, álapotok között, gyakran több lehetséges útvonal is létezik. Például, ha egy Konzervatív erőtérben (pl. gravitációs térben) leejtünk egy tárgyat,
Akkor munkát a tér csak a vertikális tengely mentén végez, horizontális elmozdulás során a munkavégzés nulla. Tehát elméletileg a tárgy végtelen útvonal közül bármelyiken eljuthat a kiinduló pontból a végpontba, ha csak a vertikális komponensét nézzük.
Felmerülhet tehát a kérdés, hogy ha kísérletet teszünk a rendszer viselkedésének megértésére, és a kezdeti paramétereket mindig azonosnak választjuk, akkor miért mindig ugyanazt az útvonalat követi a rendszer? Miért nem választ véletlenszerűen egy másik lehetséges útvonalat?

A kérdés megválaszolását először a fény útvanálanak vizsgálata indukálta, ahol Fermat felismerte, hogy a fény mindig azt az útvonalat választja, amelyen a fénynek a legkisebb idő alatt kell megtennie a távolságot. Ezt az elvet Fermat elvének nevezzük.
Itt felmerülhet a kérdés, hogy mit is jelent az, hogy "legrövidebb útvonal"? 

Arra hogy egy tetszőleges koordináta rendszerben meg tudjuk határozni a két pont közötti utat, a variációs kalkulust használjuk. A variációs kalkulus egy olyan matematikai eszköz, amely lehetővé teszi a függvények szélsőértékeinek meghatározását, amikor a függvények nem csak egy változótól, hanem egy függvénytől is függenek.
A variációs kalkulus alapvető fogalma a funkcionál, amely egy olyan függvény, amely egy függvényt rendel egy valós számhoz. Például, ha van egy $y(x)$ függvényünk, akkor egy funkcionál lehet a következő:
\begin{equation}
    S[y] = \int_{x_1}^{x_2} F(x, y, y') \, dx
\end{equation}
ahol $F$ egy adott függvény, $y'$ pedig a $y$ függvény deriváltja. A funkcionál szélsőértékének meghatározásához a variációs kalkulus a következő lépéseket követi:
\begin{itemize}
    \item Válasszunk egy kis változást a függvényben: $y(x) \to y(x) + \delta y(x)$, ahol $\delta y(x)$ egy kis perturbáció.
    \item Számítsuk ki a funkcionál változását a perturbáció hatására: $\delta S = S[y + \delta y] - S[y]$.
    \item Állítsuk be a változást úgy, hogy a funkcionál változása nulla legyen: $\delta S = 0$.
\end{itemize}
\begin{figure}
    \centering
    \includegraphics[width=0.5\textwidth]{imgs/2-tetel/CalculusofVariations.png}
    \caption{Variációs kalkulus}
    \label{fig:variational_calculus}
\end{figure}
Tehát, ha van két pontom $A = (x_1,y_1)$ és $B = (x_2,y_2)$, akkor a két pont között húzott görbe mentén a következő funkcionált definiálhatjuk:
\begin{equation}
    dS[y] = \sqrt{dx^2 + dy^2}
\end{equation}
\begin{equation}
    dS[y] = \sqrt{1 + \left(\frac{dy}{dx}\right)^2} \, dx
\end{equation}
\begin{equation}
    S[y] = \int_{x_1}^{x_2} \sqrt{1 + \left(\frac{dy}{dx}\right)^2} \, dx
\end{equation}
Ez a funkcionál a két pont közötti görbe hosszát méri. Most vezessünk be egy kis perturbációt a görbébe:
\begin{equation}
    y(x) \to y(x) + \delta y(x)
\end{equation}
Akkor a funkcionál változása a következőképpen alakul:
\begin{equation}
    \delta S = S[y + \delta y] - S[y]
\end{equation}
Ez a görbe "variálása", és a görbe hosszának változását méri a perturbáció hatására. Ha a görbe variációja nulla, akkor az azt jelenti, hogy a görbe hosszának változása a kis perturbáció hatására nulla,
vagyis a görbe egy szélsőértéket képvisel. Ez a szélsőérték lehet minimum, maximum vagy egy inflexiós pont is. Tehát, ha a görbe minimumát akarjuk megtalálni, akkor ennek a variált Funkciónálnak a változását nullára kell állítanunk:
\begin{equation}
    \delta S = \delta \int_{x_1}^{x_2} \sqrt{1 + \left(\frac{dy}{dx}\right)^2} \, dx = 0
\end{equation}
Ez a variációs számítás alapja.

\subsubsection*{Példa: Fermat-elv}
Nézzük meg, hogy akkor a Fermat-elv ismeretében hogyan tudjuk meghatározni a fény útvonalát két pont között. Tegyük fel, hogy a fény egy közegben terjed, ahol a fénysebesség $v$ és a közeg törésmutatója $n = \frac{c}{v}$, ahol $c$ a vákuumbeli fénysebesség.
Tudjuk, hogy a fény az idő "szélsőértékét keresi", tehát a két pont között a következő időt kell minimalizálnunk:
\begin{equation}
    \delta t = 0
\end{equation}
Ahol $t$ a fény által megtett idő a két pont között. A fény által megtett idő a következőképpen számítható ki:
\begin{equation}
    t = \int_{A}^{B} \frac{ds}{v} = \int_{A}^{B} \frac{n \, ds}{c}
\end{equation}
Ahol $ds$ a fény által megtett útszakasz. A fény által megtett útszakasz a következőképpen írható fel:
\begin{equation}
    ds = \sqrt{dx^2 + dy^2} = \sqrt{1 + \left(\frac{dy}{dx}\right)^2} \, dx
\end{equation}
Tehát a fény által megtett idő a következőképpen alakul:
\begin{equation}
    t = \int_{x_1}^{x_2} \frac{n}{c} \sqrt{1 + \left(\frac{dy}{dx}\right)^2} \, dx
\end{equation}
Most alkalmazzuk a variációs kalkulus elveit, és vezessünk be egy kis perturbációt a görbébe:
\begin{equation}
    y(x) \to y(x) + \delta y(x)
\end{equation}
Akkor a fény által megtett idő változása a következőképpen alakul:
\begin{equation}
    \delta t = \delta \int_{x_1}^{x_2} \frac{n}{c} \sqrt{1 + \left(\frac{dy}{dx}\right)^2} \, dx = 0
\end{equation}
Ez a variációs számítás alapja a Fermat-elv alkalmazásának a fény útvonalának meghatározására. A variációs számítás elvégzése után megkapjuk a fény útvonalát, amely a két pont között a legkisebb idő alatt teszi meg az utat.

\subsubsection*{Maupertuis-elv}
Pierre Louis Maupertuis, francia matematikus és filozófus, a 18. században dolgozott ezen a poblémán és póbálta meg általánosítani a Fermat-elvet minden fizikai rendszerre. Maupertuis felismerte, hogy a fizikai rendszerek hajlamosak arra,
hogy egy adott állapotból egy másik állapotba úgy menjenek át, hogy egy funkcionál minimális legyen. Ezt a funkcionált Maupertuis hatásnak nevezte el, és a következőképpen definiálta:
\begin{equation}
    \bold{S}[\bold{q}(t)] = \int \bold{p} \, d\bold{q}
\end{equation}
ahol $\bold{p}$ a rendszer általánosított impulzusa (nem feltétlen egyezik meg a fizikai impulzussal), és $\bold{q}$ a rendszer általánosított koordinátái. Maupertuis elve azt mondja ki, hogy egy fizikai rendszer akkor megy át egy adott állapotból egy másik állapotba,
ha a Maupertuis hatás minimális:
\begin{equation}
    \delta \bold{S} = 0
\end{equation}
Ez az elv hasonló a Fermat-elvhez, de általánosabb, mivel nem csak a fényre, hanem minden fizikai rendszerre alkalmazható. Maupertuis elve tehát egy univerzális elv, amely a fizikai rendszerek viselkedését írja le.

\subsubsection*{Hamilton-elv}
A Hamilton-elv hasonló a Maupertuis-elvhez, de egy kicsit általánosabb formában. Hamilton elve azt mondja ki, hogy egy fizikai rendszer akkor megy át egy adott állapotból egy másik állapotba,
ha a Lagrange-függvény integrálja, amelyet akciónak nevezünk, minimális:
\begin{equation}
    S = \int_{t_1}^{t_2} \mathcal{L}(q, \dot{q}, t) \, dt
\end{equation}
Itt az $S$ szintén az akciót jelöli. Ahogy hogy ezt minimalizálni tudjuk, szintén a variációs kalkulus eszközeit használjuk:
\begin{equation}
    \delta S = \delta \int_{t_1}^{t_2} \mathcal{L}(q, \dot{q}, t) \, dt = 0
\end{equation}
A Hamilton-elv tehát egy univerzális elv, amely a fizikai rendszerek viselkedését írja le, és amely a Lagrange-függvényre épül. Ez az elv lehetővé teszi a mozgásegyenletek felírását anélkül, hogy közvetlenül az erőket kellene figyelembe venni.

Ahhoz, hogy meg tudjuk oldani a Hamilton-elv által felírt variációs problémát, bontsuk fel az eseményeket egy kis időintervallumra:
\begin{figure}[H]
    \centering
    \includegraphics[width=0.5\textwidth]{imgs/2-tetel/Hamilton1.png}
    \caption{Diszkretizáció}
    \label{fig:discretization}
\end{figure}
\begin{equation}
    \int dt = \epsilon \sum_i \mathcal{L}(q_i, \frac{q_{i+1} - q_i}{\epsilon})
\end{equation}
Ekkor ha csak egy időintervallumot nézünk, akkor a következő kifejezést kapjuk:
\begin{figure}[H]
    \centering
    \includegraphics[width=0.5\textwidth]{imgs/2-tetel/Hamilton2.png}
    \caption{Diszkretizáció 2}
    \label{fig:discretization2}
\end{figure}
\begin{equation}
    \epsilon\mathcal{L}(q_i, \frac{q_{i+1} - q_i}{\epsilon}) + \epsilon\mathcal{L}(q_{i-1}, \frac{q_{i} - q_{i-1}}{\epsilon})
\end{equation}

Ennek a kifejezésnek vagyunk kiváncsiak, hogyan változik egy kis perturbáció hatására. Ezért deriváljuk a kifejezést $q_i$ szerint:
\begin{equation}
    \frac{\partial}{\partial q_i} \left( \epsilon\mathcal{L}(q_i, \frac{q_{i+1} - q_i}{\epsilon}) + \epsilon\mathcal{L}(q_{i-1}, \frac{q_{i} - q_{i-1}}{\epsilon}) \right) = 0
\end{equation}
Vezessük be a kifejezést $v_i = \frac{q_{i+1} - q_i}{\epsilon}$ sebességet, és ekkor a következő eredményt kapjuk:
\begin{equation}
    \epsilon \left( \frac{\partial \mathcal{L}}{\partial q_i} - \frac{1}{\epsilon} \frac{\partial \mathcal{L}}{\partial v_i} + \frac{1}{\epsilon} \frac{\partial \mathcal{L}}{\partial v_{i-1}} \right) = 0
\end{equation}
Ahol
\begin{equation}
    - \frac{1}{\epsilon} \frac{\partial \mathcal{L}}{\partial v_i} + \frac{1}{\epsilon} \frac{\partial \mathcal{L}}{\partial v_{i-1}} = \frac{d}{dt} \frac{\partial \mathcal{L}}{\partial v_i}
\end{equation}
Amennyiben $\epsilon$-t infinitezimálisan kicsivé tesszük. Tehát a Hamilton-elvből a következő egyenletet kapjuk:
\begin{equation}
    \frac{\partial \mathcal{L}}{\partial q_i} - \frac{d}{dt} \frac{\partial \mathcal{L}}{\partial v_i} = 0
\end{equation}
vagy a klasszikus jelöléssel:
\begin{equation}
    \frac{\partial \mathcal{L}}{\partial q} - \frac{d}{dt} \frac{\partial \mathcal{L}}{\partial \dot{q}} = 0
\end{equation}
Ez az Euler-Lagrange-féle mozgásegyenlet, amely a Hamilton-elvből származtatható. Ez az egyenlet lehetővé teszi a mozgásegyenletek felírását anélkül, hogy közvetlenül az erőket kellene figyelembe venni. Az egyenletben megadott koordináták és sebességek általánosítottak,
tehát nem feltétlenül egyeznek meg a fizikai koordinátákkal és sebességekkel.

Nézzük még meg, hogy a Hamilton-elvből hogyan vezethető le a Newton-féle mozgásegyenlet. Tegyük fel, hogy egy testre hat egy konzervatív erőtér, amelyet egy potenciális energiafüggvény ír le:
\begin{equation}
    U = U(q)
\end{equation}
Akkor a Lagrange-függvény a következőképpen írható fel:
\begin{equation}
    \mathcal{L} = K - U = \frac{1}{2}m \dot{q}^2 - U(q)
\end{equation}
Ahol $K$ a kinetikus energia, $U$ a potenciális energia, $q$ pedig a koordináta. Most alkalmazzuk az Euler-Lagrange-féle mozgásegyenletet:
\begin{equation}
    \frac{\partial \mathcal{L}}{\partial q} - \frac{d}{dt} \frac{\partial \mathcal{L}}{\partial \dot{q}} = 0
\end{equation}
Számítsuk ki az egyes tagokat:
\begin{equation}
    \frac{\partial \mathcal{L}}{\partial q} = - \frac{\partial U}{\partial q}
\end{equation}
\begin{equation}
    \frac{\partial \mathcal{L}}{\partial \dot{q}} = m \dot{q}
\end{equation}
Ahol a kinetikus energia deriváltja a sebesség szerint a kanonikus impulzus, amely ebben az esetben $p = m \dot{q}$. Most vegyük ennek a deriváltját az idő szerint:
\begin{equation}
    \frac{d}{dt} \frac{\partial \mathcal{L}}{\partial \dot{q}} = m \ddot{q}
\end{equation}
Most helyettesítsük be ezeket az egyenletbe:
\begin{equation}
    - \frac{\partial U}{\partial q} - m \ddot{q} = 0
\end{equation}
Ezt átrendezve megkapjuk a Newton-féle mozgásegyenletet:
\begin{equation}
    m \ddot{q} = - \frac{\partial U}{\partial q} = \bold{F}
\end{equation}
Ez az egyenlet azt mondja ki, hogy a test gyorsulása arányos a rá ható erővel, amelyet a potenciális energia gradiensével adhatunk meg. Tehát a Hamilton-elvből kiindulva sikerült levezetnünk a klasszikus mechanika alapegyenletét, a Newton-féle mozgásegyenletet.

\subsubsection*{Példa: Forgó koordináta rendszer}
Vegyünk egy koodináta rendszert, amely egy tengely körül forog egy szögsebességgel $\omega$. Ebben a rendszerbben a koordináták:
\begin{equation}
    x' = x \cos(\omega t) + y \sin(\omega t)
\end{equation}
\begin{equation}
    y' = -x \sin(\omega t) + y \cos(\omega t)
\end{equation}
Ekkor a Lagrange-függvény a mozgó koordinátákra a következőképpen alakul:
\begin{equation}
    \mathcal{L} = \frac{1}{2}m(\dot{x'}^2 + \dot{y'}^2) - U(x', y')
\end{equation}
Számoljuk ki a derriváltjait a gyorsuló koordinátákra:
\begin{equation}
    \dot{x'} = \dot{x} \cos(\omega t) + \dot{y} \sin(\omega t) - \omega x \sin(\omega t) + \omega y \cos(\omega t)
\end{equation}
\begin{equation}
    \dot{y'} = -\dot{x} \sin(\omega t) + \dot{y} \cos(\omega t) - \omega x \cos(\omega t) - \omega y \sin(\omega t)
\end{equation}
Most helyettesítsük be ezeket:
\begin{equation}
    \begin{aligned}
        \dot{x'}^2 + \dot{y'}^2 = & (\dot{x} \cos(\omega t) + \dot{y} \sin(\omega t) - \omega x \sin(\omega t) + \omega y \cos(\omega t))^2 + \\
        & (-\dot{x} \sin(\omega t) + \dot{y} \cos(\omega t) - \omega x \cos(\omega t) - \omega y \sin(\omega t))^2
    \end{aligned}
\end{equation}
Ezt kibontva és egyszerűsítve a következő eredményt kapjuk:
\begin{equation}
    \dot{x'}^2 + \dot{y'}^2 = \dot{x}^2 + \dot{y}^2 + \omega^2 (x^2 + y^2) + 2\omega (x \dot{y} - y \dot{x})
\end{equation}
Most helyettesítsük be ezt a Lagrange-függvénybe:
\begin{equation}
    \mathcal{L} = \frac{m}{2}(\dot{x}^2 + \dot{y}^2) + \frac{m\omega^2}{2}(x^2 + y^2) + \frac{m\omega}{2}(x \dot{y} - y \dot{x}) - U(x, y)
\end{equation}
Ahol $\frac{m\omega^2}{2}(x^2 + y^2)$ a centrifugális erő, és $\frac{m\omega}{2}(x \dot{y} - y \dot{x})$ a Coriolis erő. Most alkalmazzuk az Euler-Lagrange-féle mozgásegyenletet:
\begin{equation}
    \frac{\partial \mathcal{L}}{\partial x} - \frac{d}{dt} \frac{\partial \mathcal{L}}{\partial \dot{x}} = 0
\end{equation}
\begin{equation}
    \frac{\partial \mathcal{L}}{\partial y} - \frac{d}{dt} \frac{\partial \mathcal{L}}{\partial \dot{y}} = 0
\end{equation}
Számoljuk ki az egyes tagokat:
\begin{equation}
    \frac{\partial \mathcal{L}}{\partial x} = m\omega^2 x + \frac{m\omega}{2} \dot{y} - \frac{\partial U}{\partial x}
\end{equation}
\begin{equation}
    \frac{\partial \mathcal{L}}{\partial y} = m\omega^2 y - \frac{m\omega}{2} \dot{x} - \frac{\partial U}{\partial y}
\end{equation}
\begin{equation}
    \frac{\partial \mathcal{L}}{\partial \dot{x}} = m \dot{x} - \frac{m\omega}{2} y
\end{equation}
\begin{equation}
    \frac{\partial \mathcal{L}}{\partial \dot{y}} = m \dot{y} + \frac{m\omega}{2} x
\end{equation}
\begin{equation}
    \frac{d}{dt} \frac{\partial \mathcal{L}}{\partial \dot{x}} = m \ddot{x} - \frac{m\omega}{2} \dot{y}
\end{equation}
\begin{equation}
    \frac{d}{dt} \frac{\partial \mathcal{L}}{\partial \dot{y}} = m \ddot{y} + \frac{m\omega}{2} \dot{x}
\end{equation}
Most helyettesítsük be ezeket az egyenletbe:
\begin{equation}
    m\omega^2 x + \frac{m\omega}{2} \dot{y} - \frac{\partial U}{\partial x} - (m \ddot{x} - \frac{m\omega}{2} \dot{y}) = 0
\end{equation}
\begin{equation}
    m\omega^2 y - \frac{m\omega}{2} \dot{x} - \frac{\partial U}{\partial y} - (m \ddot{y} + \frac{m\omega}{2} \dot{x}) = 0
\end{equation}
Ezt átrendezve megkapjuk a mozgásegyenleteket:
\begin{equation}
    m \ddot{x} = - \frac{\partial U}{\partial x} + m\omega^2 x + m\omega \dot{y}
\end{equation}
\begin{equation}
    m \ddot{y} = - \frac{\partial U}{\partial y} + m\omega^2 y - m\omega \dot{x}
\end{equation}
Ezek az egyenletek azt mondják ki, hogy a test gyorsulása arányos a rá ható erővel, amelyet a potenciális energia gradiensével adhatunk meg, valamint a centrifugális és Coriolis erőkkel.
Tehát a Hamilton-elvből kiindulva sikerült levezetnünk a mozgásegyenleteket egy forgó koordináta rendszerben.

\subsubsection{Lagrange-féle elsőfajú és másodfajú egyenletek}
A 18. század második felében Jean le Rond d'Alembert francia matematikus és fizikus is vizsgálta ennek az új megközelítésnek a lehetőségeit és felismerte, hogy a virtuális munka elvét ki lehet terjeszteni dinamikai rendszerekre is.
Ő azzal az ötlettel állt elő, hogy egy dinamikai rendszerben akkor beszélhetünk egyensúlyi állapotról, ha a rendszerre ható erők és a tehetetlenségi erők egyensúlyban vannak. Ezért bevezette a tehetetlenségi erők fogalmát, amelyeket a rendszer gyorsulása okoz:
\begin{equation}
    \bold{F}_{\text{tehetetlenség}} = -m \bold{a}
\end{equation}
Ahol $m$ a test tömege, és $\bold{a}$ a test gyorsulása. Ekkor a virtuális munka elve a következőképpen írható fel dinamikai rendszerekre:
\begin{equation}
    \sum_i (\bold{F}_i + \bold{F}_{\text{tehetetlenség}}) \cdot \delta \bold{r}_i = 0
\end{equation}
\begin{equation}
    \sum_i (\bold{F}_i - m_i \bold{a}_i) \cdot \delta \bold{r}_i = 0
\end{equation}
\begin{equation}
    \sum_i (\bold{F}_i - \bold{\dot{p}}_i) \cdot \delta \bold{r}_i = 0
\end{equation}
Ez onnan ered, hogy a tehetetlenségi erőket D'Alembert azért vezette be, hogy a dinamikai rendszert "egyensúlyi rendszerként" kezelhesse, ahol a virtuális munka elve alkalmazható. Ezt az elvet D'Alembert elvének nevezzük.
Ebbe a formalizmusba beilleszthetjük a kényszerfeltételeket is a Euler-Lagrange egyenlet alapján, hasonlóan ahogy azt a virtuális munka elvében tettük:
\begin{equation}
    \sum_i (\bold{F}_i - \bold{\dot{p}}_i) \cdot \delta \bold{r}_i + \sum_j \lambda_j \sum_i \nabla_i \phi_j(\bold{r}_i) \cdot \delta \bold{r}_i = 0  
\end{equation}
\begin{equation}
    \bold{F}_i - \bold{\dot{p}}_i + \sum_j \lambda_j \nabla_i \phi_j(\bold{r}_i) = 0
\end{equation}
\begin{equation}
    \bold{\dot{p}}_i = \bold{F}_i + \sum_j \lambda_j \nabla_i \phi_j(\bold{r}_i)
\end{equation}
\begin{equation}
    m_i \bold{\ddot{r}}_i = \bold{F}_i + \sum_j \lambda_j \nabla_i \phi_j(\bold{r}_i)
\end{equation}
Ez az egyenlet a rendszer mozgásegyenleteit írja le kényszerfeltételek mellett, ahol a $\lambda_j$ multiplikátorok a kényszererőket reprezentálják. Ezt az egyenletet Lagrange-féle elsőfajú egyenletnek nevezzük.

\subsubsection*{Lagrange-féle másodfajú egyenletek}
A kényszereket nem csak Lagrange multiplikátorokkal lehet kezelni, hanem úgy is, hogy a kényszerfeltételeknek megfelelően választjuk meg a koordinátákat. Ezzel a módszerrel a kényszerfeltételek automatikusan teljesülnek, és nem kell külön kényszererőket bevezetni.
Irjuk át a D'Alembert elvét általánosított koordinátákra $q_k$:
\begin{equation}
    \sum_i (\bold{F}_i - \bold{\dot{p}}_i) \cdot \delta \bold{r}_i = 0
\end{equation}
\begin{equation}
    \sum_i (\bold{F}_i - \bold{\dot{p}}_i) \cdot \sum_k \frac{\partial \bold{r}_i}{\partial q_k} \delta q_k = 0
\end{equation}
\begin{equation}
    \sum_k \left( \sum_i (\bold{F}_i - \bold{\dot{p}}_i) \cdot \frac{\partial \bold{r}_i}{\partial q_k} \right) \delta q_k = 0
\end{equation}
Mivel a virtuális elmozdulások tetszőlegesek, ezért a fenti egyenlet csak akkor teljesülhet, ha az egyes tagok külön-külön is nulla:
\begin{equation}
    \sum_i (\bold{F}_i - \bold{\dot{p}}_i) \cdot \frac{\partial \bold{r}_i}{\partial q_k} = 0
\end{equation}
Most bontsuk fel az egyes tagokat:
\begin{equation}
    \sum_i \bold{F}_i \cdot \frac{\partial \bold{r}_i}{\partial q_k} - \sum_i \bold{\dot{p}}_i \cdot \frac{\partial \bold{r}_i}{\partial q_k} = 0
\end{equation}
Ahol az első tag az általánosított erő $Q_k$:
\begin{equation}
    Q_k = \sum_i \bold{F}_i \cdot \frac{\partial \bold{r}_i}{\partial q_k}
\end{equation}
Ezt visszahelyettesítve a D'Alembert elvébe:
\begin{equation}
    \left( Q_k - \sum_i \bold{\dot{p}}_i \cdot \frac{\partial \bold{r}_i}{\partial q_k} \right) = 0
\end{equation}
Itt vizsgáljuk meg a második tagot kicsit részletesebben. Felfedezhetjük, hogy ez a kifejezés a kinetikus energia $K$ deriváltjával kapcsolatos:
\begin{equation}
    K = \sum_i \frac{1}{2} m_i \dot{\bold{r}}_i^2
\end{equation}
Ha ezt deviáljuk $q_k$ és $\dot{q}_k$ szerint, akkor a következőt kapjuk:
\begin{equation}
    \frac{\partial K}{\partial q_k} = \sum_i m_i \dot{\bold{r}}_i \cdot \frac{\partial \dot{\bold{r}}_i}{\partial q_k}
\end{equation}
\begin{equation}
    \frac{\partial K}{\partial \dot{q}_k} = \sum_i m_i \dot{\bold{r}}_i \cdot \frac{\partial \bold{\dot{r}}_i}{\partial \dot{q_k}}
\end{equation}
Ha pedig kivonjuk egymásból a kettőt, akkor a következőt kapjuk:
\begin{equation}
    \frac{d}{dt} \frac{\partial K}{\partial \dot{q}_k} - \frac{\partial K}{\partial q_k} = \sum_i m_i \bold{\ddot{r}}_i \cdot \frac{\partial \bold{r}_i}{\partial q_k} = \sum_i \bold{\dot{p}}_i \cdot \frac{\partial \bold{r}_i}{\partial q_k}
\end{equation}
Ez pedig pont a második tag a D'Alembert elvében. Ezt visszahelyettesítve:
\begin{equation}
    Q_k - \left( \frac{d}{dt} \frac{\partial K}{\partial \dot{q}_k} - \frac{\partial K}{\partial q_k} \right) = 0
\end{equation}
\begin{equation}
    \frac{d}{dt} \frac{\partial K}{\partial \dot{q}_k} - \frac{\partial K}{\partial q_k} = Q_k
\end{equation}
Ha pedig ide behelyettesítjük a konzervatív erőt, mint a potenciális energia gradiensét, akkor a következőt kapjuk:
\begin{equation}
    Q_k = - \frac{\partial U}{\partial q_k}
\end{equation}
\begin{equation}
    \frac{d}{dt} \frac{\partial K}{\partial \dot{q}_k} - \frac{\partial K}{\partial q_k} = - \frac{\partial U}{\partial q_k}
\end{equation}
\begin{equation}
    \frac{d}{dt} \frac{\partial (K - U)}{\partial \dot{q}_k} - \frac{\partial (K - U)}{\partial q_k} = 0
\end{equation}
A Kinematikus energia és a potenciális energia különbsége pedig pont a Lagrange-függvény:
\begin{equation}
    \frac{d}{dt} \frac{\partial \mathcal{L}}{\partial \dot{q}_k} - \frac{\partial \mathcal{L}}{\partial q_k} = 0
\end{equation}
Ez pedig pont az Euler-Lagrange-féle mozgásegyenlet, amely a Hamilton-elvből is levezethető. Ezt az egyenletet Lagrange-féle másodfajú egyenletnek nevezzük dinamikai rendszerek esetén.

\subsection{Szimmetriák és megmaradási tételek}
A legkisebb hatás elvének (pontosabb stacionárius hatás elve) van egy nagyon meglepő következménye, ami a Newtoni mechanikán is túlmutató eredményt hozott, és a természet egy alapvető tulajdonságát tárta fel.
Ha ugyanis a kiválasztott pályának csak annyi a feltétele, hogy a hatás stacionárius legyen, akkor bármilyen olyan transzformáció, amely a hatást nem változtatja meg, szintén egy megengedett pályát eredményez.
Például, toljuk el a kiválasztott koordináta rendszerünket egy kicsit térben:
\begin{equation}
    \bold{r} \to \bold{r} + \delta \bold{a}
\end{equation}
Ha erre felírjuk a Lagrange-függvényt, akkor azt látjuk, hogy a Lagrange-függvény nem változik meg, hiszen a kinetikus energia csak a sebességtől függ, a potenciális energia pedig csak a relatív távolságoktól.
\begin{equation}
    \mathcal{L} = K - U = \frac{1}{2}m \dot{\bold{r}}^2 - U(\bold{r}_i - \bold{r}_j)
\end{equation}
\begin{equation}
    \mathcal{L} = K - U = \frac{1}{2}m \dot{(\bold{r} + \delta \bold{a})}^2 - U((\bold{r}_i + \delta \bold{a}) - (\bold{r}_j + \delta \bold{a})) = \frac{1}{2}m \dot{\bold{r}}^2 - U(\bold{r}_i - \bold{r}_j)
\end{equation}
Tehát a Lagrange-függvény variációja nulla:
\begin{equation}
    \delta \mathcal{L} = 0
\end{equation}
Ez pedig azt jelenti, hogy a hatás sem változik meg:
\begin{equation}
    \delta S = \delta \int \mathcal{L} \, dt = 0
\end{equation}
Ez azt jelenti, hogy a rendszer "szimmetrikus" a térbeli eltolásva nézve, nem veszi észre, ha a koordináta-rendszer pontjait eltoljuk. 

\subsubsection*{Példa: forgatás}
Vegyünk egy rendszert, melyet elforgatunk $\theta$ szöggel a középpontja körül. A koordináták a következőképpen változnak:
\begin{equation}
    x' = x \cos(\theta) + y \sin(\theta)
\end{equation}
\begin{equation}
    y' = x \sin(\theta) + y \cos(\theta)
\end{equation}
A Lagrange-függvény pedig a következőképpen alakul:
\begin{equation}
    \mathcal{L} = K - U = \frac{1}{2}m(\dot{x}^2 + \dot{y}^2) - U(\sqrt{x^2 + y^2})
\end{equation}
Ha a $\theta = \delta$ szög kicsi, akkor felírhatjuk a következőképpen:
\begin{equation}
        cos(\theta) \approx 1
\end{equation}
\begin{equation}
        sin(\theta) \approx \delta
\end{equation}
Akkor a koordiniáták a következőképpen alakulnak:
\begin{equation}
    x' = x + y \delta
\end{equation}
\begin{equation}
    y' = -x \delta + y
\end{equation}
Ezt rendezve:
\begin{equation}
    \delta x = x' - x = y \delta
\end{equation}
\begin{equation}
    \delta y = y' - y = - x \delta
\end{equation}
\begin{equation}
    \delta \dot{x} = \dot{x'} - \dot{x} = \dot{y} \delta
\end{equation}
\begin{equation}
    \delta \dot{y} = \dot{y'} - \dot{y} = - \dot{x} \delta
\end{equation}
Innen a potenciális energia változása:
\begin{equation}
    \delta \sqrt{x^2 + y^2} = \frac{1}{2\sqrt{x^2 + y^2}} 2(x \delta x + y \delta y) = \frac{1}{\sqrt{x^2 + y^2}} (xy \delta - yx \delta) = 0
\end{equation}
\begin{equation}
    \rightarrow \delta U = 0
\end{equation}
A kinetikus energia változása pedig:
\begin{equation}
    \delta K = \frac{1}{2}m(2\dot{x} \delta \dot{x} + 2\dot{y} \delta \dot{y}) = m(\dot{x} \dot{y} \delta - \dot{y} \dot{x} \delta) = 0
\end{equation}
\begin{equation}
    \rightarrow \delta K = 0
\end{equation}
Tehát a Lagrange-függvény változása:
\begin{equation}
    \delta \mathcal{L} = \delta K - \delta U = 0
\end{equation}
Ez pedig azt jelenti, hogy a hatás sem változik meg:
\begin{equation}
    \delta S = \delta \int \mathcal{L} \, dt = 0
\end{equation}
Tehát a rendszer szimmetrikus forgatásra.

\subsection{Szimmetriák általánosan}
Térjünk vissza a tanszormációkra egy kicsit. Vegyünk egy transzlációs eltolást egy részecske helyzetében. Ekkor kétféle képpen értelmezhetjük a transzformációt:
\begin{itemize}
    \item Az első megközelítés szerint az egész koordináta rendszert eltoljuk, tehát a részecske helyzete változatlan marad, de a koordináták új értéket kapnak, ezt passzív transzformációnak nevezzük.
    \item A második megközelítés szerint a koordináta rendszert változatlanul hagyjuk, de a részecske helyzete változik, ezt aktív transzformációnak nevezzük.
\end{itemize}
Mindkét megközelítés helyes, és a fizikai eredményeknek függetlennek kell lenniük attól, hogy melyik megközelítést alkalmazzuk. Viszont a két lehetőség között van egy fontos különbség: az aktív transzformáció
esetén a részecske potenciális energiája is megváltozhat, hiszen a potenciális energia a részecske helyzetétől függ. Ezzel szemben a passzív transzformáció esetén a potenciális energia nem változik, hiszen a részecske helyzete változatlan marad,
csak a koordinátákat paraméterezzük át. Ennélfogva a passzív transzormáció jelen esetben nem hordoz túl sok érdekességet magában, ezért vizsgáljuk meg az aktív transzformációt egy kicsit részletesebben.
Legyen egy részecske, amelyet a konfiguációs térben egy $q_i$ általános koordináta ír le, és mozgassuk el egy kicsit $\delta q_i$-val:
\begin{equation}
    q_k \to q_k + \delta q_k
\end{equation}
\begin{equation}
    \dot{q}_k \to \dot{q}_k + \delta \dot{q}_k
\end{equation}
Akkor a Lagrange-függvény változása:
\begin{equation}
    \delta \mathcal{L} = \sum_k \left( \frac{\partial \mathcal{L}}{\partial q_k} \delta q_k + \frac{\partial \mathcal{L}}{\partial \dot{q}_k} \delta \dot{q}_k \right)
\end{equation}
Ha a rendszer szimmetrikus a transzformációra, akkor a Lagrange-függvény változása nulla:
\begin{equation}
    \delta \mathcal{L} = 0
\end{equation}
Ezt behelyettesítve:
\begin{equation}
    \sum_k \left( \frac{\partial \mathcal{L}}{\partial q_k} \delta q_k + \frac{\partial \mathcal{L}}{\partial \dot{q}_k} \delta \dot{q}_k \right) = 0
\end{equation}
Most alkalmazzuk az Euler-Lagrange-féle mozgásegyenletet:
\begin{equation}
    \frac{\partial \mathcal{L}}{\partial q_k} = \frac{d}{dt} \frac{\partial \mathcal{L}}{\partial \dot{q}_k}
\end{equation}
Ezt behelyettesítve:
\begin{equation}
    \sum_k \left( \frac{d}{dt} \frac{\partial \mathcal{L}}{\partial \dot{q}_k} \delta q_k + \frac{\partial \mathcal{L}}{\partial \dot{q}_k} \delta \dot{q}_k \right) = 0
\end{equation}
Most alkalmazzuk a természettudományokban gyakran használt "természetes" sorrendet, ahol a deriváltakat előre hozzuk:
\begin{equation}
    \sum_k \left( \frac{d}{dt} \left( \frac{\partial \mathcal{L}}{\partial \dot{q}_k} \delta q_k \right) - \frac{\partial \mathcal{L}}{\partial \dot{q}_k} \frac{d}{dt} \delta q_k + \frac{\partial \mathcal{L}}{\partial \dot{q}_k} \delta \dot{q}_k \right) = 0
\end{equation}
\begin{equation}
    \sum_k \left( \frac{d}{dt} \left( \frac{\partial \mathcal{L}}{\partial \dot{q}_k} \delta q_k \right) + \frac{\partial \mathcal{L}}{\partial \dot{q}_k} (\delta \dot{q}_k - \frac{d}{dt} \delta q_k) \right) = 0
\end{equation}
Ahol a második tag nulla, hiszen a variáció és a deriválás felcserélhető:
\begin{equation}
    \delta \dot{q}_k = \frac{d}{dt} \delta q_k
\end{equation}
Tehát a következőt kapjuk:
\begin{equation}
    \sum_k \frac{d}{dt} \left( \frac{\partial \mathcal{L}}{\partial \dot{q}_k} \delta q_k \right) = 0
\end{equation}
\begin{equation}
    \frac{d}{dt} \left( \sum_k \frac{\partial \mathcal{L}}{\partial \dot{q}_k} \delta q_k \right) = 0
\end{equation}
Ez azt jelenti, hogy a következő mennyiség megmarad:
\begin{equation}
    \sum_k \frac{\partial \mathcal{L}}{\partial \dot{q}_k} \delta q_k = \text{konstans}
\end{equation}
Ez a Noether-tétel, amely kimondja, hogy minden szimmetria esetén megmaradási törvény létezik. Például, ha a rendszer szimmetrikus térbeli eltolásra, akkor a lendület megmarad, ha szimmetrikus időbeli eltolásra, akkor az energia megmarad,
és ha szimmetrikus forgatásra, akkor a perdület megmarad. Ez egy borzasztó fontos felismerés, hisz ez a tétel köti össze a szimmetriákat a megmaradási törvényekkel, amelyek a fizika alapvető törvényei. Tehát ha egy tetszőleges szimmetriát találunk egy fizikai rendszerben,
akkor biztosak lehetünk benne, hogy létezik egy megmaradó mennyiség is.

\subsubsection*{Példa: transzlációs szimmetria}
Legyen egy rendszer ahol van $q_1$ és $q_2$ részecske, és ezeket kis $\delta$-val eltoljuk:
\begin{equation}
    \delta q_1 = \delta
\end{equation}
\begin{equation}
    \delta q_2 = \delta
\end{equation}
\begin{equation}
    \delta q_i = f_i(q) \delta
\end{equation}
Akkor a Noether-tétel szerint a következő mennyiség megmarad:
\begin{equation}
    \sum_k \frac{\partial \mathcal{L}}{\partial \dot{q}_k} \delta q_k = \sum_k p_k f_k(q) \delta = \text{konstans}
\end{equation}
\begin{equation}
    \sum_k p_k f_k(q) = p_1 + p_2 = \text{konstans}
\end{equation}
Ez a mennyiség a rendszer lendülete, amely megmarad a térbeli eltolás szimmetriája miatt.

Eltérő eltolásokra:
\begin{equation}
    \delta q_1 = b\delta
\end{equation}
\begin{equation}
    \delta q_2 = - a\delta
\end{equation}
\begin{equation}
    \delta q_i = f_i(q) \delta
\end{equation}
Akkor a Noether-tétel szerint a következő mennyiség megmarad:
\begin{equation}
    \sum_k \frac{\partial \mathcal{L}}{\partial \dot{q}_k} \delta q_k = \sum_k p_k f_k(q) \delta = \text{konstans}
\end{equation}
\begin{equation}
    \sum_k p_k f_k(q) = b p_1 - a p_2 = \text{konstans}
\end{equation}
Ez a mennyiség a rendszer impulzusmomentuma, amely megmarad a térbeli eltolás szimmetriája miatt.

Forgásra pedig:
\begin{equation}
    f_x = y
\end{equation}
\begin{equation}
    f_y = -x
\end{equation}
\begin{equation}
    p_x y - p_y x = L_z = \text{konstans}
\end{equation}
Ez a mennyiség a rendszer perdületmomentuma, amely megmarad a forgás szimmetriája miatt.

\subsubsection*{Példa: Harmonikus oszcillátor}
Legyen egy harmonikus oszcillátor, amelynek a Lagrange-függvénye a következő:
\begin{equation}
    \mathcal{L} = \frac{1}{2}m \dot{q}^2 - \frac{1}{2}k q^2
\end{equation}
Ez a rendszer időbeli eltolásra szimmetrikus, hiszen a Lagrange-függvény nem függ kifejezetten az időtől. Legyenek $m = 1, k=1$ az egyszerűség kedvéért. Tehát alkalmazhatjuk a Noether-tételt az időbeli eltolásra:
\begin{equation}
    \delta q = \dot{q} \delta t
\end{equation}
Akkor a Noether-tétel szerint a következő mennyiség megmarad:
\begin{equation}
    \sum_k \frac{\partial \mathcal{L}}{\partial \dot{q}_k} \delta q_k = \frac{\partial \mathcal{L}}{\partial \dot{q}} \delta q = p \dot{q} \delta t = \text{konstans}
\end{equation}
\begin{equation}
    p \dot{q} = m \dot{q}^2 = 2K = \text{konstans}
\end{equation}
Ez a mennyiség a rendszer kinetikus energiája, amely megmarad az időbeli eltolás szimmetriája miatt. Mivel a teljes energia $E = K + U$ is megmarad, ezért a potenciális energia is megmarad:
\begin{equation}
    U = \frac{1}{2}k q^2 = \text{konstans}
\end{equation}

\subsection{Hamilton függvény, kanonikus egyenletek, kanonikus transzformációk}
Láthattuk, hogy a Lagrange formalizmus központi eleme a Lagrange-függvény, amit eredetileg egy olyan függvényként definiáltunk, melynek az integrálja két pont között, a hatás. De hogyan is lehet értelmezni a Lagrange-függvényt?
Egyik intuitív gondolat lehet az, hogy esetleg a Lagrange-függvény a rendszer energiáját írja le. De ahhoz, hogy ez igaz legyen, egy magmaradó mennyiségnek kell lennie időben, úgyhogy vizsgáljuk meg, hogy ez helytálló-e.
Vegyünk egy általános Lagrange-függvényt, amely nem függ kifejezetten az időtől (különben nem lenne megmaradó mennyiség) és nézzük meg a deriváltját idő szerint:
\begin{equation}
    \frac{d\mathcal{L}}{dt} = \sum_k \left( \frac{\partial \mathcal{L}}{\partial q_k} \dot{q}_k + \frac{\partial \mathcal{L}}{\partial \dot{q}_k} \ddot{q}_k \right)
\end{equation}
Most alkalmazzuk az Euler-Lagrange-féle mozgásegyenletet:
\begin{equation}
    \frac{\partial \mathcal{L}}{\partial q_k} = \frac{d}{dt} \frac{\partial \mathcal{L}}{\partial \dot{q}_k}
\end{equation}
Ezt behelyettesítve:
\begin{equation}
    \frac{d\mathcal{L}}{dt} = \sum_k \left( \frac{d}{dt} \frac{\partial \mathcal{L}}{\partial \dot{q}_k} \dot{q}_k + \frac{\partial \mathcal{L}}{\partial \dot{q}_k} \ddot{q}_k \right)
\end{equation}
hozzuk be a kanonikus impulzust $ p_k = m\dot{q}_k = \frac{\partial \mathcal{L}}{\partial \dot{q}_k} $:
\begin{equation}
    \frac{d\mathcal{L}}{dt} = \sum_k \left( \dot{p}_k \dot{q}_k + p_k \ddot{q}_k \right)
\end{equation}
Ezt átrendezve:
\begin{equation}
    \frac{d\mathcal{L}}{dt} = \sum_k \frac{d}{dt} (p_k \dot{q}_k)
\end{equation}
\begin{equation}
    \frac{d\mathcal{L}}{dt} = \frac{d}{dt} \left( \sum_k p_k \dot{q}_k \right)
\end{equation}
Tehát a Lagrange-függvény nem egy megmaradó mennyiség időben, hiszen a deriváltja nem nulla, hanem a kanonikus impulzus és sebesség szorzatának az idő szerinti deriváltja. De ha ezt átrendezzük, akkor a következőt kapjuk:
\begin{equation}
    \frac{d}{dt} \left( \sum_k p_k \dot{q}_k - \mathcal{L} \right) = 0
\end{equation}
\begin{equation}
    \sum_k p_k \dot{q}_k - \mathcal{L} = \mathcal{H} = \text{konstans}
\end{equation}
Ez a mennyiség pedig már egy megmaradó mennyiség, amelyet Hamilton-függvénynek nevezünk. A Hamilton-függvény tehát a kanonikus impulzus és sebesség szorzatának és a Lagrange-függvény különbsége.
Most nézzük meg, hogy a Hamilton-függvény valóban az energia-e. Vegyünk egy egyszerű rendszert, ahol a Lagrange-függvény a következő:
\begin{equation}
    \mathcal{L} = K - U = \frac{1}{2}m \dot{q}^2 - U(q)
\end{equation}
Akkor a kanonikus impulzus:
\begin{equation}
    p = \frac{\partial \mathcal{L}}{\partial \dot{q}} = m \dot{q}
\end{equation}
A Hamilton-függvény pedig:
\begin{equation}
    \mathcal{H} = p \dot{q} - \mathcal{L} = m \dot{q}^2 - \left( \frac{1}{2}m \dot{q}^2 - U(q) \right)
\end{equation}
\begin{equation}
    \mathcal{H} = \frac{1}{2}m \dot{q}^2 + U(q) = K + U = E
\end{equation}
Tehát a Hamilton-függvény valóban az energia, amely megmarad időben, ha a Lagrange-függvény nem függ kifejezetten az időtől.

Általános esetben, ha a Lagrange-függvény expliciten is függ az időtől, akkor a következőképpen alakul a Hamilton-függvény:
\begin{equation}
    \frac{d\mathcal{L}}{dt} = \sum_k \left( \frac{\partial \mathcal{L}}{\partial q_k} \dot{q}_k + \frac{\partial \mathcal{L}}{\partial \dot{q}_k} \ddot{q}_k \right) + \frac{\partial \mathcal{L}}{\partial t}
\end{equation}
\begin{equation}
    \frac{d\mathcal{L}}{dt} = \sum_k \left( \dot{p}_k \dot{q}_k + p_k \ddot{q}_k \right) + \frac{\partial \mathcal{L}}{\partial t}
\end{equation}
\begin{equation}
    \frac{d\mathcal{L}}{dt} = \sum_k \frac{d}{dt} (p_k \dot{q}_k) + \frac{\partial \mathcal{L}}{\partial t}
\end{equation}
\begin{equation}
    \frac{d}{dt} \left( \sum_k p_k \dot{q}_k - \mathcal{L} \right) = - \frac{\partial \mathcal{L}}{\partial t}
\end{equation}
\begin{equation}
    \frac{d\mathcal{H}}{dt} = - \frac{\partial \mathcal{L}}{\partial t}
\end{equation}
Tehát ha a Lagrange-függvény kifejezetten függ az időtől, akkor a Hamilton-függvény nem marad meg időben, tehát az energia nem lesz megmaradó mennyiség!

\subsubsection{Kanonikus egyenletek}
Most, hogy megvan a Hamilton-függvény, nézzük meg, hogy mi történik ha a Hamilton-függvénybe hozunk be egy kis változást a kanonikus impulzusok és koordináták szerint:
\begin{equation}
    \delta F(q, p) = \frac{\partial F}{\partial q} \delta q + \frac{\partial F}{\partial p} \delta p \quad \rightarrow \quad \text{Ahol F egy tetszőleges függvény}
\end{equation}
\begin{equation}
    \mathcal{H} = \sum_k p_k \dot{q}_k - \mathcal{L}
\end{equation}
\begin{equation}
    \delta \mathcal{H} = \sum_k \left( \dot{q}_k \delta p_k + p_k \delta \dot{q}_k - \frac{\partial \mathcal{L}}{\partial q_k} \delta q_k - \frac{\partial \mathcal{L}}{\partial \dot{q}_k} \delta \dot{q}_k \right)
\end{equation}
Ahol az Euler-Lagrange-féle mozgásegyenletet alkalmazva:
\begin{equation}
    \frac{\partial \mathcal{L}}{\partial q_k} = \frac{d}{dt} \frac{\partial \mathcal{L}}{\partial \dot{q}_k} = \dot{p}_k
\end{equation}
\begin{equation}
    \delta \mathcal{H} = \sum_k \left( \dot{q}_k \delta p_k + p_k \delta \dot{q}_k - \dot{p}_k \delta q_k - p_k \delta \dot{q}_k \right)
\end{equation}
\begin{equation}
    \delta \mathcal{H} = \sum_k \left( \dot{q}_k \delta p_k - \dot{p}_k \delta q_k \right)
\end{equation}
Ebből következik, hogy a Hamilton-függvény parciális deriváltjai a kanonikus koordináták és impulzusok szerint a következők:
\begin{equation}
    \frac{\partial \mathcal{H}}{\partial p_k} = \dot{q}_k \quad \quad \frac{\partial \mathcal{H}}{\partial q_k} = -\dot{p}_k
\end{equation}
Ezek a Hamilton-féle kanonikus egyenletek, amelyek a Lagrange-féle mozgásegyenletek alternatív megfogalmazásai.

Tehát összefoglalva:
\begin{equation}
    \mathcal{L}(q, \dot{q}, t) \rightarrow
\end{equation}
\begin{equation}
    p = \frac{\partial \mathcal{L}}{\partial \dot{q}} \quad \quad \dot{p} = \frac{\partial \mathcal{L}}{\partial q}
\end{equation}
\begin{equation}
    \frac{d\mathcal{L}}{dt} = \sum_k \left( \frac{\partial \mathcal{L}}{\partial q_k} \dot{q}_k + \frac{\partial \mathcal{L}}{\partial \dot{q}_k} \ddot{q}_k \right) + \frac{\partial \mathcal{L}}{\partial t}
\end{equation}
\noindent\hfil\rule{0.5\textwidth}{.4pt}\hfil
\begin{equation}
    \mathcal{H}(q, p, t) = \sum_k p_k \dot{q}_k - \mathcal{L}
\end{equation}
\begin{equation}
    \frac{d\mathcal{H}}{dt} = - \frac{\partial \mathcal{L}}{\partial t}
\end{equation}
\begin{equation}
    \frac{\partial \mathcal{H}}{\partial p_k} = \dot{q}_k \quad \quad \frac{\partial \mathcal{H}}{\partial q_k} = -\dot{p}_k
\end{equation}
\noindent\hfil\rule{0.5\textwidth}{.4pt}\hfil
\begin{equation}
    \text{Kanonikus egyenletek:}
\end{equation}
\begin{equation}
    \dot{q}_k = \frac{\partial \mathcal{H}}{\partial p_k} \quad \quad \dot{p}_k = -\frac{\partial \mathcal{H}}{\partial q_k}
\end{equation}

\subsubsection{Kanonikus transzformációk}
A Hamilton-függvény és a kanonikus egyenletek bevezetése után vizsgáljuk meg, hogy milyen transzformációkat alkalmazhatunk a kanonikus koordinátákra $(q_k, p_k)$ úgy, hogy a kanonikus egyenletek formája megmaradjon.
Ezeket a transzformációkat kanonikus transzformációknak nevezzük. Legyen egy kanonikus transzformáció, amely a következőképpen néz ki:
\begin{equation}
    (q_k, p_k) \to (Q_k, P_k)
\end{equation}
A kanonikus egyenletek formája a következő:
\begin{equation}
    \dot{q}_k = \frac{\partial \mathcal{H}}{\partial p_k} \quad \quad \dot{p}_k = -\frac{\partial \mathcal{H}}{\partial q_k}
\end{equation}
\begin{equation}
    \dot{Q}_k = \frac{\partial \mathcal{K}}{\partial P_k} \quad \quad \dot{P}_k = -\frac{\partial \mathcal{K}}{\partial Q_k}
\end{equation}
Ahol $\mathcal{K}$ az új Hamilton-függvény, amely a kanonikus transzformáció után keletkezik. A kanonikus transzformáció akkor megengedett, ha a kanonikus egyenletek formája megmarad, tehát a következő feltételnek kell teljesülnie:
\begin{equation}
    \frac{\partial \mathcal{H}}{\partial p_k} = \frac{\partial \mathcal{K}}{\partial P_k} \quad \quad -\frac{\partial \mathcal{H}}{\partial q_k} = -\frac{\partial \mathcal{K}}{\partial Q_k}
\end{equation}
Ezt a feltételt úgy is felírhatjuk, hogy a következő mátrix egyenletnek kell teljesülnie:
\begin{equation}
    \begin{pmatrix}\frac{\partial Q}{\partial q} & \frac{\partial Q}{\partial p} \\ \frac{\partial P}{\partial q} & \frac{\partial P}{\partial p} \end{pmatrix}
    \begin{pmatrix} 0 & I \\ -I & 0 \end{pmatrix}
    \begin{pmatrix}\frac{\partial Q}{\partial q} & \frac{\partial Q}{\partial p} \\ \frac{\partial P}{\partial q} & \frac{\partial P}{\partial p} \end{pmatrix}^T
    = \begin{pmatrix} 0 & I \\ -I & 0 \end{pmatrix}
\end{equation}
Ahol $I$ az egységmátrix. Ez a feltétel biztosítja, hogy a kanonikus egyenletek formája megmaradjon a kanonikus transzformáció után.
Egy kanonikus transzformációt gyakran egy generátorfüggvénnyel definiálnak, amely a következőképpen néz ki:
\begin{equation}
    F(q, P, t)
\end{equation}
A generátorfüggvény segítségével a kanonikus transzformáció a következőképpen néz ki:
\begin{equation}
    p_k = \frac{\partial F}{\partial q_k} \quad \quad Q_k = \frac{\partial F}{\partial P_k}
\end{equation}
A generátorfüggvény segítségével a kanonikus transzformációk könnyen definiálhatók és kezelhetők, és számos hasznos tulajdonsággal rendelkeznek. Például, ha a generátorfüggvény nem függ kifejezetten az időtől,
akkor a kanonikus transzformáció megőrzi a Hamilton-függvény formáját, tehát az új Hamilton-függvény is megmarad időben. Ezért a kanonikus transzformációk nagyon hasznosak a Hamilton-függvények és a kanonikus egyenletek vizsgálatában,
és számos alkalmazásuk van a fizikában és a matematikában.

\subsubsection*{Példa: Harmonikus oszcillátor}
Vegyünk egy harmonikus oszcillátort, amelynek a Hamilton-függvénye a következő (teljes energia):
\begin{equation}
    \mathcal{H} = \frac{p^2}{2m} + \frac{1}{2}k q^2
\end{equation}
A kanonikus egyenletek pedig:
\begin{equation}
    \dot{q} = \frac{\partial \mathcal{H}}{\partial p} = \frac{p}{m}
\end{equation}
\begin{equation}
    \dot{p} = -\frac{\partial \mathcal{H}}{\partial q} = -k q
\end{equation}
Az impulzust mégegyszer deriválva megkaphatjuk a $F = m\ddot{q}$ alakot:
\begin{equation}
    \ddot{q} = -\frac{k}{m} q
\end{equation}
Ez pedig pont a harmonikus oszcillátor mozgásegyenlete.

\subsubsection*{Példa: Liouville-tétel}
A kanonikus transzformációk egyik fontos tulajdonsága a Liouville-tétel, amely kimondja, hogy a kanonikus transzformációk megőrzik a fázistér térfogatát. Ez azt jelenti, hogy ha egy fázistérbeli területet kanonikus transzformációnak vetünk alá,
akkor a terület mérete nem változik meg. Matematikailag ez a következőképpen írható fel:
\begin{equation}
    \int dq_1 dp_1 \ldots dq_n dp_n = \int dQ_1 dP_1 \ldots dQ_n dP_n
\end{equation}
Ez a tétel nagyon fontos a statisztikus mechanikában, ahol a fázistér térfogatának megőrzése biztosítja, hogy a rendszer állapotainak száma nem változik meg kanonikus transzformációk alatt. Ez felfogható úgy is,
hogy a fázistérben egy "folyadék" áramlik, amely nem összenyomható, tehát a térfogat nem változik meg.

\section{Egzaktul megldható Fizikai problémák}
Csillapított és kényszerrezgések, csatolt rezgések, lineáris lánc. Kepler-probléma, bolygómozgás. Kvantummechanikai problémák: potenciálvölgy, oszcillátor, rotátor. Hidrogénatom. Keltő és eltüntető operátorok.

\subsection{Csillapított és kényszerrezgések, csatolt rezgések, lineáris lánc}
\subsubsection{Csillapított rezgés}
Csillapított rezgésről akkor beszélünk, ha egy rezgő rendszerben van valamilyen energia veszteség, amely csökkenti a rezgés amplitúdóját idővel. A csillapítás lehet lineáris vagy nemlineáris, és különböző fizikai mechanizmusok révén jöhet létre,
például súrlódás, légellenállás vagy belső súrlódás. Vegyünk példának egy olyan rendszert amikor a harmonikus oszcillátor egy közegben mozog, amely a sebességgel arányos módon csillapítja a mozgást:
\begin{figure}[H]
    \centering
    \includegraphics[width=0.4\textwidth]{imgs/3-tetel/csillapitott_rezges.jpg}
    \caption{Harmonikus oszcillátor közegben}
    \label{fig:csillapitott_rezges}
\end{figure}
Ekkor a mozgásegyenlet a következő lesz:
\begin{equation}
    m\ddot{x} = - kx - b\dot{x}
\end{equation}
Ahol $b$ a csillapítási együttható, amely a közeg tulajdonságaitól függ. Itt látható, hogy a standard harmonikus oszcillátorhoz képest van egy plusz tag, amely a sebességgel arányos és negatív előjelű, tehát csökkenti a mozgást. Ez egy másodrendű lineáris differenciálegyenlet,
amelyet megoldhatunk a klasszikus Ansatz segítségével. Először is osszunk le $m$-el és paraméterezzük át a mozgásegyenletet:
\begin{equation}
    \ddot{x} + 2\beta \dot{x} + \omega_0^2 x = 0
\end{equation}
\begin{equation}
    \text{Ahol } \beta = \frac{b}{2m} \quad \text{és} \quad \omega_0 = \sqrt{\frac{k}{m}}
\end{equation}
Mivel egy periodikus függvényt feltételezünk ezért próbálkozzunk meg egy $\sin$ fügvénnyel:
\begin{equation}
    x(t) = A(t) sin(\omega t + \varphi)
\end{equation}
Nézzük meg, hogy ez a függvény kielégíti-e a mozgásegyenletet. Ehhez szükségünk lesz az első és második deriváltjára:
\begin{equation}
    \dot{x}(t) = \dot{A}(t) sin(\omega t + \varphi) + A(t) \omega cos(\omega t + \varphi)
\end{equation}
\begin{equation}
    \ddot{x}(t) = \ddot{A}(t) sin(\omega t + \varphi) + 2\dot{A}(t) \omega cos(\omega t + \varphi) - A(t) \omega^2 sin(\omega t + \varphi)
\end{equation}
Ezt behelyettesítve a mozgásegyenletbe:
\begin{equation}
    \begin{aligned}
        & \ddot{A}(t) sin(\omega t + \varphi) + 2\dot{A}(t) \omega cos(\omega t + \varphi) - A(t) \omega^2 sin(\omega t + \varphi) + \\
        & 2\beta \left( \dot{A}(t) sin(\omega t + \varphi) + A(t) \omega cos(\omega t + \varphi) \right) + \\
        & \omega_0^2 A(t) sin(\omega t + \varphi) = 0
    \end{aligned}
\end{equation}
Most csoportosítsuk a $sin$ és $cos$ tagokat külön:
\begin{equation}
    \left( \ddot{A}(t) - A(t) \omega^2 + 2\beta \dot{A}(t) + \omega_0^2 A(t) \right) sin(\omega t + \varphi) + \left( 2\dot{A}(t) \omega + 2\beta A(t) \omega \right) cos(\omega t + \varphi) = 0
\end{equation}
Mivel a $sin$ és $cos$ függvények lineárisan függetlenek, ezért mindkét zárójelnek nullának kell lennie:
\begin{equation}
    \ddot{A}(t) - A(t) \omega^2 + 2\beta \dot{A}(t) + \omega_0^2 A(t) = 0
\end{equation}
\begin{equation}
    2\dot{A}(t) \omega + 2\beta A(t) \omega = 0
\end{equation}
Nézzük meg először az elsőrendű egyenletet:
\begin{equation}
    \dot{A}(t) + \beta A(t) = 0
\end{equation}
Ennek a megoldása triviálisan:
\begin{equation}
    A(t) = A_0 e^{-\beta t}
\end{equation}
Ezt behelyettesítve a másodrendű egyenletbe:
\begin{equation}
    \beta^2 A_0 e^{-\beta t} - A_0 e^{-\beta t} \omega^2 - 2\beta^2 A_0 e^{-\beta t} + \omega_0^2 A_0 e^{-\beta t} = 0
\end{equation}
\begin{equation}
    -\beta^2 - \omega^2 + \omega_0^2 = 0
\end{equation}
\begin{equation}
    \omega^2 = \omega_0^2 - \beta^2
\end{equation}
Tehát a csillapított rezgés megoldása a következő:
\begin{equation}
    x(t) = A_0 e^{-\beta t} sin(\omega t + \varphi)
\end{equation}
Ahol $\omega = \sqrt{\omega_0^2 - \beta^2}$. Ez azt jelenti, hogy a mozgás nagyban függ a kezdeti feltételektől, amik alapján három esetet különböztetünk meg:
\begin{itemize}
    \item Alulcsillapított rezgés ($\beta < \omega_0$): Ilyenkor a rendszer rezeg, de az amplitúdó exponenciálisan csökken idővel. A rezgés frekvenciája kisebb, mint a természetes frekvencia.
    \item Kritikus csillapítás ($\beta = \omega_0$): Ilyenkor a rendszer nem rezeg, hanem a lehető legy gyorsabban tér vissza az egyensúlyi helyzetbe anélkül, hogy túllépné azt.
    \item Túlcsillapított rezgés ($\beta > \omega_0$): Ilyenkor a rendszer szintén nem rezeg, de lassabban tér vissza az egyensúlyi helyzetbe, mint a kritikus csillapítás esetén.
\end{itemize}   
Tehát minden esetben exponenciálisan csökken az amplitúdó, de a mozgás jellege és a visszatérés sebessége az egyensúlyi helyzetbe a csillapítás mértékétől függ.
\begin{figure}[H]
\centering
\begin{subfigure}{.5\textwidth}
  \centering
  \includegraphics[width=1\linewidth]{imgs/3-tetel/csillapitas_merteke.jpg}
  \caption{Alulcsillapított rezgés alakja}
  \label{fig:alulcsillapitott}
\end{subfigure}%
\begin{subfigure}{.5\textwidth}
  \centering
  \includegraphics[width=1\linewidth]{imgs/3-tetel/csillapitott_rezgesek_tipusai.png}
  \caption{Csillapított rezgések típusai}
  \label{fig:kritikus_csillapitott}
\end{subfigure}
\caption{Csillapított rezgések}
\label{fig:csillapitott_rezgesek}
\end{figure}

\subsubsection{kényszerrezgések}
A kényszerrezgés bizonyos szempontból a Csillapított rezgés ellentéte, hiszen itt egy külső erő hat a rendszerre, amely folyamatosan energiát visz be a rendszerbe, így fenntartva a rezgést.
A kényszerrezgés akkor fordul elő, amikor egy rezgő rendszerre egy időben változó külső erő hat, amelynek frekvenciája közel van a rendszer sajátfrekvenciájához.
Ez a jelenség rezonanciaként ismert, és akkor következik be, amikor a külső erő frekvenciája megegyezik a rendszer természetes frekvenciájával, vagy nagyon közel van hozzá. Ilyenkor a rendszer válasza jelentősen megnő, és az amplitúdója is nagyobb lesz.
Vegyünk egy példát egy harmonikus oszcillátorra, amelyre egy külső erő hat:
\begin{figure}[H]
    \centering
    \includegraphics[width=0.8\textwidth]{imgs/3-tetel/kenyszerrezges.png}
    \caption{Harmonikus oszcillátor külső erő hatására}
    \label{fig:kenyszerrezges}
\end{figure}
írjuk fel a mozgásegyenletet:
\begin{equation}
    m\ddot{x} + b\dot{x} + kx = F_0 sin(\omega t)
\end{equation}
Ahol $F_0$ a külső erő amplitúdója, $\omega$ pedig a külső erő frekvenciája. Ebben az egyenletben jelen van a csillapítás is, amelyet a $b\dot{x}$ tag képvisel. Ez egy inhomogén másodrendű lineáris differenciálegyenlet,
amelynek a megoldása a homogén egyenlet megoldásából és az inhomogén megoldásból áll. Rendezzük át megint az egyenletet és paraméterezzük át:
\begin{equation}
    \ddot{x} + 2\beta \dot{x} + \omega_0^2 x = \frac{F_0}{m} sin(\omega t) 
\end{equation}
A homogén megoldás megegyezik a csillapított rezgés megoldásával, amelyet már korábban megtaláltunk:
\begin{equation}
    x_h(t) = A_0 e^{-\beta t} sin(\omega_d t + \varphi)
\end{equation}
Ahol $\omega_d = \sqrt{\omega_0^2 - \beta^2}$. Most keressük meg az inhomogén megoldást. Mivel a jobb oldalon egy szinuszos kényszerítő erő van, ezért próbálkozzunk meg egy szinuszos megoldással:
\begin{equation}
    x_p(t) = A sin(\omega t + \phi)
\end{equation}
Ezt behelyettesítve az egyenletbe:
\begin{equation}
    -A \omega^2 sin(\omega t + \phi) + 2\beta A \omega cos(\omega t + \phi) + \omega_0^2 A sin(\omega t + \phi) = \frac{F_0}{m} sin(\omega t)
\end{equation}
Csoportosítsuk a $sin$ és $cos$ tagokat külön:
\begin{equation}
    \left( -A \omega^2 + \omega_0^2 A \right) sin(\omega t + \phi) + 2\beta A \omega cos(\omega t + \phi) = \frac{F_0}{m} sin(\omega t)
\end{equation}
Most használjuk a szögösszeg képletet, hogy kifejezzük a $sin(\omega t + \phi)$ és $cos(\omega t + \phi)$-t:
\begin{equation}
    \left( -A \omega^2 + \omega_0^2 A \right) \left( sin(\omega t) cos(\phi) + cos(\omega t) sin(\phi) \right) + 2\beta A \omega \left( cos(\omega t) cos(\phi) - sin(\omega t) sin(\phi) \right) = \frac{F_0}{m} sin(\omega t)
\end{equation}
Csoportosítsuk a $sin(\omega t)$ és $cos(\omega t)$ tagokat külön:
\begin{equation}
    \begin{aligned}
        & \left( \left( -A \omega^2 + \omega_0^2 A \right) cos(\phi) - 2\beta A \omega sin(\phi) \right) sin(\omega t) + \\
        & \left( \left( -A \omega^2 + \omega_0^2 A \right) sin(\phi) + 2\beta A \omega cos(\phi) \right) cos(\omega t) = \frac{F_0}{m} sin(\omega t)
    \end{aligned}
\end{equation}
Mivel a $sin(\omega t)$ és $cos(\omega t)$ függvények lineárisan függetlenek, ezért a cos-os tagnak el kell tűnnie, és a sin-es tagnak meg kell egyeznie a jobb oldallal:
\begin{equation}
    \left( -A \omega^2 + \omega_0^2 A \right) cos(\phi) - 2\beta A \omega sin(\phi) = \frac{F_0}{m}
\end{equation}
\begin{equation}
    \left( -A \omega^2 + \omega_0^2 A \right) sin(\phi) + 2\beta A \omega cos(\phi) = 0
\end{equation}
Itt a két egyenletet négzetre emelve és összeadva kifejezhetjük az $A$ amplitúdót:
\begin{equation}
    A^2\left[\left(\left( -\omega^2 + \omega_0^2 \right) cos(\phi) - 2\beta  \omega sin(\phi)\right)^2 + \left( \left( - \omega^2 + \omega_0^2 \right) sin(\phi) + 2\beta \omega cos(\phi) \right)^2\right] = \left(\frac{F_0}{m}\right)^2
\end{equation}
\begin{equation}
    \begin{aligned}
        & A^2 \left(-\omega^2 + \omega^2_0\right)^2 cos^2(\phi) + \left(2\beta \omega\right)^2 sin^2(\phi) - 4\beta \omega \left(-\omega^2 + \omega^2_0\right) sin(\phi) cos(\phi) + \\
        & A^2 \left(-\omega^2 + \omega^2_0\right)^2 sin^2(\phi) + \left(2\beta \omega\right)^2 cos^2(\phi) + 4\beta \omega \left(-\omega^2 + \omega^2_0\right) sin(\phi) cos(\phi) = \left(\frac{F_0}{m}\right)^2
    \end{aligned}
\end{equation}
\begin{equation}
    A^2 \left[ \left(-\omega^2 + \omega^2_0\right)^2 \left( cos^2(\phi) + sin^2(\phi) \right) + \left(2\beta \omega\right)^2 \left( sin^2(\phi) + cos^2(\phi) \right) \right] = \left(\frac{F_0}{m}\right)^2
\end{equation}
\begin{equation}
    A^2 \left[ \left(-\omega^2 + \omega^2_0\right)^2 + \left(2\beta \omega\right)^2 \right] = \left(\frac{F_0}{m}\right)^2
\end{equation}
\begin{equation}
    A = \frac{\frac{F_0}{m}}{\sqrt{(\omega_0^2 - \omega^2)^2 + (2\beta \omega)^2}}
\end{equation}
Majd a második egyenletből kifejezhetjük a $\phi$ fázist:
\begin{equation}
    A\left[\left( - \omega^2 + \omega_0^2  \right) sin(\phi) + 2\beta  \omega cos(\phi)\right] = 0
\end{equation}
\begin{equation}
    \left( - \omega^2 + \omega_0^2  \right) sin(\phi) + 2\beta  \omega cos(\phi) = 0
\end{equation}
\begin{equation}
    \left( - \omega^2 + \omega_0^2  \right) sin(\phi) = - 2\beta  \omega cos(\phi)
\end{equation}
\begin{equation}
    tan(\phi) = \frac{- 2\beta  \omega}{- \omega^2 + \omega_0^2}
\end{equation}
\begin{equation}
    tan(\phi) = \frac{2\beta  \omega}{\omega^2 - \omega_0^2}
\end{equation}
Tehát a kényszerrezgés teljes megoldása a homogén és inhomogén megoldás összege:
\begin{equation}
    x(t) = A_0 e^{-\beta t} sin(\omega_d t + \varphi) + \frac{\frac{F_0}{m}}{\sqrt{(\omega_0^2 - \omega^2)^2 + (2\beta \omega)^2}} sin(\omega t + \phi)
\end{equation}
Ahol az első tag a csillapított rezgés, a második tag pedig a kényszerrezgés. Idővel a csillapított rezgés eltűnik, és a rendszer csak a kényszerrezgést mutatja. Az amplitúdó a kényszerrezgésnél a következőképpen viselkedik:
\begin{itemize}
    \item Ha a kényszerítő erő frekvenciája nagyon eltér a rendszer sajátfrekvenciájától ($\omega \ll \omega_0$ vagy $\omega \gg \omega_0$), akkor az amplitúdó kicsi lesz, mivel a rendszer nem rezonál a kényszerítő erővel.
    \item Ha a kényszerítő erő frekvenciája közel van a rendszer sajátfrekvenciájához ($\omega \approx \omega_0$), akkor az amplitúdó jelentősen megnő, mivel a rendszer rezonál a kényszerítő erővel. Ez a rezonancia jelensége.
    \item Ha a csillapítás kicsi ($\beta \to 0$), akkor az amplitúdó nagyon nagy lehet rezonancia esetén, mivel nincs elég energia veszteség a rendszerben.
    \item Ha a csillapítás nagy ($\beta$ nagy), akkor az amplitúdó kisebb lesz rezonancia esetén is, mivel a csillapítás elnyeli az energiát, és megakadályozza, hogy az amplitúdó túl nagy legyen.
\end{itemize}
\begin{figure}[H]
    \centering
    \includegraphics[width=0.6\textwidth]{imgs/3-tetel/resonance.png}
    \caption{Kényszerrezgés amplitúdója a kényszerítő erő frekvenciájának függvényében különböző csillapítási értékek mellett}
    \label{fig:rezges_amplitudo}
\end{figure}

\subsubsection{Csatolt rezgések}
A csatolt rezgések olyan rendszerek, amelyekben két vagy több rezgő elem kölcsönhatásban áll egymással, így a rezgésük nem független, hanem összefügg. A csatolt rezgések gyakran előfordulnak fizikai rendszerekben, például mechanikai rendszerekben,
elektromos rendszerekben vagy akár biológiai rendszerekben is. Vegyünk egy egyszerű példát két csatolt harmonikus oszcillátorra, amelyek egy rugóval vannak összekötve:
\begin{figure}[H]
    \centering
    \includegraphics[width=0.6\textwidth]{imgs/3-tetel/csatolt_rezges.png}
    \caption{Két csatolt harmonikus oszcillátor}
    \label{fig:csatolt_rezges}
\end{figure}
A mozgásegyenletek a következők lesznek:
\begin{equation}
    m\ddot{x}_1 = -k x_1 - k_c (x_1 - x_2)
\end{equation}
\begin{equation}
    m\ddot{x}_2 = -k x_2 - k_c (x_2 - x_1)
\end{equation}
Ahol $k$ a rugóállandó, $k_c$ a csatolási rugóállandó, $m$ a tömeg, $x_1$ és $x_2$ pedig az első és második oszcillátor helyzete. Ezeket az egyenleteket rendezzük át:
\begin{equation}
    \ddot{x}_1 + \left(\frac{k + k_c}{m}\right) x_1 - \frac{k_c}{m} x_2 = 0
\end{equation}
\begin{equation}
    \ddot{x}_2 + \left(\frac{k + k_c}{m}\right) x_2 - \frac{k_c}{m} x_1 = 0
\end{equation}
Nevezzük át ismét a paramétereket:
\begin{equation}
    \omega_0^2 = \frac{k + k_c}{m} \quad \quad \omega_c^2 = \frac{k_c}{m}
\end{equation}
Így az egyenletek a következőképpen néznek ki:
\begin{equation}
    \ddot{x}_1 + \omega_0^2 x_1 - \omega_c^2 x_2 = 0
\end{equation}
\begin{equation}
    \ddot{x}_2 + \omega_0^2 x_2 - \omega_c^2 x_1 = 0
\end{equation}
Adjuk össze és vonjuk ki az egyenleteket:
\begin{equation}
    \ddot{x}_1 + \ddot{x}_2 + \omega_0^2 (x_1 + x_2) - \omega_c^2 (x_2 + x_1) = 0
\end{equation}
\begin{equation}
    \ddot{x}_1 - \ddot{x}_2 + \omega_0^2 (x_1 - x_2) - \omega_c^2 (x_2 - x_1) = 0
\end{equation}
Nevezzük el az új változókat:
\begin{equation}
    X = x_1 + x_2 \quad \quad Y = x_1 - x_2
\end{equation}
Így az egyenletek a következőképpen néznek ki:
\begin{equation}
    \ddot{X} + (\omega_0^2 - \omega_c^2) X = 0
\end{equation}
\begin{equation}
    \ddot{Y} + (\omega_0^2 + \omega_c^2) Y = 0
\end{equation}
Ezek már független egyenletek, amelyeket könnyen megoldhatunk:
\begin{equation}
    \ddot{X} = - (\omega_0^2 - \omega_c^2) X
\end{equation}
\begin{equation}
    \ddot{Y} = - (\omega_0^2 + \omega_c^2) Y
\end{equation}
Innen triviálisan megkapjuk a megoldásokat:
\begin{equation}
    X(t) = A_1 sin(\omega_1 t + \varphi_1) \quad \quad \omega_1 = \sqrt{\omega_0^2 - \omega_c^2}
\end{equation}
\begin{equation}
    Y(t) = A_2 sin(\omega_2 t + \varphi_2) \quad \quad \omega_2 = \sqrt{\omega_0^2 + \omega_c^2}
\end{equation}
Visszaalakítva az eredeti változókra:
\begin{equation}
    x_1(t) = \frac{X(t) + Y(t)}{2} = \frac{A_1 sin(\omega_1 t + \varphi_1) + A_2 sin(\omega_2 t + \varphi_2)}{2}
\end{equation}
\begin{equation}
    x_2(t) = \frac{X(t) - Y(t)}{2} = \frac{A_1 sin(\omega_1 t + \varphi_1) - A_2 sin(\omega_2 t + \varphi_2)}{2}
\end{equation}
Ahol $\omega_1$ és $\omega_2$ a két csatolt rezgés sajátfrekvenciái, amelyek a csatolás mértékétől függnek. Az $A_1$, $A_2$, $\varphi_1$ és $\varphi_2$ a kezdeti feltételektől függő állandók.

Nézzünk meg egy speciális esetet, amikor $A_1 = A_2 = \frac{A}{2}$ és $\varphi_1 = \varphi_2 = \frac{\pi}{2}$ (azaz mindkét oszcillátor ugyanazzal az amplitúdóval és fázissal indul):
\begin{equation}
    x_1(t) = \frac{A}{2} \left( sin(\omega_1 t + \frac{\pi}{2}) + sin(\omega_2 t + \frac{\pi}{2}) \right) = \frac{A}{2} \left( cos(\omega_1 t) + cos(\omega_2 t) \right)
\end{equation}
\begin{equation}
    x_2(t) = \frac{A}{2} \left( sin(\omega_1 t + \frac{\pi}{2}) - sin(\omega_2 t + \frac{\pi}{2}) \right) = \frac{A}{2} \left( cos(\omega_1 t) - cos(\omega_2 t) \right)
\end{equation}
A $cos(x) + cos(y) = 2 cos\left(\frac{x+y}{2}\right) cos\left(\frac{x-y}{2}\right)$ és $cos(x) - cos(y) = -2 sin\left(\frac{x+y}{2}\right) sin\left(\frac{x-y}{2}\right)$ azonosságot felhasználva:
\begin{equation}
    x_1(t) = A cos\left(\frac{(\omega_1 + \omega_2)t}{2}\right) cos\left(\frac{(\omega_1 - \omega_2)t}{2}\right)
\end{equation}
\begin{equation}
    x_2(t) = A sin\left(\frac{(\omega_1 + \omega_2)t}{2}\right) sin\left(\frac{(\omega_1 - \omega_2)t}{2}\right)
\end{equation}
Ha emellé feltesszük, hogy a csatolás kicsi, azaz $\omega_1 \approx \omega_2$, akkor a sajátfrekvenciák közel lesznek egymáshoz:
\begin{equation}
    \omega = \frac{\omega_1 + \omega_2}{2} \approx \omega_1 \approx \omega_2 \quad \quad \epsilon = \frac{\omega_1 - \omega_2}{2} \ll \omega
\end{equation}
Így a megoldások a következőképpen egyszerűsödnek:
\begin{equation}
    x_1(t) \approx A cos(\omega t) cos\left(\frac{(\omega_1 - \omega_2)t}{2}\right)
\end{equation}
\begin{equation}
    x_2(t) \approx A sin(\omega t) sin\left(\frac{(\omega_1 - \omega_2)t}{2}\right)
\end{equation}
Ez azt jelenti, hogy mindkét oszcillátor rezeg a $\omega_0$ frekvenciával, de az amplitúdójuk időben változik, és egy lassú modulációt mutat a $\frac{\omega_1 - \omega_2}{2}$ frekvenciával. Ez a jelenség a lebegés a csatolt rezgések egyik jellegzetessége,
és gyakran előfordul különböző fizikai rendszerekben, például molekulákban, kristályokban vagy akár mechanikai rendszerekben is.
\begin{figure}[H]
    \centering
    \includegraphics[width=0.6\textwidth]{imgs/3-tetel/lebeges.png}
    \caption{Lebegés jelensége csatolt rezgések esetén}
    \label{fig:csatolt_rezgesek_idobeli}
\end{figure}

\subsubsection{Lineáris lánc}
A lineáris lánc a csatolt rezgés kiterjesztése egy végtelen számú oszcillátorra, amelyek egy egyenes vonal mentén helyezkednek el, és csak a szomszédos oszcillátorokkal vannak csatolva. Ez a modell gyakran használatos a kristályrácsok vibrációinak leírására,
mivel a kristályok atomjai hasonlóan rendeződnek el egy rácsban, és csak a közvetlenül szomszédos atomokkal lépnek kölcsönhatásba. Vegyünk egy végtelen lineáris láncot, ahol minden oszcillátor tömege $m$, és a csatolási rugóállandó $k$. Emelett feltételezzünk
periodikus határfeltételeket, azaz az első és utolsó oszcillátor is csatolva van egymáshoz, így a határokon is tudjuk kényelmesen kezelni a problémát:
\begin{figure}[H]
    \centering
    \includegraphics[width=0.8\textwidth]{imgs/3-tetel/linearis_lanc.png}
    \caption{Végtelen lineáris lánc}
    \label{fig:linear_chain}
\end{figure}
A mozgásegyenlet az n-edik oszcillátor számára a következő lesz:
\begin{equation}
    m\ddot{x}_n = -k (x_n - x_{n-1}) - k (x_n - x_{n+1})
\end{equation}
Ahol $x_n$ az $n$-edik oszcillátor helyzete. Ezt rendezzük át:
\begin{equation}
    \ddot{x}_n = -\frac{k}{m} (2x_n - x_{n-1} - x_{n+1})
\end{equation}
Nevezzük át a paramétert:
\begin{equation}
    \omega_0^2 = \frac{k}{m}
\end{equation}
Így az egyenlet a következőképpen néz ki:
\begin{equation}
    \ddot{x}_n = -\omega_0^2 (2x_n - x_{n-1} - x_{n+1})
\end{equation}
Hogy az összes mozgásegyenletet egyszerre tudjuk kezelni, tegyük egy mátrixba az összes $x_n$-t:
\begin{equation}
    \mathbf{x} = \begin{pmatrix}
    \vdots \\
    x_{n-1} \\
    x_n \\
    x_{n+1} \\
    \vdots
    \end{pmatrix}
\end{equation}
Így a mozgásegyenletek mátrix formában a következőképpen néznek ki:
\begin{equation}
    \ddot{\mathbf{x}} = -\omega_0^2 \mathbf{\hat{D}x}
\end{equation}
Ahol $\mathbf{\hat{D}}$ a diszkrét második differenciál operátora, amely a következőképpen néz ki:
\begin{equation}
    \mathbf{\hat{D}} = \begin{pmatrix}
    - 2 & 1 & 0 & 0 & \cdots & 1 \\
    1 & -2 & 1 & 0 & \cdots & 0 \\
    0 & 1 & -2 & 1 & \cdots & 0 \\
    0 & 0 & 1 & -2 & \cdots & 0 \\
    \vdots & \vdots & \vdots & \vdots & \ddots & \vdots \\
    1 & 0 & 0 & 0 & \cdots & -2
    \end{pmatrix}
\end{equation}
Keressük a megoldást periodikus hullámok formájában:
\begin{equation}
    x_n(t) = A_n(t) e^{i\omega t}
\end{equation}
Továbbá tudjuk, hogy az amplitúdó is periodikus, a csatolt rezgés alapján:
\begin{equation}
    A_n(t) = A e^{i q n a}
\end{equation}
Ahol $q$ a hullámszám, $a$ pedig az oszcillátorok közötti távolság. Így a teljes megoldás:
\begin{equation}
    x_n(t) = A e^{i (q n a + \omega t)}
\end{equation}
Ezt behelyettesítve a mozgásegyenletbe:
\begin{equation}
    -\omega^2 A e^{i (q n a + \omega t)} = -\omega_0^2 \left( 2A e^{i (q n a + \omega t)} - A e^{i (q (n-1) a + \omega t)} - A e^{i (q (n+1) a + \omega t)} \right)
\end{equation}
Egyszerűsítve:
\begin{equation}
    -\omega^2 = -\omega_0^2 \left( 2 - e^{-i q a} - e^{i q a} \right)
\end{equation}
\begin{equation}
    \omega^2 = \omega_0^2 \left( 2 - 2 cos(q a) \right)
\end{equation}
felhasználva a $1 - cos(x) = 2 sin^2\left(\frac{x}{2}\right)$ azonosságot:
\begin{equation}
    \omega^2 = 4 \omega_0^2 sin^2\left(\frac{q a}{2}\right)
\end{equation}
\begin{equation}
    \omega = 2 \omega_0 \left| sin\left(\frac{q a}{2}\right) \right|
\end{equation}
Ezt nevezik diszperziós relációnak, amely leírja a hullámok frekvenciáját a hullámszám függvényében egy végtelen lineáris láncban. Ez a reláció fontos információkat nyújt a rendszer dinamikájáról,
például a hullámok terjedési sebességéről és a rendszer rezgési módjairól. A diszperziós reláció alapján látható, hogy a frekvencia $\omega$ a hullámszám $q$ függvényében szinuszosan változik, és a maximális frekvencia
$\omega_{max} = 2\omega_0$ akkor érhető el, amikor $q a = \pi$. Ez azt jelenti, hogy a hullámok nem terjedhetnek végtelen sebességgel a lineáris láncban, hanem van egy felső határa frekvenciára, amelyet a rendszer fizikai tulajdonságai határoznak meg.
Emellett a reciprokrács tulajdonságai miatt a hullámszám $q$ értékei csak egy bizonyos tartományban értelmezettek, amelyet Brillouin-zónának neveznek. Az első Brillouin-zóna a $-\frac{\pi}{a} \leq q \leq \frac{\pi}{a}$ intervallumot fedi le, és ebben a tartományban
minden lehetséges hullámszám érték megtalálható. Ez  a zóna fontos szerepet játszik a kristályok és más periodikus struktúrák fizikai tulajdonságainak megértésében, mivel a hullámok terjedése és kölcsönhatása a kristályrács szerkezetével
határozza meg a rendszer viselkedését.
\begin{figure}[H]
    \centering
    \includegraphics[width=0.4\textwidth]{imgs/3-tetel/egyatomos_linearis_lanc.png}
    \caption{Példa egyatomos lineáris lánc diszperziós relációjára}
    \label{fig:linearis_lanc_diszperzio}
\end{figure}

\subsubsection*{Példa: kétatomos lineáris lánc}
A kétatomos lineáris lánc egy olyan modell, amelyben két különböző tömegű részecske (atom) váltakozva helyezkedik el egy egyenes vonal mentén, és csak a szomszédos atomokkal vannak csatolva. Ez a modell gyakran használatos a kristályrácsok vibrációinak
leírására, különösen olyan anyagok esetében, ahol két különböző típusú atom található, például sókristályokban (pl. NaCl). Vegyünk egy végtelen kétatomos lineáris láncot, ahol az egyik atom tömege $m_1$, a másik atom tömege $m_2$, és a csatolási rugóállandó $k$. Feltételezzünk
periodikus határfeltételeket, azaz az első és utolsó atom is csatolva van egymáshoz:
\begin{figure}[H]
    \centering
    \includegraphics[width=0.8\textwidth]{imgs/3-tetel/ketatomos_linearis_lanc.png}
    \caption{Végtelen kétatomos lineáris lánc}
    \label{fig:ketoatomos_linear_chain}
\end{figure}
A mozgásegyenletek az n-edik pár atom számára a következő lesz:
\begin{equation}
    m_1 \ddot{x}_n = -k (x_n - y_n) - k (x_n - y_{n-1})
\end{equation}
\begin{equation}
    m_2 \ddot{y}_n = -k (y_n - x_n) - k (y_n - x_{n+1})
\end{equation}
Ahol $x_n$ az n-edik $m_1$ tömegű atom helyzete, $y_n$ pedig az n-edik $m_2$ tömegű atom helyzete. Ezt rendezzük át:
\begin{equation}
    \ddot{x}_n = -\frac{k}{m_1} (2x_n - y_n - y_{n-1})
\end{equation}
\begin{equation}
    \ddot{y}_n = -\frac{k}{m_2} (2y_n - x_n - x_{n+1})
\end{equation}
Nevezzük át a paramétereket:
\begin{equation}
    \omega_1^2 = \frac{k}{m_1} \quad \quad \omega_2^2 = \frac{k}{m_2}
\end{equation}
Így az egyenletek a következőképpen néznek ki:
\begin{equation}
    \ddot{x}_n = -\omega_1^2 (2x_n - y_n - y_{n-1})
\end{equation}
\begin{equation}
    \ddot{y}_n = -\omega_2^2 (2y_n - x_n - x_{n+1})
\end{equation}
Keressük a megoldást periodikus hullámok formájában:
\begin{equation}
    x_n(t) = A e^{i (q n a + \omega t)}
\end{equation}
\begin{equation}
    y_n(t) = B e^{i (q n a + \omega t)}
\end{equation}
Ahol $q$ a hullámszám, $a$ pedig az atomok közötti távolság. Ezt behelyettesítve a mozgásegyenletekbe:
\begin{equation}
    -\omega^2 A e^{i (q n a + \omega t)} = -\omega_1^2 \left( 2A e^{i (q n a + \omega t)} - B e^{i (q n a + \omega t)} - B e^{i (q (n-1) a + \omega t)} \right)
\end{equation}
\begin{equation}
    -\omega^2 B e^{i (q n a + \omega t)} = -\omega_2^2 \left( 2B e^{i (q n a + \omega t)} - A e^{i (q n a + \omega t)} - A e^{i (q (n+1) a + \omega t)} \right)
\end{equation}
Egyszerűsítve:
\begin{equation}
    -\omega^2 A = -\omega_1^2 \left( 2A - B - B e^{-i q a} \right)
\end{equation}
\begin{equation}
    -\omega^2 B = -\omega_2^2 \left( 2B - A - A e^{i q a} \right)
\end{equation}
Ezeket az egyenleteket mátrix formában is felírhatjuk:
\begin{equation}
    \begin{pmatrix}
    \omega^2 - 2\omega_1^2 & \omega_1^2 (1 + e^{-i q a}) \\
    \omega_2^2 (1 + e^{i q a}) & \omega^2 - 2\omega_2^2
    \end{pmatrix}
    \begin{pmatrix}
    A \\
    B
    \end{pmatrix}
    = 0
\end{equation}
Ahol a mátrix együtthatóinak determinánsának nullának kell lennie, hogy ne triviális megoldást kapjunk:
\begin{equation}
    \left| \begin{matrix}
    \omega^2 - 2\omega_1^2 & \omega_1^2 (1 + e^{-i q a}) \\
    \omega_2^2 (1 + e^{i q a}) & \omega^2 - 2\omega_2^2
    \end{matrix} \right| = 0
\end{equation}
Kiszámolva a determinánst:
\begin{equation}
    (\omega^2 - 2\omega_1^2)(\omega^2 - 2\omega_2^2) - \omega_1^2 \omega_2^2 (1 + e^{-i q a})(1 + e^{i q a}) = 0
\end{equation}
\begin{equation}
    (\omega^2 - 2\omega_1^2)(\omega^2 - 2\omega_2^2) - \omega_1^2 \omega_2^2 (2 + 2 cos(q a)) = 0
\end{equation}
\begin{equation}
    \omega^4 - 2(\omega_1^2 + \omega_2^2) \omega^2 + 4\omega_1^2 \omega_2^2 - 2\omega_1^2 \omega_2^2 (1 + cos(q a)) = 0
\end{equation}
\begin{equation}
    \omega^4 - 2(\omega_1^2 + \omega_2^2) \omega^2 + 2\omega_1^2 \omega_2^2 (1 - cos(q a)) = 0
\end{equation}
Ez egy másodfokú egyenlet $\omega^2$-re, amelynek megoldása a következő:
\begin{equation}
    \omega^2 = (\omega_1^2 + \omega_2^2) \pm \sqrt{(\omega_1^2 + \omega_2^2)^2 - 2\omega_1^2 \omega_2^2 (1 - cos(q a))}
\end{equation}
\begin{equation}
    \omega^2 = (\omega_1^2 + \omega_2^2) \pm \sqrt{(\omega_1^2 - \omega_2^2)^2 + 4\omega_1^2 \omega_2^2 sin^2\left(\frac{q a}{2}\right)}
\end{equation}
Ez a diszperziós reláció a kétatomos lineáris lánc számára. Két különböző ágat ad meg a frekvenciára, amelyeket akusztikus és optikai ágaknak neveznek:
\begin{itemize}
    \item Az akusztikus ág ($-$ jel) alacsony frekvenciákat ír le, ahol a két atom mozgása közel azonos fázisban történik. Ez az ág a hanghullámok terjedését modellezi a kristályban.
    \item Az optikai ág ($+$ jel) magas frekvenciákat ír le, ahol a két atom mozgása ellenfázisban történik. Ez az ág a fényhullámok kölcsönhatását modellezi a kristályban.
\end{itemize}
\begin{figure}[H]
    \centering
    \includegraphics[width=0.4\textwidth]{imgs/3-tetel/ketatomos_linearis_lanc_diszperzio.png}
    \caption{Példa kétatomos lineáris lánc diszperziós relációjára}
    \label{fig:ketatomos_linearis_lanc_diszperzio}
\end{figure}

\subsection{!Kepler-Probléma, bolygómozgás}
A Kepler-probléma a klasszikus mechanikában egy olyan probléma, amely a két test közötti gravitációs kölcsönhatást vizsgálja, például egy bolygó és egy csillag között. A probléma célja meghatározni a bolygó pályáját és mozgását a csillag körül,
figyelembe véve a gravitációs erőt. A Kepler-probléma megoldása a Kepler-törvényeken alapul, amelyek Johannes Kepler német csillagász által a 17. században megfogalmazott három törvény:
\begin{itemize}
    \item \textbf{Első törvény (pályaalak törvénye)}: A bolygók ellipszis alakú pályán keringenek a Nap körül, amelynek egyik gyújtópontjában a Nap helyezkedik el.
    \item \textbf{Második törvény (területi sebesség törvénye)}: A bolygó és a Nap közötti képzeletbeli vonal egyenlő területeket súrol egyenlő idő alatt. Ez azt jelenti, hogy a bolygó gyorsabban mozog, amikor közelebb van a Naphoz, és lassabban, amikor távolabb van tőle.
    \item \textbf{Harmadik törvény (harmonikus törvény)}: A bolygók keringési idejének négyzete arányos a pálya fél nagytengelyének köbével. Matematikailag ez a következőképpen írható fel: $T^2 \propto a^3$, ahol $T$ a keringési idő, és $a$ a pálya fél nagytengelye.
\end{itemize}
A Kepler-probléma megoldása a Newton-féle gravitációs törvényen alapul, amely szerint a két test közötti gravitációs erő arányos a tömegeik szorzatával, és fordítottan arányos a távolságuk négyzetével:
\begin{equation}
    F = G \frac{M m}{r^2}
\end{equation}
Ahol $F$ a gravitációs erő, $G$ a gravitációs állandó, $M$ és $m$ a két test tömege, és $r$ a távolságuk. 

\subsubsection{Kepler-probléma megoldása}
A Kepler-problémára felirhatjuk a kinetikus és potenciális energiát:
\begin{equation}
    K = \frac{1}{2} m v^2 = \frac{1}{2} m (\dot{r}^2 + r^2 \dot{\theta}^2)
\end{equation}
\begin{equation}
    U = - G \frac{M m}{r}
\end{equation}
Ahol $r$ a bolygó és a csillag közötti távolság, $\theta$ a bolygó szöge a csillaghoz képest, és a pontok idő szerinti deriváltjai. A Lagrange-függvény a következő lesz:
\begin{equation}
    \mathcal{L} = K - U = \frac{1}{2} m (\dot{r}^2 + r^2 \dot{\theta}^2) + G \frac{M m}{r}
\end{equation}
Nézzük meg a $\theta$ koordinátára felírt Euler-Lagrange egyenletet:
\begin{equation}
    \frac{d}{dt} \left( \frac{\partial \mathcal{L}}{\partial \dot{\theta}} \right) - \frac{\partial \mathcal{L}}{\partial \theta} = 0
\end{equation}
Mivel a Lagrange-függvény nem függ $\theta$-tól, a második tag nulla lesz:
\begin{equation}
    \frac{d}{dt} \left( m r^2 \dot{\theta} \right) = 0
\end{equation}
Ez azt jelenti, hogy az $m r^2 \dot{\theta}$ mennyiség állandó, amely az impulzusmomentum $L$:
\begin{equation}
    L = m r^2 \dot{\theta} = \text{állandó}
\end{equation}
A Hamilton-függvény a következő lesz:
\begin{equation}
    \mathcal{H} = K + U = \frac{1}{2} m (\dot{r}^2 + r^2 \dot{\theta}^2) - G \frac{M m}{r}
\end{equation}
Mivel a Hamilton-függvény nem függ kifejezetten az időtől, az energia $E$ is állandó:
\begin{equation}
    E = \mathcal{H} = \text{állandó}
\end{equation}
Most fejezzük ki $\dot{\theta}$-t az impulzusmomentum segítségével:
\begin{equation}
    \dot{\theta} = \frac{L}{m r^2}
\end{equation}
Nézzük meg a $r$ koordinátára felírt Euler-Lagrange egyenletet:
\begin{equation}
    \frac{d}{dt} \left( \frac{\partial \mathcal{L}}{\partial \dot{r}} \right) - \frac{\partial \mathcal{L}}{\partial r} = 0
\end{equation}
\begin{equation}
    m \ddot{r} - m r \dot{\theta}^2 + G \frac{M m}{r^2} = 0
\end{equation}
Helyettesítsük be $\dot{\theta}$-t:
\begin{equation}
    m \ddot{r} - m r \left( \frac{L}{m r^2} \right)^2 + G \frac{M m}{r^2} = 0
\end{equation}
\begin{equation}
    \ddot{r} = \frac{L^2}{m^2 r^3} - G \frac{M}{r^2}
\end{equation}
Most fejezzük ki az energiát $E$-t:
\begin{equation}
    E = \frac{1}{2} m \dot{r}^2 + \frac{L^2}{2 m r^2} - G \frac{M m}{r}
\end{equation}
Rendezzük át:
\begin{equation}
    \frac{1}{2} m \dot{r}^2 = E - \frac{L^2}{2 m r^2} + G \frac{M m}{r}
\end{equation}
\begin{equation}
    \dot{r}^2 = \frac{2E}{m} - \frac{L^2}{m^2 r^2} + 2 G \frac{M}{r}
\end{equation}
Most fejezzük ki $\dot{r}$-t:
\begin{equation}
    \dot{r} = \frac{dr}{dt} = \frac{dr}{d\theta} \cdot \frac{d\theta}{dt} = \frac{dr}{d\theta} \cdot \frac{L}{m r^2}
\end{equation}
\begin{equation}
    \left( \frac{dr}{d\theta} \cdot \frac{L}{m r^2} \right)^2 = \frac{2E}{m} - \frac{L^2}{m^2 r^2} + 2 G \frac{M}{r}
\end{equation}
\begin{equation}
    \left( \frac{dr}{d\theta} \right)^2 = \frac{m^2 r^4}{L^2} \left( \frac{2E}{m} - \frac{L^2}{m^2 r^2} + 2 G \frac{M}{r} \right)
\end{equation}
Most helyettesítsük be $u = \frac{1}{r}$-t, így $r = \frac{1}{u}$ és $\frac{dr}{d\theta} = -\frac{1}{u^2} \frac{du}{d\theta}$:
\begin{equation}
    \left( -\frac{1}{u^2} \frac{du}{d\theta} \right)^2 = \frac{m^2 \frac{1}{u^4}}{L^2} \left( \frac{2E}{m} - \frac{L^2 u^2}{m^2} + 2 G M u \right)
\end{equation}
\begin{equation}
    \left( \frac{du}{d\theta} \right)^2 = \frac{m^2}{L^2} \left( \frac{2E}{m} - \frac{L^2 u^2}{m^2} + 2 G M u \right)
\end{equation}
\begin{equation}
    \left( \frac{du}{d\theta} \right)^2 = \frac{2 m E}{L^2} - u^2 + \frac{2 G M m}{L^2} u
\end{equation}
Most rendezzük át:
\begin{equation}
    \left( \frac{du}{d\theta} \right)^2 + u^2 - \frac{2 G M m}{L^2} u - \frac{2 m E}{L^2} = 0
\end{equation}
Most vegyük a deriváltat $\theta$ szerint:
\begin{equation}
    2 \frac{du}{d\theta} \frac{d^2 u}{d\theta^2} + 2 u \frac{du}{d\theta} - \frac{2 G M m}{L^2} \frac{du}{d\theta} = 0
\end{equation}
\begin{equation}
    \frac{d^2 u}{d\theta^2} + u - \frac{G M m}{L^2} = 0
\end{equation}
Ez egy másodrendű lineáris differenciálegyenlet, amelynek megoldása a következő:
\begin{equation}
    u(\theta) = \frac{G M m}{L^2} + A cos(\theta - \theta_0)
\end{equation}
Ahol $A$ és $\theta_0$ integrálási állandók, amelyek a kezdeti feltételektől függenek. Visszaalakítva $r$-re:
\begin{equation}
    r(\theta) = \frac{1}{\frac{G M m}{L^2} + A cos(\theta - \theta_0)}
\end{equation}
Ez az egyenlet egy ellipszis pályát ír le, amelynek egyik gyújtópontjában a csillag helyezkedik el. Az ellipszis paraméterei az energia és az impulzusmomentum függvényében határozhatók meg. Ez a megoldás megfelel Kepler első törvényének,
amely szerint a bolygók ellipszis alakú pályán keringenek a Nap körül.

\subsection{Kvantummechanikai problémák}
\subsubsection{Potenciálvölgy}
A potenciálvölgy általánosságban egy olyan fizikai rendszer, amelyben egy részecske egy adott potenciálfüggvény hatása alatt mozog, de a potenciálfüggvény egy adott tartományban a részecske energiája fölé emelkedik.
Ezt a régiót klasszikus fizikában potenciálgátként értelmezzük, amelyet a részecske nem képes átlépni, ha az energiája kisebb, mint a potenciálgát magassága. Azonban a kvantummechanikában a részecske hullámtermészete miatt létezik egy jelenség,
amelyet alagúthatásnak neveznek, amely lehetővé teszi a részecske számára, hogy áthaladjon a potenciálgáton még akkor is, ha az energiája kisebb, mint a gát magassága. Ez a jelenség a kvantummechanika egyik alapvető tulajdonsága, és számos fizikai jelenségben megfigyelhető,
például a radioaktív bomlásban, a szilárdtestfizikában és a kémiai reakciókban.
\begin{figure}[H]
    \centering
    \includegraphics[width=0.6\textwidth]{imgs/3-tetel/potencialvolgy.png}
    \caption{Potenciálvölgy Egy potenciálfüggvény esetében}
    \label{fig:potencialvolgy}
\end{figure}
\subsubsection*{Végtelen potenciálvölgy}
Nézzük meg először a végtelen potenciálvölgy esetét, ahol a potenciál-gát magasságát "effektíve" végtelennek tekintjük. Ez azt jelenti, hogy a részecske nem képes áthaladni a gátakon, és csak a potenciálvölgy belsejében mozoghat.
Bár ebben az esetben nincs alagúthatás, de a részecske hullámtermészete miatt így is váratlan eredményre fogunk jutni.
\begin{figure}[H]
    \centering
    \includegraphics[width=0.6\textwidth]{imgs/3-tetel/vegtelenpotencialvolgy.png}
    \caption{Végtelen potenciálvölgy}
    \label{fig:vegtelen_potencialvolgy}
\end{figure}
Ebben az esetben a potenciálfüggvény-t könnyen definiálhatjuk a következőképpen:
\begin{equation}
    V(x) = \begin{cases}
    0 & \text{ha } 0 < x < a \\
    \infty & \text{egyébként}
    \end{cases}
\end{equation}
Ahol $a$ a potenciálvölgy szélessége. A részecske mozgását az időfüggetlen Schrödinger-egyenlet írja le:
\begin{equation}
    -\frac{\hbar^2}{2m} \frac{d^2 \psi(x)}{dx^2} + V(x) \psi(x) = E \psi(x)
\end{equation}
Ahol $\psi(x)$ a részecske hullámfüggvénye, $E$ az energia, $m$ a részecske tömege, és $\hbar$ a redukált Planck-állandó. A falakon kívül $\psi (x) = 0$-nek kell lennie, tehát a részecskét nulla valószínűséggel találjuk meg ott.
A potenciálvölgy belsejében ($0 < x < a$) a Schrödinger-egyenlet a következőképpen egyszerűsödik:
\begin{equation}
    -\frac{\hbar^2}{2m} \frac{d^2 \psi(x)}{dx^2} = E \psi(x)
\end{equation}
\begin{equation}
    \frac{d^2 \psi(x)}{dx^2} + \frac{2mE}{\hbar^2} \psi(x) = 0
\end{equation}
Nevezzük át a paramétert:
\begin{equation}
    k \equiv \sqrt{\frac{2mE}{\hbar^2}}
\end{equation}
Itt feltételezzük, hogy az energia pozitív ($E > 0$), így $k$ valós szám lesz. Így az egyenlet a következőképpen néz ki:
\begin{equation}
    \frac{d^2 \psi(x)}{dx^2} + k^2 \psi(x) = 0
\end{equation}
A megoldás általános alakja:
\begin{equation}
    \psi(x) = A sin(kx) + B cos(kx)
\end{equation}
Ahol $A$ és $B$ integrálási állandók. Az állandókat a határfeltételek alapján tudjuk meghatározni. Általában azt feltételezzük, hogy a hullámfüggvénynek és a deriváltjának is folytonosnak kell lennie a határokon,
de a gát végtelensége miatt, most csak a hullámfüggvénytől követeljük meg a folytonosságot. Tehát a következő határfeltételeket alkalmazzuk:
\begin{equation}
    \psi(0) = \psi(a) = 0
\end{equation}
Az első határfeltétel alkalmazásával:
\begin{equation}
    \psi(0) = A sin(0) + B cos(0) = B = 0
\end{equation}
Tehát a hullámfüggvény egyszerűsödik:
\begin{equation}
    \psi(x) = A sin(kx)
\end{equation}
A második határfeltétel alkalmazásával:
\begin{equation}
    \psi(x) = A sin(kx) = 0
\end{equation}
Ezt teljesíthetné a $A = 0$, de ekkor csak annyit kaptunk, hogy $\psi (x) = 0$, ami triviális megoldás. teljesíthetné a $k = 0$ megoldás is, de ez feltételezné a $E = 0$-t, ami újból egy nem túl érdekes megoldás.
 A másik opció, hogy az x diszkrét értékeket vesz fel úgy, hogy a szinusz nulla legyen:
\begin{equation}
    k a = 0, \pm \pi, \pm 2\pi, \pm 3\pi, \ldots
\end{equation}
Itt az előjel nem számít, mivel a szinusz függvény páros. Tehát a diszkrét értékek:
\begin{equation}
    k_n = \frac{n \pi}{a} \quad \quad n = 1, 2, 3, \ldots
\end{equation}
Érdekesség, hogy a határfeltétel $x = a$ nem határozza meg az $A$ állandót, hanem a $k$-t adja meg. Ebből az eredményből pedig következik, hogy az energia csak bizonyos diszkrét értékeket vehet fel:
\begin{equation}
    E_n = \frac{\hbar^2 k_n^2}{2m} = \frac{\hbar^2 \pi^2 n^2}{2m a^2} \quad \quad n = 1, 2, 3, \ldots
\end{equation}
Ahhoz, hogy megkapjuk az $A$ állandót, normalizálnunk kell a hullámfüggvényt:
\begin{equation}
    \int_0^a |\psi_n(x)|^2 dx = 1
\end{equation}
\begin{equation}
    |A|^2 \int_0^a sin^2\left(\frac{n \pi x}{a}\right) dx = 1
\end{equation}
\begin{equation}
    |A|^2 \frac{a}{2} = 1
\end{equation}
\begin{equation}
    |A| = \sqrt{\frac{2}{a}}
\end{equation}
Így a normalizált hullámfüggvények:
\begin{equation}
    \psi_n(x) = \sqrt{\frac{2}{a}} sin\left(\frac{n \pi x}{a}\right) \quad \quad n = 1, 2, 3, \ldots
\end{equation}
Ezzel technikailag csak az $A$ nagyságát határoztuk meg, de mivel a hullámfüggvény csak a valószínűség sűrűséget határozza meg, azaz $|\psi (x)|^2$, ezért az $A$ előjelének nincs fizikai jelentősége.
Összefoglalva, a végtelen potenciálvölgy esetében a részecske energiája diszkrét értékeket vehet fel, amelyeket kvantált energiaszinteknek nevezünk. A hullámfüggvények pedig a potenciálvölgy belsejében oszcilláló
függvények, amelyek a határfeltételek miatt nullára mennek a gátakon kívül. Az $n = 1$ eset a legkisebb energiájú állapot, amelyet alapállapotnak nevezünk, míg az $n > 1$ esetek gerjesztett állapotoknak felelnek meg.
Ezeknek a $\psi_n (x)$ függvényeknek van egy pár érdekes tulajdonsága is:
\begin{itemize}
    \item Az állapotok ortogonálisak egymásra, azaz ha $n \neq m$, akkor:
    \begin{equation}
        \int_0^a \psi_n(x)^* \psi_m(x) dx = 0
    \end{equation}
    \item Ahogy megyünk feljebb az energiaszinteken (n növelése), a hullámfüggvények egyre több csomópontot tartalmaznak a potenciálvölgy belsejében. Az $n$-edik állapotban pontosan $(n-1)$ csomópont van. $n = 1$ esetében
    pedig nulla, mert a végpontok nem számítanak csomópontnak.
    \item a hullámfüggvények váltogatnak páros és páratlan függvények között, ahogy növeljük $n$ értékét. Az $n$ páratlan esetekben a hullámfüggvény páros, míg az $n$ páros esetekben páratlan függvény.
\end{itemize}
\begin{figure}[H]
    \centering
    \includegraphics[width=0.6\textwidth]{imgs/3-tetel/vegtelen_potencialvolgy_hullamfuggvenyek.png}
    \caption{Végtelen potenciálvölgy hullámfüggvényei az első néhány energiaszinten}
    \label{fig:vegtelen_potencialvolgy_hullamfuggvenyek}
\end{figure}

\subsubsection*{Véges potenciálvölgy}
A véges potenciálvölgy esetében a potenciál-gát magassága véges értékű, ami klasszikus fizikában nem jelentene nagy különbséget a végtelen potenciálvölgyhez képest, de a részecskék hullámtermészete miatt
érdekes jelenségeket tud produkálni a rendszer.

Vegyünk egy potenciálvölgyet amit a következőképpen definiálunk:
\begin{equation}
    V(x) = \begin{cases}
    -V_0 & - a < x < a \\
    0 & |x| \geq a
    \end{cases}
\end{equation}
\begin{figure}[H]
    \centering
    \includegraphics[width=0.4\textwidth]{imgs/3-tetel/vegespotencialvolgy.png}
    \caption{Véges potenciálvölgy}
    \label{fig:veges_potencialvolgy}
\end{figure}

Itt a potenciálfüggvény gátjának magasságát választjuk meg a nulla szintnek, és alatta $-V_0$-val bukik be a függvény $a$ és $-a$ között. Itt két lehetőséget határozhatunk meg:
\begin{itemize}
    \item Az energia $E$ kisebb, mint nulla ($E < 0$), ami azt jelenti, hogy a részecske kötött állapotban van a potenciálvölgyben.
    \item Az energia $E$ nagyobb, mint nulla ($E > 0$), ezt nevezik "szóródó állapotnak", ahol a részecske képes kilépni a potenciálvölgyből.
\end{itemize}
A részecske mozgását az időfüggetlen Schrödinger-egyenlet írja le:
\begin{equation}
    -\frac{\hbar^2}{2m} \frac{d^2 \psi(x)}{dx^2} + V(x) \psi(x) = E \psi(x)
\end{equation}
Ahol $\psi(x)$ a részecske hullámfüggvénye, $E$ az energia, $m$ a részecske tömege, és $\hbar$ a redukált Planck-állandó. A potenciálvölgy három régióra osztható:
\begin{itemize}
    \item Régió I: $x < -a$, ahol $V(x) = 0$
    \item Régió II: $-a < x < a$, ahol $V(x) = -V_0$
    \item Régió III: $x > a$, ahol $V(x) = 0$
\end{itemize}
Nézzük meg először a \textbf{kötött állapotot} ($E < 0$). A függvény az első régióban a következő egyenletet elégíti ki:
\begin{equation}
    \frac{d^2 \psi(x)}{dx^2} + \frac{2mE}{\hbar^2} \psi(x) = 0
\end{equation}
A megoldás a következő formát ölti:
\begin{equation}
    \psi(x) = A e^{- kx} + B e^{kx}
\end{equation}
Ahol $k = \sqrt{\frac{2mE}{\hbar^2}}$. Itt is beláthatjuk, hogy az $A$ együtthatós tag elszáll a végtelenben, így $A = 0$. Tehát az első régióban a hullámfüggvény:
\begin{equation}
    \psi_I(x) = B e^{kx}, \quad x < -a
\end{equation}
A harmadik régióban hasonlóan:
\begin{equation}
    \psi_{III}(x) = F e^{kx}, \quad x > a
\end{equation}
A második régióban a Schrödinger-egyenlet a következőképpen néz ki:
\begin{equation}
   -\frac{\hbar^2}{2m} \frac{d^2 \psi(x)}{dx^2} - V_0 \psi(x) = E \psi(x)
\end{equation}
\begin{equation}
    \frac{d^2 \psi(x)}{dx^2} + \frac{2m(E + V_0)}{\hbar^2} \psi(x) = 0
\end{equation}
Nevezzük át a paramétert:
\begin{equation}
    l \equiv \sqrt{\frac{2m(E + V_0)}{\hbar^2}}
\end{equation}
Itt megállapítjuk, hogy $E + V_0 > 0$, és bár $E < 0$, de nagyobbnak kell lennie $- V_0$-nál mert ha $E < V_{min}$ akkor a hullámfügvény nem lenne normálható, ami megszegi a hullámfüggvény valószínűség értelmezését. Így az egyenlet a következőképpen néz ki:
\begin{equation}
    \frac{d^2 \psi(x)}{dx^2} = - l^2 \psi(x)
\end{equation}
A megoldás általános alakja:
\begin{equation}
    \psi_{II}(x) = C sin(l x) + D cos(l x), \quad -a < x < a
\end{equation}
Itt élhetünk egy egyszerűsítéssel, mégpedig, hogy feltételezhetjük, hogy a hullámfüggvény vagy páros vagy páratlan lesz. Ezért, ha a potenciálvölgy szimmetrikus (és jelenleg úgy határoztuk meg),
akkor elég az egyik oldalra (pl. $+ a$-nál) megoldani as határfeltételeket, és abból automatikusan következik, hogy a másik oldalon $\psi(-x) = \pm \psi(x)$ lesz a megoldás (az előjel a paritás függvénye).
Nézzük meg a páros esetet, tehát vehetjük, hogy $C = 0$, így a teljes hullámfüggvény:
\begin{equation}
\psi(x) = \begin{cases}
    F e^{kx} & x > a \\
    D cos(l x) & 0 < x < a \\
    \psi_I(-x) & x < 0 
    \end{cases}
\end{equation}
Ezeknek az egyenleteknek a következő folytonossági határfeltételeket kell teljesíteniük az $x = \pm a$ pontokban:
\begin{equation}
    \psi_{II}(a) = \psi_{III}(a)
\end{equation}
\begin{equation}
    \psi_{II}'(a) = \psi_{III}'(a)
\end{equation}
Itt az elöbb tárgyalt párosság miatt elég, ha csak az $x = a$ pontban nézzük meg a feltételeket:
\begin{equation}
    D cos(l a) = F e^{- k a}
\end{equation}
\begin{equation}
    - D l sin(l a) = - k F e^{- k a}
\end{equation}
Ebből az $F$-et kifejezve és behelyettesítve a második egyenletbe:
\begin{equation}
    - D l sin(l a) = - k D cos(l a)
\end{equation}
\begin{equation}
    l \cdot tan(l a) = k
\end{equation}
Ez a formula mutatja a megengedett energiákat, hiszen $l$ és $k$ is az energia függvénye. Vezessünk be pár új változót a könnyebb ábrázolás érdekében:
\begin{equation}
    z \equiv l a = a \sqrt{\frac{2m(E + V_0)}{\hbar^2}}
\end{equation}
\begin{equation}
    z_0 \equiv a \sqrt{\frac{2m V_0}{\hbar^2}}
\end{equation}
Emellett a $k$ és az $l$ definíciója miatt a következő összefüggés is fennáll:
\begin{equation}
    k^2 + l^2 = \frac{2m V_0}{\hbar^2}
\end{equation}
\begin{equation}
    k^2 + \frac{z^2}{a^2} = \frac{z_0^2}{a^2}
\end{equation}
\begin{equation}
    ka = \sqrt{z_0^2 - z^2}
\end{equation}
Tehát a megengedett energiákra vonatkozó egyenlet a következő alakot ölti:
\begin{equation}
    z \cdot tan(z) = \sqrt{z_0^2 - z^2}
\end{equation}
Ez egy transzendens egyenlet, ahol $z_0$ a potenciálvölgy mélységét és szélességét határozza meg. Az egyenlet megoldásai a megengedett energiák diszkrét értékeit adják meg a kötött állapotok számára. 
Ezt a problémát csak numerikusan vagy grafikusan lehet megoldani, úgyhogy ábrázoljuk a bal és jobb oldalt külön-külön, és keressük meg azokat a pontokat, ahol a két görbe metszi egymást.
\begin{figure}[H]
    \centering
    \includegraphics[width=0.6\textwidth]{imgs/3-tetel/veges_potencialvolgy_megengedett_energiak.png}
    \caption{Véges potenciálvölgy megengedett energiák grafikus megoldása $z_0 = 8$ esetén}
\end{figure}
A megoldások közül kettő van amelyik különösen érdekes:
\begin{itemize}
    \item A potenciálvölgy \textbf{széles és mély}: Ha a potenciálvölgy nagy, akkor a metszéspontok egy kicsivel $z_n = \frac{n \pi}{2}$ fölött lesznek, ahol $n = 1. 3. 5 \ldots$ (a megoldás "másik fele" a páratlan függvény megoldásában van jelen).
    Innen belátható, hogy:
    \begin{equation}
        E_n + V_0 \approx \frac{\hbar^2 \pi^2 n^2}{2 m (2a)^2} \quad \quad n = 1, 3, 5, \ldots
    \end{equation}
    A völgy alja fölött, ami pont az mint a végtelen potenciálvölgy esetében, csak $2a$-val, vagyis pont a fele annak ($n$ páratlansága miatt csak az egyik felét kapjuk meg). Tehát a véges potenciálvölgy megoldása közeledik
    közeledik a végtelen potenciálvölgy megoldásához, ha $V_0 \to \infty$, de minden véges $V_0$ esetén véges számú kötött állapot van.
    \item A potenciálvölgy \textbf{vékony és sekély}: Ha a potenciálvölgy kicsi, akkor kevesebb és kevesebb kötött állapot lesz, míg el nem érjük a $z_0 < \frac{\pi}{2}$ értéket ahol csak egy marad. Viszont akármennyire kicsi is a potenciálvölgy,
    akkor is lesz legalább egy kötött állapot, tehát a részecske mindig képes lesz "bennragadni" a potenciálvölgyben.
\end{itemize}
A páratlan megoldáshoz nagyon hasonló eredményt kapunk:
\begin{equation}
    z \cdot cot(z) = - \sqrt{z_0^2 - z^2}
\end{equation}
Ahol a megengedett energiákra vonatkozó egyenlet a következő alakot ölti:
\begin{equation}
    E_n + V_0 \approx \frac{\hbar^2 \pi^2 n^2}{2 m (2a)^2} \quad \quad n = 2, 4, 6, \ldots
\end{equation}
Már csak annyi van hátra, hogy normalizáljuk a hullámfüggvényt, és meghatározzuk az együtthatókat. Ezt a következő integrállal tehetjük meg:
\begin{equation}
    \int_{-\infty}^{\infty} |\psi(x)|^2 dx = 1
\end{equation}
\begin{equation}
    \int_{-\infty}^{-a} |\psi_I(x)|^2 dx + \int_{-a}^{a} |\psi_{II}(x)|^2 dx + \int_{a}^{\infty} |\psi_{III}(x)|^2 dx = 1
\end{equation}
\begin{equation}
    |B|^2 \int_{-\infty}^{-a} e^{2 k x} dx + |D|^2 \int_{-a}^{a} cos^2(l x) dx + |F|^2 \int_{a}^{\infty} e^{-2 k x} dx = 1
\end{equation}
Itt $B$-t nem kell külön meghatározni, hiszen a paritás miatt $B = \pm F$ lesz, tehát a probléma tovább egyszerűsödik:
\begin{equation}
    |F|^2 \left( \int_{-\infty}^{-a} e^{2 k x} dx + \int_{a}^{\infty} e^{-2 k x} dx \right) + |D|^2 \int_{-a}^{a} cos^2(l x) dx = 1
\end{equation}
Ezek az integrálok könnyen kiszámíthatók:
\begin{equation}
    |F|^2 \left( \frac{e^{-2 k a}}{2 k} + \frac{e^{-2 k a}}{2 k} \right) + |D|^2 \left( a + \frac{sin(2 l a)}{2 l} \right) = 1
\end{equation}
\begin{equation}
    |F|^2 \frac{e^{-2 k a}}{k} + |D|^2 \left( a + \frac{sin(2 l a)}{2 l} \right) = 1
\end{equation}
A határfeltételekből pedig kifejezhetjük $F$-et $D$-ben:
\begin{equation}
    D cos(l a) = F e^{- k a}
\end{equation}
\begin{equation}
    F = D cos(l a) e^{k a}
\end{equation}
Behelyettesítve a normalizációs feltételbe:
\begin{equation}
    |D|^2 cos^2(l a) e^{2 k a} \frac{e^{-2 k a}}{k} + |D|^2 \left( a + \frac{sin(2 l a)}{2 l} \right) = 1
\end{equation}
\begin{equation}
    |D|^2 \left( \frac{cos^2(l a)}{k} + a + \frac{sin(2 l a)}{2 l} \right) = 1
\end{equation}
\begin{equation}
    |D| = \sqrt{\frac{1}{\frac{cos^2(l a)}{k} + a + \frac{sin(2 l a)}{2 l}}}
\end{equation}
\begin{equation}
    |F| = \sqrt{\frac{cos^2(l a) e^{2 k a}}{k \left( \frac{cos^2(l a)}{k} + a + \frac{sin(2 l a)}{2 l} \right)}}
\end{equation}
Ezzel megkaptuk a kötött állapotok hullámfüggvényeit és megengedett energiáit a véges potenciálvölgy esetében.
\newline
\newline
Most nézzük meg a \textbf{szóródás állapotot} ($E > 0$). Ebben az esetben is hasonló megoldásokat kapunk a három régióra:
\begin{equation}
\psi(x) = \begin{cases}
    A e^{i k x} + B e^{- i k x} & x < -a \\
    C sin(l x) + D cos(l x) & -a < x < a \\
    F e^{i k x} + G e^{- i k x} & x > a
    \end{cases}
\end{equation}
Ahol most is $k = \sqrt{\frac{2mE}{\hbar^2}}$ és $l = \sqrt{\frac{2m(E + V_0)}{\hbar^2}}$. A különbség annyi, hogy most
a falon kívül is oszcilláló megoldásokat kapunk, hiszen a részecske képes kilépni a potenciálvölgyből, így a hullámfüggvény nem megy nullára a falakon kívül.
Emellett feltételezzük, hogy a részecskenyaláb csak balról jön be, így $G = 0$. Ekkor a $x < -a$ régióban kapott hullámfüggvény
a bejövő és visszaverődő hullámokat írja le, míg a $x > a$ régióban kapott hullámfüggvény a kilépő hullámot írja le.
Tehát a hullámfüggvény most a következő alakot ölti:
\begin{equation}
\psi(x) = \begin{cases}
    A e^{i k x} + B e^{- i k x} & x < -a \\
    C sin(l x) + D cos(l x) & -a < x < a \\
    F e^{i k x} & x > a
    \end{cases}
\end{equation}
A határfeltételek alkalmazásával az $x = \pm a$ pontokban:
\begin{equation}
    \psi_{II}(-a) = \psi_{I}(-a)
\end{equation}
\begin{equation}
    \psi_{II}'(-a) = \psi_{I}'(-a)
\end{equation}
\begin{equation}
    \psi_{II}(a) = \psi_{III}(a)
\end{equation}
\begin{equation}
    \psi_{II}'(a) = \psi_{III}'(a)
\end{equation}
Tehát a folytonosság a $-a$ régióban:
\begin{equation}
    - C sin(l a) + D cos(l a) = A e^{- i k a} + B e^{i k a}
\end{equation}
A trigonometrikus függvények szimmetriája miatt lehet a negatív előjelet kivinni a szinusz elé, és eltüntetni a coszinuszban. A derivált folytonossága:
\begin{equation}
    ik \left[Ae^{- i k x} + B e^{i k x}\right] = l \left[C cos(l a) + D sin(l a)\right]
\end{equation}
A folytonosság az $+a$ régióban:
\begin{equation}
    F e^{i k a} = C sin(l a) + D cos(l a)
\end{equation}
\begin{equation}
    ik F e^{i k a} = l \left[C cos(l a) + D sin(l a)\right]
\end{equation}
Ezek segítségével eltüntethetjük $C$ és $D$-t, és kifejezhetjük $B$-t és $F$-et $A$-ban. Ebből pedig meghatározhatjuk a visszaverődési és átvitelési együtthatókat:
\begin{equation}
    B = i \frac{sin(2 l a)}{2 k l} \left(l^2 - k^2\right) F
\end{equation}
\begin{equation}
    F = \frac{e^{- i k a} A}{cos(2 l a) - i \frac{k^2 + l^2}{2 k l} sin(2 l a)}
\end{equation}
Ebből meghatározhatjuk a $T = \frac{|F|^2}{|A|^2}$ átviteli együtthatót és a $R = \frac{|B|^2}{|A|^2}$ visszaverődési együtthatót:
\begin{equation}
    T^{-1} = 1 + \frac{V_0^2}{4 E (E + V_0)} sin^2\left( \frac{2 a}{\hbar} \sqrt{2 m (E + V_0)} \right)
\end{equation}
Vegyük észre, hogy az átviteli együttható $T$ értéke $0$ és $1$ között van, hiszen a szinusz függvény négyzete sosem negatív.
és ha $T = 1$, vagyis a részecske teljesen áthalad a potenciálvölgyön anélkül, hogy visszaverődne, akkor a következő feltételnek kell teljesülnie:
\begin{equation}
    sin\left( \frac{2 a}{\hbar} \sqrt{2 m (E + V_0)} \right) = 0
\end{equation}
\begin{equation}
    \frac{2 a}{\hbar} \sqrt{2 m (E + V_0)} = n \pi \quad \quad n = 0, 1, 2, \ldots
\end{equation}
Ebből az energia értékek:
\begin{equation}
    E_n + V_0 = \frac{n^2 \pi^2 \hbar^2}{8 m a^2} \quad \quad n = 0, 1, 2, \ldots
\end{equation}
Ezek az energiák pontosan megegyeznek a végtelen potenciálvölgy esetében kapott energiákkal.
\begin{figure}[H]
    \centering
    \includegraphics[width=0.6\textwidth]{imgs/3-tetel/veges_potencialvolgy_atvitel.png}
    \caption{Véges potenciálvölgy átviteli együtthatója az energia függvényében}
    \label{fig:veges_potencialvolgy_atvitel_visszaverodes}
\end{figure}

\subsubsection{Oszcillátor}
A klasszikus mechanikai oszcillátort, ahol egy  $m$ tömeg egy  $k$ rugóállandóval rendelkező rugóhoz van kötve,
a Hooke-törvény alapján lehet felírni:
\begin{equation}
    F = - k x = m \frac{d^2 x}{dt^2}
\end{equation}
Ahol $x$ a tömeg kitérése az egyensúlyi helyzetből. A megoldás egy harmonikus mozgás lesz, ahol a tömeg $x(t) = A cos(\omega t + \phi)$ alakban mozog,
ahol $\omega = \sqrt{\frac{k}{m}}$ az oszcilláció körfrekvenciája, $A$ a amplitúdó és $\phi$ a kezdeti fázis. Az oszcillátor potenciálja a következőképpen írható fel:
\begin{equation}
    V(x) = \frac{1}{2} k x^2
\end{equation}
Ez természetesen csak egy "tökéletes" oszcillátorra igaz, ahol a rugó végtelenül rugalmas, és nincs súrlódás vagy egyéb veszteség a rendszerben.
A valóságban a Hooke-törvény gyakran már a rugó elszakadása előtt sem érvényes, ezért csak kis kitérésre lehet vele közelíteni. 
Ehhez írjuk fel a potenciál Taylor-sorát az egyensúlyi helyzet körül (V($x_0$) = 0):
\begin{equation}
    V(x) = V(x_0) + V'(x_0) (x - x_0) + \frac{1}{2} V''(x_0) (x - x_0)^2 + \ldots
\end{equation}
Itt az első tag az $V(x_0)$ defínicója miatt nulla, a magasabb rendű tagok pedig elhanyagolhatók kis kitérések esetén. Ezért a potenciál közelítőleg:
\begin{equation}
    V(x) \approx \frac{1}{2} V''(x_0) (x - x_0)^2
\end{equation}
Ami leírja a Harmonikus oszcillációt kis kitérések esetén, egy effektív rugóállandóval $k_{eff} = V''(x_0)$. Enek segítségével
virtuálisan bármilyen oszcilláló mozgást lehet harmonikus oszcillációval közelíteni, ha kicsi az amplitúdó. 
\newline
\newline
Ezek alapján egy kvantummechanikai harmonikus oszcillátort is közelítőleg felírhatunk a következő potenciállal:
\begin{equation}
    V(x) = \frac{1}{2} m \omega^2 x^2
\end{equation}
Ahol $\omega$ az oszcillátor körfrekvenciája és ezzel a nehezen definiálható kvantummechanikai rugóállandót "kivettük" a problémából.
A kvantummechanikai rendszerben nem a Newton-féle mozgásegyenletet használjuk, hanem az időfüggetlen Schrödinger-egyenletet:
\begin{equation}
    -\frac{\hbar^2}{2m} \frac{d^2 \psi(x)}{dx^2} + \frac{1}{2} m \omega^2 x^2 \psi(x) = E \psi(x)
\end{equation}
Ez egy másodrendű differenciálegyenlet, amelyet általában két módszerrrel szokás megoldani, a "brute force" hatványsor módszerrel,
ahol a hullámfüggvényt hatványsor formában írjuk fel, és a rekurzív együtthatókat határozzuk meg, vagy az elegánsabb algebrai módszerrel.
Jelen esetben nézzük meg az algebrai módszert, amely a létrehozó és megsemmisítő operátorok bevezetésén alapul.

\subsubsection*{Algebrai módszer}
Kezdjük azzal, hogy kicsit átrendezzük a Schrödinger-egyenletet:
\begin{equation}
    \frac{1}{2m} \left[\hat{p}^2 + (m \omega \hat{x})^2\right] \psi(x) = E \psi(x)
\end{equation}
Itt bevezettük az impulzus és hely operátort:
\begin{equation}
    \hat{x} = x
\end{equation}
\begin{equation}
    \hat{p} = - i \hbar \frac{d}{dx}
\end{equation}
Ez a Schrödinger-egyenlet alapján megfelel a Hamilton-operátornak:
\begin{equation}
    \hat{H}\psi = E\psi
\end{equation}
\begin{equation}
    \hat{H} = \frac{1}{2m} \left[\hat{p}^2 + (m \omega x)^2\right]
\end{equation}
Ahhoz, hogy meg tudjuk oldani az egyenletet normál esetben faktorizálnánk a Hamilton-operátort valahogy így:
\begin{equation}
    u^2 + v^2 = (iu + v)(- iu + v)
\end{equation}
Viszont itt nem egyszerű számokkal van dolgunk, mert $\hat{p}$ és $\hat{x}$ operátorok, és az operátoroknál nem feltétlenül igaz, hogy
kommutálnak egymással:
\begin{equation}
    \hat{p} \hat{x} \neq \hat{x} \hat{p} \qquad \text{általában}
\end{equation}
Ennek ellenére vezessünk be egy új operátor párt:
\begin{equation}
    \hat{a}_\pm = \frac{1}{\sqrt{2 \hbar m \omega}} \left( \mp i \hat{p} + m \omega \hat{x} \right)
\end{equation}
Nézzük meg, hogy mi a skaláris szorzatuk ezeknek az új operátoroknak:
\begin{equation}
    \begin{aligned}
    & \hat{a}_- \hat{a}_+ = \frac{1}{2 \hbar m \omega} \left( i \hat{p} + m \omega \hat{x} \right) \left( - i \hat{p} + m \omega \hat{x} \right) = \\
    & = \frac{1}{2 \hbar m \omega} \left( \hat{p}^2 + m^2 \omega^2 \hat{x}^2 + i m \omega (\hat{x} \hat{p} - \hat{p} \hat{x}) \right)
    \end{aligned}
\end{equation}
Itt látható, hogy megjelent egy extra $(\hat{x} \hat{p} - \hat{p} \hat{x})$ tag, ami kommutálás esetén eltűnne.
Ezt a tagot formálisan következőképpen írhatjuk fel:
\begin{equation}
    [\hat{x}, \hat{p}] = \hat{x} \hat{p} - \hat{p} \hat{x}
\end{equation}
És "kommutátornak" nevezzük. Ez a művelet azt méri, hogy két operátor mennyire "rosszul" kommutál egymással. Ha a kommutátor nulla, akkor a két operátor kommutál.
Most már felírhatjuk a bevezetett operátorok szorzatát a kommutátorral:
\begin{equation}
    \hat{a}_- \hat{a}_+ = \frac{1}{2 \hbar m \omega} \left( \hat{p}^2 + m^2 \omega^2 \hat{x}^2 + i m \omega [\hat{x}, \hat{p}] \right)
\end{equation}
\begin{equation}
    \hat{a}_- \hat{a}_+ = \frac{1}{\hbar \omega} \left[\hat{p}^2 + (m \omega \hat{x})^2\right] + \frac{i}{2 \hbar} [\hat{x}, \hat{p}]
\end{equation}
Itt pedig észrevehetjük a Hamilton-operátort:
\begin{equation}
    \hat{a}_- \hat{a}_+ = \frac{2}{\hbar \omega} \hat{H} + \frac{i}{2 \hbar} [\hat{x}, \hat{p}]
\end{equation}
Most már csak ki kell számolnunk a kommutátort. Ezt úgy tehetjük meg, hogy alkalmazzuk a két operátort egy tetszőleges teszt függvényre $f(x)$:
\begin{equation}
    [\hat{x}, \hat{p}] f(x) = \hat{x} \hat{p} f(x) - \hat{p} \hat{x} f(x)
\end{equation}
\begin{equation}
    = \hat{x} \left( - i \hbar \frac{d f(x)}{dx} \right) - \hat{p} \left( x f(x) \right)
\end{equation}
\begin{equation}
    = - i \hbar x \frac{d f(x)}{dx} + i \hbar \frac{d}{dx} \left( x f(x) \right)
\end{equation}
\begin{equation}
    = - i \hbar x \frac{d f(x)}{dx} + i \hbar \left( f(x) + x \frac{d f(x)}{dx} \right)
\end{equation}
\begin{equation}
    = i \hbar f(x)
\end{equation}
Mivel ez igaz minden $f(x)$ tesztfüggvényre, ezért a kommutátor maga:
\begin{equation}
    [\hat{x}, \hat{p}] = i \hbar
\end{equation}
Ezt visszahelyettesítve az előző egyenletbe:
\begin{equation}
    \hat{a}_- \hat{a}_+ = \frac{2}{\hbar \omega} \hat{H} + \frac{1}{2} [\hat{x}, \hat{p}] = \frac{2}{\hbar \omega} \hat{H} + \frac{1}{2} i \hbar
\end{equation}
vagy a Hamilton-operátorra rendezve:
\begin{equation}
    \hat{H} = \hbar \omega \left( \hat{a}_- \hat{a}_+ - \frac{1}{2} \right)
\end{equation}
Hasonlóan kiszámolhatjuk a másik sorrendű szorzatot is:
\begin{equation}
    \hat{a}_+ \hat{a}_- = \frac{1}{2 \hbar m \omega} \left( - i \hat{p} + m \omega \hat{x} \right) \left( i \hat{p} + m \omega \hat{x} \right) = 
\end{equation}
\begin{equation}
    = \frac{1}{2 \hbar m \omega} \left( \hat{p}^2 + m^2 \omega^2 \hat{x}^2 - i m \omega (\hat{x} \hat{p} - \hat{p} \hat{x}) \right)
\end{equation}
\begin{equation}
    \hat{a}_+ \hat{a}_- = \frac{1}{\hbar \omega} \left[\hat{p}^2 + (m \omega \hat{x})^2\right] - \frac{i}{2 \hbar} [\hat{x}, \hat{p}]
\end{equation}
\begin{equation}
    \hat{a}_+ \hat{a}_- = \frac{2}{\hbar \omega} \hat{H} - \frac{i}{2 \hbar} [\hat{x}, \hat{p}]
\end{equation}
\begin{equation}
    \hat{a}_+ \hat{a}_- = \frac{2}{\hbar \omega} \hat{H} - \frac{1}{2} i \hbar
\end{equation}
\begin{equation}
    \hat{H} = \hbar \omega \left( \hat{a}_+ \hat{a}_- + \frac{1}{2} \right)
\end{equation}

Látható, hogy a két eredmény pont 1-el tér el egymástól, ezért ezek az operátorok sem kommutálnak:
\begin{equation}
    [\hat{a}_-, \hat{a}_+] = 1
\end{equation}
Tehát összefoglalva a Hamilton-operátort a következőképpen írhatjuk fel:
\begin{equation}
    \hat{H} = \hbar \omega \left( \hat{a}_\pm \hat{a}_\mp \pm \frac{1}{2} \right)
\end{equation}
Ezzel a Schrödinger-egyenlet a következő alakot ölti:
\begin{equation}
    \hbar \omega \left( \hat{a}_\pm \hat{a}_\mp \pm \frac{1}{2} \right) \psi = E \psi
\end{equation}
Ekkor azt állítom, hogy ha $\psi$ kielégíti a Schrödinger-egyenletet egy adott $E$ energiára (tehát $\hat{H} \psi = E \psi$), akkor
$\hat{a}_+\psi$ is kielégíti a Schrödinger-egyenletet egy $E + \hbar \omega$ energiára: $(E + \hbar \omega)(\hat{a}_+\psi)$
\newline
\newline
Bizonyítás:
\begin{equation}
    \begin{aligned}
    & \hat{H} (\hat{a}_+ \psi) = \hbar \omega \left( \hat{a}_+ \hat{a}_- + \frac{1}{2} \right) (\hat{a}_+ \psi) = \hbar \omega \left( \hat{a}_+ \hat{a}_- \hat{a}_+ + \frac{1}{2} \hat{a}_+  \right)\psi \\
    & = \hbar \omega \hat{a}_+ \left(\hat{a}_- \hat{a}_+ + \frac{1}{2}\right)\psi = \hat{a}_+ \left[\hbar \omega \left(\hat{a}_- \hat{a}_+ + 1 + \frac{1}{2}\right)\psi\right] = \quad \quad \text{mert } \hat{a}_+ \hat{a}_- = \hat{a}_- \hat{a}_+ + 1 \\
    & = \hat{a}_+ \left( \hat{H} + \hbar \omega \right) \psi = \hat{a}_+ (E + \hbar \omega) \psi = \\
    & = (E + \hbar \omega)(\hat{a}_+ \psi)
    \end{aligned}
\end{equation}
Fontos megjegyezni, hogy a $\hat{a}_\pm$ operátorok szorzatának sorrendje számít, hiszen nem kommutálnak egymással, de az operátor és egy állandó 
szorzatánál nem, mert minden operátor kommutál minden állandóval.

Ugyanezt megtehetjük a $\hat{a}_-$ operátorral is:
\begin{equation}
    \begin{aligned}
    & \hat{H} (\hat{a}_- \psi) = \hbar \omega \left( \hat{a}_- \hat{a}_+ - \frac{1}{2} \right) (\hat{a}_- \psi) = \hbar \omega \left( \hat{a}_- \hat{a}_+ \hat{a}_- - \frac{1}{2} \hat{a}_-  \right)\psi \\
    & = \hbar \omega \hat{a}_- \left(\hat{a}_- \hat{a}_+ + \frac{1}{2}\right)\psi = \hat{a}_- \left[\hbar \omega \left(\hat{a}_- \hat{a}_+ - 1 - \frac{1}{2}\right)\psi\right] = \\
    & = \hat{a}_- \left( \hat{H} - \hbar \omega \right) \psi = \hat{a}_- (E - \hbar \omega) \psi = \\
    & = (E - \hbar \omega)(\hat{a}_- \psi)
    \end{aligned}
\end{equation}
Ezzel sikerült előállítani egy "gépezetet" amivel egy kezdő energiából tetszőleges magasabb vagy alacsonyabb energiájú állapotokat tudunk előállítani.
Ezért ezeket az operátorokat "létra operátoroknak" is szokták nevezni és a $\hat{a}_+$-at keltő és a $\hat{a}_-$-at eltüntető operátornak is hívják.
\begin{figure}[H]
    \centering
    \includegraphics[width=0.4\textwidth]{imgs/3-tetel/harmonikus_oszcillator_letra_operatorok.png}
    \caption{Harmonikus oszcillátor létra operátorok ábrázolása}
\end{figure}

De mi történik, ha ezeket az operátorokat végtelenszer alkalmazzuk? A keltő operáttorral nem lesz gond, hiszen azzal csak egyre nagyonbb és nagyobb
energiájú állapotokat állíthatunk elő. Viszont az eltüntető operátorral egy idő után el fogunk jutni egy olyan állapothoz, ahol az energia negatív lesz,
ami fizikailag értelmetlen. Ezért kell lennie egy olyan állapotnak amire ha alkalmazzuk az eltüntető operátort, akkor nullát kapunk vissza:
\begin{equation}
    \hat{a}_- \psi_0 = 0
\end{equation}
Ezt az állapotot nevezik alapállapotnak. Ahhoz, hogy meghatározzuk az alapállapot energiáját, alkalmazzuk a Hamilton-operátort az alapállapotra:
\begin{equation}
    \hat{H} \psi_0 = \hbar \omega \left( \hat{a}_+ \hat{a}_- + \frac{1}{2} \right) \psi_0 = \hbar \omega \left( 0 + \frac{1}{2} \right) \psi_0
\end{equation}
\begin{equation}
    \hat{H} \psi_0 = \frac{1}{2} \hbar \omega \psi_0
\end{equation}
Tehát az alapállapot energiája:
\begin{equation}
    E_0 = \frac{1}{2} \hbar \omega
\end{equation}
Most már csak az alapállapot hullámfüggvényét kell meghatároznunk. Ehhez használjuk fel az alapállapotra vonatkozó feltételt:
\begin{equation}
    \hat{a}_- \psi_0 = 0
\end{equation}
\begin{equation}
    \frac{1}{\sqrt{2 \hbar m \omega}} \left( i \hbar \frac{d}{dx} + m \omega x \right) \psi_0 = 0
\end{equation}
\begin{equation}
    \left( \frac{d}{dx} + \frac{m \omega}{i \hbar} x \right) \psi_0 = 0
\end{equation}
Ez egy elsőrendű differenciálegyenlet, amit könnyen meg tudunk oldani:
\begin{equation}
    \frac{d \psi_0}{\psi_0} = - \frac{m \omega}{i \hbar} x dx
\end{equation}
\begin{equation}
    \int \frac{d \psi_0}{\psi_0} = - \frac{m \omega}{i \hbar} \int x dx
\end{equation}
\begin{equation}
    ln(\psi_0) = - \frac{m \omega}{2 \hbar} x^2 + C
\end{equation}
\begin{equation}
    \psi_0(x) = A e^{- \frac{m \omega}{2 \hbar} x^2}
\end{equation}
És ha normáljuk a hullámfüggvényt:
\begin{equation}
    \int_{-\infty}^{\infty} |\psi_0(x)|^2 dx = 1
\end{equation}
\begin{equation}
    |A|^2 \int_{-\infty}^{\infty} e^{- \frac{m \omega}{\hbar} x^2} dx = 1
\end{equation}
\begin{equation}
    |A|^2 \sqrt{\frac{\pi \hbar}{m \omega}} = 1
\end{equation}
\begin{equation}
    |A| = \left( \frac{m \omega}{\pi \hbar} \right)^{1/4}
\end{equation}
Tehát az alapállapot hullámfüggvénye:
\begin{equation}
    \psi_0(x) = \left( \frac{m \omega}{\pi \hbar} \right)^{1/4} e^{- \frac{m \omega}{2 \hbar} x^2}
\end{equation}
Tehát az összes állapot előállítható az alapállapotból a keltő operátor ismételt alkalmazásával:
\begin{equation}
    \psi_n = A_n (\hat{a}_+)^n \psi_0 \quad \quad \text{ahol }  E_n = \hbar \omega \left( n + \frac{1}{2} \right) \quad \quad n = 0, 1, 2, \ldots
\end{equation}
Itt $A_n$ egy normalizáló faktor, amit a hullámfüggvény normálásával lehet meghatározni. Tehát például az első izgatott állapot hullámfüggvénye:
\begin{equation}
    \psi_1 = A_1 \hat{a}_+ \psi_0
\end{equation}
\begin{equation}
    = A_1 \frac{1}{\sqrt{2 \hbar m \omega}} \left( - i \hbar \frac{d}{dx} + m \omega x \right) \left( \frac{m \omega}{\pi \hbar} \right)^{1/4} e^{- \frac{m \omega}{2 \hbar} x^2}
\end{equation}
\begin{equation}
    = A_1 \frac{1}{\sqrt{2 \hbar m \omega}} \left( m \omega x + i \hbar \frac{m \omega}{\hbar} x \right) \left( \frac{m \omega}{\pi \hbar} \right)^{1/4} e^{- \frac{m \omega}{2 \hbar} x^2}
\end{equation}
\begin{equation}
    = A_1 \sqrt{\frac{2 m \omega}{\hbar}} x \left( \frac{m \omega}{\pi \hbar} \right)^{1/4} e^{- \frac{m \omega}{2 \hbar} x^2}
\end{equation}
Normálva ezt a hullámfüggvényt megkapjuk az első izgatott állapot a normalizáló faktort:
\begin{equation}
    \int |\psi_1(x)|^2 dx = |A_1|^2 \frac{2 m \omega}{\hbar} \left( \frac{m \omega}{\pi \hbar} \right)^{1/2} \int_{-\infty}^{\infty} x^2 e^{- \frac{m \omega}{\hbar} x^2} dx = 1
\end{equation}
\begin{equation}
    |A_1|^2 \frac{2 m \omega}{\hbar} \left( \frac{m \omega}{\pi \hbar} \right)^{1/2} \frac{\sqrt{\pi}}{2} \left( \frac{\hbar}{m \omega} \right)^{3/2} = 1
\end{equation}
\begin{equation}
    |A_1|^2 = 1
\end{equation}
Ahhoz, hogy általánosan normalizálni tudjuk a hullámfüggvényt, egy kicsit trükközni kell. Először is megállapíthatjuk, hogy
ha "feljebb" vagy "lejjebb" lépünk az  energia-létrán, akkor a hullámfüggvények arányosak a következő módon:
\begin{equation}
    \hat{a}_+\psi_{n} = c_n \psi_{n+1} \quad \quad \hat{a}_- \psi_{n} = d_n \psi_{n-1}
\end{equation}
Ahol $c_n$ és $d_n$ arányossági tényezők. Ezek meghatározásához először megjegyezzük, hogy tetszőleges függvények esetén igaz, hogy:
\begin{equation}
    \int_{-\infty}^{\infty} f^*(x) (\hat{a}_\pm g(x)) dx = \int_{-\infty}^{\infty} (\hat{a}_\mp f(x))^* g(x) dx
\end{equation}
Ahol $f^*(x)$ a komplex konjugáltja az $f(x)$ függvénynek. Ezt algebrában úgy nevezik, hogy $\hat{a}_\pm$ a hermitikus konjugáltja $\hat{a}_\mp$-nek.
Innen következik, hogy:
\begin{equation}
    \int_{-\infty}^{\infty} (\hat{a}_\pm \psi_n)^* (\hat{a}_\pm \psi_n) dx = \int_{-\infty}^{\infty} (\hat{a}_\mp \hat{a}_\pm \psi_n)^* \psi_n dx
\end{equation}
De mivel $\hat{a}_\mp \hat{a}_\pm = \frac{1}{\hbar \omega} \hat{H} \pm \frac{1}{2}$, ezért:
\begin{equation}
    \int_{-\infty}^{\infty} (\hat{a}_\pm \psi_n)^* (\hat{a}_\pm \psi_n) dx = \int_{-\infty}^{\infty} \left( \frac{E_n}{\hbar \omega} \pm \frac{1}{2} \right) \psi_n^* \psi_n dx
\end{equation}
\begin{equation}
    = \left( n + \frac{1}{2} \pm \frac{1}{2} \right) \int_{-\infty}^{\infty} \psi_n^* \psi_n dx
\end{equation}
\begin{equation}
    = n + \frac{1 \pm 1}{2}
\end{equation}
Vagyis:
\begin{equation}
    \hat{a}_+ \hat{a}_- \psi_n = n \psi_n \quad \quad \hat{a}_- \hat{a}_+ \psi_n = (n + 1) \psi_n
\end{equation}
Ezt behelyettesítve az előző egyenletbe:
\begin{equation}
    \int_{-\infty}^{\infty} (\hat{a}_+ \psi_n)^* (\hat{a}_+ \psi_n) dx = |c_n|^2 \int_{-\infty}^{\infty} |\psi_{n+1}|^2 dx = (n + 1) \int_{-\infty}^{\infty} |\psi_n|^2 dx
\end{equation}
\begin{equation}
    \int_{-\infty}^{\infty} (\hat{a}_- \psi_n)^* (\hat{a}_- \psi_n) dx = |d_n|^2 \int_{-\infty}^{\infty} |\psi_{n-1}|^2 dx = n \int_{-\infty}^{\infty} |\psi_n|^2 dx
\end{equation}
De mivel a hullámfüggvények normálva vannak, ezért az integrálok értéke 1, így:
\begin{equation}
    |c_n|^2 = n + 1 \quad \quad |d_n|^2 = n
\end{equation}
\begin{equation}
    \hat{a}_+ \psi_n = \sqrt{n + 1} \psi_{n+1} \quad \quad \hat{a}_- \psi_n = \sqrt{n} \psi_{n-1}
\end{equation}
Innen ha megnézzük az egyes állapotokra:
\begin{equation}
    \psi_1 = \hat{a}_+ \psi_0 = \sqrt{1} \psi_1 \implies \psi_1 = \hat{a}_+ \psi_0
\end{equation}
\begin{equation}
    \psi_2 = \frac{1}{\sqrt{2}} \hat{a}_+ \psi_1 = \frac{1}{\sqrt{2}} \hat{a}_+^2 \psi_0
\end{equation}
\begin{equation}
    \psi_3 = \frac{1}{\sqrt{3}} \hat{a}_+ \psi_2 = \frac{1}{\sqrt{3 \cdot 2}} \hat{a}_+^3 \psi_0
\end{equation}
\begin{equation}
    \psi_4 = \frac{1}{\sqrt{4}} \hat{a}_+ \psi_3 = \frac{1}{\sqrt{4 \cdot 3 \cdot 2}} \hat{a}_+^4 \psi_0
\end{equation}
És így tovább. Ebből következik az általános formula:
\begin{equation}
    \psi_n = \frac{1}{\sqrt{n!}} (\hat{a}_+)^n \psi_0 \quad \quad n = 0, 1, 2, \ldots
\end{equation}
Ezzel megkaptuk a harmonikus oszcillátor összes energiáját és hullámfüggvényét az algebrai módszer segítségével. A harmonikus oszcillátor
az egyik legfontosabb modell a kvantummechanikában, hiszen sok rendszer közelíthető vele, és a kvantumtérelmélet alapját is képezi. A segítségével
lehet közelíteni például a molekulákban elhelyezkedő atomok rezgéseit, vagy a kristályrácsok fononjait is.

\subsubsection*{Hatványsoros módszer}
Ezt a módszert csak nagyvonalakban tárgyaljuk mert ugyan arra az eredményre jutunk mint az előző módszerrel. Előnye,
hogy ez egy "általánosabb" módszer, amit más potenciálokra is lehet alkalmazni, míg az előző módszer csak a harmonikus oszcillátorra alkalmazható.
Kezdjük azzal, hogy írjuk át a Schrödinger-egyenletet egy kicsit:
\begin{equation}
    \frac{d^2 \psi(x)}{dx^2} = \frac{2m}{\hbar^2} \left( \frac{1}{2} m \omega^2 x^2 - E \right) \psi(x)
\end{equation}
\begin{equation}
    \frac{d^2 \psi(x)}{dx^2} = \left( \frac{m^2 \omega^2}{\hbar^2} x^2 - \frac{2mE}{\hbar^2} \right) \psi(x)
\end{equation}
Most vezessük be a következő változókat:
\begin{equation}
    \xi = \sqrt{\frac{m \omega}{\hbar}} x
\end{equation}
\begin{equation}
    \epsilon = \frac{2E}{\hbar \omega}
\end{equation}
Ezekkel a változókkal az egyenlet a következő alakot ölti:
\begin{equation}
    \frac{d^2 \psi(\xi)}{d \xi^2} = (\xi^2 - \epsilon) \psi(\xi)
\end{equation}
Most nézzük meg az egyenlet viselkedését nagy $\xi$ értékeknél. Ilyenkor a $\xi^2$ tag dominál, így az egyenlet közelítőleg:
\begin{equation}
    \frac{d^2 \psi(\xi)}{d \xi^2} \approx \xi^2 \psi(\xi)
\end{equation}
Ennek az egyenletnek a megoldása a következő alakú:
\begin{equation}
    \psi(\xi) \approx A e^{- \frac{\xi^2}{2}} + B e^{\frac{\xi^2}{2}}
\end{equation}
De mivel a hullámfüggvénynek végesnek kell lennie, ezért a $B$ együtthatót nullának kell vennünk, így:
\begin{equation}
    \psi(\xi) \approx A e^{- \frac{\xi^2}{2}}
\end{equation}
Ezért a teljes hullámfüggvényt felírhatjuk a következő alakban:
\begin{equation}
    \psi(\xi) = h(\xi) e^{- \frac{\xi^2}{2}}
\end{equation}
Ahol $h(\xi)$ egy ismeretlen függvény, amit meg kell határoznunk. Ezt visszahelyettesítve az eredeti egyenletbe:
\begin{equation}
    \frac{d^2 h(\xi)}{d \xi^2} - 2 \xi \frac{d h(\xi)}{d \xi} + (\epsilon - 1) h(\xi) = 0
\end{equation}
Most feltételezzük, hogy $h(\xi)$ egy hatványsor alakban írható fel:
\begin{equation}
    h(\xi) = \sum_{n=0}^{\infty} a_n \xi^n
\end{equation}
Ezt behelyettesítve az egyenletbe és rendezzük az egyenletet:
\begin{equation}
    \sum_{n=0}^{\infty} \left[ (n+2)(n+1) a_{n+2} - 2 n a_n + (\epsilon - 1) a_n \right] \xi^n = 0
\end{equation}
Ez akkor igaz, ha minden együttható nulla:
\begin{equation}
    (n+2)(n+1) a_{n+2} - 2 n a_n + (\epsilon - 1) a_n = 0
\end{equation}
Ebből kifejezve az $a_{n+2}$ együtthatót:
\begin{equation}
    a_{n+2} = \frac{2 n + 1 - \epsilon}{(n+2)(n+1)} a_n
\end{equation}
Most nézzük meg a sorozat viselkedését nagy $n$ értékeknél. Ilyenkor az együttható közelítőleg:
\begin{equation}
    a_{n+2} \approx \frac{2 n + 1 - \epsilon}{(n+2)(n+1)} a_n
\end{equation}
\begin{equation}
    \approx \frac{2}{n} a_n
\end{equation}
Ez azt jelenti, hogy az együtthatók növekedése olyan gyors, hogy a sorozat divergens lesz, és a hullámfüggvény nem lesz normálható. Ezért a sorozatot le kell vágni egy bizonyos $n$ értéknél, ami csak akkor lehetséges, ha a számláló nulla lesz egy adott $n$ értéknél:
\begin{equation}
    2 n + 1 - \epsilon = 0
\end{equation}
\begin{equation}
    \epsilon = 2 n + 1
\end{equation}
Ebből következik az energia kvantálódása:
\begin{equation}
    E_n = \hbar \omega \left( n + \frac{1}{2} \right) \quad \quad n = 0, 1, 2, \ldots
\end{equation}
Ez megegyezik az előző módszerrel kapott eredménnyel. A hullámfüggvényeket pedig a hatványsorozatból lehet meghatározni az együtthatók segítségével:
\begin{equation}
    \psi_n(\xi) = h_n(\xi) e^{- \frac{\xi^2}{2}} \quad \quad \text{ahol } h_n(\xi) \text{ egy n-ed fokú polinom}
\end{equation}
Ezek a polinomok a Hermite-polynomok, és a hullámfüggvények a következő alakot öltik:
\begin{equation}
    \psi_n(x) = \left( \frac{m \omega}{\pi \hbar} \right)^{1/4} \frac{1}{\sqrt{2^n n!}} H_n\left( \sqrt{\frac{m \omega}{\hbar}} x \right) e^{- \frac{m \omega}{2 \hbar} x^2}
\end{equation}
Ahol $H_n(\xi)$ az n-ed fokú Hermite-polinom. Így tehát a hatványsoros módszerrel is megkaptuk a harmonikus oszcillátor energiáit és hullámfüggvényeit.
Néhány első Hermite-polinom:
\begin{equation*}
    \begin{aligned}
    & H_0(\xi) = 1 \\
    & H_1(\xi) = 2 \xi \\
    & H_2(\xi) = 4 \xi^2 - 2 \\
    & H_3(\xi) = 8 \xi^3 - 12 \xi \\
    & H_4(\xi) = 16 \xi^4 - 48 \xi^2 + 12 \\
    & H_5(\xi) = 32 \xi^5 - 160 \xi^3 + 120 \xi \\
    \end{aligned}    
\end{equation*}

\subsubsection{!Rotátor}
A rotátor egy olyan kvantummechanikai rendszer, amely egy merev test forgását írja le egy adott tengely körül. A legegyszerűbb eset a diatomikus molekula, amely két atomot köt össze egy merev kötés. A rotátor modellje fontos szerepet játszik a molekuláris spektroszkópiában és a kvantummechanikai rendszerek forgási energiáinak leírásában.
Ahhoz, hogy a rotátort értelmezni tudjuk, ki kell terjesztenünk a Schrödinger-egyenletet három dimenzióra. Ekkor az egyenlet mind a három térkoordinátától függeni fog, ezért a derivált a Laplace operátorra fog módosulni:
\begin{equation}
    - \frac{\hbar^2}{2m} \Delta \psi(\vec{r}) + V(\vec{r}) \psi(\vec{r}) = E \psi(\vec{r})
\end{equation}
Ahol a Laplace operátor a következőképpen definiált:
\begin{equation}
    \Delta = \nabla^2 = \frac{\partial^2}{\partial x^2} + \frac{\partial^2}{\partial y^2} + \frac{\partial^2}{\partial z^2}
\end{equation}
Ezt lehet tovább egyszerűsíteni a változók szétválasztásával:
\begin{equation}
    \psi(x,y,z) = X(x) Y(y) Z(z)
\end{equation}
Ezzel jól lehet közelíteni egy 3D-s potenciálgödröt vagy harmonikus oszcillátort is. Viszont a rotátor esetében gömbi szimetriát látunk, 
így célszerűbb a szétválasztást gömbi-koordinátákban végezni. Mielőtt ezt megtennénk, megállapíthatjuk, hogy ha találnuk egy operátort, ami
kommutál a Hamilton-operátorral, akkor lehet úgy választani a hullámfüggvényeket, hogy azok az operátor sajátfüggvényei is legyenek:
\begin{equation}
    [\hat{H}, \hat{A}] = 0 \quad \text{és} \quad \hat{H} \psi = E \psi \quad \Rrightarrow \quad \hat{A} \psi = a \psi
\end{equation}
Tehát egy olyan operátorokat érdemes keresni, amelyek kommutálnak a Hamilton-operátorral és gömbi-koordinátákkal egyszerűbben felírhatók.
Egy jó választás lehet mondjuk az impulzusmomentum operátora:
\begin{equation}
    \hat{L}_i = (\hat{r} \times \hat{p})_i = \epsilon_{ijk} \hat{r}_j \hat{p}_k = - i \hbar \epsilon_{ijk} x_j \frac{\partial}{\partial x_k}
\end{equation}
Ahol az $\epsilon_{ijk}$ a Levi-Civita szimbólum. Nézzük meg, hogy ez az operátor kommutál-e a Hamilton-operátorral.
Nézzük meg, hogy hogyan kommutál az impulzusmomentum az impulzus és a hely operátorokkal:
\begin{equation}
    [\hat{L}_i, \hat{x}_j] = i \hbar \epsilon_{ijk} \hat{x}_k
\end{equation}
\begin{equation}
    [\hat{L}_i, \frac{\partial}{\partial x_j}] = i \hbar \epsilon_{ijk} \frac{\partial}{\partial x_k}
\end{equation}
\begin{equation}
    [\hat{L}_i, \hat{p}_j] = i \hbar \epsilon_{ijk} \hat{p}_k
\end{equation}
Tehát egy tetszőleges vektorra, ami a hely és az impulzus operátorok lineáris kombinációja, a következőképpen kommutál:
\begin{equation}
    [\hat{L}_i, \hat{v}_j] = i \hbar \epsilon_{ijk} \hat{v}_k
\end{equation}
Ahol $\hat{v}$ egy tetszőleges vektor operátor. Egy vektor négyzetére pedig:
\begin{equation}
    [\hat{L}_i, \hat{v}^2_j] = \hat{L}_i \hat{v}_j \hat{v}_j - \hat{v}_j \hat{v}_j \hat{L}_i = [\hat{L}_i, \hat{v}_j] \hat{v}_j + \hat{v}_j [\hat{L}_i, \hat{v}_j] = i \hbar \epsilon_{ijk} (\hat{v}_k \hat{v}_j + \hat{v}_j \hat{v}_k) = 0
\end{equation}
Mert a végén a Levi-Civita szimbólum antiszimmetrikus k-ra és j-re míg a zárójelben lévő elemek szimmetrikusak ezért az egész kifejezés nulla lesz. 
Ebből következik, hogy bármely vektor/operátor négyzetére (ami x-ből és p-ből áll) 0 lesz a kommutálás eredménye → a Hamilton operátor tartalmaz
egy négyzetes tagot, a Laplace operátort, tehát a Hamilton operátorral biztosan kommutál az impulzusmomentum operátora. És mivel önmagával
is hasonló a kommutáció, így az $L^2$-re is ugyan ez lesz az eredmény:
\begin{equation}
    [\hat{H}, \hat{L}_i] = 0
\end{equation}
\begin{equation}
    [\hat{L}_i, \hat{L}^2] = 0
\end{equation}
$\hat{L}_1$, $\hat{L}_2$ és $\hat{L}_3$ egymással mind kommutálnak, így ezek alkalmasak a sajátérték egyenlet megoldására ($\hat{L}_3$ helyett bármelyik irányt választhattuk volna, csak
azért a harmadik komponenst választjuk, mert gömbi-koordinátákban így egyszerűbb dolgunk lesz). Azért $\hat{L}_2$-t és $\hat{L}_3$-at választjuk mert ezekről könnyen
megbizonyosodhattunk, hogy kommutálnak $\hat{H}$-val, tehát lehet őket arra használni, hogy jellemezzünk egy fizikai állapotot ezeknek a
sajátállapotaival. A többi komponensét $\hat{L}$-nek azért nem használhatjuk, mert azok egymással már nem kommutálnak → nincs közös sajátfüggvény rendszerük.
Innentől a cél megtalálni $\hat{H}$, $\hat{L}_2$ és $\hat{L}_3$ közössajátfüggvény rendszerét. Ehhez először áttérünk gömbi-koordinátákra:
\begin{equation}
    x = r \sin(\theta) \cos(\phi)
\end{equation}
\begin{equation}
    y = r \sin(\theta) \sin(\phi)
\end{equation}
\begin{equation}
    z = r \cos(\theta)
\end{equation}
Innen az impulzusmomentum operátor komponensei a következő alakot öltik:
\begin{equation}
    \hat{L}_x = i \hbar \left( \sin(\phi) \frac{\partial}{\partial \theta} + \cot(\theta) \cos(\phi) \frac{\partial}{\partial \phi} \right)
\end{equation}
\begin{equation}
    \hat{L}_y = i \hbar \left( - \cos(\phi) \frac{\partial}{\partial \theta} + \cot(\theta) \sin(\phi) \frac{\partial}{\partial \phi} \right)
\end{equation}
\begin{equation}
    \hat{L}_z = - i \hbar \frac{\partial}{\partial \phi}
\end{equation}
És az impulzusmomentum négyzete pedig:
\begin{equation}
    \hat{L}^2 = - \hbar^2 \left[ \frac{1}{\sin(\theta)} \frac{\partial}{\partial \theta} \left( \sin(\theta) \frac{\partial}{\partial \theta} \right) + \frac{1}{\sin^2(\theta)} \frac{\partial^2}{\partial \phi^2} \right]
\end{equation}
$L^2$ rettentően hasonlít a Nabla operátor szögfüggő tagjaira gömbi-koordinátákban, csak egy $r^2$ osztó hiányzik.
Ha a Nabla operátor alakját nézzük gömbi-koordináta formájában, akkor a következő alakjára jutunk a Schrödinger-egyenletnek:
\begin{equation}
    - \frac{\hbar^2}{2m} \left[ \frac{1}{r^2} \frac{\partial}{\partial r} \left( r^2 \frac{\partial}{\partial r} \right) - \frac{\hat{L}^2}{\hbar^2 r^2} \right] \psi + V(r) \psi = E \psi
\end{equation}
Ezt $L^2$-re rendezve:
\begin{equation}
    \hat{L}^2 \psi = \left[ - \hbar^2 r^2 \frac{\partial}{\partial r} \left( r^2 \frac{\partial}{\partial r} \right) + \frac{2m r^2}{\hbar^2} (V(r) - E) \right] \psi
\end{equation}
Most így akkor az $\hat{L}^2$ saját vektorai és saját értékei lettek érdekesek számunkra. Tegyük fel hogy, $V(r)$ „nem nagyon szinguláris” r→0 -ban.
Ekkor $\psi(r, \theta, \phi)$ analitikus $r = 0$-ban. Ebben az esetben $\psi(x)$ felbontható komponenseire:
\begin{equation}
    \psi(r, \theta, \phi) = \sum_{ijk} C_{ijk} x_1^i x_2^j x_3^k
\end{equation}
Ha kis r-eket nézünk ($r = |\vec{x}| \to 0$) akkor a legelső tagok dominálnak, pl.:
\begin{equation}
    l = i + j + k
\end{equation}
\begin{equation}
    l = 0 \quad \Rightarrow \quad \psi(\vec{x}) = \text{konstans} + ax + by + cz + \ldots
\end{equation}
\begin{equation}
    l = 1 \quad \Rightarrow \quad \psi(\vec{x}) = a x + b y + c z + \ldots
\end{equation}
\begin{equation}
    l = 2 \quad \Rightarrow \quad \psi(\vec{x}) = a x^2 + b y^2 + c z^2 + d xy + e xz + f yz + \ldots
\end{equation}
És így tovább. Ha ezeket felírjuk gömbi koordinátákban, akkor a következő alakot kapjuk:
\begin{equation}
    \psi(r, \theta, \phi) = r^l Y_{lm}(\theta, \phi)
\end{equation}
Ha tehát megnézzük a Schrödinger-egyenletet $r \to 0$ határértékben, akkor a következő alakot kapjuk:
\begin{equation}
    r \to 0 \quad \Rightarrow \quad \hat{L}^2 Y_{lm}(\theta, \phi) = \hbar^2 \frac{\partial}{\partial r}(r^2 \frac{\partial}{\partial r} (r^l Y_{lm}))
\end{equation}
Ha elvégezzük a deriválásokat akkor a következőt kapjuk:
\begin{equation}
    \hat{L}^2 Y_{lm}(\theta, \phi) = \hbar^2 l (l + 1) Y_{lm}(\theta, \phi)
\end{equation}
És ezzel meg is kaptuk az impulzusmomentum négyzetének sajátértékét (amennyiben kis távolságokat nézünk). Itt is igaz, hogy l = 0,1,2,...
$L_3$-ra egyszerűbb a számolás mivel ott csak 1 db derivált van. Abban az esetben a sajátértékre a következő összefüggést kapjuk:
\begin{equation}
    \hat{L}_3 Y_{lm}(\theta, \phi) = - i \hbar \frac{\partial}{\partial \phi} Y_{lm}(\theta, \phi) = \hbar m Y_{lm}(\theta, \phi)
\end{equation}
Ahol m egy ismeretlen sajátérték, amit meg kell határoznunk. Ezt megoldva (egy elsőrendű inhomogén diff-egyenlet) azt kapjuk, hogy m
bármely egész szám lehet → a sajátérték $\hbar m, \quad m = 0,1,2...$ Tehát olyan megoldásokat találtunk, amik csak 2db konstanstól függnek (l,
m). Ha tovább akarnánk bontani a változókat, akkor a $\psi$ függést is leválaszthatnánk:
\begin{equation}
    \psi(r, \theta, \phi) = P_l^m (\theta) e^{i m \phi}
\end{equation}
Ahol $P_l^m (\theta)$ a Legrendre-polinomok asszociált változata. De nem lesz erre szükségünk, mert ha az előző alakot ($Y(r, \theta, \phi) = R(r) Y_{lm}(\theta, \phi)$) használjuk,
akkor a Schrödinger-egyenlet szétválasztható a következőképpen:
\begin{equation}
    \psi(r, \theta, \phi) = R(r) Y_{lm}(\theta, \phi)
\end{equation}
\begin{equation}
    - \frac{\hbar^2}{2m} \left[ \frac{1}{r^2} \frac{d}{d r} \left( r^2 \frac{d R(r)}{d r} \right) - \frac{l (l + 1)}{r^2} R(r) \right] + V(r) R(r) = E R(r)
\end{equation}
Itt látható, hogy a $Y(r, \theta, \phi)$ függvény kiesett a problémából, és csak a radiális rész maradt meg. Ebből következik, hogy a rotátor
energiái csak az l kvantumszámtól függenek, mivel az m csak a sajátállapot degenerációját határozza meg. Láthatjuk, hogy
ez nagyon hasonlít az 1D-s Schrödinger-egyenletre, csak itt van egy plusz centrális potenciál tagunk:
\begin{equation}
    V_{\text{eff}}(r) = V(r) + \frac{\hbar^2 l (l + 1)}{2m r^2}
\end{equation}

\subsubsection*{Gömbfüggvények}
Nézzük meg közelebbről a $Y_{lm}(\theta, \phi)$ függvényeket, amik az impulzusmomentum sajátfüggvényei. Ezeket gömbfüggvényeknek hívják,
és igaznak kell lenni rájuk, hogy:
\begin{itemize}
    \item $\hat{L}^2 Y_{lm}(\theta, \phi) = \hbar^2 l (l + 1) Y_{lm}(\theta, \phi)$
    \item $\hat{L}_3 Y_{lm}(\theta, \phi) = \hbar m Y_{lm}(\theta, \phi)$
\end{itemize}
Ezek a függvények a következő alakot öltik:
\begin{equation}
    Y_{lm}(\theta, \phi) = (-1)^m \sqrt{\frac{2l + 1}{4 \pi} \frac{(l - m)!}{(l + m)!}} P_l^m(\cos(\theta)) e^{i m \phi}
\end{equation}
Ahol $P_l^m(x)$ az asszociált Legrendre-polinomok, amik a következőképpen definiáltak:
\begin{equation}
    P_l^m(x) = (1 - x^2)^{m/2} \frac{d^m}{d x^m} P_l(x)
\end{equation}

\subsubsection{A Hidrogénatom}
A hidrogénatom a legegyszerűbb atom, amely egy protonból és egy elektronból áll. A proton sokkal nehezebb, mint az elektron, így
a proton mozgását elhanyagolhatjuk, és csak az elektron mozgását vizsgáljuk a proton körül (és a protont ezáltal tehetjük az origóba).
A hidrogénatom potenciálja Coulomb-potenciál, ami a következő alakot ölti:
\begin{equation}
    V(r) = - \frac{e^2}{4 \pi \epsilon_0 r}
\end{equation}
Ahol $e$ az elektron töltése, és $\epsilon_0$ a vákuum permittivitása. Ezt a potenciált behelyettesítve a radiális Schrödinger-egyenletbe:
\begin{equation}
   - \frac{\hbar^2}{2m_e} \frac{d^2 u}{dr^2} + \left[ - \frac{e^2}{4 \pi \epsilon_0 r} + \frac{\hbar^2 l (l + 1)}{2m_e r^2} \right] u(r) = E u(r)
\end{equation}
Ahol $m_e$ az elektron tömege, és $u(r) = r R(r)$ és a szögletes zárójelben lévő tagok alkotják az effektív potenciált.
A Coulomb-potenciál maga megengedi a folytonos állapotokat is (E > 0), ami az elektron-proton szóródásokat írja le. Viszont
jelen esetben a kötött állapotokat (E < 0) vizsgáljuk, ahol az elektron a proton körül kering. 
\begin{figure}[H]
    \centering
    \includegraphics[width=0.4\textwidth]{imgs/3-tetel/hidrogen_atom.png}
    \caption{A Hidrogén atom}
\end{figure}
\begin{figure}[H]
    \centering
    \includegraphics[width=0.4\textwidth]{imgs/3-tetel/hidrogen_atom_potencial.png}
    \caption{A Hidrogén atom effektív potenciálja}
\end{figure}

\subsubsection*{A radiális hullámfüggvény}
Először vezessünk be egy új változót:
\begin{equation}
    \kappa \equiv \sqrt{- \frac{2m_e E}{\hbar^2}}
\end{equation}
Ekkor az egyenlet a következő alakot ölti:
\begin{equation}
   \frac{1}{\kappa^2} \frac{d^2 u}{dr^2} = \left[1 - \frac{m_e e^2}{2\pi \epsilon_0 \hbar^2 \kappa} \frac{1}{\kappa r} + \frac{l(l + 1)}{(\kappa r)^2}\right]u
\end{equation}
Adja magát, hogy vezessünk be két további változót:
\begin{equation}
    \varrho \equiv \kappa r \quad \text{és} \quad \varrho_0 \equiv \frac{m_e e^2}{2\pi \epsilon_0 \hbar^2 \kappa}
\end{equation}
Ezekkel a változókkal az egyenlet a következő alakot ölti:
\begin{equation}
   \frac{d^2 u}{d \varrho^2} = \left[1 - \frac{\varrho_0}{\varrho} + \frac{l(l + 1)}{\varrho^2}\right]u
\end{equation}
Ez a differenciálegyenlet elég komplikált, ezért próbáljunk leválasztani róla változókat. Előszö nézzük meg, hogy hogyan viselkedik
a függvény ha $\varrho \to \infty$-be toljuk, tehát a konstans dominál a zárójelben:
\begin{equation}
   \frac{d^2 u}{d \varrho^2} = u
\end{equation}
Ennek a megoldása a következő alakú:
\begin{equation}
    u(\varrho) \approx A e^{-\varrho} + B e^{\varrho}
\end{equation}
De mivel a hullámfüggvénynek normálhatónak kell lennie, ezért a $B$ együtthatót nullának kell vennünk, így:
\begin{equation}
    u(\varrho) \approx A e^{-\varrho}
\end{equation}
Most nézzük meg az egyenletet kis $\varrho$ értékeknél ($\varrho \to 0$). Ilyenkor a $\frac{l(l + 1)}{\varrho^2}$ tag dominál, így az egyenlet közelítőleg:
\begin{equation}
   \frac{d^2 u}{d \varrho^2} = \frac{l(l + 1)}{\varrho^2} u
\end{equation}
Ennek az egyenletnek a megoldása a következő alakú:
\begin{equation}
    u(\varrho) \approx C \varrho^{l + 1} + D \varrho^{-l}
\end{equation}
De mivel a $\varrho^{-l}$ felrobban $\varrho \to 0$-ban, ezért a $D$ együtthatót nullának kell vennünk, így:
\begin{equation}
    u(\varrho) \approx C \varrho^{l + 1}
\end{equation}
Most, hogy megvannak a határfeltételeink, próbáljuk meg felírni a teljes megoldást ezek alapján:
\begin{equation}
    u(\varrho) = \varrho^{l + 1} e^{-\varrho} v(\varrho)
\end{equation}
Ahol $v(\varrho)$ egy ismeretlen függvény, amit meg kell határoznunk, abban reménykedve, hogy ez egyszerűbb lesz mint a $u(\varrho)$. Kiszámolva a deriváltakat:
\begin{equation}
   \frac{du}{d \varrho} = \varrho^{l} e^{-\varrho} \left[(l + 1 - \varrho)v + \varrho \frac{dv}{d \varrho}\right]
\end{equation}
\begin{equation}
   \frac{d^2 u}{d \varrho^2} = \varrho^{l - 1} e^{-\varrho} \left\{\left[-2l - 2 + \varrho + \frac{l(l+1)}{\varrho}\right]v + 2(l + 1 - \varrho) \frac{dv}{d \varrho} + \varrho \frac{d^2 v}{d \varrho^2}\right\}
\end{equation}
Ez egyelőre nem túl bíztató, de helyettesítsük be az eredeti egyenletbe:
\begin{equation}
    \varrho^{l - 1} e^{-\varrho} \left\{\left[-2l - 2 + \varrho + \frac{l(l+1)}{\varrho}\right]v + 2(l + 1 - \varrho) \frac{dv}{d \varrho} + \varrho \frac{d^2 v}{d \varrho^2}\right\} = \left[1 - \frac{\varrho_0}{\varrho} + \frac{l(l + 1)}{\varrho^2}\right] \varrho^{l + 1} e^{-\varrho} v
\end{equation}
\begin{equation}
   \varrho \frac{d^2 v}{d \varrho^2} + 2(l + 1 - \varrho) \frac{dv}{d \varrho} + [\varrho_0 - 2(l + 1)] v = 0
\end{equation}
Végül, tegyük fel, hogy $v(\varrho)$ felírható hatványsor alakban:
\begin{equation}
    v(\varrho) = \sum_{j=0}^{\infty} c_j \varrho^j
\end{equation}
A feladatunk most az, hogy meghatározzuk a $c_j$ együtthatókat. Nézzük meg a deriváltakat:
\begin{equation}
   \frac{dv}{d \varrho} = \sum_{j=0}^{\infty} j c_j \varrho^{j - 1} = \sum_{j=0}^{\infty} (j + 1) c_{j + 1} \varrho^{j}
\end{equation}
Itt a j "dummy indexet" átnevezzük j+1-re, hogy a hatványok megegyezzenek. Ekkor fel lehetne hozni, hogy j=-1-től kéne kezdeni a szummát,
de mivel a (j + 1)=0 tag nullát adna, így ez nem számít. Ugyanezt megtehetjük a második deriválttal is:
\begin{equation}
    \frac{d^2 v}{d \varrho^2} = \sum_{j=0}^{\infty} j (j + 1) c_{j + 1} \varrho^{j - 1}
\end{equation}
visszahelyettesítve:
\begin{equation}
    \begin{aligned}
        & \sum_{j=0}^{\infty} j(j+1)c_{j+1} \varrho^{j} + 2(l + 1) \sum_{j=0}^{\infty} j(j + 1) c_{j + 1} \varrho^{j} - \\
        & - 2 \sum_{j=0}^{\infty} j c_j \varrho^{j} + [\varrho_0 - 2(l + 1)] \sum_{j=0}^{\infty} c_j \varrho^j = 0
    \end{aligned}
\end{equation}
Ez akkor lesz igaz, ha minden együttható nulla:
\begin{equation}
    j(j + 1) c_{j + 1} + 2(l + 1) (j + 1) c_{j + 1} - 2 j c_j + [\varrho_0 - 2(l + 1)] c_j = 0
\end{equation}
Ebből kifejezve az $c_{j + 1}$ együtthatót:
\begin{equation}
    c_{j + 1} = \frac{2(j + l + 1) - \varrho_0}{(j + 1)(j + 2l + 2)} c_j
\end{equation}
Ez egy rekurziós formula, amivel az összes együttható kifejezhető az első együttható segítségével. $c_0$ egy állandó a normalizálás miatt, 
és innen kiindulva az összes többi együttható kiszámolható. Most nézzük meg a sorozat viselkedését nagy j értékeknél. Ilyenkor az együttható közelítőleg:
\begin{equation}
    c_{j + 1} \approx \frac{2j}{j(j+1)}c_j = \frac{2}{j + 1} c_j
\end{equation}
vagyis:
\begin{equation}
    c_j \approx \frac{2^j}{j!} c_0
\end{equation}
Tegyük fel, hogy ez a pontos megoldás, akkor a $v(\varrho)$ függvény a következő alakot ölti:
\begin{equation}
    v(\varrho) = c_0 \sum_{j=0}^{\infty} \frac{(2 \varrho)^j}{j!} = c_0 e^{2 \varrho}
\end{equation}
Ez pedig azt jelenti, hogy a hullámfüggvényünk a következő alakot ölti:
\begin{equation}
    u(\varrho) = \varrho^{l + 1} e^{-\varrho} v(\varrho) = c_0 \varrho^{l + 1} e^{\varrho}
\end{equation}
Tehát a $u(\varrho)$ hullámfüggvény:
\begin{equation}
    u(\varrho) = c_0 \varrho^{l + 1} e^{\varrho}
\end{equation}
Ez felrobban $\varrho \to \infty$-ban, így a hullámfüggvény nem lesz normálható. Ez pontosan az az aszimptotikus viselkedés, amit el szerettünk volna kerülni.
De nem lepődhetünk meg ezen, hiszen ez az eredmény reprezentálja valamilyen aszimptotikus megoldását a radiális egyenletnek, de ezek nem normálhatóak ezért 
nem érdekelnek minket. Mindenesetre ez azt jelenti, hogy a sorozatot le kell vágni egy bizonyos $j$ értéknél, hogy a hullámfüggvény normálható legyen.
Tehát találni kell egy megoldást ahol van egy természetes $N$ szám, amire:
\begin{equation}
    c_{N - 1} \neq 0 \quad \text{de} \quad c_N = 0
\end{equation}
És ezután minden további együttható is nulla lesz. Ebből következik, hogy a rekurziós képletben a számlálónak nullának kell lennie:
\begin{equation}
    2(N + l) - \varrho_0 = 0
\end{equation}
Ahol definiálhatjuk a következő jellemzőt:
\begin{equation}
    n \equiv N + l
\end{equation}
Ekkor a következő összefüggést kapjuk:
\begin{equation}
    \varrho_0 = 2 n \quad \text{ahol } n = 1, 2, 3, \ldots
\end{equation}
Ebből vissza tudjuk számolni az energiákat:
\begin{equation}
    E = - \frac{\hbar^2 \kappa^2}{2m} = - \frac{m_e e^4}{32 \pi^2 \epsilon_0^2 \hbar^2 \varrho_0^2}
\end{equation}
Tehát a megengedett energiák:
\begin{equation}
    E_n = - \frac{m_e e^4}{32 \pi^2 \epsilon_0^2 \hbar^2} \frac{1}{n^2} = \frac{E_1}{n^2} \quad \text{ahol } n = 1, 2, 3, \ldots
\end{equation}
Ez a híres Bohr formula, a kvantummechanika egyik legfontosabb eredménye. Bohr ezt úgy határozta meg, hogy a Schrüdinger-egyenletet még nem ismerték,
és a kvantummechanika még nem létezett mint külön tudományág. Azonban a kvantummechanika megalkotásával ez a formula is igazolást nyert.
Ha pedig megnézzük a $\kappa$ változót:
\begin{equation}
    \kappa = \frac{m_e e^2}{4 \pi \epsilon_0 \hbar^2} \frac{1}{n} = \frac{1}{a_0 n}
\end{equation}
Ahol $a_0$ a Bohr sugár, ami az atomok méretét jellemzi:
\begin{equation}
    a_0 = \frac{4 \pi \epsilon_0 \hbar^2}{m_e e^2} \approx 0.529 \times 10^{-10} \text{ m}
\end{equation}
Innen pedig küvetkezik, hogy:
\begin{equation}
    \varrho = \frac{r}{a_0 n}
\end{equation}
Tehát a térbeli hullámfügvényeket három kvantumszámmal lehet jellemezni (n, l, m):
\begin{equation}
    \psi_{n l m}(r, \theta, \phi) = R_{n l}(r) Y_{l}^m(\theta, \phi)
\end{equation}
Ahol a radiális hullámfüggvény:
\begin{equation}
    R_{n l}(r) = \frac{1}{r}\varrho^{l + 1} e^{-\varrho} v(\varrho)
\end{equation}
Ahol a $v(\varrho)$ egy polinóm $n-l-1$ fokkal $\varrho$-ban, melynek az együtthatói a rekurziós képlettel számolhatók ki:
\begin{equation}
    c_{j + 1} = \frac{2(j + l + 1) - 2n}{(j + 1)(j + 2l + 2)} c_j
\end{equation}
Tehát ha vesszük a legalacsonyabb n értéket (n=1), akkor csak l=0 és m=0 lehetséges → ez az alapállapot.
Az alapállapot hullámfüggvénye tehát:
\begin{equation}
    \psi_{100}(r, \theta, \phi) = R_{10}(r) Y_0^0(\theta, \phi)
\end{equation}
Az energiája pedig:
\begin{equation}
    E_1 = - \frac{m_e e^4}{32 \pi^2 \epsilon_0^2 \hbar^2} \approx -13.6 \text{ eV}
\end{equation}
Ez határozza meg az alapállapotot, ami a "legkisebb" energiát jelöli, amivel az elektront le tudjuk választani a protonról. Ezt kötési energiának is hívjuk.
Ekkor a rekurziós formulában egyszerűsödik az első tag (hisz $j = 0$ miatt $c_1 = 0$ lesz) tehát $v(\varrho)$ egy $c_0$ konstans lesz:
\begin{equation}
    R_ {10}(r) = \frac{1}{r} \varrho^{1} e^{-\varrho} c_0 = c_0 \frac{r}{a_0} e^{-r/a_0} \frac{1}{r} = c_0 \frac{1}{a_0} e^{-r/a_0}
\end{equation}
Ha pedig normalizáljuk ezt, akkor:
\begin{equation}
   \int_{0}^{\infty} |R_{10}(r)|^2 r^2 dr = \frac{|c_0|^2}{a_0^2} \int_{0}^{\infty} e^{-2r/a_0} r^2 dr = |c_0|^2 \frac{a_0}{4} = 1
\end{equation}
Tehát $c_0 = \sqrt{\frac{4}{a_0}}$ és $Y_0^0(\theta, \phi) = \frac{1}{\sqrt{4 \pi}}$. Így az alapállapot teljes hullámfüggvénye:
\begin{equation}
    \psi_{100}(r, \theta, \phi) = \frac{1}{\sqrt{\pi a_0^3}} e^{-r/a_0}
\end{equation}
Ez egy gömbszimmetrikus hullámfüggvény, ami az origó körül koncentrálódik, és exponenciálisan csökken a távolsággal. Ez megfelel annak a fizikai képnek, hogy az elektron a proton körül kering,
és a legvalószínűbb helye az origó közelében van.

Ha $n = 2$ a főkvantumszám, tehát az első gerjesztett állapotban vagyunk, akkor lehet $l = 0$ (ekkor $m = 0$) vagy $l = 1$ (ekkor $m = -1, 0, 1$).
Tehát az $n = 2$ állapotnak összesen 4 féle konfigurációja van, ami az állapot degenerációját jelenti. Az energiája pedig:
\begin{equation}
    E_2 = \frac{E_1}{2^2} = \frac{E_1}{4} \approx -3.4 \text{ eV}
\end{equation}
Ekkor az együtthatók a rekurziós képlet alapján kiszámolhatók, és a radiális hullámfüggvények is meghatározhatók. Például ha $l = 0$ és $m = 0$, akkor:
\begin{equation}
    c_1 = \frac{2(0 + 0 + 1) - 4}{(0 + 1)(0 + 0 + 2)} c_0 = - \frac{2}{2} c_0 = - c_0
\end{equation}
\begin{equation}
    c_2 = 0 \quad \text{(mivel a sorozatot le kell vágni $N= n - l = 1$-nél)}
\end{equation}
Szóval $v(\varrho) = c_0 (1 - \varrho)$, tehát:
\begin{equation}
    R_{20}(r) = \frac{c_0}{2a} (1 - \frac{r}{2a_0}) e^{-r/2a_0}
\end{equation}
Ha pedig $l = 1$ és $m = 0$, akkor:
\begin{equation}
    c_1 = \frac{2(0 + 1 + 1) - 4}{(0 + 1)(0 + 2 + 2)} c_0 = \frac{0}{4} c_0 = 0
\end{equation}
\begin{equation}
    c_2 = 0 \quad \text{(mivel a sorozatot le kell vágni $N= n - l = 1$-nél)}
\end{equation}
Szóval $v(\varrho) = c_0$, tehát:
\begin{equation}
    R_{21}(r) = \frac{c_0}{4 a_0^2} r e^{-r/2a_0}
\end{equation}
A $c_0$ minden esetben normalizálással határozható meg. Láthatjuk tehát, hogy egy adott $n$ értékre $l = 0, 1, 2, \dots, n-1$ lehetséges, és minden $l$ értékre $m = -l, -l+1, \dots, l-1, l$ lehetséges.
Tehát egy adott energiaszint degenerációja:
\begin{equation}
    d(n) = \sum_{l=0}^{n-1} (2l + 1) = n^2
\end{equation}
\begin{figure}[H]
    \centering
    \includegraphics[width=0.6\textwidth]{imgs/3-tetel/hidrogen_atom_energiak.png}
    \caption{A Hidrogén atom energiaszintjei és hullámfüggvényei; $n = 1$ az alapállapot $E_1 = -13.6 \text{ eV}$ enegiával;
    Végtelen számú állapot fér be $n = 5$ és $n = \infty$ közé; $E_\infty = 0$ választja el a kötött és szabad állapotokat.}
\end{figure}

A $v(\varrho)$ polinóm amit használtunk a megoldásban a matematikusok számára már jól ismert asszociált Laguerre polinómok normalizálás nélkül:
\begin{equation}
    v(\varrho) = L_{n - l - 1}^{2l + 1}(2\varrho)
\end{equation}
Ahol:
\begin{equation}
    L_p^q(x) = (-1)^q \left(\frac{d}{dx}\right)^q L_{p + q}(x)
\end{equation}
az Asszociált Laguerre polinóm és
\begin{equation}
    L_q(x) = \frac{e^x}{q!} \left(\frac{d}{dx}\right)^q (x^q e^{-x})
\end{equation}
A $q$-edik Laguerre polinóm. Innen a normalizált radiális hullámfüggvénye a Hidrogén atomnak:
\begin{equation}
    \psi_{n l m} = \sqrt{\left(\frac{2}{n a_0}\right)^3 \frac{(n - l - 1)!}{2 n [(n + l)!]^3}} e^{-r/na_0} \left(\frac{2r}{n a_0}\right)^l L_{n - l - 1}^{2l + 1} \left(\frac{2r}{n a_0}\right) Y_l^m(\theta, \phi)
\end{equation}
Ami nem egy túl szép megoldás, de a kevés, zárt alakban megoldható valós rendszer megoldásának egyike. A hullámfüggvények kölcsönösen ortogonálisak és normálhatók:
\begin{equation}
    \int \psi_{n' l' m'}^* (r, \theta, \phi) \psi_{n l m} (r, \theta, \phi) r^2 \sin(\theta) dr d\theta d\phi = \delta_{n n'} \delta_{l l'} \delta_{m m'}
\end{equation}
Ez abból következik, hogy a gömbfüggvények ortogonálisak és ($n \neq n'$) és abból, hogy ezek a Hamilton-operátor sajátfüggvényei különböző sajátértékekhez.

\subsubsection*{Hidrogén atom spektruma}
Elméletileg, ha a hidrogén atomot egy $\psi_{n l m}$ állapotba rakjuk, akkor ott is marad örökké. Ha viszont egy kicsit "megpiszkáljuk" az atomot (pl. egy fotonnal ütköztetjük),
akkor az atom gerjesztett állapotba kerülhet ($n > 1$) vagy visszaeshet egy alacsonyabb állapotba. Ezek a perturbációk a valóságban mindig jelen vannak, így az atom sosem marad egy állapotban,
időnként feljegb ugrik egyet majd visszaesik (ezeket kvantum ugrásoknak is nevezik). Amikor visszaesik egy alacsonyabb állapotba, akkor meg kell szabadulnia a fölösleges energiától,
amit általában elektromágneses sugárzás formájában tesz meg (foton kibocsátás). Ennek a kibocsájtott fotonnak pontosan annyi lesz az energiája, mint a két állapot közötti energiakülönbség:
\begin{equation}
    E_{\text{foton}} = E_i - E_f = -13.6 \text{ eV} \left(\frac{1}{n_i^2} - \frac{1}{n_f^2}\right)
\end{equation}
Azt pedig a Planck összefüggésből tudjuk, hogy a foton energiája és frekvenciája között a következő összefüggés áll fenn:
\begin{equation}
    E_{\text{foton}} = h \nu
\end{equation}
És a hullámhossz és a frekvencia között pedig:
\begin{equation}
    c = \lambda \nu
\end{equation}
Ebből következik, hogy a kibocsájtott foton hullámhossza:
\begin{equation}
    \frac{1}{\lambda} = \mathcal{R} \left(\frac{1}{n_f^2} - \frac{1}{n_i^2}\right)
\end{equation}
Ahol $\mathcal{R}$ a Rydberg állandó, ami a következő alakot ölti:
\begin{equation}
    \mathcal{R} \equiv \frac{m_e}{4 \pi c \hbar^3} \left(\frac{e^2}{4 \pi \epsilon}\right)^2 \approx 1.097 \times 10^7 \text{ m}^{-1}
\end{equation}
Ez pedig a híres Rydberg formula, amit empirikus úton fedeztek fel a hidrogén spektrumának vizsgálatakor. Itt $n_i$ az induló állapot főkvantumszáma,
és $n_f$ a végállapot főkvantumszáma. Különböző $n_f$ értékekhez különböző sorozatok tartoznak:
\begin{itemize}
    \item Lyman sorozat: $n_f = 1$, ultraibolya tartomány
    \item Balmer sorozat: $n_f = 2$, látható tartomány
    \item Paschen sorozat: $n_f = 3$, infravörös tartomány
    \item Brackett sorozat: $n_f = 4$, infravörös tartomány
    \item Pfund sorozat: $n_f = 5$, infravörös tartomány
    \item Humphreys sorozat: $n_f = 6$, infravörös tartomány
\end{itemize}
\begin{figure}[H]
    \centering
    \includegraphics[width=0.6\textwidth]{imgs/3-tetel/hidrogen_atom_spektrum.png}
    \caption{A Hidrogén atom spektrumának Energia szintjei és átmenetei.}
\end{figure}

\section{Folytonos közegek mechanikája}
Rugalmas és képlékeny alakváltozások, Hooke-törvény, speciális deformációk. A deformáció jellemzése, feszültség- és deformációs tenzor.
Folyadékok tulajdonságai, hidrosztatika, felületi feszültség, görbületi nyomás, felhajtóerő. Áramlások jellemzése, Bernoulli-egyenlet,
tökéletes folyadék áramlása, Euler-egyenletek, viszkózus folyadék áramlása, örvények, turbulencia, Reynolds-szám.
\newline
\newline
A folytonos közegek mechanikája a mechanika egy olyan ága, amely a szilárd anyagok és folyadékok viselkedését vizsgálja,
amikor külső erők hatnak rájuk. Ez a terület magában foglalja a rugalmas és képlékeny alakváltozásokat,
a folyadékok tulajdonságait, valamint az áramlások jellemzését. A "folytonos közeg" arra a határesetre utal,
amikor már nem csak 1-2 tömegpont viselkedését vizsgálom, hanem egy olyan rendszert, ahol a részecskék száma olyan nagy,
hogy az anyagot folytonosnak közelíthetjük. Ez lehetővé teszi számunkra, hogy makroszkopikus mennyiségekkel dolgozzunk,
mint például a sűrűség, nyomás és sebességmezők, ahelyett, hogy minden egyes részecskét külön-külön vizsgálnánk. A folytonos közegek mechanikája
elválaszthatatlan a statisztikus fizikától és a termodinamikától, mivel ezek a területek segítenek megérteni, hogyan viselkednek a nagy számú részecskékből álló rendszerek.

A folytonos közegek mechanikájának legfontosabb mértékegységei a következők:
\begin{itemize}
    \item Nyomás (Pascal, Pa): A nyomás az erő egységnyi felületre vetített értéke. 1 Pascal az az erő, amely 1 négyzetméter felületre 1 Newton erőt fejt ki.
    \item Sűrűség (kilogramm per köbméter, kg/m³): A sűrűség az anyag tömegének és térfogatának aránya. Ez megmutatja, hogy mennyi anyag van egy adott térfogatban.
    \item Sebesség (méter per másodperc, m/s): A sebesség a folyadék vagy gáz részecskéinek mozgási sebességét jelenti.
    \item Viskozitás (Pascal-másodperc, Pa·s): A viszkozitás a folyadék vagy gáz belső súrlódását jelenti, amely ellenáll az áramlásnak.
    \item Felületi feszültség (Newton per méter, N/m): A felületi feszültség a folyadék felületén lévő molekulák közötti kohéziós erőt jelenti.
\end{itemize}

\subsection{Rugalmas és képlékeny alakváltozások, Hooke-törvény, speciális deformációk}
Eddig a pontig főként részecskék és merev testek mozgását vizsgáltuk. Azonban a valóságban sok anyag képes alakváltozásra,
amikor külső erők hatnak rájuk. Tehát egy kiterjedt test esetén nem csak a test egészének mozgását kell figyelembe venni,
hanem azt is, hogy a test pontjainak relatív távolsága megváltozásra képes. Az alakváltozások két fő típusa a rugalmas és a képlékeny alakváltozás.
A rugalmas alakváltozás során az anyag visszatér eredeti alakjához, amikor a külső erő megszűnik.
Ezzel szemben a képlékeny alakváltozás során az anyag maradandó alakváltozást szenved, és nem tér vissza eredeti formájához.

A deformáció jellemzéséhez bevezethetjük a feszültség és deformáció fogalmát. A feszültség az anyagra ható erő egységnyi felületre vetített értéke ($Pa = \frac{N}{m^2}$ - Pascal),
a deformáció pedig az anyag alakjának megváltozását jelenti. Nézzük meg először a lineáris deformációt, ahol egy rúdra ható F erőre deformálódik:
\begin{figure}[H]
    \centering
    \includegraphics[width=0.4\textwidth]{imgs/4-tetel/linearis_deformacio.png}
    \caption{lineáris deformáció egy rúd esetén}
\end{figure}
Ahol a rúd $F$ erő hatására $\Delta L$ hosszváltozást szenved el. Ha rugalmas a deformáció (az erő nem elég erős,
hogy az atomok helyzetét permanensen megváltoztassa), akkor a következő grafikont rajzolhatjuk fel:
\begin{figure}[H]
    \centering
    \includegraphics[width=0.6\textwidth]{imgs/4-tetel/linearis_deformacio_grafikon.png}
    \caption{lineáris deformáció grafikon}
\end{figure}
Ahol a feszültség ($\sigma$) és a deformáció ($\varepsilon$) között lineáris összefüggés áll fenn:
\begin{equation}
    \sigma = E \varepsilon
\end{equation}
Ahol $E$ az anyagra jellemző Young modulus ($E \sim 100 \text{GPa}$), ami a rugalmasság mértékét jellemzi. Ezt nevezik Hooke-törvénynek.
A deformáció a lineáris deformáció esetében a relatív hosszváltozás, a feszültség pedig az erő és a keresztmetszeti terület hányadosa:
\begin{equation}
    \varepsilon = \frac{\Delta L}{L} \quad \text{és} \quad \sigma = \frac{F}{A}
\end{equation}
Ezt a jelenséget egyszerű nyújtásnak nevezik. Ilyen nyúlás esetében megfigyelhető, hogy a rúd átmérője is csökken a hossz növekedésével.
Ezt harántösszehúzásnak is nevezzük, és előszőr Poisson jellemezte matematikailag:
\begin{equation}
   \frac{\Delta d}{d} = - \nu \frac{\Delta L}{L}
\end{equation}
Ahol $\nu$ a Poisson arány, ami anyagra jellemző állandó ($\nu \sim 0.3$). A poisson szám lehet negatív is, speciális
formájú anyagoknál (auxetikus anyagok), ahol a harántirányú méretnövekedés is megfigyelhető nyújtáskor.
\begin{figure}[H]
    \centering
    \includegraphics[width=0.6\textwidth]{imgs/4-tetel/auxetikus_anyag.png}
    \caption{Auxetikus anyag viselkedése nyújtáskor}
\end{figure}
Ha a Hooke-törvényt kiterjesztjük 3D-re akkor a hossz helyett a térfogatváltozást kell figyelembe venni:
\begin{equation}
    \frac{\Delta V}{V} = \frac{(l + \Delta l)^2 (d + \Delta d)^2 - ld^2}{ld^2} = \frac{\Delta l (d + \Delta d)^2 + l 2d \Delta d + l \Delta d^2}{ld^2}
\end{equation}
\begin{equation}
    \sim \frac{\Delta l d^2 + 2 l d \Delta d}{ld^2} = \frac{\Delta l}{l} + 2 \frac{\Delta d}{d}
\end{equation}
Ezt a közelítést azért tehettem meg, mert tudom, hogy a $\Delta l$ és $\Delta d$ között kb. $10^{-3}$ nagyságrendű különbség van, vagyis kb. hasonlóak.
Helyettesítsük be ide a Poisson arányt:
\begin{equation}
    \frac{\Delta V}{V} = \frac{\Delta l}{l} - 2 \nu \frac{\Delta l}{l} = (1 - 2 \nu) \frac{\Delta l}{l} = (1 - 2 \nu) \frac{\sigma}{E}
\end{equation}
És ha itt bevezetjük a $P$ nyomást ($P = - \sigma$), akkor a következő összefüggést kapjuk:
\begin{equation}
    \frac{\Delta V}{V} = - 3 \frac{(1 - 2 \nu)}{E} P
\end{equation}
Itt a 3-as azért jelent meg, mert a térfogatváltozás három irányban történik. Ebből következik a kompressziós modulus ($K$):
\begin{equation}
    K \equiv - P \frac{V}{\Delta V} = \frac{E}{3(1 - 2 \nu)}
\end{equation}
Ez a modulus azt jellemzi, hogy egy anyag mennyire ellenálló a térfogati változásokkal szemben. Minél nagyobb az értéke, annál kevésbé hajlamos az anyag a térfogati deformációra külső nyomás hatására.
Ennek a reciprokát nevezik a kompresszibilitásnak ($\kappa$):
\begin{equation}
    \kappa \equiv \frac{1}{K} = - \frac{1}{V} \frac{\Delta V}{P}
\end{equation}
Ez a mennyiség azt jellemzi, hogy egy anyag mennyire könnyen deformálódik térfogatváltozás szempontjából külső nyomás hatására. Ebből a
kompressziós modulusból leolvasható, hogy a stabilitás feltétele, az hogy $K > 0$.

\subsubsection{Speciális deformációk}
\subsubsection*{Nyíró deformáció}
A Hooke-törvény és a feszültség-deformáció összefüggések csak akkor érvényesek, ha test oldalainak közepére hat az erő.
Ha viszont mondjuk az éleket húzzuk, akkor nyíró feszültség lép fel:
\begin{figure}[H]
    \centering
    \includegraphics[width=1\textwidth]{imgs/4-tetel/nyiro_deformacio.png}
    \caption{Nyíró deformáció}
\end{figure}
Ilyenkor a feszültség és a deformáció között a következő összefüggés áll fenn:
\begin{equation}
    \tau = \frac{F}{A} = \mu \gamma
\end{equation}
Ahol $\tau$ a nyíró feszültség, $\gamma$ a nyírási szög (kis szögnél $\gamma \approx \tan(\gamma)$), és $\mu$ a nyírási modulus (olykor $G$-vel is jelölik).

\subsubsection*{Csavarás}
Vegyünk egy henger alakú rudat, amit az egyik végénél rögzítünk, a másik végénél pedig egy $M$ nyomatékot alkalmazunk:
\begin{figure}[H]
    \centering
    \includegraphics[width=0.6\textwidth]{imgs/4-tetel/csavaras.png}
    \caption{Csavarás}
\end{figure}

Ahol a rúd hossza $L$, a sugara $r$, és a nyomaték $M$. Ekkor a rúd egy $\alpha$ szöggel elfordul a nyomaték hatására. Ekkor vegyünk egy kis téglalap szeletet
a henger oldalából és nézzük meg, hogy hogyan fordul el:
\begin{figure}[H]
    \centering
    \includegraphics[width=0.6\textwidth]{imgs/4-tetel/csavaras_2.png}
    \caption{Csavarás - kis szelet elmozdulása}
\end{figure}

Ekkor $\alpha$ és $\gamma$ között a következő összefüggés áll fenn:
\begin{equation}
    \tan(\gamma) = \frac{r \alpha}{L} \approx \gamma
\end{equation}
Az ezt az elmozdsulást okozó erő legyen $\Delta F$:
\begin{equation}
    \Delta F = \tau \Delta A = \mu \gamma \Delta r r \Delta \phi = \mu \frac{\alpha}{L} r^2 \Delta r \Delta \phi
\end{equation}
Ahol $\Delta A = r \Delta \phi \Delta r$ a kis felület, $\tau$ pedig a nyírási feszültség. Ebből pedig a nyomaték:
\begin{equation}
    \Delta M = \Delta F r = \mu \frac{\alpha}{L} r^3 \Delta r \Delta \phi
\end{equation}
Ezt össze kell adni minden kis felületre, tehát integrálni kell $r$ és $\phi$ szerint:
\begin{equation}
    M = \int_0^{2\pi} \int_0^R \mu \frac{\alpha}{L} r^3 dr d\phi = \mu \frac{\alpha}{L} 2 \pi \frac{R^4}{4} = \frac{\pi \mu R^4}{2L} \alpha
\end{equation}
Itt a konstans tagokra bevezethetünk egy új mennyiséget, a torziós moduluszt ($D^*$):
\begin{equation}
    D^* = \frac{\pi \mu R^4}{2L}
\end{equation}
Ekkor a nyomaték és az elfordulás között a következő összefüggés áll fenn:
\begin{equation}
    M = D^* \alpha
\end{equation}

\subsubsection*{Hajlítás}
Vegyünk egy kezdetben egyenes rúd alakú testet, aminek az egyik vége rögzítve van, a másik végére pedig egy erőt alkalmazunk:
\begin{figure}[H]
    \centering
    \includegraphics[width=0.6\textwidth]{imgs/4-tetel/hajlitas.png}
    \caption{Hajlítás}
\end{figure}
Ha megnézem a neutrális zónától (A rúd középvonala, kis Hajlítás esetén nem szenved hosszváltozást) $\xi$ távolságra lévő réteget, akkor ott a hosszváltozás:
\begin{equation}
    \epsilon_{xx} (\xi) = \frac{\Delta L}{L} = \frac{(R + \xi) \Delta \varphi - R \Delta  \varphi}{R \Delta \varphi} = \frac{\xi}{R}
\end{equation}
Ahol $R$ a görbületi sugár, és $\Delta \varphi$ a kis szög, amivel elfordul a rúd. Ebből a feszültség:
\begin{equation}
    \sigma_{xx} (\xi) = E \epsilon_{xx} (\xi) = E \frac{\xi}{R}
\end{equation}
Ahol $E$ a Young modulus ($\frac{1}{E} = \frac{1}{2\mu}\left(1 - \frac{\lambda}{2\mu + 3\lambda}\right)$). Ebből pedig a nyomaték:
\begin{equation}
    M = \int_A \sigma_{xx} (\xi) \xi dA = \int_A E \frac{\xi}{R} \xi dA = \frac{E}{R} \int_A \xi^2 dA
\end{equation}
Ahol az integrál a keresztmetszetre vonatkozik. Ebből következik, hogy:
\begin{equation}
    M = \frac{E I}{R}
\end{equation}
Ahol $I = \int_A \xi^2 dA$ a keresztmetset tehetetlenségi nyomatéka. Ebből a görbületi sugár:
\begin{equation}
    \frac{1}{R} = \frac{M}{E I}
\end{equation}
Ha a rúd hossza $L$, akkor a rúd végén a hajlítási szög:
\begin{equation}
    \alpha = \frac{L}{R} = \frac{M L}{E I}
\end{equation}

\subsection{A deformáció jellemzése, feszültség- és deformációs tenzor}
Deformáció esetén egy pont elmozdulását jellemezhetjük egy elmozdulási vektorral ($\bold{u}(\bold{r})$) ami az adott pont helyzetétől ($\bold{r}$) függ.
Viszont van egy probléma ezzel a megközelítéssel: ha csak az elmozdulási vektort nézzük, akkor nem tudjuk megkülönböztetni a test elforgatását és a deformációját.
Például egy test elforgatása nem jelent deformációt, de az elmozdulási vektor megváltozik. Ezért be kell vezetnünk egy olyan mennyiséget, ami csak a deformációt jellemzi.
Vegyünk tehát egymáshoz közel két pontot a testben ($\bold{r}$ és $\bold{r} + d\bold{r}$). Ezeknél az elmozdulási vektorok 
$\bold{u}(\bold{r})$ és $\bold{u}(\bold{r} + d\bold{r})$ lesznek. A kérdés pedig az, hogy az elmozdulási vektorok relatív távolsága egymáshoz képest hogyan változik.
\begin{figure}[H]
    \centering
    \includegraphics[width=0.6\textwidth]{imgs/4-tetel/eltolas_vektor.png}
    \caption{Az eltolás vektor változása két közeli pont között}
\end{figure}
Legyen a távolság $\bold{u}(\bold{r})$ és $\bold{u}(\bold{r} + d\bold{r})$ között $\Delta \bold{s}$. Ekkor a vektor:
\begin{equation}
    \Delta \bold{s} = |\bold{r} + \Delta \bold{r} + \bold{u}(\bold{r} + d\bold{r}) - \bold{r} - \bold{u}(\bold{r})|
\end{equation}
\begin{equation}
  \Delta \bold{s}  = |d\bold{r} + \bold{u}(\bold{r} + d\bold{r}) - \bold{u}(\bold{r})|
\end{equation}
Használjuk ki, hogy a potenciált kis távolságok esetén fel tudom írni, hogy:
\begin{equation}
    \phi (\bold{r} + \Delta \bold{r}) \approx \phi (\bold{r}) + grad( \phi (\bold{r})) \cdot d\bold{r}
\end{equation}
Ekkor tehát az elmozdulási vektor:
\begin{equation}
    \bold{u}_1(\bold{r} + d\bold{r}) \approx \bold{u}_1(\bold{r}) + \frac{\partial u_1}{\partial r_i} \Delta r_i
\end{equation}
Tehát általános komponensre az u-nak:
\begin{equation}
    u_j(\bold{r} + d\bold{r}) \approx u_j(\bold{r}) + \frac{\partial u_j}{\partial r_i} \Delta r_i
\end{equation}
Ezzel tehát egy deriváltal tudjuk közelíteni a relatív elmozdulást. Ezt a deriváltat helyettesíthetjük egy tenzorral:
\begin{equation}
    \beta_{ij} \equiv \frac{\partial u_j}{\partial r_i}
\end{equation}
Tehát a tenzor:
\begin{equation}
    \hat{\beta} = \frac{d\bold{u}}{\bold{r}}
\end{equation}
\begin{equation}
    \Delta \bold{u} = \hat{\beta} \Delta \bold{r}
\end{equation}
Tehát ezt behelyettesítve a $\Delta \bold{s}$ kifejezésbe:
\begin{equation}
    \Delta \bold{s} = \sqrt{(\Delta \bold{r}_i + \hat{\beta}_{ij} \Delta \bold{r}_j) \cdot (\Delta \bold{r}_i + \hat{\beta}_{ik} \Delta \bold{r}_k)}
\end{equation}
Itt a szorzatban a két fél másik összegzés alatt van, ezért van két külön index használva. Ezt kifejtve:
\begin{equation}
    \Delta \bold{s} = \sqrt{\Delta r_i \Delta r_i + \Delta r_i \beta_{ij} \Delta r_j + \Delta r_i \beta_{ik} \Delta r_k + \beta_{ij} \Delta r_j \beta_{ik} \Delta r_k}
\end{equation}
A gyök alatti utolsó tagot elhagyhatom, mert maximum $10^{-3}$ nagyságrendű deformációt akarok megengedni, és ezért a $\Delta r_i \cdot \Delta r_i$ taghoz képest is
$10^{-3}$ lesz, vagyis összességben elhanyagolhatóan kicsi. A másik két $\beta$-s tagra is igaz ez, de mivel ezeket akarom kiszámolni, ezért nem szerencsés "eldobni" őket.
Az egyenlet tehát:
\begin{equation}
    \Delta \bold{s} = \sqrt{\Delta r_i \Delta r_i + 2 \Delta r_i \beta_{ij} \Delta r_j}
\end{equation}
Itt egy másik egyszerűsítést is alkalmaztunk, mégpedig, hogy nem számít hogy milyen sorrendben összegzem és szorzom össze a tagokat, hisz skaláris mennyiségekről van szó,
ezért akár ugyan azt az indexet is használhatom az összegzéshez. Itt észrevehetjük, hogy a $\beta_{ij}$ tenzor nem szimmetrikus, hisz a $\Delta r_i \beta_{ij} \Delta r_j$ kifejezésben a $i$ és $j$ indexek felcserélhetők. Tehát bevezethetjük a szimmetrikus és antiszimmetrikus részt:
\begin{equation}
    \beta_{ij} = \underbrace{\frac{1}{2} (\beta_{ij} + \beta_{ji})}_{\varepsilon_{ij} \text{ deformációs tenzor}} + \underbrace{\frac{1}{2} (\beta_{ij} - \beta_{ji})}_{\omega_{ij} \text{ forgatási tenzor}}
\end{equation}
Ahol az első tag a deformációs tenzor, a második pedig a forgatási tenzor. A második tag nem érdekel minket, hisz az csak a test elforgatását jellemzi, és nem fog erőt létrehozni. Tehát a deformációs tenzor:
\begin{equation}
    \varepsilon_{ij} = \frac{1}{2} (\beta_{ij} + \beta_{ji}) = \frac{1}{2} \left(\frac{\partial u_j}{\partial r_i} + \frac{\partial u_i}{\partial r_j}\right)
\end{equation}
Tehát a $\Delta \bold{s}$ kifejezésbe visszahelyettesítve:
\begin{equation}
    \Delta \bold{s} = \sqrt{\Delta r_i \Delta r_i + 2 \Delta r_i \varepsilon_{ij} \Delta r_j} = |\Delta \bold{r}| \sqrt{1 + 2 \frac{\Delta \bold{r} \hat{\varepsilon} \Delta \bold{r}}{|\Delta \bold{r}|^2}}
\end{equation}
\begin{equation}
    \Delta \bold{s} = |\Delta \bold{r}| \sqrt{1 + 2 \bold{n} \cdot \hat{\varepsilon} \cdot \bold{n}}
\end{equation}
Ahol $\bold{n} = \frac{\Delta \bold{r}}{|\Delta \bold{r}|}$ az egységvektor. Mivel a második tag kicsi, ezért a gyök alatti kifejezést közelíthetjük:
\begin{equation}
    \Delta \bold{s} \approx |\Delta \bold{r}| \left(1 + \bold{n} \cdot \hat{\varepsilon} \cdot \bold{n}\right)
\end{equation}
Tehát a relatív távolságváltozás két közeli pont között:
\begin{equation}
    \frac{\Delta \bold{s} - \Delta \bold{r}}{|\Delta \bold{r}|} = \bold{n} \cdot \hat{\varepsilon} \cdot \bold{n}
\end{equation}
Ezt az $\hat{\varepsilon}$ tenzort nevezzük deformációs tenzornak, ami a test deformációját jellemzi.

Most térjünk egy kicsit vissza a $\hat{\beta}$ tenzorhoz. Ebből kiindulva meg tudjuk határozni, hogy hogyan változnak a test egyes részei a deformáció során. Vegyünk egy kis elmozdulást a testben $\Delta \bold{r}$, ami a test egy pontjától egy másik pontjához vezet. A deformáció után ez az elmozdulás megváltozik, és legyen ez $\Delta \bold{r'}$. Ekkor a következő összefüggés áll fenn:
\begin{equation}
    \Delta \bold{r'} = \Delta \bold{r} + \hat{\beta} \Delta \bold{r}
\end{equation}
Vagy komponensenként:
\begin{equation}
    \Delta x' = \Delta x + \beta \Delta x
\end{equation}
\begin{equation}
    \Delta y' = \Delta y + \beta \Delta y
\end{equation}
\begin{equation}
    \Delta z' = \Delta z + \beta \Delta z
\end{equation}
Mivel a $\Delta x$-nek csak az $x$ komponense nem nulla, ezért fel tudom írni a következőképpen:
\begin{equation}
    \Delta x' = (\Delta x + \beta_{11} \Delta x, \beta_{21} \Delta x, \beta_{31} \Delta x) = (1 + \beta_{11}, \beta_{21}, \beta_{31}) \Delta x
\end{equation}
\begin{equation}
    \Delta y' = (\beta_{12} \Delta y, \Delta y + \beta_{22} \Delta y, \beta_{32} \Delta y) = (\beta_{12}, 1 + \beta_{22}, \beta_{32}) \Delta y
\end{equation}
\begin{equation}
    \Delta z' = (\beta_{13} \Delta z, \beta_{23} \Delta z, \Delta z + \beta_{33} \Delta z) = (\beta_{13}, \beta_{23}, 1 + \beta_{33}) \Delta z
\end{equation}
Tehát a térfogatváltozás a deformáció után:
\begin{equation}
    \Delta V' = \begin{vmatrix}
        (1 + \beta_{11}) & \beta_{12} & \beta_{13} \\
        \beta_{21} & (1 + \beta_{22}) & \beta_{23} \\
        \beta_{31} & \beta_{32} & (1 + \beta_{33})
    \end{vmatrix} \underbrace{\Delta x \Delta y \Delta z}_{\Delta V}
\end{equation}
Ahol a kapott mátrix determinánsának meghatórázásával tudjuk, hogy a térfogat hogyan változik a deformáció során. Itt gyakorlatilag a tenzo segítségével az $x,y,z$ koordináta rendszerrőé egy másik rendszerbe visszük át.
Ezt kiszámolva:
\begin{equation}
    \Delta V' = (1 + \beta_{11}) + (1 + \beta_{22}) + (1 + \beta_{33}) + \beta^3 \dots  \Delta V
\end{equation}
\begin{equation}
    \Delta V' \approx \left(1 + \beta_{11} + \beta_{22} + \beta_{33}\right) \Delta V
\end{equation}
Itt a $\beta^3$ jelöli a harmadfokú tagokat, amiket elhanyagolhatunk, hisz kicsik. Tehát a relatív térfogatváltozás:
\begin{equation}
    \frac{\Delta V' - \Delta V}{\Delta V} = \beta_{11} + \beta_{22} + \beta_{33} = Tr(\hat{\beta}) = Tr(\hat{\varepsilon})
\end{equation}
Ahol a $Tr$ a tenzor nyomát jelöli (magyarul $Sp$ - spur-nak is hívják). Tehát a deformációba csak a diagonális elemek számítanak a térfogatváltozás szempontjából.
A vegyes tagok a tiszta nyírást jellemzik, ami nem változtatja meg a térfogatot. Tehát a deformáció felfogható egy koordináta-rendszerváltásként is, ahol a deformációs tenzor visz át a régi koordinátarendszerből az újba.

\subsubsection*{Feszültség tenzor}
A deformációs tenzornál a rendszerben keletkező errőkkel nem foglalkoztunk, csakis az alakváltozásra koncentáltunk. De ahoz, hogy a Newtoni formalizmusnak meg
tudjunk felelni (tehát, hogy "megtaláljuk" a deformáció $F = ma$ egyenletét) szükségünk van egy olyan mennyiségre, ami az erőket jellemzi. 
Amikor felveszünk egy pontrendszert, akkor mondhatjuk a tömegközépont gyorsulásának meg kell egyeznie a rá ható erők külső erők összegével.
Viszont ha felveszünk egy pontrendszert a pontrenceren belül akkor arra is igaznak kell lennie, viszont erre a kisebb testre hatnak belső erők is,
melyek a test szempontjából külső erők. Viszont szerencsére ezek a belső erők nagyon kis távon ($\sim \frac{1}{r^6}$) hatnak csak általában (Pl. Plazmában már nem lesz igaz ez).
\begin{figure}[H]
    \centering
    \includegraphics[width=0.6\textwidth]{imgs/4-tetel/felulet_integral.png}
    \caption{Belső erők számolása egy test szelete körül}
\end{figure}
Ezért elég lesz egy kis felületdarabkán összeadni az erőket a test körül:
\begin{equation}
    \Delta \bold{A}: \text{ felületdarab} \quad \quad \Delta \bold{F}: \text{erő a felületen}
\end{equation}
Ha kicsit kisebb felületdarabot veszek akkor kevesebb, ha nagyobb felületdarabot veszek akkor több erő lesz rajta, ezért az erő és a felületdarab között arányosságot várhatunk:
\begin{equation}
    \Delta \bold{F} \sim \Delta \bold{A}
\end{equation}
Viszont $\Delta \bold{A}$ és $\Delta \bold{F}$ vektorok nem biztos hogy páhuzamosak, ezért az arrányossági tényező nem biztos hogy skaláris lesz, hanem egy tenzor:
\begin{equation}
    \Delta \bold{F} = \hat{\sigma} \Delta \bold{A}
\end{equation}
Ezt a tenzort feszültség tenzornak nevezzük, és a testben keletkező erőket jellemzi. De erre a felüledarabra is hat a külső erő, ezért azt is figyelembe kell venni:
\begin{equation}
    \bold{f}: \text{Erősűrűség - egységnyi térfogatra ható külső erő}
\end{equation}
Ekkor a test egy kis térfogatára ható erő:
\begin{equation}
    \Delta \bold{F}_{\text{külső}} = \bold{f} \Delta V
\end{equation}
Mostmár feltudjuk írni a teljes erőt ami a kis térfogatra hat, és azt összeadva megkapom a teljes erőt a kis testdarabra:
\begin{equation}
    \bold{F}' = \int_{V'} \bold{f} dV + \oint_{F'} \hat{\sigma} \cdot d\bold{A}
\end{equation}
Ennek pedig a newtoni mozgásegyenlet szerint meg kell egyeznie az összimpulzus deriváltjával:
\begin{equation}
    \bold{F}' = \int_{V'} \bold{f} dV + \oint_{F'} \hat{\sigma} \cdot d\bold{A} = \frac{d}{dt} \int \varrho \bold{\dot{u}} dV
\end{equation}
Ezzel van egy kis probléma, mégpedig, hogy ahogy a észecskék elmozdulnak az impulzus miatt, már nem ugyanabban a felületben lesznek mint amire számolok.
Ezt a problémát most úgy hidaljuk át, hogy azt mondjuk, hogy kis deformációkat nézünk csak, így a felület nem változik jelentősen. Ekkor a következő egyenletet kapjuk:
\begin{equation}
     \int_{V'} \bold{f} dV + \oint_{F'} \hat{\sigma} \cdot d\bold{A} =\frac{d}{dt} \int \varrho \bold{\dot{u}} dV \approx \int_{V'} \varrho \bold{\ddot{u}} dV
\end{equation}
Ez az egyenlet pedig a folytonos közegek mozgásegyenlete, ami a test egy kis térfogatára vonatkozik:
\begin{equation}
    \int_{V'} \bold{f} dV + \oint_{F'} \hat{\sigma} \cdot d\bold{A} = \int_{V'} \varrho \bold{\ddot{u}} dV \quad \Longleftrightarrow \quad \bold{F} = m \bold{a}
\end{equation}
Ezt általánosan nem lehet megoldani, de speciális esetekben igen. Ebben az egyenletben a $\varrho$ sűrűség "játsza" a tömeget, a $\bold{\ddot{u}}$ a gyorsulást, 
a $\bold{f}$ a külső erőt, és a feszültség tenzor pedig a belső erőket jellemzi.

Még egy érdekességet felfedezhetünk, hogy ha megvizsgáljuk az elektromágnesességből ismert Gauss törvényt:
\begin{equation}
    \oint_{F'} \bold{E} \cdot d\bold{A} = \frac{Q}{\varepsilon_0} = \frac{1}{\varepsilon_0} \int_{V'} \varrho dV
\end{equation}
Vagy máshogy megfogalmazva:
\begin{equation}
    \int (div \bold{E}) dV = \frac{1}{\varepsilon_0} \int \varrho dV \quad \rightarrow \quad div \bold{E} = \frac{\varrho}{\varepsilon_0}
\end{equation}
Ugyanis ebben az egyenletben is erősűség és felületi integrál szerepel, mint a feszültség tenzor esetén. És bár ezt elektromágneses terekre
értelmezték eredetileg, de minden felületi térerősségre lehet alkalmazni. Vegyük a mozgásegyenletet komponensenként:
\begin{equation}
    \int_{V'} f_i dV + \oint_{F'} \sigma_{ij} dA_j = \int_{V'} \varrho \ddot{u}_i dV
\end{equation}
És alkalmazzuk erre a Gauss-tételt:
\begin{equation}
    \int_{V'} f_i dV + \int_{V'} \frac{\partial \sigma_{ij}}{\partial r_j} dV = \int_{V'} \varrho \ddot{u}_i dV
\end{equation}
Mivel ez az egyenlet tetszőleges térfogatra igaz, ezért a következő lokális egyenletet kapjuk:
\begin{equation}
    f_i + \frac{\partial \sigma_{ij}}{\partial r_j} = \varrho \ddot{u}_i
\end{equation}
Ez az egyenlet a folytonos közegek mozgásegyenlete komponensenként. 
A teljes egyenlet tehát:
\begin{equation}
    \bold{f} + div \hat{\sigma} = \varrho \bold{\ddot{u}}
\end{equation}
Azt is meg lehet mutatni, hogy ha a feszültségtenzor szimmetrikus ($\sigma_{ij} = \sigma_{ji}$), akkor az impulzusmomentum megmaradása automatikusan teljesül.
A feszültség tenzor a Newton axiómák miatt csak a deformációs tenzor első deriváltjáig függhet, tehát: 
\begin{equation}
    \hat{\sigma}(\hat{\varepsilon}, \dot{\hat{\varepsilon}}, t)
\end{equation}
De szilárd anyagok eseetében a feszültség tenzor csak a deformációs tenzor függvénye lesz:
\begin{equation}
    \hat{\sigma}(\hat{\varepsilon}) \quad \rightarrow \quad \text{szilárd anyag}
\end{equation}
Ebben az esetben a feszültség és deformációs tenzor között lineáris összefüggés áll fenn, amit általánosított Hooke-törvénynek neveznek:
\begin{equation}
    \sigma_{ij} = c_{ijkl} \varepsilon_{kl}
\end{equation}
Ahol $c_{ijkl}$ azért négyindexes tenzor, mert a feszültség és deformációs tenzor is másodrendű tenzor. Ennek alapesetben 81 független komponense lenne (mindegyik index 1-3-ig mehet és $3^4 = 81$) Azt tudjuk, hogy az $i$ és $j$ indexek felcserélhetők a feszültség tenzor szimmetriája miatt, és a $k$ és $l$ indexek is felcserélhetők
a deformációs tenzor szimmetriája miatt. Ezért 81 helyett csak 36 független komponense lesz a $c_{ijkl}$ tenzornak. De van egy további szimmetria is, mégpedig az energia megmaradás miatt:
\newline
Ha veszek egy $l$ hosszúságú $A$ keresztmetszetű rúdat amit $F$ erővel $\Delta l$-el megnyújtok, akkor a rúdban az elvégzett munka:
\begin{equation}
    W = \int_0^{\Delta l} F dl = \frac{1}{2} F \Delta l
\end{equation}
Az erő pedig:
\begin{equation}
    F = \sigma A
\end{equation}
A deformáció:
\begin{equation}
    \Delta l = \varepsilon l
\end{equation}
Ezeket behelyettesítve a munkába:
\begin{equation}
    W = \frac{1}{2} \sigma A \varepsilon l
\end{equation}
Itt pedig bevezethetem az energia sűrűséget ($w$), ami egy egységnyi térfogatra eső energia:
\begin{equation}
    w = \frac{W}{V} = \frac{W}{Al} = \frac{1}{2} \sigma \varepsilon
\end{equation}
Ezt a kifejezést általánosíthatom három dimenzióra is:
\begin{equation}
    w = \frac{1}{2} \sigma_{ij} \varepsilon_{ij}
\end{equation}
Ha most behelyettesítem ide az általánosított Hooke-törvényt:
\begin{equation}
    w = \frac{1}{2} c_{ijkl} \varepsilon_{ij} \varepsilon_{kl}
\end{equation}
Itt $c_{ijkl}$ a "rugalmatlansági" (stiffness) tenzor ($Pa$ nyomás dimenziójú, mivel feszültségtenzor is nyomás dimenziójú, az alakváltozás tenzor pedig dimenzió nélküli). Ebből pedig következik, hogy a $c_{ijkl}$ tenzornak teljesen szimmetrikusnak kell lennie az indexek felcserélésére nézve, mert ez az indexcsere az energia sűrűség
értékét nem változtathatja meg, vagyis az energiát se módosítja. A Newtoni fizika szerint pedig az energiából minden mennyiség kiszámolható, ezért az indexek felcserélésének nem szabad
megváltoztatnia a rendszert. Tehát a $c_{ijkl}$ tenzor minden indexre szimmetrikus, így már csak 21 független komponense van (Mivel $c_{ijkl} = c_{klij}$ ezért felirhatom a rugalmatlansági tenzort mint egy 6x6 mátrix $c_{IJ}$ és egy szimmetrikus 6x6 mátrix független komponensei: $\frac{6(6 + 1)}{2} = 21$).
Ez a szám még tovább csökkenhet, ha olyan anyagokat veszünk amik makroszkópikus szempontból forgatásra invariánsak (izotróp anyagok). Nem izotróp anyagok pl. a kristályos anyagok melyek kristályszerkezetük miatt lehetnek irányfüggőek.
Az izotóp anyagok energiasűsége tehát csak a következő formában írható fel:
\begin{equation}
    w = \frac{1}{2} \lambda (\hat{\varepsilon}_{lk})^2 + \mu \hat{\varepsilon}_{ij} \hat{\varepsilon}_{ij}
\end{equation}
Ahol $\lambda$ és $\mu$ a Lamé állandók, melyek anyagra jellemzőek. Ebből pedig a feszültség tenzor:
\begin{equation}
    \sigma_{op} = \lambda \delta_{op} \varepsilon_{ll} + 2 \mu \varepsilon_{op}
\end{equation}
ahol $\delta_{op}$ a Kronecker-delta. Ezt kicsit átrendezve:
\begin{equation}
    \sigma_{op} = \lambda \delta_{op} \hat{\varepsilon}_{ll} + 2 \mu \left(\varepsilon_{op} - \frac{1}{3} \delta_{op} \hat{\varepsilon}_{ll}\right) + \frac{2}{3} \mu \delta_{op} \hat{\varepsilon}_{ll}
\end{equation}
\begin{equation}
    \sigma_{op} = \left(\lambda + \frac{2}{3} \mu\right) \underbrace{\delta_{op} \hat{\varepsilon}_{ll}}_{\text{térfogati deformáció - kompresszió}} + 2 \mu \underbrace{\left(\varepsilon_{op} - \frac{1}{3} \delta_{op} \hat{\varepsilon}_{ll}\right)}_{\text{térfogati deformáció nélküli rész - Nyírás}}
\end{equation}
Innen látszik, hogy a $\mu$ a nyírási modulusz, a $\lambda + \frac{2}{3} \mu$ pedig a kompressziós modulusz (majdnem, mert az 3D-ben van).
Innen a Young modulus és a Poisson arány is kifejezhető a Lamé állandók segítségével:
\begin{equation}
    E = \frac{\mu (3 \lambda + 2 \mu)}{\lambda + \mu}
\end{equation}
\begin{equation}
    \nu = \frac{\lambda}{2(\lambda + \mu)}
\end{equation}

\subsection{Folyadékok tulajdonságai, hidrosztatika, felületi feszültség, görbületi nyomás, felhajtóerő}
A folyadékokat és a gázokat általában külön kezeljük a folytonos szilárd anyagoktól, pedig fizikailag nagyon hasonlóan lehet őket kezelni, csak 
más paraméterekkel. Tehát a folyadékok (és gázok) statikai feltétele hasonlóan néz ki a szilárd anyagokhoz:
\begin{equation}
    \bold{f} + div \hat{\sigma} = 0
\end{equation}
Ahol a $\bold{f}$ a külső erősűrűség (pl. gravitációs erő), és a $\hat{\sigma}$ a feszültség tenzor. Viszont a folyadékoknál a feszültség tenzor
másképp néz ki, hisz a folyadékokban csak normális feszültségek lehetnek, nyírófeszültségek nem (kísérleti tapasztalat). Ezért a feszültség tenzor diagonális lesz:
\begin{equation}
    \Delta \bold{F} = \hat{\sigma} \Delta \bold{A} = \begin{bmatrix}
        \sigma_{xx} & 0 & 0 \\
        0 & \sigma_{yy} & 0 \\
        0 & 0 & \sigma_{zz}
    \end{bmatrix} \begin{bmatrix}
        \Delta A_x \\
        \Delta A_y \\
        \Delta A_z
    \end{bmatrix}
\end{equation}
\begin{equation}
    \Delta \bold{F} = - p \Delta \bold{A}
\end{equation}
\begin{equation}
    \sigma_{ij} = -p \delta_{ij} \quad \rightarrow \quad \hat{\sigma} = \begin{bmatrix}
        -p & 0 & 0 \\
        0 & -p & 0 \\
        0 & 0 & -p
    \end{bmatrix}
\end{equation}
Ahol $p$ a folyadék nyomása. Ezt visszahelyettesítve a statikai egyenletbe:
\begin{equation}
    div \hat{\sigma} = \sum_{j = 0}^{3}\frac{\partial \sigma_{ij}}{\partial x_j} = - \sum_{j = 0}^{3} \frac{\partial p \delta_{ij}}{\partial x_j} = - grad(p)
\end{equation}
\begin{equation}
    \bold{f} - grad(p) = 0
\end{equation}
Ezt nevezik Pascal-törvénynek. A negatív előjel konvenció miatt van itt, hisz a nyomás a felületre merőleges befelé ható erőként van definiálva.
Ha a külső erő a gravitáció, akkor:
\begin{equation}
    \bold{f} = - \varrho \bold{g}
\end{equation}
Ekkor a Pascal-törvény:
\begin{equation}
    - \varrho \bold{g} - grad(p) = 0
\end{equation}
A gravitációs gyorsulást fel lehet írni egy potenciál segítségével is:
\begin{equation}
    \bold{g} = - grad(\phi) \quad \text{ahol } \phi = gz
\end{equation}
Ezt visszahelyettesítve a Pascal-törvénybe:
\begin{equation}
    grad(p) + \varrho grad(\phi) = 0
\end{equation}
Itt két ismeretlen van, a $p$ és a $\varrho$ ezért szükség van egy további összefüggésre, ami a $\varrho (p)$-t határozza meg.
Ehhez vehetjük a termodinamikából ismert állapotegyenleteket:
\begin{equation}
    pV = \frac{m}{M}RT \quad \quad \frac{p}{\varrho} = \frac{RT}{M} \quad \quad -V \frac{dp}{dV} = K
\end{equation}
Ahol $K$ a kompresszibilitás, ami egy anyagra jellemző állandó. Itt viszont van egy probléma, mégpedig, hogy ha nagy a kompresszió modulusz,
akkor a sűrűség nem fog jelentősen változni a nyomás hatására. Ezért a folyadékoknál és szilárd anyagoknál a sűrűség állandónak vehető,
így a Pascal-törvényből a következő egyenletet kapjuk:
\begin{equation}
    grad(p + \varrho \phi) = 0 \quad \rightarrow \quad p + \varrho \phi = \text{állandó}
\end{equation}
Ez azt jelenti, hogy a folyadékban a nyomás és a potenciál összege állandó, vagyis a folyadékok felszíne a potenciáltér ekvipotenciális felületét veszi fel.
Innen ered, hogy a víz felszíne mindig vízszintes lesz. Gravitációs térben beírhatjuk a potenciált:
\begin{equation}
    \phi = - g h
\end{equation}
\begin{equation}
    p - \varrho g h = \text{állandó} = p_0
\end{equation}
\begin{equation}
    p = p_0 + \varrho g h
\end{equation}
Ahol $p_0$ a folyadék felszínén lévő nyomás. Ebből az egyenletből látszik, hogy a folyadékban a nyomás a mélységgel lineárisan növekszik. Ezt 
hidrosztatikai nyomásnak nevezzük.

Vegyünk egy képzeletbeli kockát a folyadékban és nézzük meg, hogy a kockára ható erők hogyan néznek ki:
\begin{figure}[H]
    \centering
    \includegraphics[width=0.4\textwidth]{imgs/4-tetel/hidrosztatikai_nyomas.png}
    \caption{Hidrosztatikai nyomás egy kockán}
\end{figure}
Ekko a felületre ható erők:
\begin{equation}
    F_{\text{felül}} = \varrho_{\text{víz}} g x A \quad \quad F_{\text{alul}} = - \varrho_{\text{víz}} g (x + h) A
\end{equation}
Tehát a kockára ható eredő erő:
\begin{equation}
    F_{\text{f}} = F_{\text{felül}} + F_{\text{alul}} = - \varrho_{\text{víz}} g \underbrace{A h}_{V}
\end{equation}
Ahol $F_f$ a felhajtóerő, ami a kockára hat. Ez az erő felfelé hat, tehát ellentétes irányú a gravitációs erővel. Ez az erő megegyezik a kocka által kiszorított folyadék súlyával.
Ezt az eredményt Archimédész törvényének nevezzük. Ezt onnan is levezethetjük, hogy vesszük a kockára ható külső erő sűrűségét:
\begin{equation}
    div \hat{\sigma} = - \bold{f} = - \varrho \bold{g}
\end{equation}
Ezt integrálva a kocka térfogatára:
\begin{equation}
    \int_{V} div \hat{\sigma} dV = - \int_{V} \varrho \bold{g} dV
\end{equation}
Alkalmazva a Gauss-tételt a bal oldalon:
\begin{equation}
    \oint_{F} \hat{\sigma} \cdot d\bold{A} = - \int_{V} \varrho \bold{g} dV
\end{equation}
A bal oldalon a kockára ható összes erőt kapjuk, a jobb oldalon pedig a kocka térfogatára ható gravitációs erőt. Tehát:
\begin{equation}
    \bold{F}_f = - \varrho \bold{g} V
\end{equation}
Ez pedig pontosan az Archimédész törvénye. Itt figyelni kell arra, hogy bár a test térfogata szerint integrálunk, a víz sűrűségét kell venni,
mert a víz által kifejtett erőket számoljuk, és nem "látunk bele" a testbe.

Téjünk vissza a Pascal-törvényhez:
\begin{equation}
    \bold{f} - grad(p) = 0 \quad \quad \frac{p}{\varrho} = \frac{R}{M} T \quad \quad T = \text{állandó}
\end{equation}
Ebből a következő egyenletet kapjuk:
\begin{equation}
    grad(p) - \frac{M}{R T} p \bold{g} = 0 \quad \quad \varrho = \frac{M}{R T} p
\end{equation}
Itt $\bold{g}$ lefele mutat, tehát csak a $z$ irányú tagok számítanak:
\begin{equation}
    \frac{dp}{dz} = - \frac{M}{R T} p g
\end{equation}
Ebből kaptunk egy egyenletet a $p(z)$-re, állandó hőmérséklet mellett, ideális gázra. Ennek a megoldása:
\begin{equation}
    p(z) = p_0 e^{- \frac{M g}{R T} z}
\end{equation}
Ahol $p_0$ a $z = 0$ magasságban lévő nyomás. Ebből pedig a sűrűség:
\begin{equation}
    \varrho(z) = \frac{M}{R T} p_0 e^{- \frac{M g}{R T} z} = \varrho_0 e^{- \frac{M g}{R T} z}
\end{equation}
Ahol $\varrho_0$ a $z = 0$ magasságban lévő sűrűség. Tehát ideális gáz esetén a nyomás és a sűrűség exponenciálisan csökken a magassággal.
Ezt nevezik Barometrikus magasságformulának. Ha az $M = m A$ (moláris tömeg) átírást bevezetjük (ahol $m$ az egy részecske tömege, és $A$ a Avogadro szám), akkor a következő alakot kapjuk:
a nyomásra és sűrűségre:
\begin{equation}
    p(z) = p_0 e^{- \frac{m A g}{R T} z} \quad \quad \varrho(z) = \varrho_0 e^{- \frac{m A g}{R T} z}
\end{equation}
Ezt úgy is értelmezhetjük, mint annak a valószínűségét, hogy egy $m$ tömegű részecskét a $z$ magasságban találunk.

\subsubsection*{Felületi feszültség}
Láttuk tehát, hogy a folyadékok alakját a potenciáltér ekvipotenciális felületei határozzák meg általában. Viszont kis méretű folyadékoknál megfigyelhető
egy extra hatás is ami képes az alakot megváltoztatni. Például ha egy kis cseppet helyezünk egy sima felületre, akkor a csepp alakja nem egy lapos korong lesz,
 hanem egy gömbszelet, tehát valamilyen erő összehúzza a cseppet. Ezt az erőt meg is tudjuk mérni, ha veszünk egy kis keretet, aminek az egyik oldala mozgatható, és a keretet
belemártjuk szappanos vízbe (mert a szappan lipidjei miatt ennek a folyadéknak nagy lesz ez az erő a tiszta vízhez képest), majd kihúzzuk. Ekkor a keret egyik oldalán egy vékony folyadékréteg képződik, és a mozgó oldalt megpróbálja visszahúzni a folyadék.
\begin{figure}[H]
    \centering
    \includegraphics[width=0.5\textwidth]{imgs/4-tetel/feluleti_feszultseg.png}
    \caption{Felületi feszültség mérésének elve}
\end{figure}

Ez az erő a kísérleti tapasztalatok szerint arányos a keret hosszával:
\begin{equation}
    F = 2 \alpha l
\end{equation}
Ahol az $\alpha$ a felületi feszültség, ami anyagra jellemző mennyiség és a mértékegysége $N/m$ (vagy $J/m^2$).
Ha a keretnek a mozgó oldalát egy $dx$ távolsággal elmozdítjuk, akkor a folyadék felülete megnő, de a mért erő nem fog változni.
Tehát az elmozdíás során végzett munka:
\begin{equation}
    dW = F dx = 2 \alpha l dx = \alpha dA
\end{equation}
Ahol $dA = 2 l dx$ a folyadék felületének a növekedése. Itt látszik, hogy miért kellett a 2-es az erő képletébe, hisz a folyadékrétegnek két oldala is van és mindkettő megnő.
Innen definiálhatjuk a felületi feszültséget mint a felület növekedésre végzett munka és a felület növekedés hányadosa:
\begin{equation}
    \alpha = \frac{dW}{dA}
\end{equation}
Tehát a felületi feszültség azt méri, hogy mennyi munkát kell végezni egy adott felület növeléséhez. Ezért a folyadékok
mindig arra törekszenek, hogy a felületüket minimalizálják, hisz így a felületi feszültség miatt kevesebb munkát kell végezniük.
Ezért a kis cseppek gömb alakúak, hisz a gömbnek van a legkisebb felülete adott térfogat mellett. 

\subsubsection*{Görbületi nyomás}
Láttuk, hogy a felületi feszültség miatt a folyadékok a felületüket minimalizálni akarják. Ezért ha egy folyadék egy kis cseppben van,
akkor gömb alakú lesz. Viszont minél nagyobb a csepp, annál nehezebben tartja meg a gömb alakját, és annál inkább hajlamos a deformációra.
Tehát a cseppnek meglehet határozni egy kritikus méretét, ahol már nem tudja megtartani a gömb alakját. 
\begin{figure}[H]
    \centering
    \includegraphics[width=0.4\textwidth]{imgs/4-tetel/csepp_kritikus.png}
    \caption{Csepp kritikus mérete}
\end{figure}
Ekkor felírhatjuk a következő összefüggést:
\begin{equation}
    \alpha_{\text{üveg-víz}} \Delta l = \alpha_{\text{üveg-levegő}} \Delta l + \alpha_{\text{víz-levegő}} \cos \phi \Delta l
\end{equation}
Ahol $\alpha_{\text{üveg-víz}} \Delta l$ a csepp által a felületre kifejtett erő, az $\alpha_{\text{üveg-levegő}} \Delta l$ a felület által a cseppre kifejtett erő,
és az $\alpha_{\text{víz-levegő}} \cos \phi \Delta l$ a csepp súlya. Ebből az egyenletből kifejezve a kritikus feltételt: 
\begin{equation}
    \frac{\alpha_{\text{üveg-víz}} - \alpha_{\text{üveg-levegő}}}{\alpha_{\text{víz-levegő}}} = \cos \phi
\end{equation}
Ez az eredmény viszont nem feltétlenül létezik, hisz a jobb oldalon a $\cos \phi$ csak -1 és 1 között vehet fel értékeket. Tehát, ha nem tudjuk teljesíteni az egyenletet,
akkor a csepp már nem tudja megtartani a gömb alakját, és elfolyik a felületen.

Nézzük meg most, hogy mi szükséges ahhoz, hogy a folyadék által felvett felület alakja stabil maradjon.
Ehhez vegyünk egy görbült felületet, amin a folyadék van, és nézzük meg, hogy a felületi feszültség hogyan hat a felületre:
\begin{figure}[H]
    \centering
    \includegraphics[width=0.5\textwidth]{imgs/4-tetel/gorbulet.png}
    \caption{Görbült felület és a felületi feszültség hatása}
\end{figure}

Ekkor nekem csak a lefelé ható erőkomponenst kell figyelembe vennem, hisz a vízszintes komponensek kioltják egymást. Tehát a felületi feszültségből származó erőkomponens:
\begin{equation}
   \Delta F_{\downarrow} = 2 \alpha \Delta s_2 \cdot sin \varphi
\end{equation}
Kis szög esetén a $sin \varphi \approx \varphi$ ezért:
\begin{equation}
    \varphi \approx \frac{\Delta s_1}{2 R_1}
\end{equation}
Tehát az erőkomponens:
\begin{equation}
    \Delta F_{\downarrow} = \frac{\Delta s_1 \Delta s_2}{R_1} \alpha + \frac{\Delta s_1 \Delta s_2}{R_2} \alpha
\end{equation}
Ahol $R_1$ és $R_2$ a felület két fő görbületi sugara. Ezt az erőt kell a nyomáskülönbségnek ellensúlyoznia:
\begin{equation}
    P_g = \left(\frac{1}{R_1} + \frac{1}{R_2}\right) \alpha
\end{equation}
Ahol $P_g$ a görbületi nyomás, ami a folyadék belsejében és a külsejében lévő nyomás különbségét adja meg. Ebből látszik, hogy minél kisebb a görbületi sugár,
annál nagyobb a nyomáskülönbség, amit a felületi feszültségnek ki kell egyenlítenie.

Ezt az eredményt onnan is megkaphatjuk, hogy vesszük a feszültség tenzort a felület két oldalára:
\begin{equation}
    \Delta \bold{F_1} = \hat{\sigma_1} \Delta \bold{A} \quad \quad \Delta \bold{F_2} = - \hat{\sigma_2} \Delta \bold{A}
\end{equation}
Ahol a $\hat{\sigma_1}$ a folyadék belsejében lévő feszültség tenzor, a $\hat{\sigma_2}$ pedig a folyadék külsejében lévő feszültség tenzor és $\bold{A}$ a felület vektora.
A felület két oldalán nem ugyan akkora az erő merőleges komponense, ezért görbül meg a felület. Ezt átírhatjuk a következő alakra:
\begin{equation}
    \Delta \bold{F}_{\bot}^1 = \bold{n} \hat{\sigma_1} \bold{n} \Delta A \quad \quad \Delta \bold{F}_{\bot}^2 = -  \bold{n} \hat{\sigma_2} \bold{n} \Delta A
\end{equation}
Ahol $\bold{n}$ a felület normálvektora. Itt a $\hat{\sigma} \bold{n}$ a Cauchy-elv szerint a $\bold{T}$ "húzó" vektort adja meg ami az erőt adja meg egy egységnyi felületen ($Pa$),
 $\bold{n} \hat{\sigma} \bold{n}$ pedig az erők merőleges komponensét. Viszont ha a feszültség tenzort két oldalról beszorozzuk a normálvektorral,
akkor egy ugrást kapunk a függvényben, és ezt az ugrást a görbületi nyomás adja meg:
\begin{equation}
    \left[\bold{n} \hat{\sigma} \bold{n}\right] = \alpha \left(\frac{1}{R_1} + \frac{1}{R_2}\right) \quad \rightarrow \quad \text{Ugrás mértéke}
\end{equation}
Ezt az egyenletet nevezik Young-Laplace egyenletnek, és a görbületi nyomást szokás Laplace-nyomásnak is nevezni.

\subsubsection*{Példa: Kapilláris jelenség}
A kapilláris jelenség során egy vékony csőbe folyadékot helyezünk, és megfigyeljük, hogy a folyadék a csőben eltér a külső folyadékszinttől.
Ez a jelenség a felületi feszültség és a folyadék és a cső anyaga közötti tapadás miatt jön létre. 
\begin{figure}[H]
    \centering
    \includegraphics[width=0.5\textwidth]{imgs/4-tetel/kapillaritas.png}
    \caption{Kapilláris jelenség}
\end{figure}
A kapilláris jelenség során a folyadék a csőben egy $h$ magasságra emelkedik vagy süllyed a külső folyadékszinthez képest. Ezt a magasságot 
az energia függvényében felírhatjuk:
\begin{equation}
    E(h) = \underbrace{\varrho r^2 \pi h g \cdot \frac{h}{2}}_{\text{potenciális energia}} + \underbrace{2 \pi r h \alpha_{\text{víz-cső}} - 2 \pi r h \alpha_{\text{víz-levegő}}}_{\text{felületi energia}}
\end{equation}
A felületi energia két tagból áll, az egyik a cső és a víz közötti felület növekedésből származik, a másik pedig a víz és a levegő közötti felület csökkenéséből.
A kapilláris egyensúlyi magasságot úgy kapjuk meg, hogy az energiát minimalizáljuk:
\begin{equation}
    \frac{dE(h)}{dh} = 0
\end{equation}
\begin{equation}
    \frac{dE(h)}{dh} = \varrho r^2 \pi g h + 2 \pi r \alpha_{\text{víz-cső}} - 2 \pi r \alpha_{\text{víz-levegő}} = 0
\end{equation}
Ebből kifejezve az egyensúlyi magasságot:
\begin{equation}
    h = \frac{2 (\alpha_{\text{víz-levegő}} - \alpha_{\text{víz-cső}})}{\varrho r g}
\end{equation}

\subsection{Áramlások jellemzése, Bernoulli-egyenlet,  tökéletes folyadék áramlása, Euler-egyenletek}
A hidrosztatika vizsgálata után már felfedesztünk néhány fontos jellemzőét a folyadékoknak, most nézzük meg, hogy dinamikában milyen tuléajdonságai lesznek fontosak:
\begin{equation}
    p(\bold{r}, t), \varrho(\bold{r}, t), \bold{v}(\bold{r}, t) \quad \rightarrow \quad \text{Mozgó folyadékok legfontosabb jellemzői}
\end{equation}
Ahol $p(\bold{r}, t)$ a nyomás, $\varrho(\bold{r}, t)$ a sűrűség, és $\bold{v}(\bold{r}, t)$ a sebességmező. Ezek az értékek mind függnek a helytől és az időtől is.
Ezek a változók nem függetlenek egymástól, hisz a sűrűség és a nyomás között van egy összefüggés (állapotegyenlet), és a sebesség is hatással van a nyomásra
és a sűrűségre is. Ezeket az összefüggéseket a folyadékdinamika egyenletei írják le.
Van egy dolog amitől a folyadékok mozgása során se tudunk eltekinteni, mégpedig az anyagmegmaradás törvényétől. 
Ez azt jelenti, hogy egy adott felületen áthaladó anyagmennyiségnek meg kell egyeznie a felület előtt és után lévő anyagmennyiséggel.
\begin{figure}[H]
    \centering
    \includegraphics[width=0.5\textwidth]{imgs/4-tetel/anyagmegmaradas.png}
    \caption{Felület áthaladó anyagmennyiség}
\end{figure}
Itt $\Delta m$ a felületen áthaladó anyagmennyiség, $\Delta A$ a felület nagysága, $\Delta t$ az időintervallum.
Itt egy időegység alatt egy felületegységen áthaladó anyagmennyiség: $\varrho \bold{v} \cdot \bold{n} \Delta A$, ahol $\bold{v} \cdot \bold{n}$ a sebességnek a felületre merőleges komponense.
Ebből az anyagmegmaradás egyenlete:
\begin{equation}
    \Delta m = - \oint \underbrace{\varrho \Delta t (\bold{v} \cdot \bold{n}) dA}_{\text{kis felületen felületen áthaladó anyagmennyiség}}
\end{equation}
De a tömeget át tudom írni a sűrűség és a térfogat szorzataként is:
\begin{equation}
    \Delta \int \varrho dV = - \oint \varrho \Delta t (\bold{v} \cdot \bold{n}) dA
\end{equation}
A negatív előjel azért van, mert a felületen kívülre áramló anyagmennyiséget vonjuk ki a térfogatban lévő anyagmennyiségből. Ezt az egyenletet átírhatom a Gauss-tétel segítségével:
\begin{equation}
    \int \frac{\partial \varrho}{\partial t} dV = - \oint \varrho \Delta t (\bold{v} \cdot \bold{n}) dA = - \int div(\varrho \bold{v}) dV
\end{equation}
Mivel ez az egyenlet bármely térfogatra igaz, ezért a következő differenciálegyenletet kapjuk:
\begin{equation}
    \frac{\partial \varrho}{\partial t} + div(\varrho \bold{v}) = 0
\end{equation}
Ezt az egyenletet nevezik kontinuitási egyenletnek, és ez az anyagmegmaradás differenciálegyenlete folyadékok esetén. Ez az egyenlet nagyon hasonló 
az elektromosságtanban megismert töltésmegmaradás egyenletéhez. Itt azt is megfigyelhetjük, hogy ha a sűrűség állandó (stacionárius áramlás), akkor a divergencia
nulla lesz:
\begin{equation}
    div(\bold{v}) = 0 \quad \text{(stacionárius áramlás esetén)}
\end{equation}
Tehát a kontinuitási egyenlet tovább egyszerűsödik:
\begin{equation}
    div(\varrho \bold{v}) = 0
\end{equation}
Ennek a következménye, hogy egy áramlási csőben a keresztmetszet, sűrűség és a sebesség szorzata állandó:
\begin{equation}
    A_1 v_1 \varrho_1 = A_2 v_2 \varrho_2
\end{equation}
\begin{equation}
    A v \varrho = \text{állandó}
\end{equation}
Tehát ha a cső felülete egy részen megnő (nagyobb a keresztmetszet), akkor a sebességnek csökkennie kell, hogy az anyagmegmaradás teljesüljön.
Erre az eredményre felírhatjuk a munkatétel definícióját is:
\begin{equation}
  \frac{\Delta m}{2} (v_2^2 - v_1^2) = \Delta E_{\text{kin}}
\end{equation}
Ennek meg kell egyeznie a külső és belső erők által végzett munkával (ami az erő és az elmozdulás szorzata):
\begin{equation}
   \Delta W = p_1 A_1 v_1 \Delta t - p_2 A_2 v_2 \Delta t + \Delta \delta W_b
\end{equation}
Ahol $\Delta \delta W_b$ a belső erők által végzett munka. Még egy összefüggést fel tudunk írni a tömeg alapján:
\begin{equation}
    \Delta m = \varrho_1 A_1 v_1 \Delta t = \varrho_2 A_2 v_2 \Delta t = \varrho A \Delta V \Delta t
\end{equation}
Tehát a kinematikai energia megváltozásának egyenlőnek kell lennie a külső és belső erők által végzett munkával. De még egy közelítéssel élhetünk,
mégpedig, hogy a folyadékunk ideális (tehát nincs belső súrlódás, viszkozitás és összenyomhatatlan). Ekkor a belső erők által végzett munka nulla lesz,
és a teljes egyenlet a következő alakot ölti:
\begin{equation}
    \frac{\varrho A \Delta V \Delta t}{2} (v_2^2 - v_1^2) = p_1 A_1 v_1 \Delta t - p_2 A_2 v_2 \Delta t
\end{equation}
Ebből kifejezve a következő egyenletet kapjuk:
\begin{equation}
    \frac{v_2^2}{2} - \frac{v_1^2}{2} = \frac{p_1 A_1 v_1 \Delta t}{\varrho_1 A_1 v_1 \Delta t} - \frac{p_2 A_2 v_2 \Delta t}{\varrho_2 A_2 v_2 \Delta t}
\end{equation}
\begin{equation}
    \frac{v_2^2}{2} - \frac{v_1^2}{2} = \frac{p_1}{\varrho_1} - \frac{p_2}{\varrho_2}
\end{equation}
\begin{equation}
    \frac{p_1}{\varrho_1} + \frac{v_1^2}{2} = \frac{p_2}{\varrho_2} + \frac{v_2^2}{2}
\end{equation}
Innen pedig következik, hogy bármely két pont között a következő összefüggés fennáll:
\begin{equation}
    \frac{p}{\varrho} + \frac{v^2}{2} = \text{állandó}
\end{equation}
Ezt az egyenletet Bernoulli-egyenletnek nevezzük, és ez az ideális folyadékok áramlásának egyik legfontosabb egyenlete. Ez az egyenlet azt mondja ki, hogy egy adott pontban a nyomás
és a kinetikai energia összege állandó marad. Ez azt is jelenti, hogy ha egy pontban a sebesség megnő, akkor a nyomásnak csökkennie kell, és fordítva.
Ezt az egyenletet fel tudjuk írni egy adott magasságra is, ha figyelembe vesszük a gravitációs potenciális energiát is:
\begin{equation}
    \frac{p}{\varrho} + \frac{v^2}{2} + \phi = \text{állandó}
\end{equation}
Ezt az egyenletet is Bernoulli-egyenletnek nevezzük, és ez már a magasság hatását is figyelembe veszi. Itt a $\phi = gh$ a gravitációs potenciál.
Ez az egyenlet is azt mondja ki, hogy a nyomás, kinetikai energia és potenciális energia összege állandó marad egy adott pontban.
\newline
\newline
Most nézzük meg, hogy hogyan módosul a Bernoulli-egyenlet, ha az ideális folyadék helyett egy ideális gázzal dolgozunk.
Ehhez vegyük figyelembe a belső energia megváltozását is, hisz a gázoknál ez is fontos szerepet játszik az áramlás során.
Felírhatjuk hőmérséklet megváltozását a belső energia és a munka segítségével:
\begin{equation}
    \delta Q = \delta U + \delta W_b
\end{equation}
Ahol $\delta Q$ a gáz által felvett hőmennyiség, $\delta U$ a belső energia megváltozása, és $\delta W_b$ a belső erők által végzett munka.
Ezt átírhatjuk a következő alakra:
\begin{equation}
    \delta W_b = \delta Q - \delta U
\end{equation}
Itt feltételezzük, hogy a folyamat elég gyorsan megy végbe, így a hőcsere elhanyagolható, tehát $\delta Q = 0$. Ez egy axiomatikus közelítés az ideális gázokra.
Ekkor a belső erők által végzett munka:
\begin{equation}
    \delta W_b = - \delta U
\end{equation}
A belső energia megváltozása pedig felírható a következő alakban az állapotegyenlet segítségével:
\begin{equation}
    dU = \Delta m C_V T 
\end{equation}
Ahol $C_V$ a fajhő állandó térfogaton. A $T$ hőmérséklet egy új mennyiség bevezetését igényli, de ha felhasználom az ideális gáz állapotegyenletét,
akkor el tudom tüntetni és nem kell vele külön foglalkozni:
\begin{equation}
    pV = \frac{m}{M} RT \quad \rightarrow \frac{p}{\varrho} = \frac{RT}{M} \quad \rightarrow T = \frac{M p}{R \varrho}
\end{equation}
Tehát a belső energia megváltozása:
\begin{equation}
    dU = \Delta m C_V \frac{M}{R} \frac{p}{\varrho}
\end{equation}
A $\frac{R}{M}$ egyetemes gázállandó és a moláris tömeg hányadosától is "megszabadultunk", ha vesszük a fajhő összefüggését az ideális gázokra:
\begin{equation}
    C_p - C_V = \frac{R}{M} \quad \rightarrow \quad \frac{M}{R} = \frac{1}{C_p - C_V}
\end{equation}
Ahol $C_p$ a fajhő állandó nyomáson. Ezt beírva a belső energia megváltozásába:
\begin{equation}
    dU = \Delta m \frac{C_V}{C_p - C_V} \frac{p}{\varrho}
\end{equation}
\begin{equation}
    dU = \Delta m \frac{1}{\frac{C_p}{C_V} -1} \frac{p}{\varrho}
\end{equation}
Itt bevezethetjük a $\kappa = \frac{C_p}{C_V}$ hőkapacitás hányadost, ami ideális gázokra jellemző mennyiség:
\begin{equation}
    dU = \Delta m \frac{1}{\kappa - 1} \frac{p}{\varrho}
\end{equation}
Ezt beírva a Bernoulli-egyenletbe:
\begin{equation}
    \frac{v_2^2}{2} - \frac{v_1^2}{2} = \frac{p_1}{\varrho_1} - \frac{p_2}{\varrho_2} - \frac{1}{\kappa - 1} \frac{p_2}{\varrho_2} + \frac{1}{\kappa - 1} \frac{p_1}{\varrho_1} = \frac{\kappa}{\kappa - 1} \left(\frac{p_1}{\varrho_1} - \frac{p_2}{\varrho_2}\right)
\end{equation}
Tehát fel lehet írni hogy:
\begin{equation}
\frac{v^2}{2} + \frac{\kappa}{\kappa - 1} \frac{p}{\varrho} \left(+ \phi\right)= \text{állandó}
\end{equation}
Ez az egyenlet az ideális gázok áramlásának Bernoulli-egyenlete. Ebből látszik, hogy a nyomás és a sebesség között fordított arányosság van, tehát ha a sebesség megnő,
akkor a nyomás csökken, és fordítva.


Az anyagmegmaradás miatt itt is felírhatjuk, hogy egy áramlási csőben a következő összefüggés fennáll:
\begin{equation}
    A v \varrho = \text{állandó} \quad \rightarrow \quad d(A v \varrho) = 0
\end{equation}
Tehát:
\begin{equation}
    A v d\varrho + A \varrho dv + v \varrho dA = 0
\end{equation}
\begin{equation}
    \frac{d\varrho}{\varrho} + \frac{dv}{v} + \frac{dA}{A} = 0
\end{equation}

Végezetül nézzük meg, hogy hogyan illeszkedik ez az egész a folytonos közegek mozgásegyenletébe. Emlékezzünk, hogy az egyenlet a következő alakot ölti:
\begin{equation}
    \bold{f} + div \hat{\sigma} = \varrho \frac{d \bold{v}}{dt}
\end{equation}
Viszont itt a $\bold{v}$ nem egy sima vektor, hanem egy vektortér ($\bold{v} \rightarrow \bold{v}(\bold{x}, t)$), tehát az időbeli derivált egy komplexebb alakot ölt:
\begin{equation}
    \frac{d \bold{v}}{dt} \equiv \frac{\partial \bold{v}}{\partial t} + (\bold{v} \cdot \nabla) \bold{v}
\end{equation}
Ez az úgynevezett anyagderivált, ami figyelembe veszi, hogy a folyadék részecskéi mozognak a térben. Ez a derivált láncszabály alapján jön létre és két tagból áll: az első tag a helyi időbeli változást méri,
a második tag pedig a konvektív változást méri, ami a részecskék térbeli elmozdulásából származik. Ezt beírva a mozgásegyenletbe:
\begin{equation}
    \bold{f} + div \hat{\sigma} = \varrho \left(\frac{\partial \bold{v}}{\partial t} + (\bold{v} \cdot \nabla) \bold{v}\right)
\end{equation}
Ez pedig a mozgásegyenlet folyadékokra. Ha ideális folyadékról beszélünk, akkor a feszültség tenzor csak a nyomásból áll:
\begin{equation}
    \hat{\sigma} = - p \hat{I}
\end{equation}
Ezt beírva a mozgásegyenletbe:
\begin{equation}
    \bold{f} - \nabla p = \varrho \left(\frac{\partial \bold{v}}{\partial t} + (\bold{v} \cdot \nabla) \bold{v}\right)
\end{equation}
Ez pedig az Euler-egyenlet ideális folyadékokra.

\subsection{Viszkózus folyadék áramlása, örvények, turbulencia, Reynolds-szám}
Most vizsgáljuk meg, hogy hogyan módosulnak az áramlás egyenletei, ha a folyadék nem ideális, hanem viszkózus. A viszkozitás a folyadékok belső súrlódását méri,
ami azt jelenti, hogy a folyadék rétegei egymáshoz képest el tudnak mozdulni, de ez az elmozdulás ellenállást vált ki. 
\begin{figure}[H]
    \centering
    \includegraphics[width=0.5\textwidth]{imgs/4-tetel/viszkozitas.png}
    \caption{Viszkozitás illusztráció}
\end{figure}
Az ellenállást a súrlódás okozza amely a folyadék egyes rétegei között lép fel, amikor azok egymáshoz képest elmozdulnak. Ha a folyadék egy csőben áramlik,
akkor a cső falához közelebb lévő rétegek lassabban mozognak mivel súrlódnak a fallal, míg a cső közepén lévő rétegek gyorsabban mozognak, de ott is van súrlódás a rétegek között.
Ezt a súrlódási hatást a viszkozitás anyagi jellemzője méri, amit $\eta$-val jelölünk, és a mértékegysége $Pa \cdot s$ (Pascal szor másodperc).
A viszkozitás kísérleti tapasztalatok alapján a deformációs tenzor időbeli változásával arányos:
\begin{equation}
    \bold{\hat{\sigma}} = \underbrace{-p\bold{I}}_{\hat{\sigma} \leftarrow {\text{hidrosztatikus}}} +  \bold{\hat{\sigma}}'(\hat{\dot{\varepsilon}})
\end{equation}
Ez ugye eltér a szilárd anyagoktól, ahol a feszültség tenzor csak a deformációs tenzortól függött. Ha izotrópnak vesszük a folyadékot (tehát minden irányban ugyan olyan a viszkozitása),
akkor a viszkozitás hatását a következő alakban tudjuk felírni:
\begin{equation}
    \sigma_{ij}' = 2 \eta \dot{\varepsilon}_{ij} + \eta' \delta_{ij} \dot{\varepsilon}_{kk}
\end{equation}
Ahol az első tag a nyírófeszültségeket méri, a második tag pedig a térfogati feszültségeket. Itt az $\eta$ a nyíró viszkozitás, az $\eta'$ pedig a térfogati viszkozitás.
A deformációs tenzor időbeli változása pedig a következő alakban írható fel:
\begin{equation}
    \dot{\varepsilon}_{ij} = \frac{1}{2} \left(\frac{\partial v_i}{\partial x_j} + \frac{\partial v_j}{\partial x_i}\right)
\end{equation}
Ezt visszahelyettesítve a feszültség tenzorba:
\begin{equation}
    \sigma_{ij}' = \eta \left(\frac{\partial v_i}{\partial x_j} + \frac{\partial v_j}{\partial x_i}\right) + \eta' \delta_{ij} \frac{\partial v_k}{\partial x_k}
\end{equation}
Itt észrevehetjük, hogy $\frac{\partial v_k}{\partial x_k} = div(\bold{v})$, tehát a feszültség tenzor a következő alakot ölti:
\begin{equation}
    \sigma_{ij}' = \eta \left(\frac{\partial v_i}{\partial x_j} + \frac{\partial v_j}{\partial x_i}\right) + \eta' \delta_{ij} div(\bold{v})
\end{equation}
Mivel a folyadékok mozgásegyenletében a feszültség tenzor divergenciája szerepel, ezért vegyük mindkét oldal divergenciáját:
\begin{equation}
    \frac{\partial \sigma_{ij}'}{\partial x_j} = \eta \frac{\partial}{\partial x_j} \left(\frac{\partial v_i}{\partial x_j} + \frac{\partial v_j}{\partial x_i}\right) + \eta' \frac{\partial}{\partial x_j} \left(\delta_{ij} div(\bold{v})\right)
\end{equation}
\begin{equation}
    \frac{\partial \sigma_{ij}'}{\partial x_j} = \eta \frac{\partial^2 v_i}{\partial x_j^2} + \eta \frac{\partial}{\partial x_i} \left(\frac{\partial v_j}{\partial x_j}\right) + \eta' \frac{\partial}{\partial x_i} \left(div(\bold{v})\right)
\end{equation}
\begin{equation}
    \frac{\partial \sigma_{ij}'}{\partial x_j} = \eta \Delta v_i + (\eta + \eta') \frac{\partial}{\partial x_i} \left(div(\bold{v})\right)
\end{equation}
Tehát a teljes feszültség tenzor divergenciája:
\begin{equation}
    div \bold{\hat{\sigma}} = - \nabla p + \eta \Delta \bold{v} + (\eta + \eta')  \nabla\left(\nabla \bold{v}\right)
\end{equation}
Ezt beírva a mozgásegyenletbe:
\begin{equation}
    \bold{f} - \nabla p + \eta \Delta \bold{v} + (\eta + \eta')  \nabla\left(\nabla \bold{v}\right) = \varrho \left(\frac{\partial \bold{v}}{\partial t} + (\bold{v} \cdot \nabla) \bold{v}\right)
\end{equation}
Ez pedig a Navier-Stokes egyenlet viszkózus folyadékokra.
Látható, hogy ha a viszkozitás elhanyagolható, akkor visszaáll az Euler-egyenlet ideális folyadékokra. Ha pedig a folyadék összenyomhatatlan,
akkor a divergencia nulla lesz, és a harmadik tag is eltűnik az egyenletből:
\begin{equation}
    \bold{f} - \nabla p + \eta \Delta \bold{v} = \varrho \left(\frac{\partial \bold{v}}{\partial t} + (\bold{v} \cdot \nabla) \bold{v}\right)
\end{equation}
A Navier-Stokes egyenlet egy nemlineáris parciális differenciálegyenlet, ami azt jelenti, hogy az egyenlet megoldása nem egyszerű feladat és sok esetben nem is lehet egzaktul megoldani.
Van néhány "szép" megoldása, például a Couette-áromlás, ahol két párhuzamos lemez között áramlik a folyadék, de általánosságban véve numerikus módszerekkel kell megoldani az egyenletet.
Általánosságban azt lehet mondani, hogy ha az egyenletben a viszkozitás dominál, akkor a folyadék áramlása lamináris lesz, míg ha a nemlineáris tag dominál, akkor az áramlás turbulens lesz.
A turbulens áramlás egy olyan kaotikus áramlás amit csak statisztikai módszerekkel lehet jellemezni. 

Ahhoz hogy meg tudjuk határozni, hogy egy adott áramlás lamináris vagy turbulens lesz-e,
Bevezetünk pár közelítést a Navier-Stokes egyenletbe - a viszkozitást az $\eta$-s tag jellemzi, míg a nemlineáris tagot a $\left(\bold{v} \nabla\right)\bold{v}$ tag.
Itt bevezethetek egy karrakterisztikus hosszúságot $L$ ami a folyadék áramlásának jellemző méretét jelenti:
\begin{equation}
    \underbrace{\eta \Delta \bold{v}}_{\text{viszkozitás}} \sim \eta \frac{v}{L^2} \quad \quad \underbrace{\left(\bold{v} \nabla\right)\bold{v}}_{\text{nemlineáris}} \sim \varrho \frac{v^2}{L}
\end{equation}
A karakterisztikus hosszúság és sebesség a folyadék áramlásának jellemző méreteit és sebességeit jelenti. Ezeknek az aránya pedig megmondja, hogy melyik tag dominál:
\begin{equation}
    Re = \frac{\eta \frac{v}{L^2}}{\varrho \frac{v^2}{L}} = \frac{\varrho}{\eta} v L
\end{equation}
Ezt az arányt Reynolds-számnak nevezzük, és ez egy dimenzió nélküli mennyiség, ami megmutatja, hogy egy adott áramlás lamináris vagy turbulens lesz-e.
Általánosságban elmondható, hogy ha a Reynolds-szám kicsi (általában $Re < 2000$), akkor az áramlás lamináris lesz, míg ha a Reynolds-szám nagy (általában $Re > 4000$), akkor az áramlás turbulens lesz.
Köztes értékeknél az áramlás átmeneti állapotban van, és mindkét típusú áramlás előfordulhat. A karakterisztikus hosszúságot általában a cső átmérőjét vagy a hidraulikus sugárát veszik alapul.

Még egy fontos fogalom a viszkózus folyadékok áramlásában az örvények kialakulása. Az örvények olyan forgó mozgások a folyadékban, amelyek a viszkozitás és a sebességkülönbségek miatt jönnek létre.
Az örvények kialakulása a Navier-Stokes egyenlet nemlineáris tagjának köszönhető, és ezek a mozgások jelentős hatással lehetnek a folyadék áramlására, különösen turbulens áramlás esetén.
Az örvények jellemzéséhez alakítsuk át egy kicsit a Navier-Stokes egyenletet. Először nézzük meg ezt az azonosságot
\begin{equation}
    a \times (b \times c) = b(a \cdot c) - c(a \cdot b)
\end{equation}
Ezt alkalmazva a sebességvektorra:
\begin{equation}
    \bold{v} \times (\nabla \times \bold{v}) = \underbrace{\nabla \left(v^2\right)}_{\text{rossz!}} - \bold{v} (\bold{v} \nabla)
\end{equation}
Itt írjuk át $\bold{v} (\bold{v} \nabla)$-t $(\bold{v} \nabla)\bold{v}$-re, hogy jobban látszódjon, hogy mire hat a Nabla operátor. Illetve van egy kis probléma,
mégpedig, hogy a $\nabla \left(v^2\right)$ nem egészen ugyan az mint a $\nabla \left(\bold{v} \cdot \bold{v}\right)$, ugyanis az utóbbiban csak az egyik vektor komponenseire hat a Nabla operátor,
az előbbiben pedig a négyzetre. Ezt orvosolhatjuk egy kis trükkel:
\begin{equation}
    \bold{v} \times (\nabla \times \bold{v}) = \nabla \left(\frac{v^2}{2}\right) - (\bold{v} \nabla)\bold{v}
\end{equation}
Ezt átrendezve:
\begin{equation}
    (\bold{v} \nabla)\bold{v} = \nabla \left(\frac{v^2}{2}\right) - \bold{v} \times (\nabla \times \bold{v})
\end{equation}
Ezt beírva a Navier-Stokes egyenletbe:
\begin{equation}
    \bold{f} - \nabla p + \eta \Delta \bold{v} = \varrho \left(\frac{\partial \bold{v}}{\partial t} + \nabla \left(\frac{v^2}{2}\right) - \bold{v} \times (\nabla \times \bold{v})\right)
\end{equation}
És ez miért érdekes? Mert a $\nabla \times \bold{v} = rot(\bold{v})$ kifejezést a folyadék örvényességének nevezzük, és jelölése $\bold{\omega}$:
\begin{equation}
    \bold{\omega} = rot(\bold{v})
\end{equation}
És ez a mennyiség adja meg, hogy egy adott pontban mennyire "forog" a folyadék. Ha tudok olyan zárt görbét venni a folyadékban, ahol az örvényesség nem nulla,
akkor ott örvénylés van jelen. Ha viszont nincsenek örvények, akkor a Navier-Stokes egyenlet a következő alakot ölti:
\begin{equation}
    \bold{f} - \nabla p + \eta \Delta \bold{v} = \varrho \left(\frac{\partial \bold{v}}{\partial t} + \nabla \left(\frac{v^2}{2}\right)\right)
\end{equation}
Most tegyük fel, hogy a folyadék stacionáriusan áramlik, tehát az időbeli derivált nulla lesz és a viszkozitás is elhanyagolható és külső erő sincsen:
\begin{equation}
    - \nabla p = \varrho \nabla \left(\frac{v^2}{2}\right)
\end{equation}
Ezt átrendezve:
\begin{equation}
    \nabla \left(\frac{p}{\varrho} + \frac{v^2}{2}\right) = 0
\end{equation}
Tehát a következő összefüggést kapjuk:
\begin{equation}
    \frac{p}{\varrho} + \frac{v^2}{2} = \text{állandó}
\end{equation}
Ez pedig nem más mint a Bernoulli-egyenlet. Tehát a Bernoulli-egyenlet csak akkor érvényes, ha nincsenek örvények a folyadékban.

\subsubsection*{Példa: Folyadék áramlása csőben}
Nézzük meg egy egyszerű példán keresztül, hogy hogyan alkalmazhatjuk a Bernoulli-egyenletet és az anyagmegmaradást egy csőben áramló folyadék esetén.
Tegyük fel, hogy van egy cső amiben egy ideális folyadék áramlik, és a cső keresztmetszete állandó, de a nyomás különbözik. Ekkor a következő egyenleteket tudjuk felírni:
\begin{itemize}
    \item $\nabla \bold{p} = \eta \Delta \bold{v}$ Ahol a többi tagot elhagyjuk mivel lassan laminárisan áramló ideális folyadékról van szó.
    \item $div( \bold{v}) = 0$ Az anyagmegmaradás differenciálegyenlete.
    \item $\oint \bold{\hat{\sigma}} dA = 0$ Nincs külső erő hatás a folyadékra.
\end{itemize}
Ezekből a $div( \bold{v}) = 0$ automatikusan teljesül, ha a sebesség csak az áramlás irányába mutat és a keresztmetszet mentén állandó. Ekkor a probléma
a következő alakban rajzolható fel:
\begin{figure}[H]
    \centering
    \includegraphics[width=0.5\textwidth]{imgs/4-tetel/csoaramlas.png}
    \caption{Folyadék áramlása csőben}
\end{figure}
Ekkor a felületintergálja a deformációs tenzornak a következő alakot ölti:
\begin{equation}
    \oint \bold{\hat{\sigma}} dA = \underbrace{\eta \frac{dv_x}{dr} 2 \pi r l}_{\text{nyírófeszültség felületi integrálja}} + \underbrace{p_1 \pi r^2 - p_2 \pi r^2}_{\text{palástokon csak nyomás számít}} = 0
\end{equation}
Innentől csak a $dv$ vízszintes komponensével fogunk foglalkozni ezért elhagyjuk a $x$ indexet. Átrendezve az egyenletet:
\begin{equation}
    \frac{dv}{dr} = \frac{p_2 - p_1}{2\eta l} \cdot r
\end{equation}
Ennek a differenciálegyenletnek pedig triviális megoldása van:
\begin{equation}
    v(r) = \frac{p_2 - p_1}{4 \eta l} \cdot r^2 + C
\end{equation}
Ezzel megkaptam a sebességet egy kis $r$ sugarú cső-tartományon belül, a cső belsejében. De kérdés, hogy mi lesz a $v(R)$ megoldása? Ezt a határfertelek alapján
tudnánk megadni, de nem tudunk egzakt feltételt megadni, ugyanis a határfeltétel nagyban függ a cső anyagi tulajdonságaitól és a folyadék viszkozitásától is.
De tegyük fel, hogy akkora a surlódás a cső és a folyadék között, hogy a falnál a sebesség nulla. Tehát:
\begin{equation}
    v(R) = 0 \quad \rightarrow \quad C = - \frac{p_2 - p_1}{4 \eta l} \cdot R^2
\end{equation}
Ezt beírva a sebesség kifejezésébe:
\begin{equation}
    v(r) = \frac{p_2 - p_1}{4 \eta l} \cdot (r^2 - R^2)
\end{equation}
Ez pedig a Poiseuille-féle áramlás egyenlete egy hengeres csőben. Ebből látható, hogy a sebesség a cső közepén a legnagyobb, és a fal felé közeledve csökken,
végül a falnál nulla lesz. Ez a sebességprofil egy parabolikus görbét követ.
\begin{figure}[H]
    \centering
    \includegraphics[width=0.5\textwidth]{imgs/4-tetel/poiseuille.png}
    \caption{Poiseuille-féle áramlás egy hengeres csőben}
\end{figure}
Nézzük meg, hogy ezek alapján mennyi anyag áramlik át a csövön egy adott idő alatt. Ehhez a kontinuitási egyenletet használjuk:
\begin{equation}
    div(\varrho \bold{v}) + \frac{\partial \varrho}{\partial t} = 0
\end{equation}
Mivel ideális folyadékról van szó, ezért a sűrűség állandó, tehát a második tag eltűnik:
\begin{equation}
    div(\varrho\bold{v}) = 0
\end{equation}
Ebben a $div(\varrho\bold{v})$ adja meg az anyagmennyiséget egy adott felületen keresztül. Tehát ennek kell a felületi integrált vennünk a cső keresztmetszetén:
\begin{equation}
    \dot{\phi} = \int \varrho \bold{v} dA
\end{equation}
Ahol $\dot{\phi}$ az anyagmennyiség áramlási sebessége. Ennek a kiszámításához, vegyünk egy kis $\Delta r$ vastagságú gyűrűt a csőben, aminek a felülete:
\begin{equation}
    dA = 2 \pi r \Delta r
\end{equation}
Ezen a kis felületen a sebesség állandó, így a felületi integrál egyszerűen kiszámítható:
\begin{equation}
    \dot{\phi} = \int_0^R \varrho v(r) 2 \pi r dr
\end{equation}
Ezt behelyettesítve a sebesség kifejezésébe:
\begin{equation}
    \dot{\phi} = \int_0^R \varrho \frac{p_2 - p_1}{4 \eta l} \cdot (r^2 - R^2) 2 \pi r dr
\end{equation}
\begin{equation}
    \dot{\phi} = \frac{2\pi\varrho}{4 \eta l} (p_2 - p_1) \cdot \int_0^R (r^3 - R^2 r) dr
\end{equation}
\begin{equation}
    \dot{\phi} = \frac{2\pi\varrho}{4 \eta l} (p_2 - p_1) \cdot \left(\frac{R^4}{4} - \frac{R^4}{2}\right)
\end{equation}
\begin{equation}
    \dot{\phi} = \frac{\pi \varrho}{8 \eta l} (p_1 - p_2) R^4
\end{equation}
Ez pedig a Poiseuille-törvény, ami megadja, hogy mennyi anyag áramlik át egy hengeres csövön keresztül egy adott idő alatt, ha a csőben egy ideális folyadék áramlik.
Látható, hogy az áramlási sebesség arányos a nyomáskülönbséggel és a cső sugarának negyedik hatványával, míg fordítottan arányos a viszkozitással és a cső hosszával.
Mivel a cső sugarának negyedik hatványával arányos az áramlás, ezért a cső átmérőjének kis változása is jelentős hatással van az áramlásra, ezért fontos, hogy
nagy nyomású rendszereknél fokozottan figyeljünk a cső eltömődésére vagy szűkületére.

\section{Fenomenologikus termodinamika}
Termodinamikai állapotjelzők, hőtágulás, ideális gáz, kinetikus modell., Nyílt és zárt folyamatok, Carnot-folyamat. Főtételek.
Termodinamikai potenciálok, fundamentális egyenlet. Van der Waals- gázok. Fázisátalakulások jellemzői, típusai, Gibbs-féle fázisszabály, fázisdiagramok.
Kémiai potenciál, fázisegyensúlyok.
\newline
\newline
A termodinamika a fizika egyik ága, de sok szempontból eltér a klasszikus mechanikától vagy az elektromágnesességtől. A termodinamika során nem felállított axiómák alapján
vezetjük le a törvényeket, hanem kísérleti megfigyelések alapján fogalmazzuk meg azokat és általában nem adunk teljes magyarázatot a jelenségekre. Ezért is nevezzük fenomenologikus termodinamikának. A termodinamika alapvetően makroszkopikus jelenségekkel foglalkozik, tehát nem az egyes részecskék mozgását vizsgálja,
hanem a rendszer egészének viselkedését. Ezért a termodinamikai mennyiségek is makroszkopikus mennyiségek, mint például a hőmérséklet, nyomás, térfogat, energia stb.
A termodinamika alapvetően négy fő törvényt foglal magában, amelyeket főtételeknek nevezünk. Ezek a következők:
\begin{itemize}
    \item \textbf{Nulladik főtétel:} Ha két rendszer egyensúlyban van egy harmadik rendszerrel, akkor azok is egyensúlyban vannak egymással. Ez a főtétel alapozza meg a hőmérséklet fogalmát.
    \item \textbf{Első főtétel:} Az energia megmaradásának törvénye. Egy zárt rendszerben az energia nem keletkezik és nem vész el, csak átalakul egyik formából a másikba. Matematikailag kifejezve: $dU = \delta Q - \delta W$, ahol $dU$ a belső energia változása, $\delta Q$ a rendszerbe bevitt hő, és $\delta W$ a rendszer által végzett munka.
    \item \textbf{Második főtétel:} A hő nem áramlik spontán módon hidegebb testből melegebb testbe. Ez a főtétel bevezeti a entrópia fogalmát, amely a rendezetlenség mértéke egy rendszerben. Matematikailag kifejezve: $dS \geq \frac{\delta Q}{T}$, ahol $dS$ az entrópia változása, $\delta Q$ a rendszerbe bevitt hő, és $T$ a hőmérséklet.
    \item \textbf{Harmadik főtétel:} Ahogy a hőmérséklet közelít a nullához, az entrópia egy rendszerben közelít a minimum értékhez. Ez a főtétel kimondja, hogy abszolút nulla hőmérsékleten egy tökéletes kristály entrópiája nulla.
\end{itemize}
A termodinamika ezen főtételei alapvetőek a fizika és a kémia számos területén, és fontos szerepet játszanak a mérnöki tudományokban is. A termodinamika kiterjesztése a statisztikus fizika, ami már inkább próbál magyarázatot adni a termodinamikai jelenségekre mikroszkopikus szinten.
A termodinamika legfontosabb mértékegységei a következők:
\begin{itemize}
    \item Hőmérséklet: Kelvin (K)
    \item Nyomás: Pascal (Pa = N/m²)
    \item Térfogat: Köbméter (m³)
    \item Energia: Joule (J)
    \item Entrópia: Joule per Kelvin (J/K)
\end{itemize}

\subsection{Termodinamikai állapotjelzők, hőtágulás, ideális gáz, kinetikus modell}
A termodinamika egyik alapja a termodinamikai állapotjelzők fogalma. Ezek olyan mennyiségek, amelyek egy rendszer állapotát jellemzik, és csak a rendszer jelenlegi állapotától függenek,
nem pedig attól, hogy hogyan jutott el oda. Intuitív módon meghatáorzhatunk néhány ilyen állapotjelzőt, hisz fetételezhetjük, hogy a rendszer függ attól, hogy mennyi 
részecske van egy adott térfogategységben (sűrűség), milyen gyorsan mozognak ezek a részecskék (hőmérséklet), mekkora teret foglalnak el (térfogat) és milyen erővel nyomják egymást (nyomás).
Ezen állaptojelzők közötti összefüggésekkel pedig felépíthetjük a termodinamika alapjait. 

Nézzük meg először, hogy hogyan függ a nyomás és a térfogat egy ideális gáz esetén. Az ideális gáz olyan gáz, amelyben a részecskék közötti kölcsönhatások elhanyagolhatóak,
és a részecskék mérete is elhanyagolható a gáz térfogatáshoz képest. A legtöbb gázt (pl.: levegő) első közelítésben ideális gáznak tekinthetjük normál körülmények között.
Vegyünk tehát egy keskeny kis csövet amiben egy ideális gáz van, és a végén van egy dugattyú és egy nyomásmérő. Ha a dugattyút elkezdjük nyomni, akkor a gáz részecskéi egyre közelebb kerülnek egymáshoz,
így a sűrűségük megnő. Mivel a részecskék egyre közelebb kerülnek egymáshoz, ezért egyre gyakrabban ütköznek a dugattyúval, így a dugattyúra ható erő is megnő. Ez pedig azt jelenti, hogy a nyomás is megnő.
Tehát a dugattyú lenyomásával a térfogat csökken, míg a nyomás nő. Ebből az összefüggésből felírhatjuk az ideális gáz állapotegyenletét:    
\begin{equation}
    pV = \text{állandó}
\end{equation}
Ezt az összefüggést Boyle-Mariotte törvényének nevezzük, és azt mondja ki, hogy egy adott mennyiségű ideális gáz esetén a nyomás és a térfogat szorzata állandó marad,
ha a hőmérséklet állandó. Ez az összefüggés csak akkor érvényes, ha a hőmérséklet nem változik, tehát izoterm folyamat esetén (izo - "egyenlő", term - "hő", görög szavakból ).
kísérleti úton meg is lehet határozni a pontos értékét ennek az állandónak, és ez azt a következő alakot ölti:
\begin{equation}
    pV = nRT
\end{equation}
Ahol $n$ a gáz anyagmennyisége mol-ban, $R$ az egyetemes gázállandó ($R \approx 8.314 \frac{J}{mol \cdot K}$), és $T$ a hőmérséklet Kelvin-ben. Ez pedig az ideális gáz állapotegyenlete.
Ebből látható, hogy ha a hőmérséklet növekszik, akkor a nyomás is növekszik, ha a térfogat állandó marad. Ezért van az, hogy ha egy autó gumiabroncsát felfújjuk, akkor a nyomás is növekszik,
mivel a levegő részecskéi gyorsabban mozognak és így nagyobb erővel ütköznek az abroncs falával.

Másodiknak nézzük meg mi történik ha a nyomást állandónak tartjuk, vagyis egy izobár folyamatot vizsgálunk ("nyomás"). Tegyük fel, hogy van egy hengerünk amiben egy ideális gáz van, 
és a henger tetején van egy dugattyú ami szabadon mozoghat fel és le. Ha a gázt elkezdjük melegíteni, akkor a részecskék gyorsabban kezdenek mozogni, így nagyobb erővel ütköznek a dugattyúval.
Ez pedig azt eredményezi, hogy a dugattyú elkezd felfelé mozdulni, így a térfogat is növekszik. Tehát izobár folyamat esetén a hőmérséklet növekedésével a térfogat is növekszik, mégpedig úgy, hogy a hőmérséklet és a térfogat egyenesen arányosak egymással:
\begin{equation}
    \frac{V}{T} = \text{állandó}
\end{equation}
Ezt az összefüggést Charles vagy Gay-Lussac első törvényének nevezzük. Mivel a téfogat egyenesen nő a hőmérséklettel, ezért térfogatváltozást a következő alakban is felírhatjuk:
\begin{equation}
    V = V_0 (1 + \beta \Delta T)
\end{equation}
Ahol $V_0$ a kezdeti térfogat, $\Delta T$ a hőmérsékletváltozás, és $\beta$ a tágulási együttható ($\beta = \frac{1}{273.15\degree C}$). Ez az együttható megadja, hogy egy adott anyag térfogata mennyivel változik meg egy egységnyi hőmérséklet változás hatására.

Harmadiknak meg nézzük meg mi történik ha a térfogatot tartjuk állandónak, vagyis egy izochor folyamatot vizsgálunk ("térfogat"). Tegyük fel, hogy van egy zárt edényünk amiben egy ideális gáz van, és ezt az edényt elkezdjük melegíteni. Mivel a térfogat
állandó marad, a részecskék egyre gyorsabban kezdenek mozogni, így egyre nagyobb erővel ütköznek az edény falával. Ez pedig azt eredményezi, hogy a nyomás is növekszik. Tehát izochor folyamat esetén a hőmérséklet növekedésével a nyomás is növekszik, mégpedig úgy, hogy a hőmérséklet és a nyomás egyenesen arányosak egymással:
\begin{equation}
    \frac{p}{T} = \text{állandó}
\end{equation}
Ezt az összefüggést Gay-Lussac második törvényének nevezzük. Mivel itt is a nyomás egyenesen nő a hőmérséklettel, ezért a nyomásváltozást a következő alakban is felírhatjuk:
\begin{equation}
    p = p_0 (1 + \beta \Delta T)
\end{equation}
Ahol $p_0$ a kezdeti nyomás, $\Delta T$ a hőmérsékletváltozás, és $\beta$ itt is a tágulási együttható.
\begin{figure}[H]
    \centering
    \includegraphics[width=0.7\textwidth]{imgs/5-tetel/idealisgaz.png}
    \caption{Ideális gáz állapotváltozásai izobár és izochor folyamatok esetén}
\end{figure}
definíció szerint amelyik gáz ezeket az összefüggéseket követi, azt ideális gáznak nevezzük. Ehhez a Hélium és a Hidrogén áll a legközelebb, de a legtöbb gáz normál körülmények között ideális gáznak tekinthető.

Ezt a három törvényt összegezve vegyünk egy $V_0$, $p_0$, $T_0$ kezdeti állapotú ideális gázt, és nézzük meg, hogyan változik ha növelem a hőmérsékletét izochor rendszerben. 
Ekkor felírhatjuk, hogy a nyomás a következőképpen változik, Gay-Lussac második törvénye alapján:
\begin{equation}
    p_1 = p_0 (1 + \beta \Delta T)
\end{equation}
Ezt beírva a Boyle-Mariotte törvényébe:
\begin{equation}
    pV = p_1 V_0 = p_0 V_0(1 + \beta \Delta T)
\end{equation}
Definiáljunk ehhez egy új hőmérsékleti skálát, amit abszolút hőmérsékletnek nevezünk, és jelölése $T$. Ekkor a kezdeti hőmérsékletet a következőképpen írhatjuk fel:
\begin{equation}
    T := \Delta T + \frac{1}{\beta}
\end{equation}
Tehát a hőmérséklet változás:
\begin{equation}
    \Delta T = T - \frac{1}{\beta}
\end{equation}
Ezt beírva a nyomás kifejezésébe:
\begin{equation}
    pV = p_0 V_0 \beta T
\end{equation}
Ez pedig egy mérhető mennyiség, hisz a $p_0$, $V_0$ és $\beta$ ismertek. Most definiáljuk az egyetemes gázállandót:
\begin{equation}
    n \cdot R := p_0 V_0 \beta
\end{equation}
Ahol $R$ az egyetemes gázállandó és $n$ a gáz anyagmennyiségét jelöli. Így az ideális gáz állapotegyenlete a következő alakot ölti:
\begin{equation}
    pV = nRT
\end{equation}
Itt a $T$ pedig a Kelvin skála aminek a 0 pontja az abszolút nulla hőmérséklet ($-273.15\degree C$). Ez a skála azért hasznos, mert a termodinamika törvényei csak ezen a skálán érvényesek.
Az $n$-t pedig úgy tudjuk kiszámolni, hogy elosztjuk a gáz tömegét a moláris tömegével:
\begin{equation}
    n = \frac{m}{M} = \frac{N}{N_A}
\end{equation}
Ahol $m$ a gáz tömege kilogrammban, és $M$ a moláris tömege kilogramm per mol-ban, $N$ a részecskék száma és $N_A$ az Avogadro-szám ($N_A \approx 6.022 \times 10^{23} \frac{1}{mol}$).

Ezzel a levezetéssel csak két probléma van. Egyrészt nincsen ideális gáz, másrészt egy bizonyos hőmérséklet alatt minden gáz kondenzálódik, ahol ez már nem érvényes.
Ezért ezt a közelítést csak normál körülmények között és alacsony nyomáson lehet alkalmazni.

\subsubsection{Termodinamikai állapotjelzők}
A termodinamikában rendszereket vizsgálunk amiket különböző dolgoktól függnek, de a legegyszerűbb eset az "egyszerű rendszer", ami izotróp, homogén és egyfázisú. Emellett a belső folyamatai is elhanyagolhatóak és nincsen mágnesezettsége, töltöttsége stb.
Ezeket a rendszereket két fajta független állapotjelzővel lehet jellemezni. Az állapotjelző nem egy axiomatikus fogalom, hanem kísérleti úton meghatározott.
Az állapotjelző egy olyan fizikai mennyiség ami egy adott rendszer makroszkopikusállapotát jellemzi, és csak a rendszer jelenlegi állapotától függ, nem pedig attól, hogy hogyan jutott el oda. 
Emelett egy állapotban egyértelműen meghatározottak. Emellett végtelen sok állapotjelző van, hisz minden állapotjelző tetszőleges függvénye is állapotjelző. Például egy ideális gáz esetén a nyomás, térfogat és hőmérséklet is állapotjelzők,
de ezek tetszőleges kombinációja is állapotjelző lesz (pl.: $pV$, $\frac{p}{T}$, stb.). Az állapotjelzők két fajtája a következő:
\begin{itemize}
    \item \textbf{Intenzív állapotjelzők:} Ezek az állapotjelzők függetlenek a rendszer méretétől vagy tömegétől. Például a hőmérséklet, nyomás és sűrűség intenzív állapotjelzők.
    \item \textbf{Extenzív állapotjelzők:} Ezek az állapotjelzők arányosak a rendszer méretével vagy tömegével. Például a térfogat, tömeg és anyagmennyiség extenzív állapotjelzők.
\end{itemize}
Másik fontos fogalom az \textbf{egyensúly}. Egy rendszer akkor van termodinamikai egyensúlyban, ha az állapotjelzői időben nem változnak, és a rendszer makroszkopikus tulajdonságai állandóak maradnak. Ez azt jelenti, hogy a rendszer belső folyamatai kiegyenlítődnek, és nincs nettó energia- vagy anyagáramlás a rendszer és a környezete között.
Egy rendszer akkor van termodinamikai egyensúlyban, ha a következő feltételek teljesülnek:
\begin{itemize}
    \item \textbf{Mechanikai egyensúly:} A rendszerben nincs nettó erőhatás, így a nyomás minden pontban egyenlő.
    \item \textbf{Termikus egyensúly:} A rendszer minden részén a hőmérséklet egyenlő, így nincs nettó hőáramlás.
    \item \textbf{Kémiai egyensúly:} A rendszerben a kémiai reakciók sebességei egyenlőek, így a reakciók termékeinek és kiindulási anyagainak koncentrációi állandóak.
\end{itemize}
Az egyensúlyi állapotra ki lehet mondani a \textbf{termodinamika nulladik főtételét}, ami kimondja, hogy ha két rendszer egyensúlyban van egy harmadik rendszerrel, akkor azok is egyensúlyban vannak egymással. Ez a főtétel alapozza meg a hőmérséklet fogalmát, hisz ha két rendszer egyensúlyban van egy harmadik rendszerrel, akkor azok hőmérséklete is egyenlő lesz.
\begin{equation}
    A \Leftrightarrow B \quad \text{és} \quad A \Leftrightarrow C \quad \rightarrow \quad B \Leftrightarrow C
\end{equation}
Az állapotjelzők alapján lehet felírni az \textbf{állapotegyenleteket} ($f(p, V, T)$) ami kapcsolatot teremt az állapotjelzők között. Az állapotegyenletek kísérleti úton meghatározott összefüggések, amelyek leírják, hogyan változnak az állapotjelzők egymáshoz képest egy adott rendszerben. Például az ideális gáz állapotegyenlete a következő:
\begin{equation}
    pV = nRT \quad \quad T = \frac{pV}{nR} \quad \quad V = \frac{nRT}{p} \quad \quad \dots
\end{equation}
Ezeket az állapotjelzőket egy egyenletbe foglalva felírhatjuk azt az állapotfüggvényt, ami egy megkötést ad a lehetséges állapotok között:
\begin{equation}
    f(p, V, T) = 0
\end{equation}
\begin{equation}
    pV - nRT = 0
\end{equation}
\begin{figure}[H]
    \centering
    \includegraphics[width=0.5\textwidth]{imgs/5-tetel/idealisgaz_felulet.png}
    \caption{Ideális gáz állapotfelülete a $pVT$ térben}
\end{figure}
\begin{figure}[H]
    \centering
    \includegraphics[width=0.5\textwidth]{imgs/5-tetel/idealisgaz_felulet_2D.png}
    \caption{Ideális gáz állapotfelülete a $pV$ térben}
\end{figure}

\subsubsection{Hőtágulás, Kompresszibilitás, Feszültség}
A termodinamikai állapotjelzőknél láttuk, hogy a térfogat és a hőmérsékélet között van valamiféle kapcsolat. Ez ideális gázoknál egy sima egyenes arányosság,
de a jelenség valójában minden anyagnál megfigyelhető, a szilárd anyagoknál is. Ezt a jelenséget hőtágulásnak nevezzük, és azt jelenti,
hogy egy anyag térfogata növekszik, amikor a hőmérséklete emelkedik. Ez a jelenség a részecskék közötti kölcsönhatások miatt következik be, hisz a hőmérséklet növekedésével a részecskék gyorsabban kezdenek mozogni,
így nagyobb távolságra kerülnek egymástól, ami a térfogat növekedéséhez vezet. A legegyszerűbb ilyen hőtágulás a \textbf{lineáris hőtágulás}, ahol az anyag mérete csak egyik irányban változik meg (fontos, hogy a "lineáris" itt nem azt jelenti, hogy a hőtágulás egyenes arányos a hőmérséklettel, hanem hogy csak egy dimenzióban történik a tágulás).
Ez jó közelítés olyan testeknél mint egy drót vagy vasúti sín, ahol a hosszúság sokkal nagyobb mint a keresztmetszete. A lineáris hőtágulás kifejezhető a következő egyenlettel:
\begin{equation}
    \Delta L = \alpha L_0 \Delta T
\end{equation}
Ha $L_0$ kezdeti hosszúságot a $T = 0\degree K$-nél veszem, akkor $\Delta T$-t vehetem $T$-nek is, így a hosszúság a következőképpen változik meg:
\begin{equation}
    L = L_0 (1 + \alpha' T)
\end{equation}
Ahol $L$ a megváltozott hosszúság, $L_0$ a kezdeti hosszúság, $\Delta T$ a hőmérséklet változása, $\alpha$ a lineáris hőtágulási együttható ($\left[\alpha\right] = \frac{1}{K}$), ami anyagfüggő és $\alpha'$ pedig a lineáris hőtágulási együttható a $T = 0\degree K$-nél. Például az acél esetében $\alpha \approx 12 \times 10^{-6} \frac{1}{K}$, míg
az alumínium esetében $\alpha \approx 23 \times 10^{-6} \frac{1}{K}$. Vannak anyagok amiknél bizonyos határok között a hőtágulás negatív, tehát a hőmérséklet növekedésével a térfogat csökken (pl.: víz 0 és 4 Celsius fok között).
A $\alpha$ együtthatót általában úgy definiálják, hogy egy adott anyag hosszának relatív változását adja meg egy egységnyi hőmérséklet változásra (izobár folyamat esetén):
\begin{equation}
    \alpha := \frac{1}{L} \frac{\partial L}{\partial T}\bigg|_{p=\text{állandó}}
\end{equation}
Ekkor hosszváltozást behelyettesítve:
\begin{equation}
    \alpha = \frac{1}{L} L_0 \alpha = \frac{L_0 \alpha '}{L_0(1 + \alpha' T)} = \frac{\alpha'}{1 + \alpha' T} 
\end{equation}
Mivel $\alpha$ általában kicsi, ezért a nevező közelítőleg 1 lesz, így $\alpha \approx \alpha'$.

\subsubsection*{Térfogati hőtágulás}
A a test minden irányába kiterjesztjük a tágulást akkor lineáris hőtágulás helyett \textbf{térfogati hőtágulásról} beszélünk. Ebben az esetben a hőtágulási együtthatót a következő módon tudom definiálni:
\begin{equation}
    \beta := \frac{1}{V} \frac{\partial V}{\partial T}\bigg|_{p=\text{állandó}} \quad \quad \left[\beta\right] = \frac{1}{K}
\end{equation}
Ekkor a térfogatváltozást a következőképpen tudom kifejezni:
\begin{equation}
    V = V_0 (1 + \beta T)
\end{equation}
Ahol $V_0$ a kezdeti térfogat, $V$ a megváltozott térfogat, és $\Delta T$ a hőmérséklet változása. Ha a test izotróp, akkor a lineáris hőtágulással is felírhatjuk a térfogatváltozást:
\begin{equation}
    \begin{aligned}
        V &= L^3 = L_0^3 (1 + \alpha' \Delta T)^3 = V_0 (1 + \alpha' \Delta T)^3 = \\
        &= V_0 (1 + 3\alpha' \Delta T + \alpha'^2 \dots)  \\
        &\approx V_0 (1 + 3\alpha' \Delta T) = V_0 (1 + \beta \Delta T) \\
        &\rightarrow \beta = 3 \alpha'
    \end{aligned}
\end{equation}

Ha ezt az ereményt megnézzük ideális gázok esetén, akkor a térfogatváltozást a következőképpen tudjuk kifejezni:
\begin{equation}
    V = \frac{nR}{p} T
\end{equation}
Ebből a térfogati hőtágulási együttható:
\begin{equation}
    \beta = \frac{1}{V} \frac{\partial V}{\partial T}\bigg|_{p} = \frac{1}{V} \frac{nR}{p} = \frac{p}{nRT} \cdot \frac{nR}{p} = \frac{1}{T}
\end{equation}
Tehát ideális gázok esetén a térfogati hőtágulási együttható fordítottan arányos a hőmérséklettel, vagyis minél magasabb a hőmérséklet, annál kisebb a térfogati hőtágulás mértéke.

\subsubsection*{Kompresszibilitás}
A kompresszibilitás egy anyag azon tulajdonsága, hogy mennyire képes csökkenteni a térfogatát egy adott nyomásnövekedés hatására. A kompresszibilitást a következőképpen definiáljuk:
\begin{equation}
    \kappa := -\frac{1}{V} \frac{\partial V}{\partial p}\bigg|_{T=} \quad \quad \left[\kappa\right] = \frac{1}{Pa} = \frac{m^2}{N}
\end{equation}
Ahol $V$ a térfogat, $p$ a nyomás, és a negatív előjel biztosítja, hogy a kompresszibilitás pozitív legyen, mivel a térfogat csökken a nyomás növekedésével. Például a víz kompresszibilitása körülbelül $\kappa \approx 4.6 \times 10^{-10} \frac{1}{Pa}$, míg a levegő kompresszibilitása sokkal nagyobb, körülbelül $\kappa \approx 1.0 \times 10^{-5} \frac{1}{Pa}$.
Az ideális gázok esetén a kompresszibilitás a következőképpen számítható ki az ideális gáz állapotegyenletéből:
\begin{equation}
    pV = nRT \quad \rightarrow \quad V = \frac{nRT}{p}
\end{equation}
Ebből a kompresszibilitás:
\begin{equation}
    \kappa = -\frac{1}{V} \frac{\partial V}{\partial p}\bigg|_{T} = -\frac{1}{V} \left(-\frac{nRT}{p^2}\right) = \frac{1}{p}
\end{equation}
Tehát ideális gázok esetén a kompresszibilitás fordítottan arányos a nyomással, vagyis minél magasabb a nyomás, annál kisebb a kompresszibilitás mértéke.

\subsubsection*{Feszültségi együttható}
A feszültségi együttható egy anyag azon tulajdonsága, mennyire változik meg a nyomás benne ha fixen tartjuk a térfogatát és változtatjuk a hőmérsékletét.
Ilyen lehet például egy üvegedény aminek hirtelen felmelegítünk, de a kristályos szerkezete miatt a térfogata nem tud változni. Ilyenkor az üveg falában megnő a feszültség, ami idővel eltöri az edényt.
Hasonló probléma tud fellépni a vonatsíneknél is, ha a sín hőmérséklete nagyon megemelkedik, de a sín nem tud tágulni a rögzítések miatt. Ekkor a sínben nagy feszültség keletkezik, ami akár a sín deformálódásához vagy eltöréséhez is vezethet.
Továbbá a hosszú villanyvezetékeket is ezért "lógatják" le, hogy legyen hely a hőtágulásra, különben a nyári melegben a vezetékek elszakadhatnának a nagy feszültség miatt. 
A feszültségi együtthatót a következőképpen definiáljuk:
\begin{equation}
    \gamma := \frac{1}{p} \frac{\partial p}{\partial T}\bigg|_{V} \quad \quad \left[\gamma\right] = \frac{1}{K}
\end{equation}
Ahol $p$ a nyomás, $T$ a hőmérséklet, és $V$ a térfogat. Például a víz feszültségi együtthatója körülbelül $\gamma \approx 0.2 \frac{N}{m^2}$, míg a levegő feszültségi együtthatója sokkal kisebb, körülbelül $\gamma \approx 0.0001 \frac{N}{m^2}$.
Az ideális gázok esetén a feszültségi együttható a következőképpen számítható ki az ideális gáz állapotegyenletéből:
\begin{equation}
    pV = nRT \quad \rightarrow \quad p = \frac{nRT}{V}
\end{equation}
Ebből a feszültségi együttható:
\begin{equation}
    \gamma = \frac{1}{p} \frac{\partial p}{\partial T}\bigg|_{V} = \frac{1}{p} \left(\frac{nR}{V}\right) = \frac{1}{\frac{nRT}{V}} \cdot \frac{nR}{V} = \frac{1}{T}
\end{equation}
Tehát ideális gázok esetén a feszültségi együttható fordítottan arányos a hőmérséklettel, vagyis minél magasabb a hőmérséklet, annál kisebb a feszültségi együttható mértéke.

\subsubsection*{Összefüggés az együtthatók között}
Láttuk, hogy ezek az együtthatók mind a gázok pár alapvető állapotjelzőjével definiálható egységek. És mivel ezek az állapotjelzők függnek egymástól, ezért az együtthatók között is van összefüggés. Ezt az összefüggést a következőképpen tudjuk levezetni:
Vegyük egy tetszőleges gáz állapotegyenletét:
\begin{equation}
    f(p, V, T) = 0
\end{equation}
Ezt fel tudjuk rajzolni egy háromdimenziós felületként a $PVT$ térben. Ebből a felületből ki tudunk választani egy görbét, ami egy adott állapotváltozást ír le.
Ezt a görbét paraméterezhetjük egy paraméterrel, például $t$-vel:
\begin{figure}[H]
    \centering
    \includegraphics[width=0.5\textwidth]{imgs/5-tetel/termodinamika_felulet.png}
    \caption{Állapotegyenlet felülete és egy állapotváltozást leíró görbe}
\end{figure}

Ekkor a paraméterekre a következő összefüggéseket lehet felírni:
\begin{equation}
    \begin{aligned}
        p &= P(t, v) \\
        v &= V(p, t) \\
        t &= T(p, v)
    \end{aligned}
\end{equation}
Itt nagy $P$-vel jelöljük a nyomást, és kis $p$-vel a a paramétert amit kiválasztunk. Ez olyan mint ha különböző irányokból néznénk a felületet, és minden irányból más-más függvényt kapunk. Ezekből a függvényekből ki tudjuk számolni hogy hogyan változik egy adott állapotjelző, ha a másik kettőt fixen tartjuk.
Felírhatjuk azt is, hogy egy adott $t-v$ helyen mekkora a $p$ értéke:
\begin{equation}
    p = P(t, v)
\end{equation}
Ha pedig megnézem, hogy $v-t$ helyen mekkora a $V$ értéke:
\begin{equation}
    v = V(p, t)
\end{equation}
Viszont az előző egyenletből tudom, hogy $p = P(t, v)$, így a $V$ értékét is ki tudom számolni:
\begin{equation}
    v = V(P(t, v), t)
\end{equation}
Ez pedig egy azonosság, ami minden $t$-re igaz. Ebben benne van a két állapotjelző függvénye, és $t$ és $v$ a paraméterek. Nézzük meg, hogy mi történik, ha a $v$-t változtatjuk, miközben $t$-t fixen tartjuk:
\begin{equation}
    \frac{\partial}{\partial v} \left[V(P(t, v), t)\right] = \frac{\partial v}{\partial v}
\end{equation}
Láncszabály szerint kifejtve:
\begin{equation}
    \frac{\partial V}{\partial p}\bigg|_{t} \cdot \frac{\partial P}{\partial v}\bigg|_{t} + \frac{\partial V}{\partial t}\bigg|_{p} \cdot \frac{\partial t}{\partial v}\bigg|_{t} = 1
\end{equation}
Mivel $t$-t fixen tartjuk, ezért a második tag nullává válik, így marad:
\begin{equation}
    \frac{\partial V}{\partial p}\bigg|_{t} \cdot \frac{\partial P}{\partial v}\bigg|_{t} = 1
\end{equation}
\begin{equation}
    \frac{\partial V}{\partial p}\bigg|_{t} = \frac{1}{\frac{\partial P}{\partial v}\big|_{t}}
\end{equation}
Ha visszahelyettesítjük az eredeti összefüggéseket:
\begin{equation}
    \frac{\partial V}{\partial P}\bigg|_{T} = \frac{1}{\frac{\partial p}{\partial V}\big|_{T}}
\end{equation}
Ezt hívják inverz-függvény tételnek. Ugyanezt meg tudjuk csinálni a többi állapotjelzővel is, tehát összességében három ilyen összefüggést kapunk:
\begin{equation}
    \begin{aligned}
        \frac{\partial V}{\partial P}\bigg|_{T} &= \frac{1}{\frac{\partial p}{\partial V}\big|_{T}} \\
        \frac{\partial T}{\partial P}\bigg|_{V} &= \frac{1}{\frac{\partial p}{\partial T}\big|_{V}} \\
        \frac{\partial T}{\partial V}\bigg|_{P} &= \frac{1}{\frac{\partial V}{\partial T}\big|_{P}} 
    \end{aligned}
\end{equation}
\begin{equation}
    \frac{\partial}{\partial t} \left[V(P(t, v), t)\right] = \frac{\partial v}{\partial t}
\end{equation}
Láncszabály szerint kifejtve:
\begin{equation}
    \frac{\partial V}{\partial p}\bigg|_{t} \cdot \frac{\partial P}{\partial t}\bigg|_{v} + \frac{\partial V}{\partial t}\bigg|_{p} \cdot \frac{\partial t}{\partial t}\bigg|_{v} = 0
\end{equation}
A $\frac{\partial t}{\partial t}\big|_{v} = 1$, így marad:
\begin{equation}
    \frac{\partial V}{\partial p}\bigg|_{t} \cdot \frac{\partial P}{\partial t}\bigg|_{v} + \frac{\partial V}{\partial t}\bigg|_{p} = 0
\end{equation}
Tehát a megoldás:
\begin{equation}
     \frac{\partial V}{\partial p}\bigg|_{t} \cdot \frac{\partial P}{\partial t}\bigg|_{v} = - \frac{\partial V}{\partial t}\bigg|_{p}
\end{equation}
Itt az inverz-függvény tételt is alkalmazva és a paramétereket visszahelyettesítve kapjuk:
\begin{equation}
     \frac{\partial V}{\partial P}\bigg|_{T} \cdot \frac{\partial P}{\partial T}\bigg|_{V} = - \frac{1}{\frac{\partial T}{\partial V}\big|_{P}}
\end{equation}
Ezt átrendezve a következőt kapjuk:
\begin{equation}
     \frac{\partial V}{\partial P}\bigg|_{T} \cdot \frac{\partial P}{\partial T}\bigg|_{V} \cdot \frac{\partial T}{\partial V}\bigg|_{P} = -1
\end{equation}
Ebből az egyenletből ki tudjuk fejezni az együtthatókat:
\begin{equation}
    \begin{aligned}
        \kappa &= -\frac{1}{V} \frac{\partial V}{\partial P}\bigg|_{T} \\
        \gamma &= \frac{1}{P} \frac{\partial P}{\partial T}\bigg|_{V} \\
        \beta &= \frac{1}{V} \frac{\partial V}{\partial T}\bigg|_{P}
    \end{aligned}
\end{equation}
Ezeket beírva az előző egyenletbe:
\begin{equation}
    -V\kappa \cdot p\gamma \cdot \frac{1}{V\beta} = -1
\end{equation}
Ebből pedig az együtthatók közötti összefüggést kapjuk:
\begin{equation}
   \beta = \kappa \gamma p
\end{equation}

\subsubsection{Kinetikus Gázelmélet - Kinetikus modell}
Az ideális gáz esetben feltételeztük, hogy a gáz részecskéi kis kitejedésűek és a köztük lévő külcsünhatások nagyon rövidtávúak. Tehát a gáz részecskéi csak akkor tudnak interaktálni, ha ütköznek egymással.
Emiatt az ideális gázt elképzelhetjük úgy, mint egy nagy számú kis golyó, amelyek folyamatosan mozognak és ütköznek egymással. Ezt a modellt hívjuk kinetikus gázelméletnek vagy kinetikus modellnek.
Ebben a modellben a gáz részecskéi véletlenszerűen mozognak, és az ütközések során energiát és lendületet cserélnek egymással. A gáz nyomása a részecskék ütközéseinek eredménye az edény falával, míg a hőmérséklet a részecskék átlagos kinetikus energiájával van összefüggésben.
Ennek a modellnek adhatunk egy valószínűségi értelmezést is. Például a nyomást értelmezhetjük úgy is mint annak a valószínűségét, hogy egy adott részecske ütközik az edény falával egy adott időintervallumban.
A hőmérséklet pedig annak a valószínűségét jelenti, hogy egy adott részecske rendelkezik egy bizonyos kinetikus energiával:
\begin{figure}
    \centering
    \includegraphics[width=0.7\textwidth]{imgs/5-tetel/kinetikus_model_eloszlas.png}
    \caption{Kinetikus gázelmélet részecske sebesség eloszlása}
\end{figure}
Ezt az eloszlást Maxwell-Boltzmann eloszlásnak nevezzük, és azt mutatja meg, hogy a gáz részecskéi milyen sebességgel rendelkeznek egy adott hőmérsékleten. Az eloszlás csúcsa a legvalószínűbb sebességet jelöli, míg a szélessége azt mutatja meg, hogy mennyire változatosak a részecskék sebességei.
Ezek alapján a kinetikus gázelméletre 6 darab feltételezést lehet felírni:
\begin{itemize}
    \item A gáz részecskéi pontszerűek, tömegük $\nu$
    \item Nincs kölcsönhatás a részecskék között, kivéve az ütközéseket
    \item Minden részecske $v$ sebességgel mozog (mint ha a Boltzmann eloszlás egy Dirac-delta lenne) - nem szükséges de egyszerűség kedvéért feltesszük (másképpen megfogalmazva, a részecskék átlagos sebessége $v$)
    \item Az ütközések rugalmasak
    \item $V$ térfogatban $N$ darab részecske van
    \item A részeckék helyzete és sebessége között nincs korreláció
\end{itemize}
Ezek alapján vegyünk egy dobozt, amiben egy ideális gáz van:
\begin{figure}[H]
    \centering
    \includegraphics[width=0.5\textwidth]{imgs/5-tetel/kinetikus_model_doboz.png}
    \caption{Kinetikus gázelmélet doboz modell}
\end{figure}
Próbáljuk meg kiszámolni, hogy a gáz részecskéi milyen nyomást gyakorolnak a doboz falára. Azt tudjuk, hogy amikor as részecske $v$ sebességgel $x$ irányba nekiütközik a falnak, akkor a falra ható erő:
\begin{equation}
    F_x = \frac{\Delta I_x}{\Delta t}
\end{equation}
Itt $I$-vel jelüljük az impulzust, hogy a nyomás jelölésével ne keverjük össze. Bontsuk fel a dobozt is egységekre, úgyhogy egy kis doboz-szelet szélessége pont $v\delta t$m vagyis 
pont akkora, amekkorát a részecske egy időegység alatt megtesz. A részecske impulzusváltozása az ütközés során:
\begin{equation}
    |\Delta I_x| = |I_{x, \text{ütközés után}} - I_{x, \text{ütközés előtt}}| = |-\nu v - (\nu v)| = 2\nu v
\end{equation}
Ahol $\nu$ a részecske tömege. Már csak az a kérdés, hogy egy időegység alatt hányszor ütközik a részecske a fallal. Mivel időegység alatt csak az a részecske ütközik a fallal, ami a kis doboz-szeletben van, 
és azok közül is csak az, amelyiknek $x$ irányú sebessége van. Tehát az ütközések száma időegység alatt:
\begin{equation}
    \Delta n = v \Delta t \cdot A \cdot \varrho \cdot \frac{1}{6}
\end{equation}
Ahol $A$ a doboz fala, $\varrho$ a részecskék sűrűsége ($\varrho = \frac{N}{V}$), és az $\frac{1}{6}$ azért van, mert a részecskék hat irányba mozgáshatnak egyenlő valószínűséggel (mivel függetlenek egymástól és véletlenszerűen mozognak), de csak a pozitív $x$ irányú sebességgel rendelkező részecskék ütköznek a fallal.
Tehát az impulzusváltozás időegység alatt:
\begin{equation}
    \Delta I_x = \Delta n \cdot 2\nu v = -2\nu v \cdot v \Delta t \cdot A \cdot \varrho \cdot \frac{1}{6}
\end{equation}
Innen a nyomás:
\begin{equation}
        P = \frac{F_x}{A} = \frac{1}{3} \nu \varrho v^2 = \frac{2}{3} \frac{1}{2} \frac{N}{V} \nu v^2
\end{equation}
\begin{equation}
    P V = \frac{2}{3} N \cdot \underbrace{\frac{1}{2} \nu v^2}_{E_{kin}}
\end{equation}
Ahol $E_{kin}$ a részecskék átlagos kinetikus energiája. Ebből látható, hogy a nyomás és a térfogat szorzata arányos a részecskék összes kinetikus energiájával. Ebből az összefüggésből ki tudjuk fejezni a részecskék átlagos kinetikus energiáját is:
\begin{equation}
    E_{kin} = \frac{3}{2} \frac{P V}{N}
\end{equation}
Ha az ideális gáz állapotegyenletét is felhasználjuk, akkor:
\begin{equation}
    E_{kin} = \frac{3}{2} \frac{n R T}{N} = \frac{3}{2} k_B T
\end{equation}
Ahol $k_B$ a Boltzmann állandó ($k_B = \frac{R}{N_A}$). Ebből látható, hogy a részecskék átlagos kinetikus energiája csak a hőmérséklettől függ, és független a nyomástól és a térfogattól. Ez azt is jelenti, hogy a hőmérséklet egy mértéke a részecskék átlagos kinetikus energiájának.
Ebből pedig a rendszer teljes energiája:
\begin{equation}
    E = N \cdot E_{kin} = \frac{3}{2} N k_B T = \frac{3}{2} pV
\end{equation}
Ez viszont nem teljesen igaz, mert lehetnének további energia komponensek is, például egy két atomos molekulánál a belső energia tartalmazhatna forgási és rezgési energiákat is, viszont a nyomás továbbra is csak a kinetikus energiától függ.
Ezt az ekvipartíció-tétellel tudjuk figyelembe venni, ami kimondja, hogy minden szabadsági fokra egyenlő mértékben oszlik el az energia. Tehát ha egy részecskének $f$ szabadsági foka van, akkor az átlagos energiája:
\begin{equation}
    E = \frac{f}{2} k_B T
\end{equation}
Így a teljes energia pedig:
\begin{equation}
    E = N \cdot E_{kin} = \frac{f}{2} N k_B T = \frac{f}{2} pV
\end{equation}
A szabadsági fokok a részecskék összetételétől és köztük lévő kölcsönhatásoktól függenek. Például egy egyatomos gáz esetén csak a három transzlációs szabadsági fok van jelen, így $f = 3$. Egy kétatomos molekulánál viszont már vannak forgási és rezgési szabadsági fokok is, így $f$ értéke nagyobb lehet (például $f = 5$ vagy $f = 7$). 
Szilárd anyagoknál általában $f = 6$ (három transzlációs és három rezgési szabadsági fok).

Ez a kinetikus kázelmélet alapján le lehet vonni a konklúziót, hogy a hőmérséklet tulajdonképpen rendezetlen mozgás.

\subsubsection*{Van Der Waals állapotegyenlet}
Az ideális gáz állapotegyenletét úgy kaptuk meg, hogy feltételeztük, hogy a gáz részecskéi pontszerűek és nincs kölcsönhatás közöttük. Viszont a valóságban a gáz részecskéi rendelkeznek véges térfogattal és kölcsönhatásban állnak
egymással, például vonzó erők hatnak rájuk. Ekkor azt tudjuk csinálni, hogy vesszük a $pV$ állapotegyenlet által meghatározott görbét, és elkezdem csökkenteni a térfogatot, miközben a részecskék száma és a hőmérséklet fix.
Ha a részecskéknek van kiterjedése (és feltesszük, hogy összenyomhatatlanok), akkor van egy minimum térfogat, ami alá nem lehet csökkenteni, mert a részecskét az összes rendelkezésre álló teret kitöltik.
Ekkor módosíthatjuk az állapotegyenletet úgy, hogy a térfogatból kivonjuk a részecskék által elfoglalt térfogatot:
\begin{equation}
    p (V - b n) = n R T
\end{equation}
Ahol $b$ egy anyagfüggő állandó, ami a részecskék kiterjedését veszi figyelembe. Például a hidrogén esetében $b \approx 0.0265 \frac{L}{mol}$, míg a nitrogén esetében $b \approx 0.0391 \frac{L}{mol}$.
Ezen felül feltételezhetjük hogy a részecskék között kölcsönhatás is van. Ez általában vonzó erő, ami csökkenti a részecskék közötti távolságot, így a nyomást is csökkenti.
Ezt úgy tudjuk figyelembe venni, hogy a nyomásból levonunk egy olyan tagot, ami a részecskék közötti kölcsönhatást veszi figyelembe. De hogyan becsüljük meg, hogy mekkora ez a tag?
Gondoljunk arra, hogy a részecskék közötti kölcsönhatás erőssége arányos a részecskék sűrűségének négyzetével, hisz minél több részecske van egy adott térfogatban, annál nagyobb az esélye annak, hogy két részecske kölcsönhatásba lépjen egymással.
Tehát ennek a korrekciós tagnak arányosnak kell lennie a részecskék sűrűségével:
\begin{equation}
    f_1 \sim \varrho = \frac{n}{V}
\end{equation}
Mivel a nyomás egy erő per egység terület, ezért az erő is arányos kell legyen a részecskék sűrűségével, viszont ha veszek egy kis felületet a dobozban és megnézem, hogy mennyi erő hat rá a részecskék közötti kölcsönhatás miatt, akkor az erőnek arányosnak kell lennie a felület mögötti részecskék számával is. Tehát az erőnek arányosnak kell lennie a részecskék sűrűségének négyzetével:
\begin{equation}
    f_2 \sim \varrho^2 = \left(\frac{n}{V}\right)^2
\end{equation}
Tehát a "nyomása" a részecskék közötti kölcsönhatásnak arányosnak kell lennie a részecskék sűrűségének négyzetével:
\begin{equation}
    \frac{F}{A} = a \varrho^2 = a \left(\frac{n}{V}\right)^2
\end{equation}
Ahol $a$ egy arányossági tényező. Innen tehát a nyomás:
\begin{equation}
    p_{\text{mért}} = p_{\text{valós}} - a \left(\frac{n}{V}\right)^2
\end{equation}
Tehát amikor a nyomást megmérem, akkor ezt fogom kapni, ezért a tényleges nyomást úgy tudom kifejezni, hogy hozzáadom ezt a korrekciós tagot:
\begin{equation}
    p_{\text{valós}} = p_{\text{mért}} + a \left(\frac{n}{V}\right)^2
\end{equation}
Ezt a két korrekciós tagot be tudom helyettesíteni az ideális gáz állapotegyenletébe, így megkapom a Van Der Waals állapotegyenletét:
\begin{equation}
    \left(p + a \frac{n^2}{V^2}\right) (V - b n) = n R T
\end{equation}
Ahol $a$ és $b$ anyagfüggő állandók, amelyek a részecskék közötti kölcsönhatást és kiterjedést veszik figyelembe. Például a hidrogén esetében $a \approx 0.244 \frac{L^2 \cdot atm}{mol^2}$ és $b \approx 0.0265 \frac{L}{mol}$, míg a
nitrogén esetében $a \approx 1.370 \frac{L^2 \cdot atm}{mol^2}$ és $b \approx 0.0391 \frac{L}{mol}$. Ezeket a konstansokat úgy tudjuk meghatározni,
hogy ha felrajzoljuk a Van Der Waals izotermákat, akkor van egy kritikus pont, ahol a görbe inflexiós ponttá válik. Ebből a kritikus pontból ki tudjuk számolni az $a$ és $b$ értékét:
\begin{equation}
    \begin{aligned}
        a &= 3\frac{p_c V_c^2}{n^2} \\
        b &= \frac{V_c}{3n} \\
        R &= \frac{8}{3} \frac{p_c V_c}{T_c n}
    \end{aligned}
\end{equation}
Ahol $T_c$, $V_c$ és $p_c$ a kritikus hőmérséklet, térfogat és nyomás. Ez a kritikus pont az a pont, ahol a gáz és folyadék fázisok közötti határvonal eltűnik, és a gáz és folyadék fázisok egyetlen fázisba olvadnak össze, amit szuperkritikus fluidumnak nevezünk.
Ezt visszahelyettesítve a Van Der Waals állapotegyenletébe kapjuk:
\begin{equation}
    \left(p + 3\frac{p_c V_c^2}{n^2} \frac{n^2}{V^2}\right) \left(V - \frac{V_c}{3n} n\right) = n R T
\end{equation}
Ahol bevezethetjük a redukált változókat is:
\begin{equation}
    \begin{aligned}
       p_r &= \frac{p}{p_c} \\
       V_r &= \frac{V}{V_c} \\
       T_r &= \frac{T}{T_c}
    \end{aligned}
\end{equation}
Így a Van Der Waals állapotegyenlet redukált formája:
\begin{equation}
    \left(p_r + \frac{3}{V_r^2}\right) \left(3 V_r - 1\right) = 8 T_r
\end{equation}
Ezzel a relatív skálával az összes gáz állapotegyenletét azonos alakra tudjuk hozni, így könnyebben összehasonlíthatók egymással.

Ez a kritikus pont a fázisátalakulások vizsgálata során lesz rendkívül fontos, hiszen ez a pont alatt kezd el a gáz folyadékká kondenzálódni.

\subsection{Nyílt és zárt folyamatok, Carnot-folyamat, Főtételek}
Most, hogy megértettük a hőmérséklet fizikai értelmezését, nézzük meg, hogy hogyan viselkednek a termodinamikai rendszerek különböző folyamatok során.
Azt tudjuk, hogy a hő egy intenzív állapotjelző, ami azt jelenti, hogy ha két rendszert összakapcsolunk, amelyeknek különböző a hpmérséklete, akkor a hő megpróbál kiegyenlítődni.
Ez azt jelenti, hogy a hő valamilyen formán képes áramolni és ennek a jellemzésére vezette be a "kalória" mértékegységet Nicolas Clément 1824-ben. A kalóriát ő 
úgy definiálta, hogy az 1 gramm víz hőmérsékletét 1 Celsius fokkal emeli meg. Ebből adódóan a kalória egy energia mértékegység is, hiszen a hőenergia mértékegysége.
Később aztán James Prescott Joule-nak sikerült kimutatnia, hogy súrlódás segítségével mechanikai munkát is át lehet alakítani hővé, ebből pedig kiderült,
hogy a hőmérsékletet át lehet váltani energiává és fordítva. A kettő közötti pontos relációt a Joule-féle kísérlet alapján határozták meg, ahol azt mérték, hogy
egy edényben lévő vizet mennyire tudnak felmelegíteni egy súly segítségével, ami leesik és a vízben lévő lapátokat forgatja. Ebben a kísérletben a súly által végzett
$mgh$ munka átalakult rotációs munkává ($\tau \theta$, ahol $\tau$ a forgatónyomaték, $\theta$ pedig a szögelfordulás), ami pedig hővé alakult át a vízben. Ebből pedig pontosan meg tudták határozni a mechanikai munka és a hő közötti átváltási tényezőt, amit Joule-féle állandónak neveznek:
\begin{equation}
    1 \text{ cal} = 4.186 \text{ J}
\end{equation}
Ez azt jelenti, hogy 1 kalória hőenergia 4.186 Joule mechanikai munkának felel meg. Ebből adódóan a hőmennyiség mértékegysége a Joule is lehet, és a hőmennyiség jele $Q$.
\begin{figure}[H]
    \centering
    \includegraphics[width=0.6\textwidth]{imgs/5-tetel/joule_kiserlet.jpeg}
    \caption{Joule-féle kísérlet a mechanikai munka és hő közötti átváltás kimutatására}
\end{figure}

\subsubsection{Temodinamika I. főtétele}
Most, hogy láttuk, hogy a hőmennyiség és a munka között van kapcsolat, nézzük meg, hogy hogyan viselkedik egy termodinamikai rendszer, amikor hőt vesz fel vagy ad le, illetve amikor munkát végez vagy végeztetnek vele.
Nézzünk meg egy abiatikusan zárt rendszert, ami azt jelenti, hogy a rendszer nem tud hőt cserélni a környezetével, de munkát tud végezni vagy végeztetnek vele:
\begin{figure}[H]
    \centering
    \includegraphics[width=0.5\textwidth]{imgs/5-tetel/adiabatikus_zart_rendszer.png}
    \caption{Adiabatikus zárt rendszer}
\end{figure}
Ebben az esetben a rendszer belső energiája csak a munkavégzés miatt változik meg. Viszont többféle munkavégzés is lehetséges, például lehet egy propeller bent ami kavarja a folyadékot,
vagy egy réz drót amin áramot keringetek, de lehet egy dugattyú is, amivel a térfogatot tudom változtatni. Nézzük meg, hogy hogyan tudjuk kifejezni a dugattyúval végzett munkát.
Tegyük fel, hogy a dugattyú egy kis elmozdulást végez $\Delta s$-szel, miközben a dugattyú keresztmetszete $A$. Ekkor a dugattyú által végzett munka:
\begin{equation}
    W = F \cdot \Delta s
\end{equation}
Ahol $F$ a dugattyúra ható erő. Ezt az erőt pedig ki tudjuk fejezni a nyomás segítségével:
\begin{equation}
    F = p \cdot A
\end{equation}
Ezt visszahelyettesítve a munkára:
\begin{equation}
    W = p \cdot A \cdot \Delta s
\end{equation}
Az $A \cdot \Delta s$ pedig a dugattyú által elmozdított térfogat, vagyis $\Delta V$:
\begin{equation}
    W = p \cdot \Delta V
\end{equation}
Ha a dugattyút összenyomjuk, akkor a térfogat csökken, így a dugattyú által végzett munka negatív lesz:
\begin{equation}
    W = - p \cdot \Delta V
\end{equation}
Ez akkor igaz, ha lassan nyomom be a dugattyút, mert akkor a folyamat kvázi-statikus lesz. Ha gyorsan nyomom be, akkor turbulens áramlás alakulhat ki, és a munka nem csak a térfogatváltozás miatt fog megváltozni.
Most vegyük fel a $pV$ grafikont, és jelöljünk ki rajta két pontot ($A$, $B$):
\begin{figure}[H]
    \centering
    \includegraphics[width=0.5\textwidth]{imgs/5-tetel/pV_diagram_1.png}
    \caption{$pV$ diagram két állapot között}
\end{figure}
Mivel nem tudjuk pontosan milyen anyag van a rendszerben ezért nem tudjuk, pontosan hogyan változik a nyomás a térfogat függvényében. A kezdő és a végpontját viszont ismerjük.
Viszont fel tudok venni olyan $B$ pontot, ahová kvázisztatikusan nem tudom eljuttatni a rendszert, csak mondjuk egy $C$ pontba. Ekkor jönnek be az extra
munkát végezni képes dolgok mint a propeller vagy a fűtőszál, amikkel "fel tudom tolni" a rendszert a $B$ pontba:
\begin{figure}[H]
    \centering
    \includegraphics[width=0.5\textwidth]{imgs/5-tetel/pV_diagram_2.png}
    \caption{$pV$ diagram két állapot között, extra munkavégzéssel}
\end{figure}
A kvázisztatikus munkavégzést a $A \to C$ úton tudom kiszámolni, hogy a $pV$ függvény alatti területet meghatározom:
\begin{equation}
    \delta W = - p dV
\end{equation}
\begin{equation}
    W_{A \to C} = \int_{V_A}^{V_C} - p dV
\end{equation}
Azt meg hogy mennyi extra munkát végeztem a $C \to B$ úton, azt külön meg tudom mérni, például Joule módszerével. Így a teljes munka:
\begin{equation}
    W_{A \to B} = W_{A \to C} + W_{C \to B}
\end{equation}
De ebbe a $B$ állapotba nem csak úgy tudok eljutni, hogy összenyomom a dugattyút majd melegítek, hanem úgy is, hogy először melegítek, majd összenyomom a dugattyút,
vagy nyomok egy kicsit, melegítek egy kicsit majd megint nyomok stb. Tehát a $B$ állapotba végtelen úton el tudok jutni, és az, hogy melyik úton teszem,
nem befolyásolja a teljes munkavégzés értékét. Ez a termodinamika I. főtétele, ami kimondja, hogy:
 \begin{center}
    \textbf{Termodinamika I. főtétele:} \\
    \emph{Egy abiatikusan zárt rendszer esetén, ha a rendszer A-ból B állapotba jut, akkor a munkavégzés csak a végállapotoktól függ.}
 \end{center}
 Vagy matematikailag kifejezve:
 \begin{equation}
    W_{A \to B} = U(B) - U(A)
 \end{equation}
Ahol $U$ a rendszer belső energiája, ami csak a rendszer állapotától függ. Ebből az következik, hogy a belső energia egy extenzív állapotjelző.
A belső energiát is ki kell tudnom fejezni a rendszer állapotjelzőivel, pl.:
\begin{equation}
    U(p, V) \qquad U(p, T) \qquad U(V, T)
\end{equation}
Tehát felírható a belső energia változása a rendszer állapotjelzőinek változásával:
\begin{equation}
    dU = \frac{\partial U}{\partial p}\bigg|_{V} dp + \frac{\partial U}{\partial V}\bigg|_{p} dV
\end{equation}
Két pont között tehát a belső energia változása:
\begin{equation}
    \oint dU = \oint \frac{\partial U}{\partial p}\bigg|_{V} dp + \frac{\partial U}{\partial V}\bigg|_{p} dV = \oint \left(\frac{\partial U}{\partial p}\bigg|_{V}, \frac{\partial U}{\partial V}\bigg|_{p}\right)
    \begin{pmatrix}
        dp \\
        dV
    \end{pmatrix}
\end{equation}
Ha ezt egy zárt görbén veszem, akkor a belső energia változása nulla kell legyen, hiszen ugyanabba az állapotba térek vissza:
\begin{equation}
    \oint dU = \oint \nabla E d \bold{r} = 0
\end{equation}
Ez pedig azt jelenti, hogy a belső energia egy jól definiált állapotjelző, hiszen a belső energia változása csak a kezdő és végponttól függ, és nem az úttól.

Nézzük meg most azt az esetet, amikor a rendszer nem abiatikusan zárt, hanem képes hőt cserélni a környezetével is:
\begin{figure}[H]
    \centering
    \includegraphics[width=0.5\textwidth]{imgs/5-tetel/nem_adiabatikus_zart_rendszer.png}
    \caption{Abiatikus és Nem abiadatikus zárt rendszer pV grafikonja}
\end{figure}
Ha nincs szigetelve a rendszer, akkor kívülről hő áramolhat be vagy ki a rendszerből, ami módosíthatja a pályát a $pV$ grafikonon $A$ és $B$ állapot között. 
Továbbra is kvázistatikus folyamatokat vizsgálunk, ezért lehet sima görbékkel jellemezni a problémát. Ekkor a munkavégzés egyenletét kiegészítjük egy hőmennyiség taggal:
\begin{equation}
    U(B) - U(A) = W_1 + Q_1 = W_2 + Q_2
\end{equation}
Ahol $Q$ a rendszer által felvett hőmennyiség, melynek a mértékegysége szintén Joule. Ha $A$ és $B$ között a távolságot csökkentem, akko felírható a belső energia változása differenciális formában is:
\begin{equation}
    dU = \delta W + \delta Q
\end{equation}
Azt tudjuk, hogy zát görbén továbbra is igaz, hogy a belső energia változása nulla:
\begin{equation}
    \oint dU = 0
\end{equation}
Viszont a munkavégzés és a hőmennyiség tagjai már nem lehetnek külön-külön nullák, hiszen a környezet is végez munkát:
\begin{equation}
    \oint DU = 0 = \oint \delta W + \oint \delta Q = \underbrace{\oint - p dV}_{\neq 0} + \underbrace{\oint \delta Q}_{\neq 0}
\end{equation}
Itt azért van $\delta$ jel a $W$ és $Q$ tagok előtt, mert ezzel jelöljük, hogy ezek nem biztosan nullák zárt görbén.
Ebből következik az is, hogy $Q$ és $W$ sem lehetnek állapotjelzők, hiszen külön-külön nem csak a kezdő és végponttól függenek, hanem az úttól is.
Ezek alapján az első főtételt úgy is lehet értelmezni, hogy nem lehet elsőfajú örökmozgót készíteni, ami azt jelenti, hogy nem lehet olyan gépet készíteni,
ami több munkát képes végezni, mint amennyi energiát vesz fel a környezetéből. Ez az energia megmaradás törvénye termodinamikai rendszerekre is kiterjesztve.

\subsubsection{Termodinamikai folyamatok}
A termodinamikában a "folyamat" kifejezést olyan változások leírására használjuk, amelyek során egy rendszer egyik állapotból a másikba kerül. Ezt a megváltozást
egy tetszőleges állapotjelzők által leírható görbével tudjuk jellemezni, leggyakrabban a $pV$ diagramot használva. A folyamatok lehetnek kvázisztatikusak, ahol a rendszer egyensúlyi állapotban van minden pillanatban, vagy irreverzibilisek, ahol a rendszer nem éri el az egyensúlyt.
Az egyszerűség kedvéért mi eddig csak kvázisztatikus folyamatokat vizsgáltunk. Egy folyamatot jellemezhetünk a nyíltáságval is, nyílt rendszernek azt a rendszert nevezzük, amely képes anyagot vagy energiát is cserélni a környezetével, míg adiabatikusan zárt rendszerben a környezet semmilyen munkát nem tud végezni a rendszeren, és anyagot sem tud kicserélni vele.
Lehet még csak zárt rendszerről is beszélni ahol anyagot nem tud kicserélni a környezettel, de hőt igen.
Ezen felül a folyamatokat jellemezhetjük azzal is, hogy milyen állapotjelzők maradnak állandóak a folyamat során. Néhány fontosabb folyamat típus:
\begin{itemize}
    \item \textbf{Izoterm folyamat:} A hőmérséklet állandó marad a folyamat során ($dT = 0$). Ilyen folyamat például egy gáz lassú összenyomása, miközben hőt ad le a környezetének.
    \item \textbf{Izobár folyamat:} A nyomás állandó marad a folyamat során ($dp = 0$). Ilyen folyamat például egy gáz fűtése egy nyitott edényben, ahol a gáz térfogata növekszik.
    \item \textbf{Izokor folyamat:} A térfogat állandó marad a folyamat során ($dV = 0$). Ilyen folyamat például egy gáz fűtése egy zárt edényben, ahol a nyomás növekszik.
    \item \textbf{Adiabatikus folyamat:} Nincs hőcsere a rendszer és a környezet között ($Q = 0$). Ilyen folyamat például egy gáz gyors összenyomása vagy tágulása, ahol a hőmérséklet változik.
\end{itemize}
A folyamatok jellemzéséhez szükség van a belső energiák ismeretére:
\begin{itemize}
    \item \textbf{ideális gáz}: $U = \frac{f}{2} n R T$
    \item \textbf{Van Der Waals gáz}: $U = \frac{f}{2} n R T - a \frac{n^2}{V}$
\end{itemize}

Emellett bezethetünk pár extra fogalmat is:
\begin{itemize}
    \item \textbf{Hőkapacitás}: A hőkapacitás azt mutatja meg, hogy mennyi hő szükséges egy rendszer hőmérsékletének egy egységnyi mértékben történő megváltoztatásához.
    \begin{equation}
        \hat{C} = \frac{\delta Q}{dT} \qquad \left[\hat{C}\right] = \frac{J}{K}
    \end{equation}
    \item \textbf{Fajhő}: A fajhő azt mutatja meg, hogy mennyi hő szükséges egy egységnyi tömegű anyag hőmérsékletének egy egységnyi mértékben történő megváltoztatásához.
    \begin{equation}
        c = \frac{1}{m} \frac{\delta Q}{dT} \qquad \left[c\right] = \frac{J}{kg \cdot K}
    \end{equation}
    \item \textbf{Moláris hőkapacitás}: A moláris hőkapacitás azt mutatja meg, hogy mennyi hő szükséges egy mól anyag hőmérsékletének egy egységnyi mértékben történő megváltoztatásához.
    \begin{equation}
        C = M \cdot c = \frac{M}{n} \frac{\delta Q}{dT} = \frac{1}{n} \frac{\delta Q}{dT} \qquad \left[C\right] = \frac{J}{mol \cdot K}
    \end{equation}
\end{itemize}
Nézzük meg ezeket a fogalmakat egy ideális gáz példáján keresztül. Vegyünk egy ideális gázt, ami egy zárt edényben van, és melegítsük fel izokor folyamat során.
Ekkor a belső energia változása:
\begin{equation}
    dU = \frac{\partial U}{\partial V}\bigg|_{T} dV + \frac{\partial U}{\partial T}\bigg|_{V} dT
\end{equation}
Ez még minden folyamatra igaz. Viszont $V = áll$ , így az első tag eltűnik: 
\begin{equation}
    dU = \frac{\partial U}{\partial T}\bigg|_{V} dT
\end{equation}
A termodinamika I. főtétele szerint pedig:
\begin{equation}
    dU = \delta W + \delta Q
\end{equation}
Viszont mivel izokor folyamatot vizsgálunk, így $dV = 0 \to W = pdV = 0$, tehát a munkavégzés is nulla lesz:
\begin{equation}
    dU = \delta Q
\end{equation}
Ebből következik, hogy:
\begin{equation}
    \delta Q = \frac{\partial U}{\partial T}\bigg|_{V} dT
\end{equation}
Tehát a moláris hőkapacitás izokor folyamat során:
\begin{equation}
    C_V = \frac{1}{n} \frac{\delta Q}{dT} = \frac{1}{n} \frac{\partial U}{\partial T}\bigg|_{V}
\end{equation}
Ez eddig bármilyen gázra igaz, de egy ideális gáz esetén a belső energia csak a hőmérséklet függvénye, így:
\begin{equation}
    U(V, T) = \frac{f}{2} n R T
\end{equation}
Ahol $f$ a részecskék szabadsági fokainak száma. Itt megjegyezzük, hogy a belső energia lehetne más állapotjelzők függvényeként is felírva, de itt a parciális
deriváltnál jelezzük, hogy melyik függvényére vagyunk épp kíváncsiak. Ebből következik, hogy:
\begin{equation}
    \frac{\partial U}{\partial T}\bigg|_{V} = \frac{f}{2} n R
\end{equation}
Így a moláris hőkapacitás izokor folyamat során:
\begin{equation}
    C^{\text{id.}}_V = \frac{f}{2} R
\end{equation}
Ez szilárd testek esetén például tipikusan $f = 6 \to C_V = 3R$. Ezt a Dulong-Petit törvénynek nevezzük.

Hasonlóan ki lehet számolni az izobár folyamatra is a molhőt, ahol a térfogat változik, így a munkavégzés nem lesz nulla:
\begin{equation}
    dU = \frac{\partial U}{\partial p}\bigg|_{T} dp + \frac{\partial U}{\partial T}\bigg|_{p} dT
\end{equation}
De mivel a nyomást tartjuk állandóan, így az első tag eltűnik:
\begin{equation}
    dU = \frac{\partial U}{\partial T}\bigg|_{p} dT
\end{equation}
A termodinamika I. főtétele szerint pedig:
\begin{equation}
    dU = \delta W + \delta Q = \delta Q - p dV
\end{equation}
Ha a téfogat változását akarjuk kifejezni a nyomás és hőmérséklet változásával, felírhatjuk, hogy:
\begin{equation}
    dV = \frac{\partial V}{\partial p}\bigg|_{T} dp + \frac{\partial V}{\partial T}\bigg|_{p} dT
\end{equation}
De mivel izobár folyamatot vizsgálunk, így $dp = 0$, tehát:
\begin{equation}
    dV = \frac{\partial V}{\partial T}\bigg|_{p} dT
\end{equation}
Ennek segítségével fel tudjuk írni a belső energia változását, mint a hőmérséklet és nyomás függvényét:
\begin{equation}
    dU = \delta Q - p \frac{\partial V}{\partial T}\bigg|_{p} dT
\end{equation}
Azért jó, hogy ezekre az állapotjelzőkre írtuk fel a belső energiát, mert a $p$ állandó és ideális gáznál a belső energia csak $T$-től függ.
Innen a moláris hőkapacitás izobár folyamat során:
\begin{equation}
    C_p = \frac{1}{n} \frac{\delta Q}{dT} = \frac{1}{n} \left(\frac{\partial U}{\partial T}\bigg|_{p} + p \frac{\partial V}{\partial T}\big|_{p}\right)
\end{equation}
Itt észrevehetjük, hogy a zárójelben lévő deriváltakat össze lehet vonni, hisz ugyanazon változó szerint deriválunk, és az állandó tag is ugyan az:
\begin{equation}
    C_p = \frac{1}{n} \frac{\partial \left(U + pV\right)}{\partial T}\bigg|_{p}
\end{equation}
Az $U + pV$ kifejezést entalpiának nevezzük, és $H$-val jelöljük:
\begin{equation}
    H = U + pV
\end{equation}
Így a moláris hőkapacitás izobár folyamat során:
\begin{equation}
    C_p = \frac{1}{n} \frac{\partial H}{\partial T}\bigg|_{p}
\end{equation}
Ideális gáz esetén a belső energia és a térfogat kifejezése:
\begin{equation}
    U = \frac{f}{2} n R T \qquad V = \frac{n R T}{p}
\end{equation}
Ebből következik, hogy az entalpia:
\begin{equation}
    H = \frac{f}{2} n R T + p \cdot \frac{n R T}{p} = \left(\frac{f}{2} + 1\right) n R T
\end{equation}
Ezt $T$ szerint deriválva és $\frac{1}{n}$-nel szorozva kapjuk a moláris hőkapacitást izobár folyamat során:
\begin{equation}
    C^{\text{id.}}_p = \left(\frac{f}{2} + 1\right) R
\end{equation}
Ebből következik, hogy az izobár és izokor moláris hőkapacitások közötti különbség:
\begin{equation}
    C^{\text{id.}}_p - C^{\text{id.}}_V = R
\end{equation}
Ha pedig ezt leosztjuk a Moláris tömeggel, akkor megkapjuk a fajhőkre vonatkozó összefüggést is:
\begin{equation}
    c^{\text{id.}}_p - c^{\text{id.}}_V = \frac{R}{M}
\end{equation}
Ezt az összefüggést Robert-Mayer-relációként ismerjük.

Fontos, hogy a molárris hő definíciójában szereplő $U$ nem ugyanaz a függvény, hisz más állapotjelzők szerint van definiálva a parciális deriváltban:
\begin{equation}
    \begin{aligned}
        U_{V, T} = U(V, T) \\
        U_{p, T} = U(p, T)
    \end{aligned}
\end{equation}
A kettő közötti kapcsolatot a következő összefüggés adja meg:
\begin{equation}
U_{p, T} (p, T)= U_{V, T} (V(p, T), T)
\end{equation}
Ami egy állapotegyenlet, mivel két független állapotjelző van benne, amik között kapcsolatot teremt egy egyenlet:
\begin{equation}
    \frac{\partial U}{\partial T}\bigg|_{p} = \frac{\partial E}{\partial V}\bigg|_{T} \frac{\partial V}{\partial T}\bigg|_{p} + \frac{\partial U}{\partial T}\bigg|_{V}
\end{equation} 
Ebből következik, hogy a moláris hőkapacitások közötti különbség általános esetben:
\begin{equation}
    C_p - C_V = \frac{1}{n} \left[\frac{\partial U}{\partial T}\bigg|_{p} - \frac{\partial U}{\partial T}\bigg|_{V} + \frac{\partial V}{\partial T}\bigg|_{p}\right] = \frac{1}{n} \left[\frac{\partial U}{\partial V}\bigg|_{T} + p \right] \underbrace{\frac{\partial V}{\partial T}\bigg|_{p}}_{v \beta}
\end{equation}
Ezeket hívják Maxwell relációknak. Ebből pedig következik:
\begin{equation}
    C_p - C_V = \frac{v T \beta^2}{\kappa} = \frac{T \cdot \frac{RT}{T} \cdot (\frac{1}{T})^2}{\frac{1}{p}} = R
\end{equation}
Ahol $\beta = \frac{1}{T}$ a térfogati hőtágulási együttható, és $\kappa = \frac{1}{p}$ a izotermikus összenyomhatósági együttható és $v$ a moláris térfogat ($v = \frac{V}{n}$).
Ez pedig a Robert-Mayer reláció általános esete.

\subsubsection{Adiabatikus folyamatok}
Egy adiabatikus folyamat során nincs hőcsere a rendszer és a környezet között, így $Q = 0$. Ebből következik, hogy a termodinamika I. főtétele szerint:
\begin{equation}
    dU = \delta W = - p dV
\end{equation}
A belső energiát pedig kifejezhetjük a nyomás és térfogat függvényében is:
\begin{equation}
    dU = \frac{\partial U}{\partial p}\bigg|_{V} dp + \frac{\partial U}{\partial V}\bigg|_{p} dV = - p dV
\end{equation}
Ebből következik, hogy:
\begin{equation}
   \frac{\partial U}{\partial p}\bigg|_{V} dp = - \left(p + \frac{\partial U}{\partial V}\bigg|_{p}\right) dV
\end{equation}
Ami egy differenciálegyenlet a $p$ és $V$ között. Ezt meg tudjuk oldani, ha kifejezzük a belső energia parciális deriváltját a nyomás és térfogat függvényében:
\begin{equation}
    \frac{dp}{dV} = p'(V) = f(p, V)
\end{equation}
Ahol $f(p, V)$ egy ismeretlen függvény. Ezt a differenciálegyenletet meg tudjuk oldani, ha felírjuk a belső energiát ideális gáz esetén:
\begin{equation}
    U = \frac{f}{2} n R T = \frac{f}{2} p V
\end{equation}
Ebből következik, hogy:
\begin{equation}
    \frac{\partial U}{\partial V}\bigg|_{p} = \frac{f}{2} p \qquad \text{és} \qquad \frac{\partial U}{\partial p}\bigg|_{V} = \frac{f}{2} V
\end{equation}
Ezt visszahelyettesítve a differenciálegyenletbe:
\begin{equation}
    \frac{f}{2} V dp = - \left(p + \frac{f}{2} p\right) dV
\end{equation}
Ahol az egyenletet átrendezve:
\begin{equation}
    \frac{dp}{p} = - \left(\frac{f}{2} + 1\right) \frac{dV}{V}
\end{equation}
Ezt az egyenletet most integráljuk ki:
\begin{equation}
    \int_{p_1}^{p_2} \frac{dp}{p} = - \left(\frac{f}{2} + 1\right) \int_{V_1}^{V_2} \frac{dV}{V}
\end{equation}
Ahol az integrálás után:
\begin{equation}
    \ln{\frac{p_2}{p_1}} = - \left(\frac{f}{2} + 1\right) \ln{\frac{V_2}{V_1}}
\end{equation}
Ezt exponenciálva:
\begin{equation}
    \frac{p_2}{p_1} = \left(\frac{V_1}{V_2}\right)^{\frac{f}{2} + 1}
\end{equation}
Ahol bevezethetjük a $\kappa$ kitevőt, ami a moláris hőkapacitások aránya:
\begin{equation}
    \kappa = \frac{C_p}{C_V} = \frac{\frac{f}{2} + 1}{\frac{f}{2}} = \frac{f + 2}{f}
\end{equation}
Így az adiabatikus folyamatokra a következő összefüggést kapjuk:
\begin{equation}
    p V^{\kappa} = \text{állandó}
\end{equation}
Ez az adiabatikus folyamatok egyenlete ideális gázokra.

\subsubsection{Politrop folyamatok}
A politrop folyamatok olyan termodinamikai folyamatok, amelyek során a rendszer állapotjelzői között egy adott hatványkapcsolat áll fenn. Ezek a folyamatok általánosítják az izoterm, izobár, izokor és adiabatikus folyamatokat.
A politrop folyamatokat a következő egyenlettel jellemezhetjük:
\begin{equation}
    p V^n = \text{állandó}
\end{equation}
Ahol $n$ a politrop index, amely meghatározza a folyamat típusát. Néhány speciális eset:
\begin{itemize}
    \item Ha $n = 0$, akkor az izobár folyamatot kapjuk ($p = \text{állandó}$).
    \item Ha $n = 1$, akkor az izoterm folyamatot kapjuk ($p V = \text{állandó}$).
    \item Ha $n = \kappa$, akkor az adiabatikus folyamatot kapjuk ($p V^{\kappa} = \text{állandó}$).
    \item Ha $n \to \infty$, akkor az izokor folyamatot kapjuk ($V = \text{állandó}$).
\end{itemize}
A politrop folyamatok során a rendszer hőmérséklete, nyomása és térfogata is változhat, és a politrop index értéke befolyásolja, hogy milyen mértékben történik ez a változás. A politrop folyamatok hasznosak lehetnek különböző termodinamikai rendszerek modellezésében és elemzésében, mivel lehetővé teszik a különböző folyamatok közötti átmenetek vizsgálatát.
Ideális gázokra a politrop folyamat a következőképpen írható fel:
\begin{equation}
    C = \text{állandó}
\end{equation}
Ahol $C$ a politrop hőkapacitás, a hőmennyiség változása:
\begin{equation}
    \delta Q = n C dT
\end{equation}
A belső energia változása pedig:
\begin{equation}
    dU = \delta Q - pdV = n C dT - p dV
\end{equation}
A belső energiát fejezzük ki a hőmérséklet és a térfogat függvényében ($U(V, T)$):
\begin{equation}
    dU = \frac{\partial U}{\partial V}\bigg|_{T} dV + \frac{\partial U}{\partial T}\bigg|_{V} dT = n C_V dT
\end{equation}
Ebből következik, hogy:
\begin{equation}
    n C_V dT = n C dT - p dV
\end{equation}
Ahol átrendezve:
\begin{equation}
    n (C_V - C) dT = - p dV
\end{equation}
Most fejezzük ki a hőmérséklet változását az ideális gáz állapotegyenletének segítségével:
\begin{equation}
    p = \frac{nR T}{V} \qquad \to \qquad
    n (C_V - C) dT = - \frac{nR T}{V} dV
\end{equation}
Ahol átrendezve:
\begin{equation}
    \frac{dT}{T} = - \frac{R}{C_V - C} \frac{dV}{V}
\end{equation}
Ezt az egyenletet most integráljuk ki:
\begin{equation}
    \int_{T_1}^{T_2} \frac{dT}{T} = - \frac{R}{C_V - C} \int_{V_1}^{V_2} \frac{dV}{V}
\end{equation}
Ahol az integrálás után:
\begin{equation}
    \ln{\frac{T_2}{T_1}} = - \frac{R}{C_V - C} \ln{\frac{V_2}{V_1}}
\end{equation}
Ezt exponenciálva:
\begin{equation}
    \frac{T_2}{T_1} = \left(\frac{V_1}{V_2}\right)^{\frac{R}{C_V - C}}
\end{equation}
Ebből következik, hogy a politrop folyamatokra az alábbi összefüggés érvényes ideális gázokra:
\begin{equation}
    p V^{\frac{C_P - C}{C_V - C}} = \text{állandó}
\end{equation}
Ahol a politrop index:
\begin{equation}
    k = \frac{C_P - C}{C_V - C}
\end{equation}
\begin{equation}
    \boxed{p V^{k} = \text{állandó}}
\end{equation}
Ezek alapján a politrop index segítségével különböző termodinamikai folyamatokat tudunk jellemezni, és az ideális gázokra vonatkozó összefüggések segítségével elemezni a rendszerek viselkedését:
\begin{itemize}
    \item Izobár folyamat: $C = C_p \to k = 0 \to pV^0 = \text{állandó}$
    \item Izoterm folyamat: $C_T = \infty \to k = 1 \to pV^1 = \text{állandó}$
    \item Izokor folyamat: $C = C_V \to k \to \infty \to pV^{\infty} = \text{állandó}$
\end{itemize}

\subsubsection*{Folyamatok összefoglás}
Összefoglalva a termodinamikai folyamatokat ideális gázokra amikor $(p_1, V_1, T_1) \to (p_2, V_2, T_2)$ állapotváltozás történik:
\begin{table}[H]
    \centering
    \begin{tabular}{|c|c|c|c|}
        \hline
        Folyamat típusa & Belső energia: $\Delta U_{12}$ & Rendszer munkavégzése: $W_{12}$ & Hő változása $Q_{12}$ \\
        \hline
        Izoterm $T = \text{áll}$ & $0$ & $nRT \ln{\frac{V_1}{V_2}}$ & $- nRT \ln{\frac{V_2}{V_1}}$ \\
        \hline
        Izobár $p = \text{áll}$ & $nC_V (T_2 - T_1)$ & $- p (V_2 - V_1)$ & $n C_p (T_2 - T_1)$ \\
        \hline
        Izokor $V = \text{áll}$ & $nC_V (T_2 - T_1)$ & $0$ & $n C_V (T_2 - T_1)$ \\
        \hline
        Adiabatikus $Q = 0$ & $nC_V (T_2 - T_1)$ & $- \frac{p_1V_1}{\kappa-1}\left[1- (\frac{V_1}{V_2})^\kappa\right]$ & $0$ \\
        \hline
        Politrop $C = \text{áll}$ & $nC_V (T_2 - T_1)$ & $- \frac{p_1V_1}{k-1}\left[1- (\frac{V_1}{V_2})^k\right]$ & $n C (T_2 - T_1)$ \\
        \hline
    \end{tabular}
    \caption{Termodinamikai folyamatok összefoglalása ideális gázokra}
\end{table}

\subsubsection{Körfolyamatok, irreverzibilitás}
Eddig olyan rendszereket vizsgáltunk, amikor a rendszer kvázistatikus folyamatokon keresztül egyik állapotból a másikba jutott. Ezeknek a folyamatoknak van egy 
speciális tulajdonsága, mégpedig, hogy reverzibilisek, tehát vissza is tudnak jutni az eredeti állapotba ugyanazon úton. A kvázistatikus folyamatoknál a rendszer egyensúlyi állapotokon halad keresztül, és ezen egyensúlyi állapotok között lehet "mozogni".
Nem kvázisztatikus folyamatok esetén viszont csak a kezdő és a végpontot ismerem, a kettő közötti utat nem, ezért ezek a folyamatok irreverzibilisek. Ez azt jelenti, hogy
a végállapotból a kezdő állapotba nem tudom ugyanazon az úton visszajutni, ha beáll a végállapotba, akkor csak extra munkavégzéssel tudok visszajutni a kezdő állapotba.
Ilyen például az amikor két eltérő hőmérsékletű testet összekötöm, és a hő a melegebb testből a hidegebb testbe áramlik. Ekkor a két test hőmérséklete kiegyenlítődik, és magától sose fog visszaállni az eredeti állapotba.
Ilyen lehet még ha veszek két gázt amiknek más a nyomása, és közéjük teszek egy dugattyút. Ha elengedem a dugattyút, akkor a két oldalán a nyomás ki fog egyenlítődni, a dugattyú elmozdulásával.
\begin{figure}[H]
    \centering
    \includegraphics[width=0.5\textwidth]{imgs/5-tetel/irreverzibilis_folyamatok.png}
    \caption{Néhány irreverzibilis folyamat schematikus ábrázolása}
\end{figure}
kvázistatikus folyamatok is lehetnek irreverzibilisek, például ha két eltérő hőmérsékletű testet egy vékony rézdróttal kötök össze, akkor a hőcsere lassan és egyensúlyi állapotokon keresztül történik,
de a folyamat akkor is irreverzibilis, hiszen a hő nem fog magától visszafolyni a hidegebb testből a melegebbe. Tehát az irreverzibilitást nem lehet csak a kvázistatikussággal definiálni.

Nézzük meg, hogy akkor mi is egészen pontosan az irreverzibilitás. Vegyünk egy gázzal töltött dugattyús hengert, ami egy hőtartályban van. Nézzük meg mi történik, ha lassan (kvázisztatikusan) összenyomom:
\begin{figure}[H]
    \centering
    \includegraphics[width=0.5\textwidth]{imgs/5-tetel/reverzibilis_folyamat.png}
    \caption{Reverzibilis folyamat egy dugattyús hengerben}
\end{figure}
Látható, hogy a gáz nyomása mindig megegyezik a dugattyúra ható külső nyomással, így a rendszer egyensúlyi állapotokon halad keresztül. A rendszer belső energiája nem változik, 
Ezért a teljes munka a gázon hővé alakul ami a tartályba távozik. Ha pedig lassan kitolom a dugattyút, akkor a gáz hőt vesz fel a tartályból, és a gáz végzi el a munkát a dugattyún. Így vissza tudok jutni az eredeti állapotba ugyanazon az úton.
Fontos megjegyezni, hogy izotermán mozgunk, tehát a hőmérséklet nem változik, de hőcsere történik a gáz és a tartály között.

Most nézzük meg, hogy mi van ha "kicsit gyorsan" nyomom össze a dugattyút: Ha gyorsan nyomom a dugattyút, akkor egy adiabatán menne végig a folyamat, de ha a "kettő között" nyomom valahol, akkor ott fog a folyamat 
végbe menni:
\begin{figure}[H]
    \centering
    \includegraphics[width=0.5\textwidth]{imgs/5-tetel/irreverzibilis_folyamat_dugattyu.png}
    \caption{Irreverzibilis folyamat egy dugattyús hengerben}
\end{figure}
Ekkor látható, hogy valamennyi munkát be kellett fektetni a gázba, hogy összenyomjam, de a gáz nem tudta az összes munkát hővé alakítani, hiszen a folyamat nem izoterm volt. Így a gáz belső energiája megnőtt, és a gáz hőt adott le a tartálynak.
Ezt a munkát már nem tudom visszanyerni, hiszen a gáz magasabb belső energiájú állapotban van, így ha ki akarom tolni a dugattyút, akkor csak annyi munkát tudok kivenni a gázból, amennyi a belső energia növekedésének megfelelő. Tehát a folyamat nem reverzibilis, hiszen nem tudok visszajutni az eredeti állapotba ugyanazon az úton.
Ebből levonhatjuk a következtetést, hogy nem az a folyamat irreverzibilis ahol hőáramlás történik, hanem az, ahol hőcsere két különböző hőmérsékletű test között történik.
Ez igaz a többi állapotra is, tehát egy folyamat akkor is irreverzibilis, ha két különböző nyomású gáz között történik térfogatcsere.

\subsubsection*{Termodinamika II. főtétele}
Eddig olyan folyamatokat vizsgáltunk, ahol a rendszer egyik állapotból a másikba jutott. De mi van akkor, ha egy rendszer vissza tud jutni a kezdeti állapotába?
Ezt a fajta folyamatot körfolyamatnak nevezzük és az a speciális tulajdonságuk, hogy a kezdeti és végállapot megegyezik. De kérdés lehet, hogy ezek a folyamatok reverzibilisek vagy irreverzibilisek?
A válasz az, hogy ez a folyamattól függ, hiszen bár a körfolyamat során végzünk munkát a rendszeren (amit a körfolyamat által kijelölt terület jelez), de a reverzibilitás nem csak ettől függ, hanem a környezettől is.
uUgyanis a reverzibilitás feltétele, hogy az állapotcsere (legyen az hő vagy térfogatcsere) ne történjen meg különböző hőmérsékletű vagy nyomású testek között. Tehát egy körfolyamat lehet reverzibilis, ha minden állapotcsere egyensúlyi állapotokon keresztül történik, és a környezet is ugyanazon állapotban van.
Ez alapján kimondhatjuk a \textbf{termodinamika II. főtételét}, amit Clausius tételnek is nevezünk:
\begin{quote}
    "Hő nem mehet alacsonyabb hőmérsékletű helyről magasabb hőmérsékletű helyre anélkül, hogy közben a környezetben valamilyen változás vissza ne maradjon."
\end{quote}
Ezt a tételt Kelvin és Planck is finomította a következőképpen:
\begin{quote}
    "Nem konstruálható olyan periodikusan működő gép, mely csupán egyetlen hőtartállyal áll kapcsolatban és munkát végez."
\end{quote}
Ez azt jelenti, hogy egy körfolyamat során nem lehet csak hőt felvenni egy hőtartályból és azt teljes egészében munkává alakítani, hiszen a környezetben valamilyen változásnak kell történnie.
A harmadik és legegyszerűbb megfogalmazása a termodinamika II. főtételének a következő:
\begin{quote}
    "Nem konstruálható másodfajú örökmozgó"
\end{quote}
\begin{figure}[H]
    \centering
    \includegraphics[width=0.5\textwidth]{imgs/5-tetel/II_fotetel.png}
    \caption{Lehetetlen gépek a termodinamika II. főtételének értelmében}
\end{figure}
Felmerülhet az a kérdés, hogy hogyan lehetséges az, hogy vannak irreverzibilis folyamatok a termodinamikában, amikor a fizika alapvető törvényei mind szimmetrikusak az idővel szemben?
A választ a termodinamika statisztikus jellegében kell keresni, ugyanis ez az irreverzibilitás nem egy törvény, hanem egy következménye annak, hogy a makroszkopikus rendszerekben rengeteg részecske van.
Ha elképzelünk például egy tartályt, aminek az egyik felén gáz van, a másik felén pedig vákuum, akkor a gáz egy idő után ki fogja tölteni az egész tartályt és az az állapot nem fog spontán beállni, hogy a gázrészecskék megint csak a tartály egyik felébe mennek.
Ennek az az oka, hogy az az állapot amikor a gáz csak az egyik oldalon van jelen, egy specifikus állapot a több milliárd, azonosan valószínű állapot közül. Azokból az állapotokból amikor a gázrészecskék egyenletesen eloszlanak a tartályban,
rengeteg van, míg az az állapot amikor a gáz csak az egyik felében van, csak egyetlen egy van. Így a statisztikus valószínűsége annak, hogy a gáz visszakerüljön az eredeti állapotba, gyakorlatilag nulla.

\subsubsection*{Carnot-körfolyamat}
A Carnot-körfolyamatot először Nicolas Léonard Sadi Carnot írta le 1824-ben, és ez a körfolyamat egy ideális hőerőgép modellje, amely két hőtartály között működik. A Carnot-körfolyamat lényege, hogy
hogy két izoterma között mozgatjuk a rendszert, úgy, hogy köztük adiabatákon kereszül mozgunk. A folyamat végére pedig ugyanabba az állapotba jutunk vissza, ahonnan elindultunk és
a folyamat során képes munkát végezni a környezetén, vagy ha megfordítjuk a folyamatot, akkor hőt vesz fel a környezetből és hőszivattyúként működik. A Carnot-körfolyamat reverzibilis és kvázistatikus folyamat.
\begin{figure}[H]
    \centering
    \includegraphics[width=0.5\textwidth]{imgs/5-tetel/carnot_korfolyamat.png}
    \caption{Carnot-körfolyamat $p-V$ diagramon ábrázolva}
\end{figure}
A Carnot-körfolyamat négy lépésből áll:
\begin{table}[H]
    \centering
    \begin{tabular}{|c|c|c|c|}
        \hline
        Folyamat & Munka & Belső energia változás & Hő \\
        \hline
        $1 \to 2$ Izoterm tágulás & $- nRT_H \ln{\frac{V_2}{V_1}}$ & $0$ & $nRT_1 \ln{\frac{V_2}{V_1}} = Q_1$ \\
        \hline
        $2 \to 3$ Adiabatikus tágulás & $\Delta U_{23}$ & $- n C_V (T_1 - T_2) =\Delta U_{23}$ & $0$ \\
        \hline
        $3 \to 4$ Izoterm összenyomás & $n R T_C \ln{\frac{V_3}{V_4}}$ & $0$ & $- n R T_C \ln{\frac{V_3}{V_4}} = Q_2$ \\
        \hline
        $4 \to 1$ Adiabatikus összenyomás & $\Delta U_{41}$ & $n C_V (T_1 - T_2) = \Delta U_{41}$ & $0$ \\
        \hline
    \end{tabular}
    \caption{A Carnot-körfolyamat lépései}
\end{table}
A Carnot körfolyamatot egy schematikus Carnot-géppel is ábrázolhatjuk, ahol a hőerőgép két hőtartály között működik:
\begin{figure}[H]
    \centering
    \includegraphics[width=0.5\textwidth]{imgs/5-tetel/carnot_gep.png}
    \caption{Carnot-gép schematikus ábrázolása}
\end{figure}

Ezek alapján felírhatjuk a Carnot-körfolyamat belső energiájának változását:
\begin{equation}
    \Delta U = W + Q = 0
\end{equation}
Mivel a körfolyamat visszatér a kiindulási állapotba, így a belső energia változása nulla. Ebből következik, hogy a körfolyamat során végzett munka egyenlő a felvett hővel:
\begin{equation}
    W = - Q
\end{equation}
A munkát viszont fel lehet írni az egyes lépések munkáinak összegzésével:
\begin{equation}
    W = W_{12} + W_{23} + W_{34} + W_{41}
\end{equation}
Ahol az egyes lépések munkái a következők:
\begin{equation}
    \begin{aligned}
        W_{12} = - n R T_H \ln{\frac{V_2}{V_1}} \\
        W_{23} = \Delta U_{23} = - n C_V (T_1 - T_2) \\
        W_{34} = n R T_C \ln{\frac{V_3}{V_4}} \\
        W_{41} = \Delta U_{41} = n C_V (T_1 - T_2)
    \end{aligned}
\end{equation}
Ezeket összeadva a munkát a következőképpen kapjuk meg:
\begin{equation}
    W = - n R T_H \ln{\frac{V_2}{V_1}} + n R T_C \ln{\frac{V_3}{V_4}} = - Q = - (Q_1 + Q_2)
\end{equation}
Látható, hogy az adiabatikus lépések munkái kiesnek egymással, hiszen azok a belső energia változását adják, ami a körfolyamat során nulla.
Számoljuk ki a Carnot-körfolyamat hatásfokát, ami megmutatja, hogy a felvett hő mekkora részét tudja munkává alakítani a gép. A hatásfokot úgy definiáljuk, hogy a mennyi munkát végez a gép ($-W$) a felvett hő arányában ($Q_1$):
\begin{equation}
    \eta = \frac{- W}{Q_1} = \frac{n R T_H \ln{\frac{V_2}{V_1}} - n R T_C \ln{\frac{V_3}{V_4}}}{n R T_H \ln{\frac{V_2}{V_1}}}
\end{equation}
Értelemszerűenn ha $\eta = 1$, akkor a gép az összes felvett hőt munkává tudja alakítani, tehát veszteség nélkül működik.
Most tegyük fel, hogy ideális gázról van szó, vagyis fel lehet írni:
\begin{equation}
    p V^{\kappa} = \text{állandó} \qquad pV = n R T
\end{equation}
Vagy átrendezve:
\begin{equation}
    \frac{nRT}{V} V^{\kappa} = \text{állandó}
\end{equation}
Ebből következik, hogy:
\begin{equation}
    T V^{\kappa - 1} = \text{állandó}
\end{equation}
Ez igaz a Carnot-körfolyamat egyes lépéseire is, így felírhatjuk az adiabatikus lépésekre:
\begin{equation}
    T_1 V_2^{\kappa - 1} = T_2 V_3^{\kappa - 1} \qquad T_1 V_1^{\kappa - 1} = T_2 V_4^{\kappa - 1}
\end{equation}
Ebből következik, hogy:
\begin{equation}
    \frac{V_2^{\kappa - 1}}{V_1^{\kappa - 1}} = \frac{V_3^{\kappa - 1}}{V_4^{\kappa - 1}} \qquad \Rightarrow \qquad \frac{V_2}{V_1} = \frac{V_3}{V_4}
\end{equation}
Ezt visszahelyettesítve a hatásfok kifejezésébe:
\begin{equation}
    \eta = \frac{T_1 - T_2}{T_1} = 1 - \frac{T_2}{T_1}
\end{equation}
Ez pedig a Carnot-körfolyamat hatásfoka ideális gázokra. Látható, hogy a hatásfok csak a két hőtartály hőmérsékletétől függ, és minél nagyobb a hőmérsékletkülönbség, annál nagyobb a hatásfok.
Ha visszahelyettesítjük a hőt a hatásfok kifejezésébe:
\begin{equation}
    \frac{Q_1 + Q_2}{Q_1} = 1 - \frac{T_2}{T_1} \qquad \Rightarrow \qquad \boxed{\frac{Q_1}{T_1} + \frac{Q_2}{T_2} = 0}
\end{equation}
Ez pedig a Carnot-körfolyamat hőmérlege ideális gázokra.
A reverzibilitás következménye, hogy mindkét irányba le tud folyni a folyamat, vagyis ha munkát fektetek be a gépbe, akkor hőt tudok kivenni a hidegebb tartályból és át tudom adni a melegebb tartálynak.
Ez a folyamat a hőszivattyú, és a hatásfoka a következőképpen írható fel:
\begin{equation}
    \eta_{\text{szivattyú}} = \frac{|Q_1|}{W} = \frac{|Q_1|}{Q_1 + Q_2} = \frac{T_1}{T_1 - T_2}
\end{equation}
Fontos még megjegyezni, hogy a termodinamika II. főtételéből következik, hogy a hatásfok mindig kisebb mint 1, vagyis egy hőerőgép sosem tudja az összes felvett hőt munkává alakítani, hiszen mindig van valamilyen veszteség a folyamat során.
Ebből következik, hogy a hőszivattyú hatásfoka mindig nagyobb mint 1, hiszen a felvett hő mindig nagyobb mint a befektetett munka:
\begin{equation}
    \eta_{\text{szivattyú}} = \frac{|Q_1|}{W} > 1
\end{equation}
Fel lehet még írni a hűtőgépek hatásfokát is, ami megmutatja, hogy a hidegebb tartályból mennyi hőt tudunk kivenni a befektetett munka arányában:
\begin{equation}
    \eta_{\text{hűtőgép}} = \frac{|Q_2|}{W} = \frac{|Q_2|}{Q_1 + Q_2} = \frac{T_2}{T_1 - T_2} = \eta_{\text{szivattyú}} - 1
\end{equation}
A Carnot-körfolyamat ezáltal egy ideális hőerőgép, mivel a hatásfoka csakis a két hőtartály hőmérsékletétől függ, és nem függ a gép anyagától vagy kialakításától, emiatt egy körfolyamat hatásfoka sose lesz nagyobb mint a Carnot-körfolyamaté.
A Carnot-körfolyamat lehet irreverzibilis is, ha gyorsan hajtjuk végre a folyamatot, vagy ha a hőcsere nem egyensúlyi állapotokon keresztül történik. Ilyenkor a hatásfok kisebb lesz mint a reverzibilis esetben, és a hőmérleg sem lesz pontosan azonos a reverzibilis esettel.

De mi van akkor, ha a körfolyamat nem ideális gázokra történik? Meglepő módon ugyan az marad a hatásfok, ugyanis a hatásfok csak a két hőtartály hőmérsékletétől függ, és nem függ a rendszer anyagától vagy kialakításától.
Ezt egy gondolatkísérlettel is be lehet látni: Tegyük fel, hogy van egy nem ideális gázból készült hőerőgépünk, ami nagyobb hatásfokkal működik mint a Carnot-körfolyamat. Ekkor ezt a gépet összekapcsolhatjuk egy reverzibilis Carnot-géppel,
ami ugyanolyan hőtartályok között működik. Ekkor a két gép összességében egy olyan gépet alkot, ami csak egy hőtartállyal van kapcsolatban, és munkát végez, ami ellentmond a termodinamika II. főtételének.
Ezért a nem ideális gázokra is ugyan az a hatásfok érvényes mint az ideális gázokra. Ha pedig egy olyan gázt néznénk aminek kisebb a hatásfoka mint az ideális gázoké, akkor a folyamat visszafelé sértené a termodinamika II. főtételét, hiszen a körfolyamat során kevesebb hőt adnánk le a hidegebb tartálynak mint amennyit felvettünk a melegebb tartályból.
\begin{figure}[H]
    \centering
    \includegraphics[width=0.5\textwidth]{imgs/5-tetel/carnot_id_gaz_hatasfok.png}
    \caption{A Carnot-körfolyamat hatásfoka ideális és nem ideális gázokra}
\end{figure}

Most nézzük meg, hogy mi van akkor, ha több hőtartályt használunk a Carnot-körfolyamat során. Tegyük fel, hogy van 3 darab hőtartályunk, amik különböző hőmérsékletűek: $T_1 > T_2 > T_3$. Ekkor a körfolyamat során először a legmelegebb tartályból
veszünk fel hőt, majd a középső tartálynak adunk le hőt, végül pedig a leghidegebb tartálynak adunk le hőt.
\begin{figure}[H]
    \centering
    \includegraphics[width=0.5\textwidth]{imgs/5-tetel/carnot_tobb_hotartaly.png}
    \caption{Carnot-körfolyamat $pV$ grafikonja több hőtartállyal}
\end{figure}
Ekkor a Carnot-gép schematikus ábrázolása a következő lesz:
\begin{figure}[H]
    \centering
    \includegraphics[width=0.5\textwidth]{imgs/5-tetel/carnot_tobb_hotartaly_gep.png}
    \caption{Carnot-gép schematikus ábrázolása több hőtartállyal}
\end{figure}
Ekkor a hatásfokot a következőképpen számolhatjuk ki:
\begin{equation}
    \eta = \frac{- W}{Q_1 + \Delta Q} = \frac{-W_1 - W_2}{Q_1 + \Delta Q} = \frac{Q_1 + \Delta Q - \frac{T_2}{T_3}(\frac{T_2}{T_1}Q_1 + \Delta Q)}{Q_1 + \Delta Q}
\end{equation}
Ahol $\Delta Q$ a középső tartálynak átadott hőmennyiség. Vezessünk be egy segédváltozót:
\begin{equation}
    x = \frac{\Delta Q}{Q_1}
\end{equation}
Ekkor a hatásfok kifejezése a következő lesz:
\begin{equation}
    \eta = \frac{1 + x - \frac{T_2}{T_1} - \frac{T_2}{T_3}x}{1 + x} = \frac{\eta_{12}^c + \eta_{32}x}{1 + x} = \eta_{12}^c \frac{1 + \frac{\eta_{32}}{\eta_{12}^c} x}{1 + x}
\end{equation}
Ahol $\eta_{12}^c$ a Carnot-körfolyamat hatásfoka a $T_1$ és $T_2$ hőtartályok között, míg $\eta_{32}$ a $T_3$ és $T_2$ hőtartályok között. Látható, hogy mivel $\eta_{32} < \eta_{12}^c$, így a hatásfok mindig kisebb lesz mint a két legextrémebb hőtartály közötti hatásfok:
\begin{equation}
    \eta < \eta_{12}^c
\end{equation}
Ez azt jelenti, hogy ha több hőtartályt használunk a Carnot-körfolyamat során, akkor a hatásfok mindig kisebb lesz mint ha csak a két hőtartályt használnánk.

\subsubsection*{Példa: Stirling Motor hatásfoka}
\begin{figure}[H]
    \centering
    \includegraphics[width=0.5\textwidth]{imgs/5-tetel/stirling_motor_schematikus.png}
    \caption{Stirling motor schematikus ábrázolása}
\end{figure}
A Stirling motor egy olyan hőerőgép, amely külső égésű motor, és a Carnot-körfolyamat egy speciális esete. A Stirling motor négy lépésből áll:
\begin{figure}[H]
    \centering
    \includegraphics[width=0.5\textwidth]{imgs/5-tetel/stirling_motor.png}
    \caption{Stirling motor $pV$ diagramon ábrázolva}
\end{figure}
A Stirling motor négy lépése a következő:
\begin{itemize}
    \item Izoterm tágulás: A gáz felveszi a hőt a meleg tartályból, miközben a térfogata növekszik.
    \item Izokor hűtés: A gáz térfogata állandó marad, miközben a hőmérséklete csökken.
    \item Izoterm összenyomás: A gáz leadja a hőt a hideg tartálynak, miközben a térfogata csökken.
    \item Izokor fűtés: A gáz térfogata állandó marad, miközben a hőmérséklete növekszik.
\end{itemize}
A Stirling motor hatásfokát a következőképpen számolhatjuk ki:
\begin{equation}
    \eta_{\text{Stirling}} = \frac{-W_{12} - W_{14}}{Q_{12} + Q_{41}} = \frac{n R (T_H - T_C) \ln{\frac{V_2}{V_1}}}{n R T_H \ln{\frac{V_2}{V_1}} + n C_V (T_H - T_C)}
\end{equation}
Ahol $W_{12}$ az izoterm tágulás munkája, $W_{14}$ az izokor fűtés munkája, $Q_{12}$ a meleg tartályból felvett hő, és $Q_{41}$ a hideg tartálynak leadott hő.
Egyszerűsítve a hatásfok kifejezést:
\begin{equation}
    \eta_{\text{Stirling}} = \frac{(T_H - T_C) \ln{\frac{V_2}{V_1}}}{T_H \ln{\frac{V_2}{V_1}} + \frac{C_V}{R} (T_H - T_C)}
\end{equation}
Ha ebbe belehelyettesítjük a gyakran használt értékeket:
\begin{equation}
    \begin{aligned}
        T_H &= 600 \, K \\
        T_C &= 300 \, K \\
        \frac{V_2}{V_1} &= 2 \\
        C_V &= \frac{3}{2} R \quad \text{(monatomikus gáz esetén)}
    \end{aligned}
\end{equation}
Akkor a Stirling motor hatásfoka a következő lesz:
\begin{equation}
    \eta_{\text{Stirling}} = \frac{(600 - 300) \ln{2}}{600 \ln{2} + \frac{3}{2} (600 - 300)} \approx 0.24
\end{equation}
Ez azt jelenti, hogy a Stirling motor a felvett hő 24\%-át tudja munkává alakítani ezekkel a paraméterekkel.

\subsubsection{Entrópia, Főtételek egyesített alakja}
Láttuk, hogy a Carnot-körfolyamat során a hőmérleg a következőképpen írható fel ideális gázokra:
\begin{equation}
    \frac{Q_1}{T_1} + \frac{Q_2}{T_2} = 0
\end{equation}
Ezt a kifejezést fel lehet írni egy körintegrál segítségével is, ahol a körintegrál a körfolyamat során felvett és leadott hőmennyiségeket jelenti:
\begin{equation}
    \oint \frac{\delta Q}{T} = 0
\end{equation}
De mi van, ha egy komplexebb körfoyamatot nézek? Nézzük meg hogy néz ki, ha három hőtartály között megy végbe a körfolyamat:
\begin{figure}[H]
    \centering
    \includegraphics[width=0.5\textwidth]{imgs/5-tetel/ket_lepes_korfolyamat.png}
    \caption{Két lépéses körfolyamat $pV$ diagramon ábrázolva}
\end{figure}
Látható, hogy mivel a körintegál 0 azon a szakaszon ahol a folyamat mintkét irányba megy, így a teljes folyamat olyan mint két különálló Carnot-körfolyamat összege, tehát a teljes körintegrál továbbra is 0 lesz:
\begin{equation}
    \oint \frac{\delta Q}{T} = \oint_{1} \frac{\delta Q}{T} + \oint_{2} \frac{\delta Q}{T} = 0 + 0 = 0
\end{equation}
Most nézzük meg egy teljesen tetszőleges körfolyamatot:
\begin{figure}[H]
    \centering
    \includegraphics[width=0.5\textwidth]{imgs/5-tetel/tetszoleges_korfolyamat.png}
    \caption{Tetszőleges körfolyamat $pV$ diagramon ábrázolva}
\end{figure}
Ekkor a körfolyamatot fel lehet bontani több kisebb Carnot-körfolyamatra, amik mindegyike egyensúlyi állapotokon keresztül megy végbe. Mivel mindegyik kis körfolyamatra igaz, hogy a körintegrálja 0, így a teljes körfolyamat körintegrálja is 0 lesz:
\begin{equation}
    \oint \frac{dQ}{T} = \sum_{i} \oint_{i} \frac{\delta Q}{T} = 0
\end{equation}
Ebből következik, hogy bármilyen körfolyamatra igaz, hogy:
\begin{equation}
    \oint \frac{dQ}{T} = 0
\end{equation}
Az állapotjelzőt pedig pont úgy definiáltuk, hogy egy olyan mennyiség ami csak a folyamat végpontjaitól függ, vagyis ha ugyan abba az állapotba jutunk vissza, akkor az állapotjelző változása nulla lesz.
Ebből következik, hogy ez a mennyiség amit itt kaptunk is egy állapotjelző lesz:
\begin{equation}
    S(A) := \int_{0}^{A} \frac{\delta Q}{T}
\end{equation}
Ennek az állapotjelzőnem a megváltozása:
\begin{equation}
    \Delta S = S(B) - S(A) = \int_{A}^{B} \frac{\delta Q}{T} = \int_{A}^{B} dS
\end{equation}
Tehát a hő változását fel lehet írni az állapotjelző változásaként is:
\begin{equation}
   dS = \frac{\delta Q}{T} \qquad \delta Q = T dS
\end{equation}
Ezt az állapotjelzőt entrópiának nevezzük, és valamilyen módon a hőmérséklet áramlásával kapcsolatos rendszertulajdonságot fejez ki.
\newline
\newline
Fontos, hogy ez eddig csak kvázistatikus és reverzibilis folyamatokra volt igaz, de mi van akkor, ha irreverzibilis folyamatot nézünk?
Nézzük meg egy egyszerű példát: Két különböző hőmérsékletű gázt összeengedünk egy tartályban:
\begin{figure}[H]
    \centering
    \includegraphics[width=0.5\textwidth]{imgs/5-tetel/irreverzibilis_entrópia.png}
    \caption{Irreverzibilis folyamat két gáz között}
\end{figure}
Ekkor az entrópia változása a következő lesz:
\begin{equation}
    \delta Q = m c dT
\end{equation}
\begin{equation}
    \Delta S = \Delta S_1 + \Delta S_2 = \int_{T_1}^{T^*} \frac{m c dT}{T} + \int_{T_2}^{T^*} \frac{m c dT}{T} = m c \ln{\frac{T^*}{T_1}} + m c \ln{\frac{T^*}{T_2}}
\end{equation}
\begin{equation} 
    \Delta S = m c \ln{\frac{T^{*2}}{T_1 T_2}} = m c \ln{\frac{(T_1 + T_2)^{2}}{4T_1 T_2}} > 0
\end{equation}
Tehát a hőmérséklet kiegyenlítődésével zárt rendszerben az Entrópia nő. Ebből az is látszik, hogy az entrópia egy extenzív állapotjelző, mert függ a rendszer méretétől.
Fontos azt is megjegyezni, hogy az entrópia csak az egész rendszert nézve nő, hiszen ha csak az egyik gázra nézzük, akkor annak az entrópiája csökkenhet is a hőmérséklet csökkenésével.
\newline
\newline
Meg lehet nézni az entrópiaváltozást a Gay-Lussac folyamat során is, ahol egy gázt összenyomunk, majd kiengedünk.
Ekkor a gáz entrópiaváltozása az összenyomás során:
\begin{equation}
    \Delta S_{21} = \int_{1}^{2} \frac{\delta Q}{T} = \int_{2V_0}^{V_0} \frac{pdV}{T} = \int_{2V_0}^{V_0} \frac{nR dV}{V} = n R \ln{\frac{V_0}{2V_0}} = - n R \ln{2} < 0
\end{equation}
Tehát a gáz entrópiája csökken, amikor összenyomom a gázt. Viszont amikor kitágul a gáz, akkor pont ugyan abba az állapotba kell visszajutnom mint az összenyomás előtt, így az entrópia változása a gáznak nulla lesz.
\begin{equation}
    \Delta S_{12} = - \Delta S_{21} = n R \ln{2} > 0
\end{equation}
Így a teljes entrópiaváltozás a gáz és a tartály között:
\begin{equation}
    \Delta S_{\text{teljes}} = \Delta S_{\text{gáz}} + \Delta S_{\text{tartály}} = - n R \ln{2} + n R \ln{2} = 0
\end{equation}
Tehát a teljes entrópiaváltozás nulla. Az entrópia a Boltzmann-féle megfogalmazásban a következőképpen néz ki:
\begin{equation}
    \Delta S_{12} = N k_B ln2 = k_B ln2^{N} 
\end{equation}
Ahol $N$ a részecskék száma, és $k_B$ a Boltzmann állandó ($\left[k_B\right] = \frac{J}{K}$). Ez azt jelenti, hogy amikor a gáz kitágul, akkor a részecskék eloszlása a tartályban megduplázódik, így az entrópia növekszik.
\newline
\newline
Ezek alapján felírhatjuk a termodinamika II. főtételének egyesített alakját:
\begin{equation}
     dS \geq \frac{\delta Q}{T} \qquad dS T \geq \delta Q
\end{equation}
\begin{equation}
    \boxed{\Delta S \geq \int \frac{\delta Q}{T}}
\end{equation}
Ahol az egyenlőség csak kvázistatikus és reverzibilis folyamatokra igaz, míg az egyenlőtlenség irreverzibilis folyamatokra érvényes.
Ez azt jelenti, hogy egy zárt rendszer entrópiája sosem csökken, vagyis az entrópia mindig nő vagy állandó marad. Ebből az következik,
hogy az univerzum entrópiája folyamatosan növekszik, hiszen az univerzum egy zárt rendszer. Ez az entrópianövekedés irányát is meghatározza, vagyis megmondja, hogy egy folyamat milyen irányban mehet végbe spontán módon.
Ebből a felismerésből született meg a hőhalál elmélete, ami azt mondja ki, hogy az univerzum egyre nagyobb entrópiájú állapotok felé halad, és végül eléri a maximális entrópiájú állapotot, ahol már nem lesz többé energiaátadás vagy munka végzésére alkalmas folyamat.

Az entrópia értelmezésénél meg kell jegyezni, hogy egy lokális rendszer entrópiája csökkenhet is, de csak úgy, hogy közben a környezete entrópiája nagyobb mértékben növekszik. Például
a földön végbe mehetnek olyan folyamatok, ahol az entrópia csökken, mint a szél kialakulása, amit eltérő légnyomású zónák kialakulása okoz, de ezt a folyamatot a nap energiája hajtja, ami növeli a föld környezetének entrópiáját.

\subsubsection*{Entrópia statisztikus értelmezése}
Az entrópiát nem csak termodinamikai értelemben lehet értelmezni, hanem statisztikus mechanikai értelemben is. Ezt először Ludwig Boltzmann fogalmazta meg a következő képlettel:
\begin{equation}
    S = k_B \ln{\Omega}
\end{equation}
Ahol $S$ az entrópia, $k_B$ a Boltzmann állandó, és $\Omega$ az adott makroszkopikus állapothoz tartozó mikroszkopikus állapotok száma.
Ez azt jelenti, hogy minél több mikroszkopikus állapot felel meg egy adott makroszkopikus állapotnak, annál nagyobb az entrópia értéke.
Például egy gáz esetén, ha a gáz részecskéi egyenletesen eloszlanak a tartályban, akkor rengeteg mikroszkopikus állapot felel meg ennek a makroszkopikus állapotnak, míg ha a gáz csak az egyik felében van jelen, akkor csak egyetlen mikroszkopikus állapot felel meg ennek a makroszkopikus állapotnak.
Ezért az egyenletes eloszlású állapotnak nagyobb az entrópiája, mint a gáz csak az egyik felében lévő állapotnak.
\newline
\newline
Ez a statisztikus értelmezés összhangban van a termodinamikai entrópia fogalmával, hiszen a termodinamikai entrópia is azt fejezi ki, hogy egy rendszer mennyire rendezetlen vagy mennyire valószínű egy adott állapot.
Minél több mikroszkopikus állapot felel meg egy makroszkopikus állapotnak, annál valószínűbb az adott állapot, és annál nagyobb az entrópia értéke.
\newline
\newline
Ez a statisztikus értelmezés segít megérteni, hogy miért növekszik az entrópia egy zárt rendszerben. Mivel a rendszerek hajlamosak olyan állapotok felé haladni, amelyek több mikroszkopikus állapotnak felelnek meg,
ezért az entrópia növekedése természetes következmény. Például egy gáz esetén, ha a gáz részecskéi kezdetben egy kis térfogatban vannak összegyűjtve, akkor idővel a részecskék szétterjednek a rendelkezésre álló térfogatban,
mivel ez az állapot több mikroszkopikus állapotnak felel meg, és így az entrópia növekszik.
\newline
\newline
Ebből az is következik, hogy az entrópia spontán csökkenése nem teoretikus lehetetlenség egy adott időpillanatban, de elég hosszú időskálán nézve a valószínűsége effektíve nulla.

Az entrópia statisztikus értelmezését szemléltetve nézzünk meg egy egyszerű példát: Képzeljünk el egy dobozt, amiben pár részecske van, és a dobozt két részre osztjuk egy elválasztó fallal. Kezdetben az összes részecske az egyik oldalon van,
majd az elválasztó falat eltávolítjuk, és a részecskék szabadon mozoghatnak a doboz mindkét felében. Ekkor az entrópia növekedését a következőképpen értelmezhetjük statisztikus mechanikai szempontból:
\begin{figure}[H]
    \centering
    \includegraphics[width=0.5\textwidth]{imgs/5-tetel/entropia_statisztikus.png}
    \caption{Az entrópia statisztikus értelmezése}
\end{figure}
Kezdetben, amikor az összes részecske az egyik oldalon van, csak egyetlen mikroszkopikus állapot felel meg ennek a makroszkopikus állapotnak, hiszen minden részecske ugyanazon az oldalon van. Ezért az entrópia értéke alacsony.
A várható értéke a bal oldalon lévő részecskék számának:
\begin{equation}
    \langle N_b \rangle = N \cdot \frac{1}{2} = \frac{N}{2}
\end{equation}
Ahol $N$ a részecskék teljes száma. Ez alapján azt is fel lehet írni, hogy:
\begin{equation}
    N_b = \frac{N}{2} \pm \sqrt{N}
\end{equation}
Ez azt jelenti, hogy a részecskék száma a bal oldalon az idő előrehaladtával ingadozni fog a várható érték körül, de a teljes részecskeszám feléhez fog közelíteni.
Ha mondjuk feltesszük, hogy $N = 10^{24} $ részecske van a dobozban, akkor a várható érték $5 \times 10^{23}$ lesz, és az ingadozás mértéke körülbelül $10^{12}$ lesz:
\begin{equation}
    N_b = 5 \times 10^{23} \pm 10^{12}
\end{equation}
\begin{equation}
    \frac{\Delta p}{p} = \frac{\Delta N}{N} = \frac{10^{12}}{5 \times 10^{23}} = 2 \times 10^{-12}
\end{equation}
Az errntrópiára visszavetítve kimondhatjuk, hogy az entrópia a mikroállapotok számától függ:
\begin{equation}
    S = S(\Omega)
\end{equation}
Ahol $\Omega$ a mikroszkopikus állapotok száma. Erre Boltzmann a következő képletet adta meg:
\begin{equation}
    S = k_B \ln{\Omega} \qquad \left[S\right] = \frac{J}{K}
\end{equation}
Ahol $k_B$ a Boltzmann állandó. Ez alapján az entrópia változása:
\begin{equation}
    \Delta S = k_B \ln{\frac{\Omega_2}{\Omega_1}}
\end{equation}
Vagyis a mi példánkra:
\begin{equation}
    \Delta S = k_B \ln{\frac{2^{N}\Omega_0}{\Omega_0}} = k_B \ln{2^{N} \Omega_0} - k_B \ln{\Omega_0} = N k_B \ln{2}
\end{equation}
Ami pontosan az mint amit a Carnot-körfolyamatnál is kaptunk. Ezt úgy is értelmezhetjük, hogy az az állapot ami több mikroszkopikus állapotnak felel meg, az entrópiában is nagyobb értéket képvisel.
A legtöbb mikroszkopikus állapottal rendelkező állapot pedig a legrendezetlenebb, vagyis az entrópiát lehet a rendezetlenség mérőszámaként is értelmezni.
\newline
\newline
Végezetül pedig nézzük meg a Carnot-körfolyamat TS diagramon való ábrázolását:
\begin{figure}[H]
    \centering
    \includegraphics[width=0.5\textwidth]{imgs/5-tetel/carnot_ts_diagram.jpg}
    \caption{Carnot-körfolyamat TS diagramon ábrázolva}
\end{figure}
Ezen a grafikonon a körfolyamat által kijelölt terület a felvett hőmennyiséget jelenti:
\begin{equation}
    -W = Q = \oint T dS
\end{equation}
A hatásfok pedig a görbe által közrefogott terület osztva a görbe alatti területtel:
\begin{equation}
    \eta = \frac{-W}{Q_1} = \frac{\oint T dS}{\int_{S_1}^{S_2} T_1 dS} = \frac{T_1 (S_2 - S_1) - T_2 (S_2 - S_1)}{T_1 (S_2 - S_1)} = \frac{T_1 - T_2}{T_1}
\end{equation}

\subsubsection*{Összefoglalás}
Mindent összegezve a következő fontos eredményeket kaptuk a Carnot-körfolyamatról és az entrópiáról:
\begin{itemize}
    \item A Carnot-körfolyamat hatásfoka csak a két hőtartály hőmérsékletétől függ, és nem függ a rendszer anyagától vagy kialakításától:
    \begin{equation}
        \eta = 1 - \frac{T_2}{T_1}
    \end{equation}
    \item A Carnot-körfolyamat hőmérlege ideális gázokra:
    \begin{equation}
        \frac{Q_1}{T_1} + \frac{Q_2}{T_2} = 0
    \end{equation}
    \item Az entrópia egy állapotjelző, ami a hőmérséklet áramlásával kapcsolatos rendszertulajdonságot fejez ki:
    \begin{equation}
        dS = \frac{\delta Q}{T}
    \end{equation}
    \item A termodinamika II. főtétele egyesített alakja:
    \begin{equation}
        \Delta S \geq \int \frac{\delta Q}{T}
    \end{equation}
    \item Az entrópia statisztikus értelmezése Boltzmann szerint:
    \begin{equation}
        S = k_B \ln{\Omega}
    \end{equation}
\end{itemize}

Ezek alapján az állapotegyenletet a következő általános formában írhatjuk fel:
\begin{equation}
    \boxed{dU = T dS - p dV}
\end{equation}

\subsection{Kémiai Potenciál}
A termodinamika két főtétele alapján fel tudtuk írni az állapotegyenletet reverzibilis folyamatokra:
\begin{equation}
    dU = T dS - p dV
\end{equation}
Ez alapján az állapotjelzőkre fel tudjuk írni a következő kifejezéseket:
\begin{equation}
    T = \frac{\partial U}{\partial S}\bigg|_V \qquad p = - \frac{\partial U}{\partial V}\bigg|_S
\end{equation}
De eddig csak olyan rendszereket néztünk, amikben az anyagmennyiség állandó volt. Mi van akkor, ha egy olyan rendszert nézünk, ahol az anyagmennyiség is változhat?
Ekkor az állapotegyenletet ki kell egészíteni egy új taggal, ami az anyagmennyiség változását veszi figyelembe:
\begin{equation}
    U = U(S, V, n)
\end{equation}
\begin{equation}
    dU = \frac{\partial U}{\partial S}\bigg|_{V,n} dS + \frac{\partial U}{\partial V}\bigg|_{S,n} dV + \frac{\partial U}{\partial n}\bigg|_{S,V} dn
\end{equation}
Az új tagot pedig a következőképpen definiáljuk:
\begin{equation}
    \mu := \frac{\partial U}{\partial n}\bigg|_{S,V} - TR \cdot ln{\left(\frac{V \cdot T^{\frac{f}{2}}}{n \psi}\right)} \qquad \text{kémiai potenciál}
\end{equation}
Ez a mennyiség a kémiai potenciál, ezt visszahelyettesítve az állapotegyenletbe:
\begin{equation}
    dU = T dS - p dV + \mu dn \qquad \mu: \text{kémiai potenciál} \qquad \left[\mu\right] = \frac{J}{mol}
\end{equation}
A kémiai potenciál azt fejezi ki, hogy mennyi belső energia változás történik egy rendszerben, amikor az anyagmennyiségét egy molal egységgel megváltoztatjuk,
miközben a rendszer entrópiája és térfogata állandó marad.
\newline
\newline
A kémiai potenciált fel lehet írni olyan rendszerekre is, ahol több komponens van jelen. Ekkor az állapotegyenlet a következőképpen néz ki:
\begin{equation}
    U = U(S, V, n_1, n_2, \dots, n_k)
\end{equation}
\begin{equation}
    dU = T dS - p dV + \sum_{i=1}^{k} \mu_i dn_i \qquad \mu_i := \frac{\partial U}{\partial n_i}\bigg|_{S,V,n_{j \neq i}}
\end{equation}
Ahol $k$ a komponensek száma, és $\mu_i$ az $i$-edik komponens kémiai potenciálja.
\newline
\newline
Kérdés, hogy a kémiai potenciál hogyan illeszkedik az állapotegyenletekbe? Fel tudunk írni állapotegyenleteket bármilyen állapotjelzőkre?
A válasz nem, mert az állapotegyenleteknek mindig teljesíteniük kell a főtételeket. Emiatt vannak kitüntetett állapotjelzők, amik szerepelnek a
belső energia megváltozásának egyenletében. Azt is észrevehetjük, hogy ezek az állapotjelzők párban vannak, úgy, hogy tipikusan az egyik extenzív, míg a másik intenzív állapotjelző.
Ezeket \textbf{Konjugált állapotjelzőknek} nevezzük. Ilyen konjugált állapotjelző párok például:
\begin{itemize}
    \item Entrópia ($S$) - Hőmérséklet ($T$)
    \item Térfogat ($V$) - Nyomás ($p$)
    \item Anyagmennyiség ($n$) - Kémiai potenciál ($\mu$)
\end{itemize}
Ezeket az állapotjelzőket ki tudjuk egymásból fejezni állapotegyenletek segítségével:
\begin{equation}
    U(S, V, n) \quad \to \quad T = \frac{\partial U}{\partial S}\bigg|_{V,n} \qquad p = - \frac{\partial U}{\partial V}\bigg|_{S,n} \qquad \mu = \frac{\partial U}{\partial n}\bigg|_{S,V}
\end{equation}
Nézzük meg például, hogy mit kapunk ha a belső energia a következő alakban van megadva:
\begin{equation}
    U(S, V, n) = \phi \cdot e^{\frac{2S}{fnR}} \cdot V^{-\frac{2}{f}} \cdot n^{\frac{f + 2}{f}} \qquad \phi: \text{állandó}
\end{equation}
Ekkor a különböző állapotjelzők a következőképpen néznek ki:
\begin{equation}
    \begin{aligned}
        T &= \frac{\partial U}{\partial S}\bigg|_{V,n} = \frac{2}{f n R} \cdot U(S, V, n) \quad \Rightarrow \quad U(S, V, n) = \frac{f}{2} n R T \\
        p &= - \frac{\partial U}{\partial V}\bigg|_{S,n} = \frac{2}{f} \cdot U(S, V, n) \cdot V^{-1} \quad \Rightarrow \quad U(S, V, n) = \frac{f}{2} p V \\
        \mu &= \frac{\partial U}{\partial n}\bigg|_{S,V} = \left(\frac{f + 2}{f n} - \frac{2S}{f n^2 R}\right) \cdot U(S, V, n) \quad \Rightarrow \quad U(S, V, n) = \frac{f}{f + 2} \cdot RT = C_P T
    \end{aligned}   
\end{equation}
Tehát ez a "random" függvény visszaadja a jól ismert állapotegyenleteket ideális gázokra. Tehát ez a függvény nem is annyira random, hanem az úgynevezett fundamentális egyenlet.
Tehát a belső enerrgia függvényének csak olyan függvényt választhatok meg, ami visszaadja a jól ismert állapotegyenleteket. Ezeket a függvényeket termodinamikai potenciáloknak nevezzük.
\newline
\newline
Ezeket az egyenleteket más változókkal is fel tudjuk írni, például az entrópiával:
\begin{equation}
    S(U, V, n) \quad \to \quad dS = \underbrace{\frac{\partial S}{\partial U}\bigg|_{V,n}}_{\frac{1}{T}} dU + \underbrace{\frac{\partial S}{\partial V}\bigg|_{U,n}}_{\frac{p}{T}} dV + \underbrace{\frac{\partial S}{\partial n}\bigg|_{U,V}}_{- \frac{\mu}{T}} dn
\end{equation}
Tehát az entrópia megváltozásának egyenlete a következő lesz:
\begin{equation}
    dS = \frac{1}{T} dU + \frac{p}{T} dV - \frac{\mu}{T} dn
\end{equation}
Tehát az Entrópia függvénye:
\begin{equation}
    S(U, V, n) = n R \ln{\left[\left(\frac{U}{p}\right)^{\frac{f}{2}} V \cdot n^{-\frac{f + 2}{2}} \right]} = k_B \ln{\Omega}
\end{equation}
Ezt úgy mondjuk, hogy $U, V, n$ az entrópia természetes változói.
\newline
\newline
Fontos megjegyezni, hogy nem az $U$ maga a termodinamikai potenciál, hanem az $U(S, V, n)$ függvény, ami a belső energiát adja meg az entrópia, térfogat és anyagmennyiség függvényében,
mert más felírásban (pl.: $U(T,V, n)$) nem adja feltétlen vissza az állapotegyenleteket. Például nézzük meg az $U(T, V, n)$ függvényt:
\begin{equation}
    U(T, V, n) = \frac{f}{2} n R T \quad \to \quad dU = \frac{f}{2} n R dT + \frac{f}{2} R T dn
\end{equation}
Ez pedig nem adja vissza az állapotegyenleteket, mert az entrópiát és a nyomást nem tudom származtatni belőle. Tehát ha ezt a függvényt ismerem csak,
akkor nem ismerem az egész rendszert. Kérdés, hogy akkor át tudok-e térni egy másik potenciálra, ami más természetes változókkal van megadva?
Igen, ezt meg tudom tenni a Legendre-transzformáció segítségével. Ahhoz, hogy ezt megértsük nézzünk meg egy tetszőleges $f(X)$ függvényt. Ekkor fel tudjuk írni a függvény differenciálját:
\begin{equation}
    u(X) = f'(X) = \frac{df}{dX} \qquad df = u(X) dX
\end{equation}
És ha az inverz függvényét nézzük meg $u(X)$-nek:
\begin{equation}
    x(U) \quad \rightarrow \quad u(x(U)) = U \quad \rightarrow \quad x = u^{(-1)}
\end{equation}
Kérdés, hogy létezik-e olyan $g(U)$ függvény, amire igaz, hogy:
\begin{equation}
    g'(U) = x(U)
\end{equation}
A válasz, hogy igen:
\begin{equation}
    g(U) = f(x(U)) - U \cdot x(U)
\end{equation}
Mert ha ezt deriváljuk:
\begin{equation}
    g'(U) = f'(x(U)) \cdot x'(U) - \left[ U \cdot x'(U) + x(U) \cdot 1 \right] = u(x(U)) \cdot x'(U) - U \cdot x'(U) - x(U) = - x(U)
\end{equation}
Tehát a Legendre-transzformáció segítségével át tudunk térni különböző függvények között, amik különböző természetes változókkal vannak megadva.
Tehát, ha van egy $f(X)$ függvényünk, és át akarunk térni egy $g(U)$ függvényre, ahol az argumentumok között van egy összefüggés $X = U^{(-1)}$, akkor a következőképpen tudjuk megtenni:
\begin{equation}
    g(U) = f(X) - U \cdot X \qquad \left[g\right] = \left[f\right]
\end{equation}
Tehát ha van egy $U(S,V,n)$ függvényünk, és át akarunk térni egy $F(T,V,n)$ függvényre, akkor a következőképpen tudjuk megtenni:
\begin{equation}
    F(T,V,n) = U(S,V,n) - T \cdot S \qquad \left[F\right] = J
\end{equation}
Ezt a függvényt Helmholtz szabad energiának nevezzük (van, hog $A$-val jelölik).


\subsection{Termodinamikai potenciálok}
Láttuk a kémai potenciálnál, hogy a Legendre-transzformáció segítségével át tudunk térni különböző állapotegyenletek között. Azok az állapotegyenletek pedig,
amelyekből ki tudjuk fejezni az anyag összes többi tulajdonságát (a termodinamika egyesített főtételének segítségével), azok a termodinamikai potenciálok. Tehát:
\begin{itemize}
    \item termodinamika egyesített főtétele:
    \begin{equation}
        dU = T dS - p dV + \mu dn
    \end{equation}
    \item termodinamikai potenciálok (fundamentális egyenletek):
    \begin{itemize}
        \item Belső energia: $U(S, V, n)$
        \item Entrópia: $S(U, V, n)$
        \item Helmholtz szabad energia: $F(T, V, n) = U - T S$
        \item ...
    \end{itemize}
    \item Állapotegyenletek:
    \begin{equation}
        T = \frac{\partial U}{\partial S}\bigg|_{V,n} \qquad p = - \frac{\partial U}{\partial V}\bigg|_{S,n} \qquad \mu = \frac{\partial U}{\partial n}\bigg|_{S,V}
    \end{equation}
\end{itemize}
Ha pedig megvan egy állapotegyenlet, akkor azt átrendezve bármilyen állapotjelzőre fel tudjuk írni a többi állapotjelzőt is. Például ha megvan a $U(T,V,n)$ függvényünk, akkor
A $T$-t ki tudom fejezni $S$-el az állapotegyenlet segítségével:
\begin{equation}
    U(S(T), V n) = U'(T, V, n)
\end{equation}
De $U'$ és $U$ nem ugyan az a két függvény, mert a $U'(T, V, n)$ nem egy termodinamikai potenciál, hiszen nem adja vissza az állapotegyenleteket. 
A Legrendre-transzofrmáció segítségével viszont át tudok térni egy olyan potenciálra, ami $T$-t használja természetes változóként:
\begin{equation}
    F(T, V, n) := U(S(T, V, n), V, n) - T S
\end{equation}
Ekkor pedig az állapotegyenletek a következőképpen néznek ki:
\begin{equation}
    dF = - S dT - p dV + \mu dn
\end{equation}
A parciális deriváltak pedig:
\begin{equation}
    \frac{\partial F}{\partial T}\bigg|_{V,n} = \underbrace{\frac{\partial U}{\partial S}\bigg|_{V,n}}_{T} \cdot \frac{\partial S}{\partial T}\bigg|_{V,n} - S - T \cdot \frac{\partial S}{\partial T}\bigg|_{V,n} = - S(T, V, n)
\end{equation}
\begin{equation}
    \frac{\partial F}{\partial V}\bigg|_{T,n} = \underbrace{\frac{\partial U}{\partial S}\bigg|_{V,n}}_{T} \cdot \frac{\partial S}{\partial V}\bigg|_{T,n} + \frac{\partial U}{\partial V}\bigg|_{S,n} - T \cdot \frac{\partial S}{\partial V}\bigg|_{T,n} = - p(T, V, n)
\end{equation}
\begin{equation}
    \frac{\partial F}{\partial n}\bigg|_{T,V} = \underbrace{\frac{\partial U}{\partial S}\bigg|_{V,n}}_{T} \cdot \frac{\partial S}{\partial n}\bigg|_{T,V} + \frac{\partial U}{\partial n}\bigg|_{S,V} - T \cdot \frac{\partial S}{\partial n}\bigg|_{T,V} = \mu(T, V, n)
\end{equation}
Tehát a Helmholtz szabad energia egy olyan termodinamikai potenciál, ami a hőmérsékletet használja természetes változóként az entrópia helyett. 
A szabad energia pedig a következőképpen néz ki ideális gázokra:
\begin{equation}
    F(T, V, n) = n R T \left\{\frac{f}{2} - ln \left[\left(\frac{fk}{2\phi}T\right)^{\frac{f}{2}}\cdot \frac{V}{n} \right]\right\}
\end{equation}
A differenciális alakja pedig:
\begin{equation}
    dF = \frac{\partial F}{\partial T}\bigg|_{V,n} dT + \frac{\partial F}{\partial V}\bigg|_{T,n} dV + \frac{\partial F}{\partial n}\bigg|_{T,V} dn = - S dT - p dV + \mu dn
\end{equation}
\newline
\newline
Ezek a lapján a következő \textbf{termodinamikai potenciálokat} tudjuk definiálni:
\begin{table}[H]
    \centering
    \begin{tabular}{|p{2cm}|p{4cm}|p{2cm}|c|c|}
        \hline
        Név & Jelölés & Természetes változók & Állapotegyenletek & Differenciális alak \\
        \hline
        Belső energia & $U(S, V, n)$ & $S, V, n$ & $T = \frac{\partial U}{\partial S}\big|_{V,n}$ & $dU = T dS - p dV + \mu dn$ \\
         &  &  & $p = - \frac{\partial U}{\partial V}\big|_{S,n}$ &  \\
         &  &  & $\mu = \frac{\partial U}{\partial n}\big|_{S,V}$ &  \\
        \hline
        Entrópia & $S(U, V, n)$ & $U, V, n$ & $\frac{1}{T} = \frac{\partial S}{\partial U}\big|_{V,n}$ & $dS = \frac{1}{T} dU + \frac{p}{T} dV - \frac{\mu}{T} dn$ \\
         &  &  & $\frac{p}{T} = \frac{\partial S}{\partial V}\big|_{U,n}$ &  \\
         &  &  & $- \frac{\mu}{T} = \frac{\partial S}{\partial n}\big|_{U,V}$ &  \\
        \hline
        Helmholtz szabad energia & $F(T, V, n) = U - TS$ & $T, V, n$ & $- S = \frac{\partial F}{\partial T}\big|_{V,n}$ & $dF = - S dT - p dV + \mu dn$ \\
         &  &  & $- p = \frac{\partial F}{\partial V}\big|_{T,n}$ &  \\
         &  &  & $\mu = \frac{\partial F}{\partial n}\big|_{T,V}$ &  \\
        \hline
        Entalpia & $H(S, p, n) = U + pV$ & $S, p, n$ & $T = \frac{\partial H}{\partial S}\big|_{p,n}$ & $dH = T dS + V dp + \mu dn$ \\
         &  &  & $V = \frac{\partial H}{\partial p}\big|_{S,n}$ &  \\
         &  &  & $\mu = \frac{\partial H}{\partial n}\big|_{S,p}$ &  \\
        \hline
        Gibbs szabad energia & $G(T, p, n) = U - TS + pV$ & $T, p, n$ & $- S = \frac{\partial G}{\partial T}\big|_{p,n}$ & $dG = - S dT + V dp + \mu dn$ \\
         &  &  & $V = \frac{\partial G}{\partial p}\big|_{T,n}$ &  \\
         &  &  & $\mu = \frac{\partial G}{\partial n}\big|_{T,p}$ &  \\
        \hline
    \end{tabular}
    \caption{Fontosabb termodinamikai potenciálok}
\end{table}
A termodinamikai főtételek alapján csak olyan anyagok létezhetnek, amelyek minden tulajdonságát ki lehet fejezni egy termodinamikai potenciál segítségével. De a termodinamikai potenciálok
ekvivalensek egymással, tehát szabadon megválaszthatom melyiket használom egy rendszer leírására.
\newline
\newline
De hogyan tudok a különböző potenciálok között átváltani? Erre lehet felhasználni a parciális deriváltak tulajdonságait.
Először is, a parciális derviáltak definíciójából következően egy tetszőleges $f(x,y)$ függvényre igaz, hogy:
\begin{equation}
    \frac{\partial}{\partial y} \left(\frac{\partial f}{\partial x}\bigg|_y\right)\bigg|_x = \frac{\partial}{\partial x} \left(\frac{\partial f}{\partial y}\bigg|_x\right)\bigg|_y
\end{equation}
Ezt a termodinamikai potenciálokra alkalmazva, például a belső energiára:
\begin{equation}
    \frac{\partial}{\partial V} \underbrace{\left(\frac{\partial U}{\partial S}\bigg|_{V,n}\right)}_{T}\bigg|_{S,n} = \frac{\partial}{\partial S} \underbrace{\left(-\frac{\partial U}{\partial V}\bigg|_{S,n}\right)}_{p}\bigg|_{V,n}
\end{equation}
\begin{equation}
    \frac{\partial T}{\partial V}\bigg|_{S,n} = - \frac{\partial p}{\partial S}\bigg|_{V,n}
\end{equation}
Hasonlóan a többi termodinamikai potenciálra is kapunk ilyen összefüggéseket. Ezeket a relációkat \textbf{Maxwell-relációknak} nevezzük, és
a segítségükkel tetszőleges állapotjelzők parciális deriváltjait tudjuk kifejezni egymás segítségével. Tehát az állapotegyenleteket a Maxwell-relációk segítségével
felírhatjuk a következő alakban is:
\begin{table}[H]
    \centering
    \begin{tabular}{|c|c|}
        \hline
        Term. Pot. & Maxwell-reláció \\
        \hline
        $U(S, V, n)$ & $\frac{\partial T}{\partial V}\big|_{S,n} = - \frac{\partial p}{\partial S}\big|_{V,n}$ \\
        \hline
        $S(U, V, n)$ & $\frac{\partial (1/T)}{\partial V}\big|_{U,n} = \frac{\partial (p/T)}{\partial U}\big|_{V,n}$ \\
        \hline
        $F(T, V, n)$ & $\frac{\partial S}{\partial V}\big|_{T,n} = \frac{\partial p}{\partial T}\big|_{V,n}$ \\
        \hline
        $H(S, p, n)$ & $\frac{\partial T}{\partial p}\big|_{S,n} = \frac{\partial V}{\partial S}\big|_{p,n}$ \\
        \hline
        $G(T, p, n)$ & $\frac{\partial S}{\partial p}\big|_{T,n} = - \frac{\partial V}{\partial T}\big|_{p,n}$ \\
        \hline
    \end{tabular}
    \caption{Maxwell-relációk}
\end{table}
Ezekkel a Maxwell-relációkkal megkaphatjuk a például a Joules-Thompson kísérlet eredményét is, szimplán a megfelelő parciális deriváltakat kifejezve a Maxwell-relációk segítségével:
\begin{equation}isér
    \frac{\partial H}{\partial p}\bigg|_{T} = \frac{\partial U}{\partial p}\bigg|_{T} + V + p \frac{\partial V}{\partial p}\bigg|_{T} 
\end{equation}
\begin{equation}
    \frac{\partial U}{\partial p}\bigg|_{T} = \frac{\partial U}{\partial V}\bigg|_{S} \cdot \frac{\partial V}{\partial p}\bigg|_{T} + \frac{\partial U}{\partial S}\bigg|_{V} \cdot \frac{\partial S}{\partial p}\bigg|_{T}
\end{equation}
\begin{equation}
    \frac{\partial U}{\partial p}\bigg|_{T} = -p \cdot \frac{\partial V}{\partial p}\bigg|_{T} + T \cdot \left(- \frac{\partial V}{\partial T}\bigg|_{p}\right)
\end{equation}
\begin{equation}
    \frac{\partial H}{\partial p}\bigg|_{T} = V - p \cdot \frac{\partial V}{\partial p}\bigg|_{T} + T \cdot \left(- \frac{\partial V}{\partial T}\big|_{p}\right) + p \frac{\partial V}{\partial p}\bigg|_{T} = V - T \cdot \frac{\partial V}{\partial T}\bigg|_{p}
\end{equation}
Ez pedig pontosan az a kifejezés, amit a Joules-Thompson kísérletnél használtunk.
\newline
\newline
Ugyanigy megkaphatjuk a Robert-Mayer egyenletet is:
\begin{equation}
    dU = TdS -pdV
\end{equation}
\begin{equation}
    C_P - C_V = \frac{1}{n} \left[p + \left(\frac{\partial U}{\partial V}\bigg|_{T}\right)\right] \cdot \underbrace{\frac{\partial V}{\partial T}\bigg|_{p}}_{\beta V} = \frac{1}{n} \left[T \cdot \frac{\partial S}{\partial V}\bigg|_{T}-p+p \right] \beta V
\end{equation}
\begin{equation}
    C_P - C_V = \frac{1}{n} T \cdot \underbrace{\frac{\partial p}{\partial T}\bigg|_{V}}_{p \cdot \gamma = \frac{\beta}{\kappa}} \beta V \qquad p \cdot \gamma \cdot \kappa = \beta
\end{equation}
\begin{equation}
    C_P - C_V = \frac{1}{n} \frac{TV\beta^2}{\kappa} = \frac{Tv\beta^2}{\kappa}
\end{equation}
Ez pedig pontosan a Robert-Mayer egyenlet.

\subsection{Fundamentális egyenlet}
Láttuk, hogy a fundamentális egyenletek azok az egyenletek amiknek a szabad változói szerinti parciális deriváltjaik szerint vissza tudjuk kapni az állapotegyenleteket.
Ez a fundamentális egyenlet sokféle formában létezhet, de az ideális gázokra már felírtuk egyszer:
\begin{equation}
    U_{\text{id}}(S, V, n) = \phi \cdot e^{\frac{2S}{fnR}} \cdot V^{-\frac{2}{f}} \cdot n^{\frac{f + 2}{f}} \qquad \phi: \text{állandó}
\end{equation}
Most nézzük meg, hogy milyen egyéb tulajdonságai vannak ennek az egyenletnek. Először vegyünk egy anyagot amit $S, V, n$ állapotjelzőkkel jellemezhetünk,
tehát az állapotegyenlete $U(S, V, n)$. Most nézzük meg mi történik ha veszek mégegyszer ennyit és összeengedném őket?:
\begin{equation}
    U_1(S, V, n) + U_2(S, V, n) = U(2S, 2V, 2n) = 2 U(S, V, n)
\end{equation}
Ez csak abban az esetben igaz, ha rövidtávúak a kölcsönhatások az anyagban, és nincs köztük potenciális energia. Ezért az ideális gázokra igaz ez a tulajdonság. (Ekkor sem teljesen igaz,
mert a határon lévő részecskék kölcsönhatnak egymással, de elhanyagolható mértékben ha elég nagy a rendszer). Ezt a tulajdonságot \textbf{homogenitásnak} nevezzük. Ez nem Csak
megkétszereződésnél igaz, hanem bármilyen $\lambda$ skalárral történő szorzásnál:
\begin{equation}
    U(\lambda S, \lambda V, \lambda n) = \lambda U(S, V, n)
\end{equation}
Ezt a tulajdonságot tehát az ideális gáz fundamentális egyenletének is tudnia kell. Nézzük meg hogyan változik a függvény ha $\lambda$-vel felskálázzuk a rendszert:
\begin{equation}
    T = \frac{\partial U}{\partial S}\bigg|_{V,n} \quad \to \quad \frac{\partial U}{\partial S}\bigg|_{V,n} (\lambda S, \lambda V, \lambda n) \cdot \underbrace{\lambda}_{\text{belső függvény deriváltja}} \stackrel{?}{=} \frac{\partial U}{\partial S}\bigg|_{V,n} (S, V, n) \cdot \lambda
\end{equation}
\begin{equation}
    T(\lambda S, \lambda V, \lambda n) \cdot \lambda= \lambda \cdot T (S, V, n) \quad \Rightarrow \quad T(\lambda S, \lambda V, \lambda n) = T (S, V, n)
\end{equation}
Ebből pedig következik, hogy a hőmérséklet intenzív állapotjelző, hiszen nem változik a rendszer felskálázásakor, ez pedig pont az az eredmény, mint amit vártunk.
Hasonlóan a többi intenzívállapotjelzőre is megkaphatjuk ugyanezt az eredményt:
\begin{equation}
    p(\lambda S, \lambda V, \lambda n) = p (S, V, n)
\end{equation}
\begin{equation}
    \mu(\lambda S, \lambda V, \lambda n) = \mu (S, V, n)
\end{equation}
De mi van akkor, ha a $\lambda$-t deriváljuk? Ezt úgy tudjuk megtenni, hogy a teljes differenciált vesszük:
\begin{equation}
    \frac{\partial}{\partial \lambda}: \quad \frac{\partial U}{\partial S}\bigg|_{V,n} (\lambda S, \lambda V, \lambda n) \cdot S + \frac{\partial U}{\partial V}\bigg|_{S,n} (\lambda S, \lambda V, \lambda n) \cdot V + \frac{\partial U}{\partial n}\bigg|_{S,V} (\lambda S, \lambda V, \lambda n) \cdot n
\end{equation}
De ezeket pont most számoltuk ki:
\begin{equation}
    U(S, V, n) = T(S, V, n) \cdot S - p(S, V, n) \cdot V + \mu(S, V, n) \cdot n
\end{equation} 
\begin{equation}
    \boxed{U = T S - p V + \mu n}
\end{equation}
Ez az egyenlet pedig az \textbf{Euler-egyenlet}, ami a fundamentális egyenlet homogenitásából következik. Ez az egyenlet bármilyen anyagra igaz,
ami homogenitást mutat. Az Euler-egyenletből pedig le tudjuk vezetni az \textbf{Gibbs-Duhem egyenletet} is. Ehhez vegyük az Euler-egyenlet teljes differenciáltját:
\begin{equation}
    dU = T dS + S dT - p dV - V dp + \mu dn + n d\mu
\end{equation}
De a termodinamika egyesített főtétele szerint:
\begin{equation}
    dU = T dS - p dV + \mu dn
\end{equation}
Ebből következik, hogy:
\begin{equation}
    S dT - V dp + n d\mu = 0
\end{equation}
Ez pedig a Gibbs-Duhem egyenlet, ami azt mondja ki, hogy az intenzív állapotjelzők nem függetlenek egymástól, hanem egy összefüggés köti őket össze.
Tehát ha egy rendszerben megváltoztatunk egy intenzív állapotjelzőt, akkor a többi is megváltozik ennek megfelelően.

\subsubsection{Összetett rendszerek egyensúlyi állapota}
Eddig olyan rendszerekkel foglalkoztunk amik egyensúlyban voltak. Ezeket a rendszereket lehetett jellemezni egy termodinamikai potenciálok segítségével.
De mi történik akkor, ha van egy összetett rendszerünk, ami több alrendszerből áll, és ezek az alrendszerek nincsenek egyensúlyban egymással? Hol állnak be egyensúlyi állapotba?
\newline
Azt tudjuk, hogy egyensúlyi állapotban igazak a következő feltételek:
\begin{equation}
    U(S, V, n), S(U, V, n) \quad \to \quad \frac{1}{T} = \frac{\partial S}{\partial U}\bigg|_{V,n} \quad \frac{p}{T} = \frac{\partial S}{\partial V}\bigg|_{U,n} \quad - \frac{\mu}{T} = \frac{\partial S}{\partial n}\bigg|_{U,V}
\end{equation}
Most vegyünk egy rendszert amely két alrendszerből áll, amiknek külön-külön ismerem az állapotjelzőit:
\begin{figure}[H]
    \centering
    \includegraphics[width=0.4\textwidth]{imgs/5-tetel/osszetett_rendszer.png}
    \caption{Összetett rendszer két alrendszerrel}
\end{figure}
Tegyük fel, hogy a rendszerek egy szigetelt tartályban vannak és egymás között nem tudnak anyagot vagy térfogatot cserélni, de hőt tudnak cserélni egymással.
Ekkor az állapotjelzők a következők lesznek:
\begin{equation}
    V_1 = \text{állandó} \qquad V_2 = \text{állandó} \qquad n_1 = \text{állandó} \qquad n_2 = \text{állandó}
\end{equation}
Feltesszük, hogy a hőcsere lassú, vagyis a folyamat kvázistatikus. Ekkor a teljes rendszer belső energiája pedig a két alrendszer belső energiájának összege lesz:
\begin{equation}
    U_{\text{össz}} = U_1 + U_2 = \text{állandó}
\end{equation}
Itt a két energia külön-külön nem állandó, de az összegük igen. Ekkor a teljes entrópia pedig a két alrendszer entrópiájának összege lesz:
\begin{equation}
    S_{\text{össz}} (U_{\text{össz}}, U_1, V_1, V_2, n_1, n_2) = S_1(U_1, V_1, n_1) + S_2(U_{\text{össz}} - U_1, V_2, n_2)
\end{equation}
Vegyük az össz entrópia deriváltját az $U_1$ változó szerint:
\begin{equation}
    \frac{\partial S_{\text{össz}}}{\partial U_1}\bigg|_{U_{\text{össz}}, V_1, V_2, n_1, n_2} = \frac{\partial S_1}{\partial U_1}\bigg|_{V_1, n_1} + \frac{\partial S_2}{\partial U_2}\bigg|_{V_2, n_2} \cdot \underbrace{\frac{\partial U_2}{\partial U_1}\bigg|_{U_{\text{össz}}}}_{-1}
\end{equation}
És mivel $\frac{1}{T} = \frac{\partial S}{\partial U}\bigg|_{V,n}$, ezért az egyenlet a következőképpen alakul:
\begin{equation}
    \frac{\partial S_{\text{össz}}}{\partial U_1}\bigg|_{U_{\text{össz}}, V_1, V_2, n_1, n_2} = \frac{1}{T_1} - \frac{1}{T_2}
\end{equation}
Mivel az entrópia maximumra törekszik, ezért a derivált értékének nullának kell lennie egyensúlyi állapotban:
\begin{equation}
    \frac{\partial S_{\text{össz}}}{\partial U_1}\bigg|_{U_{\text{össz}}, V_1, V_2, n_1, n_2} = 0 \quad \to \quad \frac{1}{T_1} - \frac{1}{T_2} = 0 \quad \to \quad T_1 = T_2
\end{equation}
Tehát az egyensúlyi állapotban a két alrendszer hőmérséklete megegyezik. 
\newline
\newline
Hasonlóan, ha a két alrendszer között térfogatcsere is lehetséges. Például ha fal a két rendszer között mozgatható, akkor a térfogat is változhat:
\begin{equation}
    V_{\text{össz}} = V_1 + V_2 = \text{állandó}
\end{equation}
Ekkor az össz entrópia a következőképpen alakul:
\begin{equation}
    S_{\text{össz}} (U_{\text{össz}}, U_1, V_{\text{össz}}, V_1, n_1, n_2) = S_1(U_1, V_1, n_1) + S_2(U_{\text{össz}} - U_1, V_{\text{össz}} - V_1, n_2)
\end{equation}
Vegyük az össz entrópia deriváltját a $V_1$ változó szerint:
\begin{equation}
    \frac{\partial S_{\text{össz}}}{\partial V_1}\bigg|_{U_{\text{össz}}, U_1, V_{\text{össz}}, n_1, n_2} = \frac{\partial S_1}{\partial V_1}\bigg|_{U_1, n_1} + \frac{\partial S_2}{\partial V_2}\bigg|_{U_2, n_2} \cdot \underbrace{\frac{\partial V_2}{\partial V_1}\bigg|_{V_{\text{össz}}}}_{-1}
\end{equation}
És mivel $\frac{p}{T} = \frac{\partial S}{\partial V}\bigg|_{U,n}$, ezért az egyenlet a következőképpen alakul:
\begin{equation}
    \frac{\partial S_{\text{össz}}}{\partial V_1}\bigg|_{U_{\text{össz}}, U_1, V_{\text{össz}}, n_1, n_2} = \frac{p_1}{T_1} - \frac{p_2}{T_2}
\end{equation}
Mivel az entrópia maximumra törekszik, ezért a derivált értékének nullának kell lennie egyensúlyi állapotban:
\begin{equation}
    \frac{\partial S_{\text{össz}}}{\partial V_1}\bigg|_{U_{\text{össz}}, U_1, V_{\text{össz}}, n_1, n_2} = 0 \quad \to \quad \frac{p_1}{T_1} - \frac{p_2}{T_2} = 0 \quad \to \quad \frac{p_1}{T_1} = \frac{p_2}{T_2}
\end{equation}
De mivel már tudjuk, hogy egyensúlyi állapotban $T_1 = T_2$, ezért következik, hogy $p_1 = p_2$. Tehát egyensúlyi állapotban a két alrendszer nyomása is megegyezik.
\newline
\newline
Összegezve azt lehet általánosan felírni, hogy egy általános belső változóra (vagy belső szabadsági-fok):
\begin{equation}
    \frac{\partial S_{\text{össz}}}{\partial X_i}\ = 0 \quad \quad \frac{\partial S}{\partial U} = \frac{1}{T}
\end{equation}
Tehát egy belső változó változásakor az entrópia egy extrémumot (maximumot) vesz fel, ha pedig ezt az entrópiát az energia parciális deriváltjával fejezzük ki, akkor az intenzív állapotjelzők egyenlőségét kapjuk egyensúlyi állapotban.
Az entrópia maximuma miatt pedig a második deriváltaknak negatívnak kell lennie egyensúlyi állapotban:
\begin{equation}
    \frac{\partial^2 S_{\text{össz}}}{\partial X_i^2} < 0
\end{equation}
\newline
\newline
Most nézzük meg, hogy mivan ha az entrópia tud áramolni a két rendszer között:
\begin{equation}
    U_1(S, V, n), U_2(S, V, n) \quad \to \quad S_{\text{össz}} = S_1 + S_2 = \text{állandó}
\end{equation}
\begin{equation}
    V_1 = \text{állandó} \qquad V_2 = \text{állandó} \qquad n_1 = \text{állandó} \qquad n_2 = \text{állandó}
\end{equation}
Ilyen eset nincs a valóságban, mert kell hőcsere az entrópia áramlásához, de nézzük meg elméletben. Ekkor a következő állapotegyenletek lesznek érvényesek:
\begin{equation}
    \frac{\partial U}{\partial S}\bigg|_{V,n} = T \qquad \qquad \frac{\partial U}{\partial V}\bigg|_{S,n} = - p \qquad \frac{\partial U}{\partial n}\bigg|_{S,V} = \mu
\end{equation}
Ekkor az össz entrópia a következőképpen alakul:
\begin{equation}
    U_{\text{össz}} (S_{\text{össz}}, S_1, V_1, V_2, n_1, n_2) = U_1(S_1, V_1, n_1) + U_2(S_{\text{össz}} - S_1, V_2, n_2)
\end{equation}
Vegyük az össz entrópia deriváltját az $S_1$ változó szerint:
\begin{equation}
    \frac{\partial U_{\text{össz}}}{\partial S_1}\bigg|_{S_{\text{össz}}, V_1, V_2, n_1, n_2} = \frac{\partial U_1}{\partial S_1}\bigg|_{V_1, n_1} + \frac{\partial U_2}{\partial S_2}\bigg|_{V_2, n_2} \cdot \underbrace{\frac{\partial S_2}{\partial S_1}\bigg|_{S_{\text{össz}}}}_{-1}
\end{equation}
És mivel $T = \frac{\partial U}{\partial S}\big|_{V,n}$, ezért az egyenlet a következőképpen alakul:
\begin{equation}
    \frac{\partial U_{\text{össz}}}{\partial S_1}\bigg|_{S_{\text{össz}}, V_1, V_2, n_1, n_2} = T_1 - T_2
\end{equation}
Mivel az energia minimumra törekszik, ezért a derivált értékének nullának kell lennie egyensúlyi állapotban:
\begin{equation}
    \frac{\partial U_{\text{össz}}}{\partial S_1}\bigg|_{S_{\text{össz}}, V_1, V_2, n_1, n_2} = 0 \quad \to \quad T_1 - T_2 = 0 \quad \to \quad T_1 = T_2
\end{equation}
Tehát ebben a virtuális esetben is visszakaptuk az eredményt amit a hőcsere esetén is, vagyis egyensúlyi állapotban a két alrendszer hőmérséklete megegyezik.
Viszont ebben a rendszerben az energia minimumra törekszik, nem pedig az entrópia maximumra (ezt most nem számoljuk ki):
\begin{equation}
    \frac{\partial^2 U_{\text{össz}}}{\partial S_1^2} > 0
\end{equation}
Ezt nevezik az energia minimum elvének, ami ekvivalens az entrópia maximum elvével, de nem tartozik valós fizikai folyamat hozzá. Fontos, hogy Ezek
a minimum és maximum elvek csak akkor érvényesek, ha a belső változók szerint nézem a változást. Ha az egész rendszerre nézem meg, akkor nem ugyan azt kapjuk:
\begin{equation}
    \frac{\partial U_{\text{össz}}}{\partial S_{\text{össz}}} = \frac{\partial U_1}{\partial S} (S_{\text{össz}} - S_1, V_2, n_2) = T_2 = T_1 \neq 0 \qquad \text{Egyensúlyban}
\end{equation}
A többi termodinamikai potenciálra is lehet ilyen minimum elveket felírni zárt rendszerek esetén:
\begin{itemize}
    \item Ha a rendszer egy hőfürdővel érintkezik (állandó $T$), akkor a Helmholtz szabad energia ($F$) minimumra törekszik egyensúlyi állapotban.
    \item Ha a nyomást tartjuk állandóan (állandó $p$), akkor az entalpia ($H$) minimumra törekszik egyensúlyi állapotban.
    \item Ha a hőmérsékletet és a nyomást tartjuk állandóan (állandó $T, p$), akkor a Gibbs szabad energia ($G$) minimumra törekszik egyensúlyi állapotban.
\end{itemize}

\subsection{Fázisátalakulások, Fázisátalakulások jellemzői, típusai, Gibbs-féle fázisszabály, fázisdiagramok, fázisegyensúlyok}
Az anyagok eddigi vizsgálata során feltételeztük, hogy az anyag egy adott halmazállapotban van, és az állapotjelzők folyamatosan változnak.
Azonban a valóságban minden anyagra vannak jellemző kritikus állapotjelző értékek, ahol az anyag fundamentális fizikai tulajdonságai megváltoznak és ezáltal 
egy új halmazállapotba kerülnek. Ezt a folyamatot \textbf{fázisátalakulásnak} nevezzük. A fázisátalakulások során az anyag szerkezete, sűrűsége,
hőkapacitása, vezetőképessége és más fizikai tulajdonságai is megváltozhatnak. A fázisátalakulások során az anyag belső energiája és entrópiája is változik, ami a termodinamikai potenciálok
változását eredményezi.

\subsubsection{Egykomponensű rendszerek egyensúlyi állapota}
Korábban megvizsgáltuk olyan rendszerek egyensúlyi állapotát, amelyek több alrendszerből álltak, amik két különböző anyagból álltak, és csak
bizonyos állapotjelzők cserélődhettek közöttük. Most nézzük meg egy olyan rendszert, amely egykomponensű, tehát csak egyféle anyagot tartalmaz. 
Ha viszont feltételezek homogenitást (ahogy eddig tettük) ebben a rendszerben akkor meg is van az egyensúlyi állapot, többet ezzel már nem is kell foglalkozni.
A valóságban azonban egy komponensű rendszerek se mindig homogének, van, hogy egyes részeinek más a belső energiája, entrópiája, stb. Ilyen esetben a rendszer
részei között is létrejöhet anyag- és energiaáramlás, amíg el nem érik az egyensúlyi állapotot. 

Most vegyünk egy olyan egykomponensű rendszert, amely két részből áll, amik között hő és ezáltal energia áramlás is lehetséges:
\begin{figure}[H]
    \centering
    \includegraphics[width=0.4\textwidth]{imgs/5-tetel/osszetett_rendszer_egykomponensu.png}
    \caption{Egykomponensű összetett rendszer két alrendszerrel}
\end{figure}
Ekkor az Entrópia-Energia grafikon a következőképpen néz ki:
\begin{figure}[H]
    \centering
    \includegraphics[width=0.6\textwidth]{imgs/5-tetel/egykomponensu_fazisdiagram.png}
    \caption{Egykomponensű rendszer Entrópia-Energia grafikonja}
\end{figure}
Tegyük fel, hogy egy kis $\Delta U$ energia átáramlik az egyik oldalról a másikra. Ekkor a teljes entrópia változása a következőképpen alakul:
\begin{equation}
    2S(U_0, V, n) \quad \to \quad S(U_0 - \Delta U, V, n) + S(U_0 + \Delta U, V, n)
\end{equation}
A kapott entrópia pedig nagyobb lesz, mint a kezdeti entrópia:
\begin{figure}[H]
    \centering
    \includegraphics[width=0.6\textwidth]{imgs/5-tetel/entropia_valtozas_egykomponens.png}
    \caption{Az új entrópia nagyobb, mint a kezdeti entrópia}
\end{figure}
Ebből következik, hogy inhomogén rendszereknél az energia áramlása növelheti az entrópiát és ezáltal egy olyan folyamat ami be tud következni.
Egy homogén rendszernél a függvény mindig konkáv lenne, így ott az energia áramlása csökkentené az entrópiát, ami nem történhet meg:
\begin{equation}
    \frac{\partial^2 S}{\partial U^2}\bigg|_{V,n} < 0 \quad \to \quad \text{konkáv}
\end{equation}
És mivel tudjuk, hogy az a rendszer valósul meg, ami a legnagyobb entrópiával rendelkezik, ezért az inhomogén rendszereknél az energia áramlása addig fog tartani, amíg el nem éri a
maximális entrópiát. Ez az állapot pedig olyan lesz ami két különböző "energiaszintet" tartalmaz, hiszen a görbe konkáv részeinél az entrópia növelhető energia áramlással.
Ezért az egykomponensű rendszerek egyensúlyi állapota általában több fázisból áll, amik különböző belső energiával rendelkeznek:
\begin{figure}[H]
    \centering
    \includegraphics[width=0.6\textwidth]{imgs/5-tetel/egykomponensu_fazisdiagram_egyensuly.png}
    \caption{Egykomponensű rendszer két nergiacsúccsal egyensúlyi állapotban}
\end{figure}
Erre azt tudom felírni, hogy a két rész belső energiája:
\begin{equation}
    U_1 = U_0 - \Delta U_1 \qquad U_2 = U_0 + \Delta U_2
\end{equation}
Eddig feltételeztük, hogy a két rész azonos mennyiségű anyagot tartalmaz, de ezt semmi nem követeli meg, ezét vezessünk be $x_1$ és $x_2$ mennyiségeket, amik az anyagmennyiség arányát adják meg:
\begin{equation}
    0 < x_1, x_2 < 1 \qquad x_1 + x_2 = 1 \quad \rightarrow \quad x_1 = 1 - x_2
\end{equation}
Azt is tudom, hogy a teljes belső energia állandó, ezért a belső energia változás 0:
\begin{equation}
    -x_1 \Delta U_1 + x_2 \Delta U_2 = 0
\end{equation}
\begin{equation}
    (x_2 - 1) \Delta U_1 + x_2 \Delta U_2 = 0 \quad \to \quad x_2 (\Delta U_1 + \Delta U_2) = \Delta U_1
\end{equation}
Ebből pedig ki tudjuk fejezni az arányokat:
\begin{equation}
    x_2 = \frac{\Delta U_1}{\Delta U_1 + \Delta U_2} \qquad x_1 = \frac{\Delta U_2}{\Delta U_1 + \Delta U_2}
\end{equation}
Ha az entrópia eloszlását nézzük akkor arra is felírhatjuk a következő összefüggést:
\begin{equation}
    S_{\text{össz}} = x_1 S(U_1, V, n) + x_2 S(U_2, V, n)
\end{equation}
Itt is kifejezhetem $x_1$-et:
\begin{equation}
    \begin{aligned}
        S_{\text{össz}} &= (1 - x_2) S(U_1, V, n) + x_2 S(U_2, V, n) = \\
        &= S_1 + x_2 \left(S_2 - S_1\right) = \\
        &= S_1 + \frac{\Delta U_1}{\Delta U_1 + \Delta U_2} \left(S_2 - S_1\right)
    \end{aligned}
\end{equation}
Tehát az össz entrópia $S_1$ és $S_2$ között van, mégpedig pont olyan arányban, mint ahogy a $\Delta U_1$ aránylik az össz energiaváltozáshoz:
\begin{equation}
    S_{\text{össz}} = S_1 + \frac{\Delta U_1}{\Delta U_1 + \Delta U_2} \left(S_2 - S_1\right)
\end{equation}
Ez az összefüggés pedig a \textbf{mérleg szabály}. Ha megnézzük a grafikonon akkor azt fogjuk látni, hogy az össz entrópia pont azon a vonalon helyezkedik el, ami összeköti $S_1$-et és $S_2$-t:
\begin{figure}[H]
    \centering
    \includegraphics[width=0.6\textwidth]{imgs/5-tetel/merleg_szabaly.png}
    \caption{Mérleg szabály grafikus ábrázolása}
\end{figure}
Ebből az következik, hogy egy inhomogén egykomponensű rendszer egyensúlyi állapota állhat több fázisból, ha van egy konvex inflexiós pont az entrópia-görbében.
Ezek a fázisok saját belső energiával és entrópiával rendelkeznek, és az össz entrópia a mérleg szabály szerint alakul ki közöttük. Felmerülhet a kérdés, hogy
ha más a belső energiájuk, akkor a komponensek hőmérsékete is más lesz-e? A válasz az, hogy nem, hiszen a hőmérséklet (reciproka) az entrópia deriváltja az energia szerint,
az pedig a két csúcsban pont ugyanaz lesz:
\begin{equation}
    \frac{1}{T} = \frac{\partial S}{\partial U}\bigg|_{V,n}
\end{equation}
Ez megfelel a várakozásnak, hisz a hőmérséklet egy intenzív állapotjelző, tehát arra törekszik, hogy mindenhol ugyanaz legyen egyensúlyi állapotban. Ugyanigy 
van ez a nyomással is és a kémiai potenciállal is:
\begin{equation}
    \frac{p}{T} = \frac{\partial S}{\partial V}\bigg|_{U,n} \qquad - \frac{\mu}{T} = \frac{\partial S}{\partial n}\bigg|_{U,V}
\end{equation}
\begin{equation}
    p_1 = p_2 \qquad \mu_1 = \mu_2
\end{equation}
Tehát egykomponensű rendszerek egyensúlyi állapotában minden intenzív állapotjelző megegyezik a különböző fázisok között.
Ez azt jelenti, hogy ez a belső energia eltérés valamilyen más tulajdonságot fejez ki, nem pedig állapotjelzők eltérését. 
\newline
\newline
Ha felírjuk a stabilitás feltételét az entrópia második deriváltjára, akkor azt kapjuk, hogy:
\begin{equation}
    \frac{\partial^2 S}{\partial U^2}\bigg|_{V,n} < 0
\end{equation}
De mivel:
\begin{equation}
    \frac{\partial S}{\partial U}\bigg|_{V,n} = \frac{1}{T} \quad \to \quad \frac{\partial^2 S}{\partial U^2}\bigg|_{V,n} = \frac{\partial}{\partial U} \left(\frac{1}{T}\right)\bigg|_{V,n} = - \frac{1}{T^2} \cdot \frac{\partial T}{\partial U}\bigg|_{V,n}
\end{equation}
Ezért a stabilitás feltétele a következőképpen alakul:
\begin{equation}
    - \frac{1}{T^2} \cdot \frac{\partial T}{\partial U}\bigg|_{V,n} < 0 \quad \to \quad \frac{\partial T}{\partial U}\bigg|_{V,n} > 0
\end{equation}
Ez azt jelenti, hogy a hőkapacitásnak pozitívnak kell lennie egyensúlyi állapotban:
\begin{equation}
    C_{V,n} = \frac{\partial U}{\partial T}\bigg|_{V,n} > 0
\end{equation}
Ha ez nem teljesül, akkor a rendszer instabil, és az entrópia nem lesz konkáv, így a rendszer inhomogén állapotba kerülhet, ahol több fázis is jelen van.
Ezért a pozitív hőkapacitás egy szükséges feltétel a homogenitás és stabilitás biztosításához egykomponensű rendszerek esetén. Ezt nevezik \textbf{Le Chatelier-elvnek}.
\newline
\newline
Van még stabilitási feltétel, mert a térfogat is változhat egyensúlyi állapotban és a térfogatra nézve is maximumra törekszik az entrópia:
\begin{equation}
    \frac{\partial^2 S}{\partial V^2}\bigg|_{U} < 0
\end{equation}
Ez azért kell mert nem csak egy dimenzióban kell stabilnak lennie a rendszernek, hanem mind a három állapotjelző irányában.
Tehát ha van egy $S(U, V)$ entrópia függvényünk, akkor annak mind a két irányban konkávnak kell lennie egyensúlyi állapotban. Erre felírhatunk egy mátrixot is:
\begin{equation}
    \begin{pmatrix}
    \frac{\partial^2 S}{\partial U^2}\big|_{V} & \frac{\partial^2 S}{\partial U \partial V}\big|_{V} \\
    \frac{\partial^2 S}{\partial V \partial U}\big|_{V} & \frac{\partial^2 S}{\partial V^2}\big|_{U}
    \end{pmatrix}
\end{equation}
És erre az a feltételünk matematikaileg, hogy a mátrixnak negatív definitnek kell lennie egyensúlyi állapotban, ami azt jelenti, hogy a mátrix által kijelölt
felület "lefelé hajlik". Ez pedig csak akkor igaz, ha a mátrix minden sajátértéke negatív, vagyis a determinánsa pozitív és a főátló elemei váltakozó előjelűek:
\begin{equation}
    \begin{vmatrix}
    \frac{\partial^2 S}{\partial U^2}\big|_{V} & \frac{\partial^2 S}{\partial U \partial V}\big|_{V} \\
    \frac{\partial^2 S}{\partial V \partial U}\big|_{V} & \frac{\partial^2 S}{\partial V^2}\big|_{U}
    \end{vmatrix} > 0
\end{equation}
Ebből a feltételből pedig arra lehet jutni, hogy a kompresszibilitási együtthatónak is pozitívnak kell lennie egyensúlyi állapotban:
\begin{equation}
    \kappa = - \frac{1}{V} \cdot \frac{\partial V}{\partial p}\bigg|_{T} > 0
\end{equation}
Tehát egy egykomponensű rendszer stabilitásának a feltételei a következők:
\begin{equation}
    \boxed{\text{pozitív hőkapacitás: } C_{V} > 0 \qquad \text{pozitív kompresszibilitás: } \kappa > 0}
\end{equation}
Ezt nevezik \textbf{Le Chatelier-Braun-elvnek}. 
\newline
\newline
Ezt az jelenséget, ahol az anyag "szétesik" két részre úgy tudjuk szemléletesen elképzelni mint például egy tartályt aminek az alján víz van és fölötte vízgőz.
Ekkor a tartályban lévő anyag két fázisban van jelen, egy folyékony és egy gáz halmazállapotban. A két fázis között a közeghatár jelenti a falat, amin anyag nem,
de hő és energia tud áramolni. A két rész pedig úgy áll be egyensúlyi állapotba, hogy a hőmérsékletük, nyomásuk és kémiai potenciáljuk megegyezik, 
a hőkapacitásuk és kompresszibilitásuk pedig pozitív értékeket vesz fel. A belső energia és az entrópia különbsége pedig az eltérő halmazállapotokból adódik.

\subsubsection{Fázisátalakulások, fázisdiagramok}
Láttuk, hogy két fázis tehát akkor van egyensúlyban, ha a következő feltételek teljesülnek:
\begin{equation}
    T_1 = T_2 \qquad p_1 = p_2 \qquad \mu_1 = \mu_2
\end{equation}
\begin{equation}
   C_{V} > 0 \qquad \kappa > 0
\end{equation}
Emellett a fundamentális egyenlet is igaz marad:
\begin{equation}
    dU = T dS - p dV + \mu dn \qquad U = TS - pV + \mu n
\end{equation}
De a belső energia változását teljes differenciális formában is fel tudjuk írni:
\begin{equation}
    dU = SdT + TdS - Vdp - pdV + \mu dn + n d\mu
\end{equation}
A két egyenletet kivonva egymásból kapjuk, hogy:
\begin{equation}
    0 = SdT - Vdp + n d\mu
\end{equation}
Ebből kifejezve a kémiai potenciál differenciálisát:
\begin{equation}
    d\mu = - \frac{S}{n} dT + \frac{V}{n} dp
\end{equation}
Bevezetve a moláris entrópiát és moláris térfogatot:
\begin{equation}
    s = \frac{S}{n} \qquad v = \frac{V}{n}
\end{equation}
Az egyenlet a következőképpen alakul:
\begin{equation}
    d\mu = - s dT + v dp
\end{equation}
Ez az egyenlet a \textbf{Gibbs-Duhem-egyenlet}. Ez az egyenlet azt fejezi ki, hogy egykomponensű rendszerekben a kémiai potenciál hogyan változik a hőmérséklet és
a nyomás függvényében. Ebből az egyenletből ki tudjuk fejezni a kémiai potenciál parciális deriváltjait is:
\begin{equation}
    \frac{\partial \mu}{\partial T}\bigg|_{p} = - s = - \frac{S}{n} \qquad \qquad \frac{\partial \mu}{\partial p}\bigg|_{T} = v = \frac{V}{n}
\end{equation}
Ebből pedig látszódi, hogy ha állandó a hőmérséklet, akkor a kémiai potenciál a nyomással nő, hiszen a moláris térfogat pozitív:
\begin{equation}
    T = \text{áll.} \quad \Rightarrow \quad d\mu = vdp
\end{equation}
Tehát a kémiai potenciált meghatározhatom úgy is, hogy integrálom a nyomás függvényében:
\begin{equation}
    \mu (p, v) = \mu_{(1)} + \int_{p_0}^{p} v dp
\end{equation}
Ábrázoljuk mondjuk egy Van der Waals-gáz esetén a kémiai potenciált a $pV$ grafikonon állandó hőmérsékleten:
\begin{figure}[H]
    \centering
    \includegraphics[width=0.6\textwidth]{imgs/5-tetel/kemia_potencial_vdw.png}
    \caption{Kémiai potenciál ábrázolása Van der Waals-gáz esetén}
\end{figure}
De a kémiai potenciálban van egy integrál a nyomás szerint, ezért váltsunk át a $V-p$ grafikonra és nézzük meg az integrál értelmezését ott:
\begin{figure}[H]
    \centering
    \includegraphics[width=0.6\textwidth]{imgs/5-tetel/kemia_potencial_vdw_vp.png}
    \caption{Kémiai potenciál ábrázolása Van der Waals-gáz esetén $V-p$ grafikonon és a hozzátartozó $\mu-p$ grafikonja}
\end{figure}
Látható, hogy a kémai potenciál grafikonján van egy metszéspont, ahol a két kémiai potenciál tartozik egy nyomás értékhez.
Ez a pont ott van, ahol a $V-p$ grafikonon a csúcsok területe pont egyenlő (ábrán kékkel jelölve). Ezt visszavezeve a $pV$ grafikonra:
\begin{figure}[H]
    \centering
    \includegraphics[width=0.6\textwidth]{imgs/5-tetel/azonos_kemia_potencial_vdw_pv.png}
    \caption{Egyenlő kémiai potenciálok a Van der Waals-gáz esetén $pV$ grafikonon}
\end{figure} 
Ez a pont pedig a két fázis egyensúlyi pontja, ahol a két fázis kémiai potenciálja megegyezik. Ezt a pontot \textbf{gőznyomásnak} nevezzük.
Ez a nyomásérték az a pont, ahol a folyadék és a gőz fázisok egyensúlyban vannak egymással adott hőmérsékleten. 
Ez a pont grafikus megtalálását Maxwell-szerkesztésnek nevezzük. Ha pedig a nyomás és a hőmérséklet állandó, akkor a Gibbs-potenciál fog minimalizálódni, és adja meg az egyensúlyi arányát a két fázisnak.
amit úgy tudunk felírni, hogy:
\begin{equation}
    G = H - TS = U + pV - TS = \mu n
\end{equation}
Ebből pedig látszik, hogy ha a kémiai potenciál egyenesen arányos a Gibbs-potenciállal, Ezért ebben a rendszerben a kémiai potenciál minimuma fogja megadni az egyensúlyi állapotot.
A Gibbs-potenciál minimalizálása miatt meg, ha csökkentem a nyomást vagy növelem a hőmérsékletet, akkor a folyadék két alrendszerre fok szétszakadni, ami a fázisátalakulásnak felel meg.
Ez a $pV$ és a $V-p$ grafikonokon egy ugrásként jelenik meg, amit \textbf{fázisátalakulásnak} nevezünk.
\begin{figure}[H]
    \centering
    \includegraphics[width=0.6\textwidth]{imgs/5-tetel/fazisatalakulas_vdw.png}
    \caption{Fázisátalakulás Van der Waals-gáz esetén $pV$ és $V-p$ grafikonokon}
\end{figure}
Ezek alapján fel tudjuk írni a fázisátalakulás jellemzőit:
\begin{itemize}
    \item A Gibbs-potenciál és a kémiai potenciál minimalizálódik az egyensúlyi állapotban.
    \item A $G$ és a $\mu$ folytonosan mennek át.
    \item A $U$, $S$, $H$ és $F$ termodinamikai potenciáloknak ugrása van.
    \item A $V$, $c$, $\alpha$, $\kappa$, $\gamma$ állapotjelzőknek ugrása van.
\end{itemize}
Megnézhetjük, hogy a fázisátalakulás hogyan is néz ki a hőmérséklet függvényében:
\begin{figure}[H]
    \centering
    \includegraphics[width=0.6\textwidth]{imgs/5-tetel/fazisatalakulas_tp.png}
    \caption{Fázisátalakulás hőmérséklet-nyomás grafikonon}
\end{figure}
Ezt a grafikont nevezik \textbf{Fázisdiagramnak}. Nézzük meg egy komplexebb anyag, például a szén-dioxid és a víz fázisdiagramját is:
\begin{figure}[H]
    \centering
    \begin{subfigure}{0.45\textwidth}
        \includegraphics[width=\linewidth]{imgs/5-tetel/viz_fazisdiagram.jpg}
        \caption{Víz fázisdiagramja}
    \end{subfigure}
    \begin{subfigure}{0.45\textwidth}
        \includegraphics[width=\linewidth]{imgs/5-tetel/co2_fazisdiagram.jpg}
        \caption{Szén-dioxid fázisdiagramja}
    \end{subfigure}
\end{figure}
Látható, hogy ezeken a diagramokon megjelenik az úgynevezett Hármas pont is, ahol az anyag egyszerre három fázisban van jelen (szilárd, folyékony, gáz).
Ez a pont egy adott hőmérséklet és nyomás értéknél található, és az anyag ezen a ponton egyszerre létezhet mindhárom halmazállapotban. 
Ezen kívül megjelenik a kritikus pont is, ahol a folyadék és gáz fázisok közötti különbség eltűnik, és az anyag egy szuperkritikus folyadék állapotba kerül.
Ez a pont is egy adott hőmérséklet és nyomás értéknél található, és az anyag ezen a ponton már nem különböztethető meg folyadék és gáz fázisokra.

A Fázisátalakuláskora még egy mértéket be lehet vezetni, ugyanis az átalakulás közben munkavégzés is történik, ezt pedig látens hőnek nevezzük ($L$, $\left[L\right] = J$), és azt adja meg, hogy
mennyi hő szükséges az anyag egy adott mennyiségének átalakításához egyik fázisból a másikba anélkül, hogy a hőmérséklete megváltozna.
Ezt a következőképpen tudjuk kiszámolni:
\begin{equation}
    \frac{dp}{dT} = \frac{L}{T \cdot \Delta V} \qquad : \text{Clausius-Clapeyron egyenlet}
\end{equation}
A látens hő pozitív érték, ha az anyag alacsonyabb energiájú fázisból magasabb energiájú fázisba megy át (pl. szilárdból folyékonyba vagy folyékonyból gázba),
és negatív érték, ha az anyag magasabb energiájú fázisból alacsonyabb energiájú fázisba megy át (pl. gázból folyékonyba vagy folyékonyból szilárdba). A víz esetén például
a látens hő értéke a következő:
\begin{itemize}
    \item Olvadáshő (szilárd $\to$ folyékony): $L_f = 334 \, \text{kJ/kg}$
    \item Párolgáshő (folyékony $\to$ gáz): $L_v = 2260 \, \text{kJ/kg}$
\end{itemize}
Ezek az értékek azt jelentik, hogy 1 kg jég olvasztásához 334 kJ hő szükséges, míg 1 kg víz pároltatásához 2260 kJ hő szükséges anélkül, hogy a hőmérséklete megváltozna.

\subsubsection{folytonos fázisátalakulások, Gibbs-féle fázisszabály}
A fázisátalakulásoknak két fő típusa van: diszkrét (elsőrendű) és folytonos (másodrendű) fázisátalakulások.
A diszkrét fázisátalakulások során az anyag hirtelen változtatja meg a halmazállapotát, például amikor a víz fagy vagy forr. Ezeknél az átalakulásoknál a termodinamikai potenciál
ugrásokat mutat, és a fázisátalakulás során látens hő is felszabadul vagy elnyelődik.
Ezzel szemben a folytonos fázisátalakulások során az anyag fokozatosan változtatja meg a halmazállapotát, például amikor egy mágneses anyag elveszíti a mágneses tulajdonságait a Curie-pont
közelében. Ezeknél az átalakulásoknál a termodinamikai potenciálok folyamatosan változnak, és nincs látens hő felszabadulás vagy elnyelődés.
\newline
\newline
Eddig mi elsőrendű fázisátalakulásokat vizsgáltunk, de minden anyagnál elő lehet idézni másodrendű fázisátalakulásokat is, a kritikus pont közelében.
A kritikus pontnál az anyag tulajdonságai drasztikusan megváltoznak, és a fázisok közötti különbség eltűnik. Ilyenkor a hőkapacitás, kompresszibilitás és más állapotjelzők
divergenssé válnak, vagyis elszállnak végtelenbe. Ez azt jelenti, hogy az anyag rendkívül érzékennyé válik a külső hatásokra, és kis változások is nagy hatással lehetnek a tulajdonságaira.
\begin{equation}
    \frac{\partial p}{\partial V}\bigg|_{T} = 0 \qquad \text{kritikus pontban}
\end{equation}
\begin{equation}
    \kappa = - \frac{1}{V} \cdot \frac{\partial V}{\partial p}\bigg|_{T} \quad \to \quad \kappa \to \infty
\end{equation}
Tehát a kritikus pont közelében összenyomásra nincs "visszatartó erő", vagyis szabadon össze lehet nyomni az anyagot, ezért nagy sűrűségfluktuációk jönnek létre.
Ezt a jelenséget \textbf{kritikus opaleszcenciának} nevezzük, mert az anyag sűrűsége befolyásolja a fényáteresztő képességét is, ami a kritikus pont közelében nagyon leesik emiatt.
\newline
\newline
Láttuk, hogy fázisátalakulások során, ha a nyomás és a hőmérséklet állandó, akkor a Gibbs-potenciál minimalizálódik.
Ez azt jelenti, hogy a rendszer egyensúlyi állapotban van, amikor a Gibbs-potenciál értéke a legalacsonyabb. Ha több fázis is jelen van,
akkor a Gibbs-potenciál függ a különböző komponensek anyagmennyiségétől:
\begin{equation}
    k: \text{Komponensek száma} \qquad m: \text{Fázisok száma}
\end{equation}
\begin{equation}
    G = G(T, p, n_1, n_2, \ldots, n_k)
\end{equation}
Itt bevezethetjük a koncentráció fogalmát ami megadja az egyes komponensek anyagmennyiségének arányát a teljes anyagmennyiséghez képest:
\begin{equation}
    c_i = \frac{n_i}{\sum_i n_i} \qquad n = \sum_i n_i \qquad c_i = \frac{n_i}{n} \qquad 1 \leq 1 \leq k
\end{equation}
Ekkor a Gibbs-potenciált fel tudjuk írni a koncentrációk függvényében is:
\begin{equation}
    G = G \underbrace{(T, p, c_1, c_2, \ldots, c_{k-1}, n)}_{k + 2 \text{ változó}} = n \cdot g \underbrace{(T, p, c_1, c_2, \ldots, c_{k-1})}_{k + 1 \text{ változó}}
\end{equation}
Ahol $g$ a moláris Gibbs-potenciál, ami az anyagmennyiségre normalizált Gibbs-potenciált jelenti:
\begin{equation}
    g = \frac{G}{n} \qquad \left[g\right] = \frac{J}{mol}
\end{equation}
A Gibbs-potenciált a kémiai potenciál alapján is ki tudjuk fejezni:
\begin{equation}
    G = \sum_{i=1}^{k} n_i \mu_i \qquad \mu_i = \frac{\partial G}{\partial n_i}\bigg|_{T, p, n_{j \neq i}}
\end{equation}
Mivel a kémiai potenciál egy intenzív állapotjelző, ezért nem függ az anyagmennyiségtől, ezért a kémiai potenciál ugyan attól függ mint moláris Gibbs-potenciál:
\begin{equation}
    \mu_i = \mu_i (T, p, c_1, c_2, \ldots, c_{k-1})
\end{equation}
Fontos még megjegyezni, hogy a koncentrációk nem feltétlenül egyeznek meg a különböző fázisokban, hiszen egyes komponensek előnyben részesíthetik az egyik fázist a másikkal szemben.
Például desztilláció során is a víz sokkal gyorsabban alakúl gázzá mint a benne oldott sók, ezért azokat hátrahagyva, a vízgőzben sokkal kisebb lesz az oldott anyag koncentráció.
Tehát a kémiai potenciál függését a következőképp kell módósítani:
\begin{equation}
    \mu_i^{(m)} = \mu_i (T, p, c_1^{(j)}, c_2^{(j)}, \ldots, c_{k-1}^{(j)}) \quad j = 1, 2, \ldots, m \quad 2 + m(k - 1) \text{ db. változó}
\end{equation}
Egyensúlyban pedig a kémiai potenciálnak meg kell egyeznie a különböző fázisok között:
\begin{equation}
    \mu_i^{(1)} = \mu_i^{(2)} = \ldots = \mu_i^{(m)} \qquad i = 1, 2, \ldots, k
\end{equation}
Ebből összesen $k(m - 1)$ egyenletet kapunk, amik a kémiai potenciál egyenlőségét írják le a különböző fázisok között.
Ezek az egyenletek pedig korlátozzák a szabadsági fokok számát, amit a következőképpen tudunk kiszámolni:
\begin{equation}
    F = \text{változók száma} - \text{összefüggések száma}
\end{equation}
\begin{equation}
    F = 2 + km - m - km + k = 2 + k - m
\end{equation}
Ez a \textbf{Gibbs-féle fázisszabály}. Ez az egyenlet megadja, hogy egy adott rendszerben hány szabadsági fok van jelen, figyelembe véve a komponensek és fázisok számát.
Például egy egykomponensű rendszerben ($k=1$), ha két fázis van jelen ($m=2$), akkor a szabadsági fokok száma:
\begin{equation}
    F = 2 + 1 - 2 = 1
\end{equation}
Ez azt jelenti, hogy egy intenzív állapotjelzőt szabadon változtathatunk anélkül, hogy a rendszer fázisainak száma megváltozna. Ha viszont három fázis van jelen ($m=3$), akkor a szabadsági fokok száma:
\begin{equation}
    F = 2 + 1 - 3 = 0
\end{equation}
Ez azt jelenti, hogy nincs szabadsági fok, vagyis az intenzív állapotjelzők értékei teljesen meghatározottak a rendszer fázisainak száma alapján, tehát csak a kritikus pontban lehetek.

\subsection{Termodinamika III. Főtétele}
Ha egy rendszerben a hőmérséklet és a nyomás állandó, de valamilyen kémiai reakció megy végbe, akkor az Gibbs-potenciál fog minimalizálódni:
\begin{equation}
    \Delta G \leq 0 \qquad \text{áll. } T, p
\end{equation}
\begin{equation}
    G = H - TS \quad \to \quad \Delta G = \Delta H - T \Delta S = Q - T \Delta S < 0
\end{equation}
Az entalpia változást fel tudjuk írni a következőképpen:
\begin{equation}
    \Delta H =  \Delta (U + pV) = \Delta U + p\Delta V = Q - p \Delta V + p \Delta V = Q
\end{equation}
Vannak olyan kémiai reakciók amelyek hőt termelnek (exoterm reakciók, $Q < 0$), és vannak olyanok amelyek hőt nyelnek el (endoterm reakciók, $Q > 0$). 
Ezek alapján a következő egyenlőtlenséget kapjuk:
\begin{equation}
    Q - T \Delta S < 0 \quad \to \quad Q < T \Delta S
\end{equation}
Ez a \textbf{Thomsen-Berthelot-szabály}. Ez az egyenlőtlenség azt fejezi ki, hogy egy kémiai reakció akkor megy végbe spontán módon, ha a felszabaduló hő mennyisége kisebb, mint a hőmérséklet és az entrópia változásának szorzata.
Ez azt jelenti, hogy egy exoterm reakció akkor lesz spontán, ha a hőmérséklet és az entrópia változásának szorzata nagyobb, mint a felszabaduló hő mennyisége.
Ezzel szemben egy endoterm reakció akkor lesz spontán, ha a hőmérséklet és az entrópia változásának szorzata kisebb, mint a felszabaduló hő mennyisége. Ezt Nernst dolgozta tovább
1907-ben amikor kimondta a Nernst-tételt:
\begin{quote}
    Egy izoterm folyamatban ha a hőmérséklet közelít a nullához, akkor az entrópia változás is közelít a nullához.
\end{quote}
Ez azt jelenti, hogy ha egy rendszer hőmérséklete nagyon alacsony, akkor az entrópia változása is nagyon kicsi lesz egy izoterm folyamat során. Ez a tétel a termodinamika harmadik főtételének alapja.
Ez a tétel már olyan 100\degree C alatti hőmérsékleteknél is érvényesül, ahol az anyagok már közelítik a kristályos állapotot. További kisérleti megfigyelés, hogy
a molhők is tartanak a nullához, ha a hőmérséklet közelít a nullához:
\begin{equation}
    \lim_{T \to 0} C_{V,n} = 0 \qquad \lim_{T \to 0} C_{p,n} = 0
\end{equation}
Ebből következik, hogy a hőkapacitás integrálja is véges lesz a nullához közelítve, így az entrópia is véges értéket fog felvenni a nullánál:
\begin{equation}
    S(T, p) = S(0, p) + \int_{0}^{T} \frac{C_{p,n}}{T} dT
\end{equation}
\begin{equation}
    S(0, p) = \lim_{T \to 0} S(T, p) < \infty
\end{equation}
Ebből következik a termodinamika III. főtétele:
\begin{quote}
    Tetszőleges tiszta, kristályos anyag entrópiája nulla, ha a hőmérséklet abszolút nullához közelít.
\end{quote}
Ez azt jelenti, hogy ha egy anyag teljesen rendezett kristályos állapotban van, akkor az entrópiája nulla lesz a nullához közelítő hőmérsékleten.
Ez a tétel fontos következményekkel jár a termodinamika és a kémia területén, például segít megérteni a kémiai reakciók irányát és a termodinamikai egyensúlyokat.
Ennek a tételnek van pár következménye is:
\begin{itemize}
    \item Abszolút nulla fokon az adiabata és az izoterma egybe esik, mivel az entrópia változás nulla.
    \item Abszolút nulla fokot csak végtelen sok lépésben lehet megközelíteni, mivel nincs egy adiabata amivel közvetlenül elérhető lenne.
    \item Abszolút nulla fokon a lehetséges állapotok száma egy, mivel az entrópia nulla: $S = k_B \ln \Omega = 0 \quad \to \quad \Omega = 1$. (Planck-tétel)
\end{itemize}

\section{Elektro- és magnetosztatika, áramkörök}
Coulomb- és Gauss-törvény, szuperpozíció elve. Vezetők, szigetelők, dielektromos polarizáció. Kondenzátor. Magnetosztatika, Lorentz-erő. Stacionárius áram, áramköri törvények: Kirchhoff-törvények, Ohm-törvény.

\subsection{Coulomb- és Gauss-törvény, szuperpozíció elve}












\end{document}